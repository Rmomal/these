\documentclass[a4paper,10pt]{article}
\usepackage[utf8]{inputenc}
\usepackage{amsmath,amssymb,amsthm,mathrsfs,amsfonts,dsfont,bm} 
\usepackage[margin=2.5cm]{geometry}

%opening
\title{EM algorithm}
\author{}
\DeclareMathOperator*{\argmax}{arg\,max}

\begin{document}

\maketitle



\section{Context}

We have observed data $Y$ and unobserved data $Z$. The goal is to compute the likelihood of the data, $p_\theta (Y)$.
\[ \log(p_\theta (Y)) = \log (p_\theta (Y,Z)) - \log(p_\theta (Z|Y)).\]

The advantage of this is to link $p_\theta(Y)$ with $p_\theta (Y,Z)$ which is easier to compute in general. We now take the expectation,
conditioned on the data $Y$ :
\[ \log(p_\theta (Y) = \mathds{E}_\theta \left(\log(p_\theta (Y,Z))|Y \right) \underbrace{- \mathds{E}_\theta \left(\log(p_\theta (Y|Z))|Y \right)}_{\text{\normalsize{$\mathcal{H}(p_\theta (Y|Z))$}}} \]

\paragraph{E step :}
In this step, we must be very cautious on to what is varying. During this computation, data $Y$ is fixed, leading the entropy term to be fixed as well.
We compute this expectation with fixed parameters, only the hidden part is varying. So from now on we are interested in the first term, which is the conditional expectation of the complete log-likelihood. \\

\paragraph{M step :}
We consider that  $\theta$ is varying and we want to maximise the expectation with respect to these parameters.


\section{Example for Gaussian mixture models}
The data Y is an array of dimension $n\times d$, being for example $n$ samples of $d$ different species. Let
$Y_i$ be the $i^{th}$ row (i.e. sample) of Y. We then assume that data from the species follow a mixture of K multivariate Gaussians :
\[\forall k\in\{1,..K\}, f_k(Y_i) = \frac{1}{\sqrt{(2\pi)^d\det(\Sigma_k)}}\exp\left(-\frac{1}{2}(Y_i - \mu_k)^T\Sigma_k^{-1}(Y_i - \mu_k)\right)\]
With $d$ being the size of both $Y_i$ and $\mu_k$. The covariance matrix $\Sigma_k$ has size $d\times d$.\\

\paragraph{E step :}



\begin{align*}
\log(p_\theta(Y,Z)) &= \sum_{i,k} \mathds{1}_{\{Z_i = k\}} \times \log(\pi_k f_k(Y_i)|Y)\\
\mathds{E}_\theta (\log(p_\theta(Y,Z))|Y)&= \sum_{i,k} \mathds{E}_\theta \left( \mathds{1}_{\{Z_i = k\}} |Y_i\right)[\log(\pi_k) + \log(f_k(Y_i)) ]\\
\intertext{We can estimate the expectation with $\tau_{ik} = \frac{\pi_k f_k(Y_i)}{\sum_{l} \pi_l f_l(Y_i)}$ :}
&= \sum_{i,k}\tau_{ik}[\log(\pi_k) + \log(f_k(Y_i)) ] \\
&= \sum_{i,k}\tau_{ik}\left[\log(\pi_k)-\frac{1}{2}\log\left((2\pi)^d\det(\Sigma_k)\right) - \frac{1}{2}(Y_i - \mu_k)^T\Sigma_k^{-1}(Y_i - \mu_k)\right] \\
\end{align*}


\paragraph{M step :}
Maximising the last expression, we get after some algebraic manipulations :
\begin{itemize}
\item \large{$\hat{\mu}_k = \frac{\sum_i \tau_{ik} y_i}{\sum_i \tau_{ik}}$}\normalsize
\item \large{$\hat{\Sigma}_k = \frac{\sum_i \tau_{ik} (y_i-\mu_k)^T(y_i-\mu_k)}{\sum_i \tau_{ik}}$}\normalsize
\item \large{$\hat{\pi}_k = \frac{1}{n} \sum_i \tau_{ik}$}
\end{itemize}

\section{Example for mixtures of Gaussian Dependence Trees}
Let $T$ be a standard gaussian dependence tree : all means are null and all variances are equal to 1. We are considering a mixture of hidden trees, $k$ an $l$  are nodes of the trees (
i.e. variables or species).\\

\[ \mathds{P}(T) = \frac{1}{B}\prod_{k,l\in T} \beta_{kl} \text{ , with } B = \sum_T \prod_{k,l\in T} \beta_{kl} \]
\begin{align*}
\mathds{P}(Y|T) &=\mathds{P}(y_1|T)\prod_{i=2}^n \frac{\mathds{P}(y_i,y_{a_i}|T)}{\mathds{P}(y_{a_i}|T)}\\
&=\prod_{i=2}^n \mathds{P}(y_i|T)\prod_{i=1}^n \frac{\mathds{P}(y_i,y_{a_i}|T)}{\mathds{P}(y_i|T)\times \mathds{P}(y_{a_i}|T)}\\
&=\underbrace{\prod_{i=1}^n \mathds{P}(y_i|T)}_{\text{A}}\prod_{i=1}^n\prod_{k,l \in T}\psi_{kl}(Y_i)
\end{align*}

We know (cf. Chow gaussian document) that 
\[\log(A) = \sum_{i=1}^n-\frac{1}{2}\left(\log(2\pi\sigma_i^2)-\frac{y_i^2}{\sigma_i^2}\right),\]
and this quantity is independant from the tree structure. We also know the explicit form of $\log(\psi_{kl})$ :
\[\log(\psi_{kl}(Y_i))=\frac{-1}{2}\left(\log\left(1-\frac{\sigma_{kl}^2}{\sigma_k^2 \sigma_{l}^2}\right)+\frac{(y_k^2\sigma_{l}^2+y_{l}^2\sigma_k^2-2\sigma_{kl}y_ky_{l})}
{\det(\Sigma_{kl})}-\left(\frac{y_k^2}{\sigma_k^2}+\frac{y_{l}^2}{\sigma_{l}^2}\right)\right)\]

Remembering that we work with standard normal distributions, the last two expressions are greatly simplified. The correlation $\rho_{kl}$ 
between $y_k$ and $y_l$ is now their covariance too, and after some manipulations:

\[\log(A) = \sum_{i=1}^n-\frac{1}{2}\left(\log(2\pi)-y_i^2\right)\]
\[\log(\psi_{kl}(Y_i))=\log\left(\frac{1}{\sqrt{1-\rho_{kl}^2}}\right)+\frac{\rho_{kl}}{1-\rho_{kl}^2}\cdot y_{ik}y_{il} - 
\frac{\rho_{kl}^2}{1-\rho_{kl}^2}\cdot \frac{y_{ik}^2 + y_{il}^2}{2}\]

\subsection{E step :}
\[ \mathds{P}(Y,T) = \mathds{P}(T)\times\mathds{P}(Y|T)\]
\begin{align*}
 \log(\mathds{P}(Y,T)) &= \sum_{(k,l)\in T} \log(\beta_{kl})  + \sum_{(k,l)\in T} \log(\psi_{kl}(Y))- \log (B)+\log(A)\\
 &=\sum_{k,l} \mathds{1}_{\{(k,l) \in T\}} \left(\log(\beta_{kl})  +  \log(\psi_{kl}(Y))\right)- \log (B)+\log(A)
\end{align*}
 Conditional expectation :
\[ \mathds{E}_\theta[\log(\mathds{P}(Y,T))|Y] \\
= \sum_{k,l}  \mathds{P}((k,l)\in T | Y) \times \left[ \log(\beta_{kl}) + \log(\psi_{kl}(Y)) \right]
 -\log(B)+\log(A)\]
 Computation of conditional probability : using Bayes, we specially consider the proportion of trees which contain an edge between the nodes $k$ and $l$.
 \begin{align*}
 \mathds{P}((k,l)\in T | Y)  &= \frac{\sum_{(k,l)\in T} \mathds{P}(T)\mathds{P}(Y|T)}{\sum_{T} \mathds{P}(T)\mathds{P}(Y|T)}\\
 &=1-\frac{\sum_{(k,l)\notin T} \prod \beta_{uv} \prod \psi_{uv}}{\sum_{T} \prod \beta_{uv} \prod \psi_{uv}}
 \end{align*}
 Defining $\Delta$ as the Laplacian $W_{\beta}$, the matrix  of weights $\beta_{kl}$, in can be shown that :

\begin{align*}
B &= \sum_{T \in \mathcal{T}} \prod_{k,l\in T} \beta_{kl}\\
 &=|\Delta^{\{u,v\}}(W_{\beta})|,
\end{align*}
where $\Delta^{\{u,v\}}$ is any minor of the Laplacien. We then get :

\[ \mathds{P}((k,l)\in T | Y) =1-\frac{|\Delta^{\{u,v\}}(W_{\beta}^{-kl}\bigodot\psi)|}{|\Delta^{\{u,v\}}(W_{\beta}\bigodot\psi)|}\]

Where the notation $W_{\beta}^{-kl}$ means that the entry at the $k^{\text{th}}$ line and $l^{\text{th}}$ has been set to zero
(concreatly, we erased the edge between nodes $k$ and $l$). This last quantity will be computed using the Krishner theorem, allowing for a great gain in computation time.


\subsection{M step :\\}
Moving to the M step, the quantity $\tau_{kl} = \mathds{P}((k,l)\in T | Y)$ has been computed and is now considered as fixed.
We maximise the conditional expectation with respect to parameters $\beta_{kl}$.

\[\argmax_{\beta_{kl}} \left\{\sum_{k,l} \tau_{kl}\times \left[ \log(\beta_{kl}) + \log(\psi_{kl}(Y)) \right]
 -\log(B)+\log(A)\right\}\]

 We derive with respect to $\beta_{kl}$:
 \[\frac{\partial\mathds{E}_\theta[\log(\mathds{P}(Y,T))|Y]}{\partial\beta_{kl}} =\frac{ \mathds{P}((k,l)\in T | Y)}{\beta_{kl}} - \frac{1}{B}
 \sum_{\substack{T\in \mathcal{T}:\\T \ni \{k,l\}}} \prod_{\substack{\{u,v\} :\\ u \ne k,\\v\ne l}} \beta_{uv}\]

 We notice that if we multiply the last part of the right term by $\beta_{kl}$, we obtain the probability for the edge $\{k,l\}$ to be in the tree $T$.
Thus we can write :

 \begin{equation}
 \label{1}
 \frac{\partial\mathds{E}_\theta[\log(\mathds{P}(Y,T))|Y]}{\partial\beta_{kl}} =\frac{ \mathds{P}((k,l)\in T | Y)}{\beta_{kl}} 
 -\frac{\mathds{P}((k,l)\in T)}{\beta_{kl}}  
 \end{equation}
\subsubsection{Update formula}
 We now stand in this awkward situation where the $\beta_{kl}$ will disappear if we want to optimise the quantity according to it. But let's 
 remember that we are in an EM algorithm here. Going from iteration $h$ to iteration $h+1$, we model the update on $\beta_{kl}$ as:
 \[\beta_{kl}^{h+1} = \lambda^h \beta_{kl}^h\]
Therefore, we want to determine $\lambda^h$. We adopt the following notations:
\[ \sum_{\substack{T\in \mathcal{T}:\\T \ni \{k,l\}}} \prod_{\{u,v\}} \beta_{uv}^h= S_{kl}^h\]
\[ \sum_{\substack{T\in \mathcal{T}:\\\{k,l\} \notin T }} \prod_{\{u,v\}} \beta_{uv}^h= S_{\backslash kl}^h\]

We now use a decomposition of the constant of normalisation $B$ :
\begin{align*}
B^{h+1}  &=\sum_{\{k,l\} \in T } \left(\prod_{\substack{u,v\in T\\ (u,v) \neq (k,l)}} \beta_{uv}^{h}\right) \times\beta_{kl}^{h+1} +\sum_{\{k,l\} \notin T }
 \prod_{u,v\in T} \beta_{uv}^{h}\\
 &=\lambda^h S_{kl}^h + S_{\backslash kl}^{h}
\end{align*}

Then, remembering \ref{1}, the optimal update of $\beta_{kl}$ would be such that :

\begin{align*}
 \mathds{P}((k,l)\in T | Y)&=\mathds{P}((k,l)\in T)\\
 &=\frac{\lambda^h S_{kl}^h}{\lambda^h S_{kl}^h+S_{\backslash kl}^{h}}
\end{align*}
\[\Leftrightarrow \lambda^h = \frac{S_{\backslash kl}^{h}}{S_{kl}^h}\times\frac{\mathds{P}((k,l)\in T | Y) }{1-\mathds{P}((k,l)\in T | Y)}\]

The conditional probability has just been computed at this point in the algorithm. As for the ratio of sums, Krishner's theorem allows us
to compute it in a reasonable amout of time, after using the Matrix tree theorem which says :


\begin{align*}
  \begin{array}{r@{}l}
    B & =|\Delta^{\{u,v\}}(W_{\beta})| \\
    S_{\backslash kl}^{h}& =|\Delta^{\{u,v\}}(W_{\beta}^{- kl})|
  \end{array}
 \end{align*}
 
 \subsubsection{Direct formula}
 We are also able to give a direct formula for $\beta_{kl}$, by noticing the following relation between a matrix minors :
 \[|\Delta^{\{a,b\}}(A)|=|\Delta^{\{a,b\}}(A^{- ab})| - [\Delta^{\{a,b\}}(A)]_{ab}\times Cof_{ab}(\Delta^{\{ab\}}(A)).\]\\
 In the last expression we used $Cof_{ab}(A)$ to refer to the $(a,b)$ term of the matrix of the cofactor matrix of the $A$ matrix. The minus in 
 the second term comes from the signs of minors in the development of the determinent. As A row and a column of the Laplacian matrix have
 already be removed, the signing in the formula is shifted and a minus appears. In our case with the matrix of weights $W_\beta$, we obtain this expression :
\[|\Delta^{\{k,l\}}(W_\beta)|=|\Delta^{\{k,l\}}(W_\beta^{- kl})| + \beta_{kl}\times Cof_{kl}(\Delta^{\{kl\}}(W_\beta)).\]\\

By definition we have :
\[\mathds{P}((k,l)\in T ) = \frac{|\Delta^{\{k,l\}}(W_\beta)|-|\Delta^{\{k,l\}}(W_\beta^{- kl})|}{|\Delta^{\{k,l\}}(W_\beta)|}\]
\[\Leftrightarrow  \text{\hspace{0.5cm}}\mathds{P}((k,l)\in T ) =\frac{\beta_{kl}\times Cof_{kl}(\Delta^{\{kl\}}(W_\beta))}{|\Delta^{\{k,l\}}(W_\beta)|}\]
Replacing this expression in \ref{1}, we get:

\begin{align*}
 &\frac{ \mathds{P}((k,l)\in T | Y)}{\beta_{kl}} -\frac{\mathds{P}((k,l)\in T)}{\beta_{kl}} =0 \\
 \Leftrightarrow \text{\hspace{0.5cm}}& \frac{ \mathds{P}((k,l)\in T | Y)}{\beta_{kl}} -\frac{Cof_{kl}(\Delta^{\{kl\}}(W_\beta))}{|\Delta^{\{k,l\}}(W_\beta)|}=0\\
 \Leftrightarrow \text{\hspace{0.5cm}}& \hat{\beta}_{kl} = \mathds{P}((k,l)\in T | Y) \times \frac{|\Delta^{\{k,l\}}(W_\beta)|}{Cof_{kl}(\Delta^{\{kl\}}(W_\beta))}
\end{align*}

The drawback of this formula is its high computational complexity. However we can use it for ponctual checks.
\end{document}
