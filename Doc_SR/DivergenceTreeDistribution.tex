\documentclass[a4paper, 10pt]{article}

%% Language and font encodings
\usepackage[english]{babel}
\usepackage[utf8x]{inputenc}
\usepackage[T1]{fontenc}

%% Sets page size and margins
\usepackage[a4paper,top=2cm,bottom=2cm,left=2cm,right=2cm,marginparwidth=1.75cm]{geometry}

%% Useful packages
\usepackage{amsfonts,amssymb,amsmath,amscd,amsthm,latexsym}
\usepackage{tikz}
\usepackage{graphicx}
\usepackage{enumerate}
\usepackage[colorinlistoftodos]{todonotes}
\usepackage[colorlinks=true, allcolors=blue]{hyperref}

\newcommand{\Tcal}{\mathcal{T}}

\title{Divergence between distributions over spanning trees}
\author{SR}
\date{\today}

%%%%%%%%%%%%%%%%%%%%%%%%%%%%%%%%%%%%%%%%%%%%%%%%%%%%%%%%%%%%%%%%%%%%%%%%%%%%%%%
%%%%%%%%%%%%%%%%%%%%%%%%%%%%%%%%%%%%%%%%%%%%%%%%%%%%%%%%%%%%%%%%%%%%%%%%%%%%%%%
\begin{document}
%%%%%%%%%%%%%%%%%%%%%%%%%%%%%%%%%%%%%%%%%%%%%%%%%%%%%%%%%%%%%%%%%%%%%%%%%%%%%%%

\maketitle

%%%%%%%%%%%%%%%%%%%%%%%%%%%%%%%%%%%%%%%%%%%%%%%%%%%%%%%%%%%%%%%%%%%%%%%%%%%%%%%
\section{Problem} 
%%%%%%%%%%%%%%%%%%%%%%%%%%%%%%%%%%%%%%%%%%%%%%%%%%%%%%%%%%%%%%%%%%%%%%%%%%%%%%%

Consider two distributions $F$ and $G$ over the set of spanning $\Tcal$ of a complete graph with $p$ nodes. Both $F$ and $G$ are supposed to be factorizable, that is:
$$
\forall T \in \Tcal, \qquad
f(T) = \prod_{j, k \in T} a_{jk} / A, \qquad
g(T) = \prod_{j, k \in T} b_{jk} / B.
$$

\paragraph{Some facts.}
\begin{enumerate}
 \item The number of spanning trees (i.e. the cardinal of $\Tcal$) is $p^{p-2}$.
 \item The number of spanning trees containing a given edge is $2p^{p-3}$.
 \item The normalizing constants $A$ and $B$ can be computed using the Matrix-Tree theorem \cite{Cha82}, which enables to compute efficiently any quantity of the form
 \begin{equation} \label{eq:MTT}
 \sum_{T \in \Tcal} \prod_{jk \in T} x_{jk}.
 \end{equation}
 \item We denote by $F_{jk}$ (resp. $G_{jk}$) the probability for edge $(j, k)$ to be part of $T$ under $F$ (resp. $G$), that is
 $$
 F_{jk} = \sum_{T \ni jk} f(T).
 $$
 All $F_{jk}$ and $G_{jk}$ can be computed using an avatar of the Matrix-Tree theorem \cite{Kir07}.
\end{enumerate}

\paragraph{Problem:} 
Compute some distance or divergence between $F$ and $G$.

%%%%%%%%%%%%%%%%%%%%%%%%%%%%%%%%%%%%%%%%%%%%%%%%%%%%%%%%%%%%%%%%%%%%%%%%%%%%%%%
\section{Some computable divergences} 
%%%%%%%%%%%%%%%%%%%%%%%%%%%%%%%%%%%%%%%%%%%%%%%%%%%%%%%%%%%%%%%%%%%%%%%%%%%%%%%

%%%%%%%%%%%%%%%%%%%%%%%%%%%%%%%%%%%%%%%%%%%%%%%%%%%%%%%%%%%%%%%%%%%%%%%%%%%%%%%
\subsection{$\alpha$-divergences} 

\paragraph{Definition.} \cite{Min05}
$$
D_\alpha(F || G) = \frac1{\alpha (1-\alpha)} \left(1 - \sum_{T \in \Tcal} f(T)^\alpha g(T)^{1 - \alpha} \right)
$$

\paragraph{Property.}  All $\alpha$-divergences are comptable via the Matrix-Tree theorem. Indeed
$$
\sum_{T \in \Tcal} f(T)^\alpha g(T)^{1 - \alpha} 
 = \frac{B}{A} \sum_{T \in \Tcal} \prod_{jk \in T} (a_{jk})^{\alpha} (b_{jk})^{1 - \alpha},
$$
which has the form \eqref{eq:MTT}.

%%%%%%%%%%%%%%%%%%%%%%%%%%%%%%%%%%%%%%%%%%%%%%%%%%%%%%%%%%%%%%%%%%%%%%%%%%%%%%%
\subsection{Bregman divergences} 

\paragraph{Definition.} 
\url{https://en.wikipedia.org/wiki/Bregman_divergence}

\paragraph{Examples.} 

\paragraph{Property.}  The Küllback-Leibler divergence is comptable via the Matrix-Tree theorem. Indeed
\begin{align*}
 KL(F || G) 
 & = \sum_{T \in \Tcal} f(T) \log \frac{f(T)}{g(T)} 
 \qquad = \sum_{T \in \Tcal}  f(T) \sum_{jk \in T} \log \frac{a_{jk}}{b_{jk}} + \log \frac{B}{A} \\
 & = \sum_{jk} \log \frac{a_{jk}}{b_{jk}} \left(\sum_{t \ni jk} f(T) \right) 
 + \log \frac{B}{A} 
 \qquad = \sum_{jk} F_{jk} \log \frac{a_{jk}}{b_{jk}} + \log \frac{B}{A}.
\end{align*}


\paragraph{Property.}  The Itakura-Saito divergence is comptable via the Matrix-Tree theorem. Indeed
$$
IS(F || G) 
= \sum_{T \in \Tcal} \left( \frac{f(T)}{g(T)} - \log \frac{f(T)}{g(T)} - 1 \right)
$$
where
$$
\sum_{T \in \Tcal} \frac{f(T)}{g(T)}
= \frac{B}{A} \sum_{T \in \Tcal}  \prod_{jk \in T} \frac{a_{jk}}{b_{jk}}
$$
which has the form \eqref{eq:MTT}, and
\begin{align*}
 \sum_{T \in \Tcal} \log \frac{f(T)}{g(T)} 
 & = \sum_{T \in \Tcal} \log \frac{B}{A} + \sum_{T \in \Tcal} \sum_{jk \in T} \log \frac{a_{jk}}{b_{jk}} 
 \qquad = p^{p-2} \log \frac{B}{A} + \sum_{jk} \#\{T \in T: jk \in T\} \log \frac{a_{jk}}{b_{jk}}  \\
 & = p^{p-2} \log \frac{B}{A} + 2p^{p-3}\sum_{jk} \log \frac{a_{jk}}{b_{jk}} 
\end{align*}
so
$$
IS(F || G) 
= \frac{B}{A} \sum_{T \in \Tcal} \prod_{jk \in T} \frac{a_{jk}}{b_{jk}} 
- p^{p-2} \log \frac{B}{A} - 2p^{p-3}\sum_{jk} \log \frac{a_{jk}}{b_{jk}}
- p^{p-2}.
$$


%%%%%%%%%%%%%%%%%%%%%%%%%%%%%%%%%%%%%%%%%%%%%%%%%%%%%%%%%%%%%%%%%%%%%%%%%%%%%%%
\bibliography{/home/robin/Biblio/BibGene}
\bibliographystyle{plain}

%%%%%%%%%%%%%%%%%%%%%%%%%%%%%%%%%%%%%%%%%%%%%%%%%%%%%%%%%%%%%%%%%%%%%%%%%%%%%%%
%%%%%%%%%%%%%%%%%%%%%%%%%%%%%%%%%%%%%%%%%%%%%%%%%%%%%%%%%%%%%%%%%%%%%%%%%%%%%%%
\end{document}
%%%%%%%%%%%%%%%%%%%%%%%%%%%%%%%%%%%%%%%%%%%%%%%%%%%%%%%%%%%%%%%%%%%%%%%%%%%%%%%
