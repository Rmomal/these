\documentclass[10pt,a4paper]{article}
\usepackage[utf8]{inputenc}
\usepackage{amsmath}
\usepackage{amsfonts, dsfont}
\usepackage{amssymb}
\usepackage{xcolor}
\usepackage[margin=3.5cm]{geometry}
\author{Raphaelle Momal}
\title{PLN spatial}


\newcommand{\esp}{\mathds{E}}
\newcommand{\var}{\mathds{V}}
\newcommand{\RM}{\textcolor{purple}}
\begin{document}
\maketitle
A variogram in geostatistics is defined for an intrinsically stationary spatial process Z as follows, for sites $s$ and $t$ separated by a distance $d$:
\begin{align*}
 \gamma (d) &=  \frac{1}{2} \esp[(Z(s) - Z(t) )^2]\\
 &=\frac{1}{2} \left(\var(Z(s)) + \esp[Z(s))^2 + \var(Z(t)) + \esp[Z(t))^2 \right) - \esp(Z(s)Z(t))
\end{align*}

We wish to write the theoretical variogram formula for the PLN model.
\section{Variogram for  Y}
We recall the PLN model :
\begin{align*}
Z_s &\sim \mathcal{N}(0,\Sigma)\\
Y_{sj}|Z_{sj} &\sim \mathcal{P}(e^{x_s \beta_j + Z_{sj}}) 
\end{align*}

In what follows we want to account for a spatial structure of tha data. This means we will need to differentiate between two sites and two species at the same time. Therefore we write $Cov(Z) = [\Gamma]_{stjk}$ where $s$ and $t$ are sites and $j$ and $k$ are species. Two models are under consideration here:
\begin{enumerate}
\item $\Gamma_{stjk} = \delta_{st} \sigma_{jk}$
\item $\Gamma_{stjk} = \delta_{st} + \mathds{1}_{s=t}\sigma_{jk}$
\end{enumerate}
Where $\delta$ and $\sigma$ represent respectively the spatial and inter-species dependence.  We  assume 
\begin{itemize}
\item [(a)] a stationarity in spatial variance, that is that $\delta_{uu} = cst$ for any site $u$. This means that  a behavior can be spatially translated. 
\item [(b)] The inter-species dependence only  intervenes  on a same site, so furthermore two species in two different sites will behave independently.  

\end{itemize}For clarity sake, we write $\Gamma_{ssjj}$ as $\Gamma_{j}$, according to the assumption of spatial variance stationarity. 

\RM{\begin{itemize}
\item Est-ce que (b) implique que $\Gamma_{stj} = \Gamma_{st}$ ? \item En suivant (b), devrait-on écrire le modèle 1 comme 
$$ \Gamma_{stjk} =   \mathds{1}_{s\neq t} \delta_{st} + \mathds{1}_{s=t} \delta_{st} \sigma_{jk}$$  
\end{itemize} }
\subsection{Y characteristics}
We know the expectation, variance and covariance expressions for the PLN model. When only one species is considered, we have:
\begin{align*}
\esp[Y_{sj}] &= e^{x_s\beta _j- \Gamma_{j}^2/2} = \mu_{sj}\\
\var(Y_{sj}) &= \mu_{sj}+\mu_{sj}^2(e^{\Gamma_{j}^2} - 1)\\
\esp[Y_{sj}Y_{tj}]&= \mu_{sj}\mu_{tj} e^{\Gamma_{stj}}
\end{align*}

\subsection{Same species in two sites}
For sites $s$ and $t$ at a distance $d$:

\begin{align*}
\gamma_Y(d) &= \frac{1}{2} \left[ \mu_s + \mu_s^2(e^{\Gamma^2_j}-1) + \mu_s^2 + \mu_t+\mu_t^2(e^{\Gamma^2_j}-1) +\mu_t\right] - \mu_s\mu_te^{\Gamma_{stj}}\\
&= \frac{1}{2} \left[ ( \mu_s^2+\mu_t^2)e^{\Gamma^2_j}+ \mu_s+ \mu_t\right] - \mu_s\mu_te^{\Gamma_{stj}}\\
\end{align*}
We then define $\tilde{Y}_{sj}= Y_{sj}/\mu_{sj}$.
It  comes that 
\begin{align*}
 \hat{\gamma}_{\tilde{Y}}(d) &= \frac{1}{2} \esp \left[(\tilde{Y}_{sj}-\tilde{Y}_{sj})^2 - (\frac{1}{\mu_s}+\frac{1}{\mu_t})\right]\\
 &=e^{\Gamma_j^2} - e^{\Gamma_{st}}
\end{align*}
$ \hat{\gamma}_{\tilde{Y}}$ resembles the classical expression for the semi-variogram of a stationary spatial process, except it is a difference of exponentials rather than a difference of direct variance and covariance. 
\RM{indice en s ou en sj, il faut choisir}
\subsection{Two species in two sites}
We recall that
$$\esp[Y_{sj}Y_{tk}]= \mu_{sj}\mu_{tk} e^{\Gamma_{stjk}}.$$


Similarly to the above paragraph, we get:
$$ \esp[(Y_{sj} - Y_{tk})^2] = \left[ \mu_{sj}+\mu_{tk} + \mu_{sj}^2 e^{\Gamma_{sj}^2} + \mu_{tk}^2 e^{\Gamma_{tk}^2}\right] - \mu_{sj}\mu_{tk}e^{\Gamma_{stjk}}$$
Then, $$\hat{\gamma}_{\tilde{Y}_{jk}}(d) = \frac{1}{2} \left(e^{\Gamma_{sj}^2}+ e^{\Gamma_{tk}^2}\right) - e^{\Gamma_{stjk}}$$

\section{Perspectives}
\begin{itemize}
\item Culture sur INLA, approximation de Laplace et courbure Gaussienne
\item Vraisemblance composite, éventuellement utilisée sur les arbres, pour approcher $p(Z|Y)$.
\end{itemize}
\end{document}