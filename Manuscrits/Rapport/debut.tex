\documentclass[12pt]{phd}
 %%%%%%%%% PAckage %%%%%%%%%
%% intro 

\usepackage{enumerate}
\usepackage{smartdiagram}
\usepackage{graphicx,amsmath,amssymb,stmaryrd, amsthm}
\usepackage{ulem}
\usepackage{pifont}
\usepackage{amsfonts}
\usepackage{fancyhdr} 
\usepackage{color} % où xcolor selon l'installation
\usepackage{mdframed}
\usepackage{multirow} %% Pour mettre un texte sur plusieurs rangées
\usepackage{multicol} %% Pour mettre un texte sur plusieurs colonnes
\usepackage{tikz}
\usepackage[absolute]{textpos} 
\usepackage{colortbl}
\usepackage{array}
\usepackage{lineno}
\usepackage[ruled,vlined]{algorithm2e}
 \usepackage{float}
%% JMVA 
\usepackage{enumitem,hyperref,ifthen,dsfont}

%%SAGMB
\usepackage{microtype}
\usepackage{epstopdf}

%%block
\usepackage{multirow}
\setlength\parindent{0pt}

% Functions
\def\independenT#1#2{\mathrel{\rlap{$#1 #2$}\mkern2mu{#1 #2}}}
\DeclareMathOperator*{\argmax}{arg\,max}
\DeclareMathOperator*{\argmin}{arg\,min}
%\DeclareMathOperator*{\Esp}{\mathbb{E}}
%\DeclareMathOperator*{\Var}{\mathbb{V}}
\DeclareMathOperator*{\Cov}{\mathbb{C}\text{ov}}
\DeclareMathOperator*{\prob}{\mathds{P}}
\DeclareMathOperator*{\probt}{\widetilde{\mathds{P}}}
\DeclareMathOperator{\sign}{sign}
\def\lambdaMax{\lambda_{\rm{max}}}
\def\lambdaMin{\lambda_{\rm{min}}}

% math
\newtheorem{theorem}{Theorem}[chapter]
\newtheorem{prop}{Proposition}[chapter]
\newtheorem{lemma}{Lemma}[chapter]
\newtheorem{definition}{Definition}[chapter]
\newtheorem{remark}{Remark}[chapter]
\newtheorem{example}{Example}[chapter]
\newcommand{\Esp}{{\mathds{E}}}
\newcommand{\Var}{{\mathds{V}}}
\newcommand{\tr}[1]{\ensuremath{\rm tr}\left( #1 \right)}
\newcommand{\ve}{\ensuremath{\rm vec}}
\newcommand{\entr}{\mathcal{H}}
\newcommand{\corHTemp}{{\rho(H, \text{temp})}}
\newcommand{\corKTemp}{{\rho(k, \text{temp})}}
\newcommand{\STAB}[1]{\begin{tabular}{@{}c@{}}#1\end{tabular}}
\newcommand*\xbar[1]{%
   \hbox{%
     \vbox{%
       \hrule height 0.5pt % The actual bar
       \kern0.5ex%         % Distance between bar and symbol
       \hbox{%
         \kern-0.1em%      % Shortening on the left side
         \ensuremath{#1}%
         \kern-0.1em%      % Shortening on the right side
       }%
     }%
   }%
} 


% Notations pairs
\newcommand\jk{{jk}}

% Notations cal
\newcommand\Ccal{\mathcal{C}}
\newcommand\Ncal{\mathcal{N}}
\newcommand\Pcal{\mathcal{P}}
\newcommand\Tcal{\mathcal{T}}
\newcommand\G{\mathcal{G}}
%\newcommand\Ccal{{\mathcal{C}}}
\newcommand\Hcal{{\mathcal{H}}}
\newcommand\Jcal{{\mathcal{J}}}
%\newcommand\Ncal{{\mathcal{N}}}
%\newcommand\Pcal{{\mathcal{P}}}
%\newcommand\Tcal{{\mathcal{T}}}
\newcommand{\bound}{{\mathcal{J}}} 

% Notations bf
\newcommand\gammab{{\boldsymbol{\gamma}}}
\newcommand\betab{{\boldsymbol{\beta}}}
\newcommand\thetab{{\boldsymbol{\theta}}}
\newcommand\lambdab{{\boldsymbol{\lambda}}}
\newcommand\Lambdab{{\boldsymbol{\Lambda}}}
\newcommand\Sigmab{{\boldsymbol{\Sigma}}}
\newcommand\Omegab{{\boldsymbol{\Omega}}}
\newcommand\cst{\text{cst}}
\newcommand\Ob{{\bf O}}
\newcommand\Hb{{\bf H}}
\newcommand\Mbf{{\bf M}}
\newcommand\Qbf{{\bf Q}}
\newcommand\Abf{{\bf A}}
\newcommand\Wbf{{\bf W}}
\newcommand\Mb{{\bf M}}
\newcommand\Qb{{\bf Q}}
\newcommand\Wb{{\bf W}}
\newcommand\Xb{{\bf X}}
\newcommand\xb{{\boldsymbol{x}}}
\newcommand\Yb{{\bf Y}}
\newcommand\Zb{{\bf Z}}
\newcommand\zb{{\boldsymbol{z}}}
\newcommand\yb{{\boldsymbol{y}}}
\newcommand\Ibb{\mathbb{I}}
\newcommand{\betabf}{{\boldsymbol{\beta}}}
\newcommand{\thetabf}{{\boldsymbol{\theta}}}
\newcommand{\sigmabf}{{\boldsymbol{\sigma}}}
\newcommand{\Omegabf}{{\boldsymbol{\Omega}}}
\newcommand{\Sigmabf}{{\boldsymbol{\Sigma}}}
\newcommand{\Gammabf}{{\boldsymbol{\Gamma}}}
\newcommand{\zerobf}{{\boldsymbol{0}}}
\newcommand{\Xbf}{{\boldsymbol{X}}}
\newcommand{\xbf}{{\boldsymbol{x}}}
\newcommand{\Ybf}{{\boldsymbol{Y}}}
\newcommand{\Zbf}{{\boldsymbol{Z}}}
\newcommand{\hbf}{{\boldsymbol{h}}}
\newcommand{\Hbf}{{\boldsymbol{H}}}
\newcommand{\Ubf}{{\boldsymbol{U}}}
%\newcommand{\Mbf}{{\boldsymbol{M}}}
%\newcommand{\Qbf}{{\boldsymbol{Q}}}
\newcommand{\Rbf}{{\boldsymbol{R}}}
\newcommand{\Sbf}{{\boldsymbol{S}}}
\newcommand{\mbf}{{\boldsymbol{m}}}

% Notations tilde
\newcommand\Pt{\widetilde{P}}
\newcommand\pt{\widetilde{p}}
\newcommand\et{\widetilde{\mathds{E}}}
\newcommand\e{{\mathds{E}}}
\newcommand{\betabft}{{\widetilde{\betabf}}}
\newcommand{\Mbft}{{\widetilde{\Mbf}}}
\newcommand{\Sbft}{{\widetilde{\Sbf}}}
\newcommand\mt{\widetilde{m}}
\newcommand\St{\widetilde{S}}
\newcommand\mbt{\widetilde{\bf m}}
\newcommand\Sbt{\widetilde{\bf S}}
\newcommand{\betat}{{\widetilde{\beta}}}
\newcommand{\Bt}{{\widetilde{B}}}
%\newcommand{\Pt}{{\widetilde{P}}}

% TikZ
\newcommand{\nodesize}{0.4em}
\newcommand{\edgeunit}{6*\nodesize}
\tikzstyle{covariate}=[draw, rectangle, minimum width=\nodesize, minimum height=\nodesize, inner sep=0, color=black]
\tikzstyle{covmiss}=[draw, minimum width=\nodesize, minimum height=\nodesize, inner sep=0, color=gray, text=gray]
\tikzstyle{observed}=[draw, circle, minimum width=\nodesize, inner sep=0, color=black]
\tikzstyle{edge}=[-, line width=1pt, color=black]
\tikzstyle{edgemiss}=[-, line width=1pt, dashed, color=gray]
\tikzstyle{basic}=[draw, circle, minimum width=\nodesize, inner sep=0, color=black, fill=black]
\newcommand{\length}{1}
 \tikzstyle{clique}=[draw, rectangle, minimum width=2*\nodesize, minimum height=2*\nodesize, inner sep=0, color=black]
 \tikzstyle{variable}=[scale=0.9,rectangle,draw=white,transform shape,fill=white,font=\Large]

%coloring
\newcommand{\review}[1]{\textcolor{black}{#1}}
\newcommand{\validSR}[1]{\textcolor{black}{#1}}
\newcommand{\questSR}[1]{\textcolor{black}{[#1]}}
\newcommand{\newmodif}[1]{\textcolor{black}{#1}}
\newcommand{\modif}[1]{\textcolor{black}{#1}}
\newcommand{\remove}[1]{\textcolor{gray}{#1}}
\newcommand\independent{\protect\mathpalette{\protect\independenT}{\perp}}

%%%%%%%%%%%%%%%%%%%%%%
% for vignette from Rmd files
%\usepackage{lmodern}
%\usepackage{amssymb,amsmath}
%\usepackage{ifxetex,ifluatex}
\usepackage{fixltx2e} % provides \textsubscript
%\ifnum 0\ifxetex 1\fi\ifluatex 1\fi=0 % if pdftex
%  \usepackage[T1]{fontenc}
%  \usepackage[utf8]{inputenc}
%\else % if luatex or xelatex
%  \ifxetex
%    \usepackage{mathspec}
%  \else
%    \usepackage{fontspec}
%  \fi
%  \defaultfontfeatures{Ligatures=TeX,Scale=MatchLowercase}
%\fi
% use upquote if available, for straight quotes in verbatim environments
%\IfFileExists{upquote.sty}{\usepackage{upquote}}{}
% use microtype if available
%\IfFileExists{microtype.sty}{%
%\usepackage{microtype}
%\UseMicrotypeSet[protrusion]{basicmath} % disable protrusion for tt fonts
%}{}
%\urlstyle{same}  % don't use monospace font for urls
\usepackage{color}
\usepackage{fancyvrb}
\newcommand{\VerbBar}{|}
\newcommand{\VERB}{\Verb[commandchars=\\\{\}]}
\DefineVerbatimEnvironment{Highlighting}{Verbatim}{commandchars=\\\{\}}
% Add ',fontsize=\small' for more characters per line
\usepackage{framed}
\definecolor{shadecolor}{RGB}{248,248,248}
\newenvironment{Shaded}{\begin{snugshade}}{\end{snugshade}}
\newcommand{\KeywordTok}[1]{\textcolor[rgb]{0.13,0.29,0.53}{\textbf{#1}}}
\newcommand{\DataTypeTok}[1]{\textcolor[rgb]{0.13,0.29,0.53}{#1}}
\newcommand{\DecValTok}[1]{\textcolor[rgb]{0.00,0.00,0.81}{#1}}
\newcommand{\BaseNTok}[1]{\textcolor[rgb]{0.00,0.00,0.81}{#1}}
\newcommand{\FloatTok}[1]{\textcolor[rgb]{0.00,0.00,0.81}{#1}}
\newcommand{\ConstantTok}[1]{\textcolor[rgb]{0.00,0.00,0.00}{#1}}
\newcommand{\CharTok}[1]{\textcolor[rgb]{0.31,0.60,0.02}{#1}}
\newcommand{\SpecialCharTok}[1]{\textcolor[rgb]{0.00,0.00,0.00}{#1}}
\newcommand{\StringTok}[1]{\textcolor[rgb]{0.31,0.60,0.02}{#1}}
\newcommand{\VerbatimStringTok}[1]{\textcolor[rgb]{0.31,0.60,0.02}{#1}}
\newcommand{\SpecialStringTok}[1]{\textcolor[rgb]{0.31,0.60,0.02}{#1}}
\newcommand{\ImportTok}[1]{#1}
\newcommand{\CommentTok}[1]{\textcolor[rgb]{0.56,0.35,0.01}{\textit{#1}}}
\newcommand{\DocumentationTok}[1]{\textcolor[rgb]{0.56,0.35,0.01}{\textbf{\textit{#1}}}}
\newcommand{\AnnotationTok}[1]{\textcolor[rgb]{0.56,0.35,0.01}{\textbf{\textit{#1}}}}
\newcommand{\CommentVarTok}[1]{\textcolor[rgb]{0.56,0.35,0.01}{\textbf{\textit{#1}}}}
\newcommand{\OtherTok}[1]{\textcolor[rgb]{0.56,0.35,0.01}{#1}}
\newcommand{\FunctionTok}[1]{\textcolor[rgb]{0.00,0.00,0.00}{#1}}
\newcommand{\VariableTok}[1]{\textcolor[rgb]{0.00,0.00,0.00}{#1}}
\newcommand{\ControlFlowTok}[1]{\textcolor[rgb]{0.13,0.29,0.53}{\textbf{#1}}}
\newcommand{\OperatorTok}[1]{\textcolor[rgb]{0.81,0.36,0.00}{\textbf{#1}}}
\newcommand{\BuiltInTok}[1]{#1}
\newcommand{\ExtensionTok}[1]{#1}
\newcommand{\PreprocessorTok}[1]{\textcolor[rgb]{0.56,0.35,0.01}{\textit{#1}}}
\newcommand{\AttributeTok}[1]{\textcolor[rgb]{0.77,0.63,0.00}{#1}}
\newcommand{\RegionMarkerTok}[1]{#1}
\newcommand{\InformationTok}[1]{\textcolor[rgb]{0.56,0.35,0.01}{\textbf{\textit{#1}}}}
\newcommand{\WarningTok}[1]{\textcolor[rgb]{0.56,0.35,0.01}{\textbf{\textit{#1}}}}
\newcommand{\AlertTok}[1]{\textcolor[rgb]{0.94,0.16,0.16}{#1}}
\newcommand{\ErrorTok}[1]{\textcolor[rgb]{0.64,0.00,0.00}{\textbf{#1}}}
\newcommand{\NormalTok}[1]{#1}
\usepackage{graphicx,grffile}
%\makeatletter
%\def\maxwidth{\ifdim\Gin@nat@width>\linewidth\linewidth\else\Gin@nat@width\fi}
%\def\maxheight{\ifdim\Gin@nat@height>\textheight\textheight\else\Gin@nat@height\fi}
%\makeatother
% Scale images if necessary, so that they will not overflow the page
% margins by default, and it is still possible to overwrite the defaults
% using explicit options in \includegraphics[width, height, ...]{}
%\setkeys{Gin}{width=\maxwidth,height=\maxheight,keepaspectratio}
%\IfFileExists{parskip.sty}{%
%\usepackage{parskip}
%}{% else
%\setlength{\parindent}{0pt}
%\setlength{\parskip}{6pt plus 2pt minus 1pt}
%}
\setlength{\emergencystretch}{3em}  % prevent overfull lines
\providecommand{\tightlist}{%
  \setlength{\itemsep}{0pt}\setlength{\parskip}{0pt}}
\setcounter{secnumdepth}{0}
% Redefines (sub)paragraphs to behave more like sections
%\ifx\paragraph\undefined\else
%\let\oldparagraph\paragraph
%\renewcommand{\paragraph}[1]{\oldparagraph{#1}\mbox{}}
%\fi
%\ifx\subparagraph\undefined\else
%\let\oldsubparagraph\subparagraph
%\renewcommand{\subparagraph}[1]{\oldsubparagraph{#1}\mbox{}}
%\fi