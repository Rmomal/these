\paragraph{ideas}

 
\section{Unresolved details of the presented approach}
\begin{itemize}
\item model selection: not working and why (pb théorique stat selection dans modele inference variationnel)
\item offsets: very important and not so obvious in some cases (pb modélisation, pas forcément facile à avoir)
\item un mot sur la clique, pb algo
\end{itemize} 

\section{natural extensions}
state the model clearly, as it holds for the whole section
\subsection{Other free results}
estimation de la vraie matrice omega à partir de sigma et de la structure inférée, transformée en graph chordal si besoin (sinon pas décomposable et pas lauritzen)
contribution d'une arête
\subsection{Network comparison}
comparing networks = comparing laws on trees, avec divergence KL symmétrisée par ex
(si temps pour exemple, MDS 2 premiers axes)

\subsection{Emission law}
 loi d'émission différente : données mixtes (popovic), données multidimensionnelles sur les noeuds (au lieu d'univarié) article martina, extension pour les comptages multiatribute gaussian gm
 
la mécanique d'inférence de réseau reste la même, parce qu'on conserve la couche Gaussienne. En revanche boulot pour estimation des paramètres de cette loi d'émission.

\subsection{Spatial dependence}
how to not model the spatial dependency:
\begin{itemize}
\item covariates. Because close sites are similar (environmental similarity)
\item missing actor
\end{itemize}
if it's an absolute necessity, modèle kronecker en regardant les sites 2 à 2 pour chaque espèces et les espèces 2 à 2 pour chaque site. Vraisemblance composite.
ovaskainen hmsc fait le job, à vérifier

\section{Network inference in the observed layer}
Sortie en sentiers non-gaussiens
\begin{itemize}
\item comment simuler des comptages avec structure de dépendance
\item par paires avec arbre maximal chow \& liu
\item factorization par paire avec n'importe quelle loi de comptage bivariée. 
\item Moyenne d'arbre possible et vraisemblance composite pour évaluer les modèles
\end{itemize}
