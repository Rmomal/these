%chap3

 % intro
%Les réseaux :
%- [ ] C’est quoi
%- [ ] À quoi ça sert (les enjeux, appliqué)
%- [ ] Quelles sont les questions qui se posent (que dit la recherche)
%
%- [ ] Lauritzen pour les nuls
%
%L’inférence de réseaux :
%- [ ] Quelle méthode pour quel réseau
%- [ ] Auxquelles on se compare, pourquoi elles sont comme ça comment elles marchent et ce qui manque

  \section*{Biological context}
 \subsection*{Networks}
 A network is an intuitive object which anyone can easily relate to. It is first of all a graphical tool representing the links between different entities. This helps understand how a system organizes and have a direct image of it. It is also an analysis tool which can unravel sensible information about the system, its structure and the different roles in its organization. Networks are versatile tools that are  used in many domains (e.g. sociology, linguistics, computer sciences, neurosciences, climatology, psychology, etc.) and can take various forms to adapt to each problem.  They can be directed or undirected. Some link entities with multiple kinds of edges (multidimensional), or have different layers (multiplex), others link groups of disconnected objects (multipartite). In biology, the most typical networks simply represent the species (nodes) and their relationships (edges).

Various types of species interactions are studied with networks. For example, their is a rich literature of networks for plant-pollinator and host-parasite relationships in ecology. These species interactions are clearly defined and directly observed in the field. Contacts of pollination or parasitism are counted and networks constructed from these interaction abundances.
\begin{figure}
\centering
\includegraphics[width=0.7\linewidth]{figs/pocock.png}
\caption{Species interaction networks at Norwood Farm, Somerset, UK \citep{PED12,BRM13}.}
\label{pocock}
\end{figure}
 However many mechanisms cannot be observed and may not be well defined. One way to discover them may then be to resort to a more mathematical definition of species interactions. Working with the latter allows the study of community assembly mechanisms with the inference of networks representing guilds of species in community ecology. This type of network is extensively used in genomics for protein-protein interaction network, or gene regulatory networks, or in microbiology to study the output of a metabarcoding experiment assessing the composition of a microbiome. 
 
 \begin{figure}
\centering
\includegraphics[width=0.7\linewidth]{figs/PPInetwork.png}
\caption{Protein-protein interactions between genes involved in schizophrenia \citep{GTH16}.}
\label{PPI}
\end{figure}

  \subsection*{Statistical interactions}
The correlations between the species measures first come to mind as a statistical characterization of an interaction. These are easily obtained, yet their corresponding networks are hard to interpret. Indeed, two covariates correlating with a same third one will appear correlated, even if they have no direct effect on each other\footnote{e.g. the number of covid 19 cases detected correlates with both the real number of cases and the number of tests done on the population, which induces a spurious correlation between the two latter where obviously there is no direct effect of one on the other.}. This phenomena of spurious correlations complicates both the analysis and the interpretation by inducing a very high number of edges  which cannot be categorized as direct or indirect associations between the species. 
 
 Conditional dependencies are then very useful measures of interaction. They describe dependencies between each pair of species conditional on all others. That is, all other species measures kept fixed, their measures should still be correlated. A link between two species can then be interpreted as a direct association. This yields a network of species conditional dependence (link) and independence (absence of link), which is interpretable and falls within the well-studied mathematical framework of graphical models.
 
 \subsection*{Measures on species}
 Networks of statistical interactions are obtained from datasets of repeated measures on a set of species, which can be of various types. Measures can be continuous, as for example the output of gene expression profiling experiments using DNA microarrays, which  are  fluorescences measures from targeted genes. Using a Gaussian approximation, these measures of genes expressions can be used to derive genes regulatory networks.  Measures can also be binary, as in co-occurrence data in ecology, which record the joint presences and absences of a set of species in several sites. \citet{CAM16}.
%développer les données de co-occurence

Abundance data  are joint counts of species in a series of sites (also known as assemblages data in ecology). Recent technologies made this type of data increasingly available. Assemblages data were rare in ecology as they implied intensive sampling efforts, which is now greatly facilitated by camera traps and sensors. In metagenomics, high-throughput sequencing technologies for metabarcoding experiments made it possible  to get joint counts of pseudo-species (operational taxonomic units (OTUs)) abundances.  Both domains work with the same type of output: a dataset of joint (pseudo-)species abundances from different sites or samples.\\
%développer les metabarcoding et OTU


Once the data has been collected, it is very likely that not all species or covariates were observed: there exists missing actors and data is incomplete. In the network, the existence of a missing actor translates into appearance of edges between all the species it should connect with, creating dense cliques of species which are not actually conditionally dependent on one another. A second objective of this work is to take missing actors into account during network inference in order to get more accurate interpretations.

\section*{Network inference}
% ce qui existe et motivation du sujet
\section*{Objectives}
 \subsection*{Graphical models and Trees}
The dedicated framework for the representation of conditional dependency structures are graphical models. Gaussian Graphical Models (GGM) in particular provide with appealing algebraic properties for network inference, which are detailed in \citet{Lau96}. Exploring the set of possible graphs is a non-ending task, and we chose to reduce the searching space to that of spanning trees. This is the sparsest connected structure, and enjoys specific algebraic properties allowing to sum on all possible spanning trees at the cost of a determinant calculus. Our network inference methodology relies on spanning trees and \citet{Lau96} maximum likelihood estimators for multivariate Gaussian graphical models.

%caser le missing actor qq part
 \subsection*{Modeling abundance data}
 As this ideal framework of GGM is not directly applicable to abundance data, there exist two possible ways to proceed: either apply a Gaussian transformation to the data, or rely on Gaussian latent vectors in the framework of Joint Species Distribution Models (JSDM). Our methodology resorts to the latter, and more specifically to the Poisson log-normal distribution to model the counts. This distribution allows easy handling of both covariates and offsets, as well as take overdispersion into account thanks to its Gaussian random parameters. 
 
 
 \subsection*{Estimation procedure} %or algorithm
The central model we adopt involves a Gaussian layer of parameters which is a mixture on all spanning trees. Each component of the mixture is a Gaussian distribution which dependency structure is a spanning tree. This represents two hidden layers of parameters, which we first estimates with an Estimation-Maximization (EM) algorithm. Then to infer missing actors of the network, we resort to a variational (VEM) approach.

\section*{Outline} 
  \subsection*{Chapter 1}
The first chapter details the mathematical tools and technical results for the statistical modeling and the network inference used in Chapters 2 and 3.

   \subsection*{Chapter 2}
   This chapter details a method to infer undirected networks representing conditional statistical dependencies between the species from their joint abundance measures.  The proposed methodology, implemented in the R package \texttt{EMtree},  is compared to state of the art approaches and applied to two empirical datasets from ecology and metagenomics. This chapter has been published as an article in the journal \textit{Methods in Ecology and Evolution} \citep{MRA20}.
   
    \subsection*{Chapter 3}
This chapter details a variational approach to the inference of a missing actor in the network. The reconstruction of missing actor(s) is implemented in the R package \texttt{nestor} and illustrated on two  empirical datasets from ecology. This chapter has been submitted for publication in the \textit{Journal of the Royal Statistical Society: series C (applied statistics)}.
 
  \subsection*{Chapter 4}
This final chapter introduces some natural perspectives of this work. After concluding on the specifics of the developed methodology, natural extension are presented, including network comparison. The inclusion of spatialized data is discussed, as well as the possibility of network inference directly from the observed counts.
 \section*{Notations}
 
 \begin{description}
 \item[Operations:]  \begin{itemize}
     \item[]
 \item[] $|\cdot|$ : matrix determinant
 \item[] $\odot$ : Hadamard product
 \end{itemize}
 \item[Variables:] \begin{itemize}
     \item[]
 \item[] $\Ybf$ :  matrix  of observed counts
 \item[] $\Zbf$ : matrix of latent Gaussian parameters 
 \item[] $\Ubf$ : matrix of latent normalized Gaussian parameters 
 \item[] $\Xbf$ : matrix of covariates 
 \item[] $O$ : matrix of measured offsets
 \end{itemize}
 \item[Dimensions:]\begin{itemize}
     \item[]
 \item[] $n$ : number of samples
 \item[] $p$: number of observed species
 \item[] $r$: number of unobserved species
 \item[] $d$: number of covariates
 \end{itemize}
 \end{description} 
 

%%%%%%%%%%%%%%%%%%%%%%%%%%%%%%%%%%%%%%%%%%%%%%%%%%%%%%%%%%%%%%%%%%%%%%%%%%%%%%%%
\section{Model} \label{sec:Model}
%%%%%%%%%%%%%%%%%%%%%%%%%%%%%%%%%%%%%%%%%%%%%%%%%%%%%%%%%%%%%%%%%%%%%%%%%%%%%%%%
%%%%%%%%%%%%%%%%%%%%%%%%%%%%%%%%%%%%%%%%%%%%%%%%%%%%%%%%%%%%%%%%%%%%%%%%%%%%%%%%

%%%%%%%%%%%%%%%%%%%%%%%%%%%%%%%%%%%%%%%%%%%%%%%%%%%%%%%%%%%%%%%%%%%%%%%%%%%%%%%%
\subsection{Poisson log-normal and tree-shaped graphical models}
%%%%%%%%%%%%%%%%%%%%%%%%%%%%%%%%%%%%%%%%%%%%%%%%%%%%%%%%%%%%%%%%%%%%%%%%%%%%%%%%

%%%%%%%%%%%%%%%%%%%%%%%%%%%%%%%%%%%%%%%%%%%%%%%%%%%%%%%%%%%%%%%%%%%%%%%%%%%%%%%%
\subsubsection*{Poisson log-normal model.} 
We start with a reminder on the multivariate Poisson log-normal model, with the example of abundance data. The abundances of $p$ species observed on $n$ sites are gathered in the $n \times p$ matrix $\Ybf$ where $Y_ {ij}$ is the count of species $j$ in site $i$, and the row $i$ of $\Ybf$, denoted $\Ybf_i$, is the abundance vector collected on site $i$. A covariate vector $\xbf_i $ with dimension $d$ is also measured on each site $i$ and all covariates are gathered in the $n \times d$ matrix  $\boldsymbol X$. The PLN model states that a (latent) Gaussian vector $\Ubf_i$ of size $p$ with variance matrix $\Rbf = (\rho_{kl})_{kl}$ is associated with each site:
\begin{equation} \label{eq:PLN-Z}
\{\Ubf_i\}_{1 \leq i \leq n} \text{ iid}, \qquad 
\Ubf_1 \sim \Ncal_p(\zerobf, \Rbf),
\end{equation}
the sites being assumed to be independent. To ensure identifiability, we let the diagonal of $\Rbf$ be made of 1's, so $\Rbf$ is actually a correlation matrix.
All latent vectors $\Ubf_i$ are gathered in the $n \times p$ matrix $\Ubf$. The PLN model further assumes that species abundances in all sites are conditionally independent, and that their respective distribution only depends on the environment and the associated latent variable:
\begin{equation} \label{eq:PLN-Y.Z}
\{Y_{ij}\}_{1 \leq i \leq n, 1 \leq j \leq p} \mid \Ubf \text{ independent}, \quad 
Y_{ij} \mid U_{ij} \sim \Pcal\left(\exp(o_{ij} + \xbf_i^\intercal \thetabf_j + \sigma_j U_{ij})\right),
\end{equation}
where $o_{ij}$ is a known offset term which typically accounts for the sampling effort, and $\sigma_j$ is the latent standard deviation associated with species $j$. The vector $d \times 1$ of regression coefficients $\thetabf_j$ describes the environmental effects on species $j$. An important feature of the PLN model is that the sign of the correlation between the observed counts is the same as this of correlation between the latent variables \citep{AiH89}: $\text{sign}(\text{Cor}(Y_{ij}, Y_{ik})) = \text{sign}(\text{Cor}(U_{ij}, U_{ik}))$. 
% The dependence between the species abundances is entirely controlled by the latent dependency structure encoded in the precision matrix $\Omegabf:=\Rbf^{-1}$.

%%%%%%%%%%%%%%%%%%%%%%%%%%%%%%%%%%%%%%%%%%%%%%%%%%%%%%%%%%%%%%%%%%%%%%%%%%%%%%%%
\subsubsection*{Tree-shaped graphical models.} 
Network inference relies on the assumption that few species are directly dependent on one another, meaning that the underlying graphical model is sparse. In the framework of the PLN model, the graphical model of interest rules the distribution of the latent vectors $\Ubf_i$ and is  encoded in the precision matrix $\Omegabf:=\Rbf^{-1}$. A way to foster sparsity is to impose $\Omegabf$ to be faithful to a spanning tree $T$, that is: $\Ubf_1 \sim \Ncal_p(\zerobf, \Omegabf_T^{-1})$ where the non-zero terms of $\Omegabf_T$ correspond to the edges of the tree $T$ . However this hypothesis is very restrictive  as it allows only $p-1$ links among $p$ species \citep{ChowLiu}. A more flexible approach consists in assuming that the latent vectors are drawn from a mixture of Gaussian distributions, each faithful to a tree $T$ \citep{MixtTrees,MeilaJaak,kirshner,SRS19}:
\begin{equation} \label{eq:mixt-Z}
\Ubf_1 \sim \sum_{T \in \Tcal_p} p(T) \Ncal_p(\zerobf, \Omegabf_T^{-1}),
\end{equation}
where $\Tcal_p$ is the set of spanning trees with $p$ nodes.
We further assume that the tree distribution $\{p(T)\}_{T \in \Tcal_p}$ can be written as a product over the edges:
\begin{equation} \label{eq:prob-T}
p(T) = B^{-1} \prod_{jk \in T} \beta_{jk}, \qquad
\text{with} \quad B = \sum_{T \in \Tcal_p} \prod_{jk \in T} \beta_{jk}.
\end{equation}
The weights $\beta_{jk}$ are gathered in the $p \times p$ symmetric matrix $\betabf$ with diagonal zero. Observe that these weights are defined up to a multiplicative constant, so that only $p(p-1)/2 - 1$ of them may vary independently. This PLN model with latent tree-shaped dependency structure is similar to that considered by \cite{MRA20}.

%%%%%%%%%%%%%%%%%%%%%%%%%%%%%%%%%%%%%%%%%%%%%%%%%%%%%%%%%%%%%%%%%%%%%%%%%%%%%%%%
\subsection{Introducing the missing actor} \label{sec:missActor}
%%%%%%%%%%%%%%%%%%%%%%%%%%%%%%%%%%%%%%%%%%%%%%%%%%%%%%%%%%%%%%%%%%%%%%%%%%%%%%%%

%%%%%%%%%%%%%%%%%%%%%%%%%%%%%%%%%%%%%%%%%%%%%%%%%%%%%%%%%%%%%%%%%%%%%%%%%%%%%%%%
\subsubsection*{PLN model with missing actors.} 
We now introduce the concept of missing actors, which corresponds to variables that are involved in the graphical model but are not associated with observed variables. To involve such actors in the model, we assume that a complete latent vector $\Ubf_i$ with dimension $p+r$ is associated with site $i$, where $r$ is the number of missing actors. This complete vector can be decomposed as $\Ubf_i^\intercal = [\Ubf_{Oi}^\intercal \; \Ubf_{Hi}^\intercal]$ where $\Ubf_{Oi}$ (with dimension $p$) corresponds to observed species and $\Ubf_{Hi}$ (with dimension $r$) corresponds to the missing actors.
The complete $n \times (p+r)$ latent matrix $\Ubf$ can be decomposed in the same way as $\Ubf = [\Ubf_O \; \Ubf_H]$, $\Ubf_O$ and $\Ubf_H$ having dimension $n \times p$ and $n \times r$, respectively. \\ 
The model we consider states that
\begin{enumerate}[label=\roman*]
\item the complete latent vectors $\Ubf_i$ are all iid and distributed according to a mixture similar to \eqref{eq:mixt-Z} and \eqref{eq:prob-T} but with Gaussian distributions (and matrices $\Omegabf_T$ and $\betabf$) of dimension $(p+r)$, and trees drawn from $\Tcal_{p+r}$;
\item  the abundances $Y_{ij}$ of the  $p$ observed species are distributed according to \eqref{eq:PLN-Y.Z}, replacing $\Ubf$ with $\Ubf_O$,
\end{enumerate}

\begin{figure}[H]
 \begin{center}
	\begin{tikzpicture}	
      \tikzstyle{every edge}=[-,>=stealth',auto,thin,draw]
		\node (A1) at (0.625*\length, 2*\length) {$T$};
		\node (A2) at (0*\length, 1*\length) {$\Ubf_O$};
		\node (A3) at (1.25*\length, 1*\length) {$\Ubf_H   $};
		\node (A4) at (0*\length, 0*\length) {$\Ybf$};
		\draw (A1) edge [->] (A2);
        \draw (A1) edge [->] (A3);
        \draw (A2) edge  (A3);
        \draw (A2) edge [->] (A4);
	\end{tikzpicture} 
 \caption{Graphical model linking the count data $\Ybf$, the latent layer of Gaussian parameters $\Ubf=(\Ubf_O,\Ubf_H)$, and the latent tree $T$.}
  \label{fig:MGmodel}
    \end{center}
\end{figure}

In the sequel, we shall refer to the elements of $\Ubf_O$ and $\Ubf_H$ respectively as 'observed' and 'hidden' (or 'missing') latent variables, whereas obviously none of them are actually observed. Figure \ref{fig:MGmodel} displays the graphical model of the quadruplet $(T, \Ubf_O, \Ubf_H, \Ybf)$. The observed data $\Ybf$ still arise from an PLN model, but the graphical model of the observed latent $\Ubf_O$ may not be sparse due to the marginalization over the hidden latent $\Ubf_H$. Our main goal is to infer the dependency structure of the complete latent vectors, that is to estimate the elements of the matrices $\Omegabf_T$ and the edges weights $\betabf$. The latent dependency structure is similar to this considered by \cite{RAR19}, but the inference strategy much differs, because of the additional hidden layer.

%%%%%%%%%%%%%%%%%%%%%%%%%%%%%%%%%%%%%%%%%%%%%%%%%%%%%%%%%%%%%%%%%%%%%%%%%%%%%%%%
\subsubsection*{Identifiability restriction.} 
The proposed model only makes sense because the graphical model of the complete latent vectors $\Ubf_i^\intercal = [\Ubf_{Oi}^\intercal \; \Ubf_{Hi}^\intercal]$ is supposed to be sparse. Missing actors could obviously not be identified from a regular PLN model, without restriction on the precision matrix $\Omegabf$, as only the marginal precision matrix of the $\Ubf_{Oi}$ could be recovered. Still, to ensure identifiability we impose the same restriction as \cite{RAR19} that  missing latent variables are not connected with each other (the block corresponding to $\Ubf_H \times \Ubf_H$ is diagonal in each $\Omegabf_{T}$).
%\CA{However \SR{here}{} we do not need the additional assumption that all precision matrices $\Omegabf_T$ borrow their non-null elements from a same matrix.}{} \SR{}{[{\sl Faut-il mettre cette dernière phrase alors que rien ne le suggère ? Ou alors préciser, 'as opposed to \cite{RAR19}'}]}
%\begin{description}
%\item[(A)] All the precision matrices $\Omegabf_T$ (for $T \in \Tcal_{p+r}$) borrow their elements from a same matrix $\Omegabf$. Namely:
%$$\forall T \in \Tcal_{p+r}, \qquad [\Omegabf_T]_{j,k} = 
%\left\{ \begin{array}{ll}
%    [\Omegabf]_{j, k} & \text{if } (j, k) \in T \\
%    0 & \text{otherwise}.
%\end{array}\right.$$
%and their conditional variance is set to one. Namely, 
%$$\forall p+1 \leq h, \ell \leq p+r, \qquad [\Omegabf]_{h, \ell} = 
%\left\{ \begin{array}{ll}
%    1 & \text{if } h = \ell \\
%    0 & \text{otherwise}.
%\end{array}\right.$$
 
%\end{description}
%Assumption ({\bf A}) obviously avoids to infer $|\Tcal_{p+r}| = (p+r)^{p+r-2}$ independent (sparse) precision matrices. Assumption ({\bf B}) ensures identifiability, especially regarding the scaling of the missing latent variables.

%To summarize, the model defined in Section \ref{sec:missActor} involves $p d$ regressions coefficients (gathered in $\thetabf$), $
%(p+r)(p+r+1)/2 - r(r+1)/2 = 
%p(p+1)/2 + r p$ conditional covariances (gathered in $\Omegabf$), and $(p+r)(p+r-1)/2 - 1$ independent edge weights (gathered in $\betabf$).

%Finally, the original PLN model only concerns covariates $\Ybf$ and $\Ubf_O$. The use of a dependency structure with mixture of trees allows a sparse and efficient inference, and missing actors are accounted for in covariate $\Ubf_H$.


We now describe how to infer the model parameters. We gather the edges weights into the $p \times p$ matrix $\betab$ and the vectors of regression coefficients into a $d \times p$ matrix $\thetab$. The $p \times p$ matrix $\Sigmab$ contains the variances and covariances between the coordinates of each latent vector $\Zb_i$. Hence, the set of parameters to be inferred is $(\betab, \Sigmab, \thetab)$.

\paragraph{Likelihood.} 
The model described above is an incomplete data model, as it involves two hidden layers: the random tree $T$ and the latent Gaussian vectors $\Zb_i$. The most classical approach to achieve maximum likelihood inference in this context is to use the Expectation-Maximization algorithm \citep[EM:][]{DLR77}. Rather than the likelihood of the observed data $p(\Yb)$, the EM algorithm deals with the often more tractable likelihood $p(T, \Zb, \Yb)$ of the complete data (which consists of both the observed and the latent variables). It can be decomposed as 
 
\begin{equation} \label{eq:PTZY}
    p_{\betab, \Sigmab, \thetab}(T, \Zb, \Yb) = p_{\betab}(T) \times p_{\Sigmab}(\Zb \; | \; T) \times p_{\thetab}(\Yb \; | \; \Zb),
\end{equation}
 
where the subscripts indicate on which parameter each distribution depends. \\
Observe that the dependency structure between the species is only involved in the first two terms, whereas the third term only depends on the regression coefficients $\thetab$. 
We take advantage of this decomposition to propose a two-stage estimation algorithm. The first stage deals with the observed layer $p_{\thetab}(\Yb \; | \; \Zb)$, the second with the two hidden layers $p_{\betab}(T)$ and  $p_{\Sigmab}(\Zb \; | \; T)$. The network inference itself takes place in the second step.

\paragraph{Inference in the observed layer.} 
The variational EM (VEM) algorithm that provides an estimate of the regression coefficients matrix $\thetab$ is described in Appendix \ref{app:VEM} (along with a reminder on EM and VEM). It also provides the (approximate) conditional means $\Esp(Z_{ij} | \Yb_i)$, variances $\Var(Z_{ij} | \Yb_i)$ and covariances $\Cov(Z_{ij}, Z_{ik} | \Yb_i)$ required for the inference in the hidden layer. As a consequence, this first step provides the estimates $\widehat{\thetab}$ and $\widehat{\Sigmab}$.

\paragraph{Inference in the hidden layer.} The second step is dedicated to the estimation of $\betab$. The EM algorithm actually deals with the conditional expectation of the complete log-likelihood, namely $\Esp\left(\log p_{\betab, \Sigmab, \thetab}(T, \Zb, \Yb) \; | \; \Yb\right)$. 
As shown in Appendix \ref{app:EM}, this  reduces to
 
\begin{equation} \label{expectation}
    \Esp\left(\log p_{\betab, \Sigmab, \thetab}(T, \Zb, \Yb) \; | \; \Yb\right)
    \simeq
    \sum_{1 \leq j < k \leq p} P_\jk \log \left(\beta_\jk \widehat{\psi}_\jk\right) - \log B + \cst
\end{equation}
 
where $\widehat{\psi}_\jk$ is the estimate of $\psi_\jk$ defined in Eq.~\eqref{eq:pZfact}, and the '$\cst$' term depends on $\thetab$ and $\Sigmab$ but not on $\betab$. 
$P_\jk$ is the approximate conditional probability (given the data) for the edge $(j, k)$ to be part of the network:
$P_\jk \simeq \prob\{jk \in T \; | \; Y\}$.
It is also shown in Appendix~\ref{app:EM} that $\widehat{\psi}_\jk = (1-\widehat{\rho}_\jk^2)^{-n/2}$, where the estimated correlation $\widehat{\rho}_\jk$ depends on the conditional mean, variance and covariances of the $Z_{ij}$'s provided by the first step.
 Eq.~\eqref{expectation} is maximized via an EM algorithm iterating the calculation of the $P_\jk$ and the maximization with respect to the $\beta_\jk$:
\begin{description}

\item[Expectation step: Computing the $P_\jk$ with tree averaging.] The conditional probability of an edge is simply the sum of the conditional probabilities of the trees that contain this edge. Hence, computing $P_\jk$ amounts to averaging over all spanning trees.
Fig.~\ref{fig:treeaveraging} illustrates the principle of tree averaging for a toy network with $p=4$ nodes. Here, five arbitrary spanning trees $t_1$ to $t_5$ (among the $p^{p-2} = 16$ spanning trees) are displayed, with their respective conditional probability $p(T \mid Y)$. 
The edge $(1, 3)$ has a high conditional probability $P_{13}$ because it is part of likely trees such as $t_3$ and $t_4$, whereas $P_{23}$ is small because the edge $(2, 3)$ is only part of unlikely trees (e.g. $t_1$, $t_2$). \\
Averaging over all spanning trees at the cost of a determinant calculus (i.e. with complexity $O(p^3)$) is possible using the Matrix Tree theorem \citep[][recalled as Theorem~\ref{thm:MTT2} in Appendix~\ref{app:MTT}]{matrixtree}. 
\citet{kirshner} further shows that all the $P_\jk$'s can be computed at once with the same complexity $O(p^3)$, although the calculation may lead to numerical instabilities for large $n$ and $p$.



\begin{figure}%[H]
   \begin{center}
    \begin{tabular}{cccccc}
        \input{figs/FigTreeAveraging-p4-tree1-seed2} &
        \input{figs/FigTreeAveraging-p4-tree2-seed2} &
        \input{figs/FigTreeAveraging-p4-tree3-seed2} &
        \input{figs/FigTreeAveraging-p4-tree4-seed2} &
        \input{figs/FigTreeAveraging-p4-tree5-seed2} \\
        $t_1: 2.1\%$ & 
        $t_2: 3.5\%$ & 
        $t_3: 34.1\%$ & 
        $t_4: 15.6\%$ & 
        $t_5:  <.1\%$ \\ \\
        & 
        \input{figs/FigTreeAveraging-p4-avgtree-seed2} &
        \qquad \qquad &
        \input{figs/FigTreeAveraging-p4-graph-seed2} \\
        \multicolumn{3}{c}{Edge conditional probabilities} & Estimated graph \\
    \end{tabular}
    \caption{Tree averaging principle. 
    \textit{Top:} 5 spanning trees with 4 nodes  $(t_1, \dots t_5)$, with their respective conditional probability given the data $P(T = t \mid Y)$.
    \textit{Bottom left:} Weighted graph resulting from tree averaging. Each edge  has width proportional to its conditional probability. \textit{ Bottom right:} Estimated graph (obtained by thresholding edge probabilities) is not a tree.}
    \label{fig:treeaveraging}
   \end{center}
\end{figure}

\item[Maximization step: Estimating the $\beta_\jk$.] 
This step is not straightforward, as the normalizing constant $B = \sum_T \prod_{jk \in T} \beta_\jk$ involves all $\beta_\jk$'s. We propose an exact maximization built upon the Matrix Tree theorem (see Appendix~\ref{app:EM}). 
\end{description}


\paragraph{Algorithm output: edge scoring and network inference} 
%As a side product, 
EMtree provides the (approximate) conditional probability $P_\jk$ for each edge $(j, k)$ to be part of the network. 
Whenever an actual inferred network $\widehat{G}$ is needed (e.g. for a graphical purpose), it can be obtained by thresholding the $P_\jk$ (see Fig.~\ref{fig:treeaveraging}, bottom right). Because we are dealing with trees, a natural threshold is the density of a spanning tree, which is  $2/p$.
More robust results can be obtained using a resampling procedure similar to the stability selection proposed by \citet{LRW10}. It simply consists in sampling a series of subsamples $s = 1 \dots S$, to get an estimate $\widehat{G}^s$ from each of them and to collect the selection frequency for each edge. Again, these edge selection frequencies can be thresholded if needed.


\section{Simulations} \label{sec:Simul}
%%%%%%%%%%%%%%%%%%%%%%%%%%%%%%
\subsection{Count datasets}
For the simulation study, 300 count datasets of $15$ species in total including one missing actor are generated, thus $p=14$ and $r=1$. 
Data is generated as follows.
We generate a scale-free structure $\mathcal{G}$ (which degree distribution is a power law) with $p+1$ nodes using the R package \texttt{huge} \citep{zhao2012huge} available on CRAN. 
The missing species $h$ is chosen as the one with highest degree. We measure the {\sl influence} of the missing actor with its degree, distinguishing three influence classes: \textit{Minor} (degree $\leq 5$), \textit{Medium} ($5<$ degree $\leq 7$) and \textit{Major} (degree $\geq 8$). For each replicate, the latent layer $\Ubf$ and the observed abundances $\Ybf$ are simulated according to the model defined in Section \ref{sec:Model}.


\subsection{Experiment \& Measures}
% \SR{
% For each simulated dataset, the set of 4 likely initial cliques is identified as detailed in Section 
%\ref{init}. 
%Simulated datasets involve a unique missing actor. A quick exploration of the space of likely cliques consists in keeping the two first principal components of sPCA, and their complements. This way, we obtain a set of 4 $C_h$ candidates, which size we impose to be between 2 and $(p-1)$ (the missing actor should have at least two neighbors, but not all the network).
%}{
For each simulated dataset, the VEM algorithm is initialized as described in Section \ref{init}. 
More specifically and because we only look for one missing actor, we consider the cliques corresponding to each of the first two principal components of sPCA, and their respective complements, which provides us with four cliques.
Then four VEM algorithms, as described in Section \ref{algo}, are run starting from each of the four candidate cliques, and the one yielding the highest lower bound $\Jcal$ is kept. 
For all simulations, we set the precision of the convergence criterion to $\varepsilon=10^{-3}$, the tempering parameter to $\alpha=0.1$ and the maximal number of iterations to $100$. 
The  inference quality is assessed regarding the global network inference, the missing actor's position in the network, and its values along the $n$ sites. We refer to this first procedure as the \textit{blind} procedure. Additionally, we define the \textit{oracle} procedure as running the VEM with the set of true neighbors of the missing actor as initial clique.\\

For each procedure, a general measure of the whole network inference quality is first given by comparing the inferred edge probabilities to the original dependency structure. This is done using the Area Under the ROC Curve (AUC) criteria.  Then, to be more specific and target the neighbors of node $h$ specifically, the probabilities of edges involving $h$ are transformed into binary values using the 0.5 threshold. The values are then compared to the original links of $h$ and yield quantities of true/false ($T$/$P$) positives/negatives ($P$/$N$), from which are built the \textit{precision} (also known as the positive predictive value, ${TP}/({TP+FP)}$) and the \textit{recall} (also known as the true positive rate, ${TP}/({TP+FN})$) criteria. Finally, we assess the ability to reconstruct the missing actor across the sites  by computing the absolute correlation between its inferred vector of means ($M_h$) and its original latent Gaussian vector $\Ubf_h$.


%%%%%%%%%%%%%%%%%%%%%%%%%%%%%%
\subsection{Results}
Simulations performance measures are gathered in Table \ref{tab:perf} and Table \ref{tab:oracle} for blind and oracle procedures respectively. The distributions of the quality measures are displayed in Figure \ref{fig:densities}. \\

Table \ref{tab:perf} shows the network is well inferred, as all AUC means are above 0.85, with almost perfect inference when the influence of the missing actor is major. Its neighbors and values per site are very well retrieved in these cases with mean recall values above 0.9 and mean correlation above 0.8, with a great confidence in the algorithm outputs as mean precision is above 0.95. However, there exists a clear deterioration of all performance as the influence decreases with lower means are greater deviations, down to about 0.6 mean values for all measures when the influence is minor. Moreover, the algorithm takes more and more time to converge as the influence decreases, although it stays at about $3s$ for minor cases which is reasonable. Figure \ref{fig:densities} shows that as the influence decreases, the densities present with several modes and dilute towards 0, illustrating that even if some networks are still well-inferred, there also are more and more cases where the algorithm fails. In particular, the performance decrease of medium cases seems to be only due to a greater number of failed inferences.\\

All these elements point to minor cases being harder problems to solve, unsurprisingly. Yet as oracle results show in Table \ref{tab:oracle}, it is possible to carry out almost-perfect inference in all cases, if the algorithm is initialized with the true clique; the deterioration is still present in all measures, but stays marginal. Thus the harsh decrease in the blind procedures seems to be mainly due to the proposed initialization method failing at correctly finding some of the small cliques of neighbors.\\

%\textcolor{red}{Faut-il autant s'etendre sur ce point ? Pour l'instant, nous semblons dire que l'initialisation fait tout et que la solution que nous proposons ne fonctionne que dans les cas facile...}
%\textcolor{red}{Si on le garde : ajouter un titre 'About initialization'?}

% point sur les FN dans les initializations
\paragraph{About intialization.}
Figure \ref{fig:perfinit} compares the initialization quality and the corresponding final inferred neighbors, in terms of initial (-i) and final (-f) false negative (FNR, also 1-TPR) and positive rates (FPR). It clearly appears that final measures  mostly increase with false negatives of the initial clique. This means that not including a neighbor in the initialization is much worse for the inference than falsely including a node. The increase of FNR-f is bigger than that of FPR-f, meaning that a wrong initialization leads to a set of inferred neighbors which most part can be trusted, but which will be largely incomplete. This advocates for bigger initialization cliques when no prior information is available.


\begin{table}
\centering
\begin{tabular}{lrrrrrr}
  \hline
  & N & AUC & Precision & Recall & Correlation & Time (s) \\ 
  \hline
 Major & 100 & 0.98 (0.06) & 0.96 (0.14) & 0.94 (0.17) & 0.83 (0.10)& 2.36 (0.91)  \\ 
 Medium & 132 & 0.93 (0.12) & 0.83 (0.26) & 0.81 (0.30) & 0.73 (0.17)& 2.69 (1.15)  \\ 
 Minor &  68 & 0.89 (0.10) & 0.61 (0.34) & 0.66 (0.36) & 0.59 (0.21) & 3.08 (1.14) \\ 
   \hline
\end{tabular}
\caption{\label{tab:perf}Blind procedure using cliques from initialization. The influence of the missing actor is measured with its degree, distinguishing three influence classes: \textit{Minor} (degree $\leq 5$), \textit{Medium} ($5<$ degree $\leq 7$) and \textit{Major} (degree $\geq 8$).  For each class of influence, the following quantities are reported:  number of simulated graphs (N), means and standard deviations of AUC, Precision, Recall, Correlation between missing actor inferred vector of means and original latent vector, and running times in seconds. AUC measures the retrieval of the dependence structure between all variables (observed and missing), whereas precision and recall are specific to the missing actor links.} 

\end{table}
 
 

\begin{figure}[H]
    \centering
    \includegraphics[width=11cm]{figs/simu_densities.png}
    \caption{
    %\CA{}
    The influence of the missing actor is measured with its degree, distinguishing three influence classes: \textit{Minor} (degree $\leq 5$), \textit{Medium} ($5<$ degree $\leq 7$) and \textit{Major} (degree $\geq 8$). 
    %\CA{Distributions of performance measures}
    The distributions of performance measures are displayed for each class of influence: AUC measures the retrieval of the dependence structure between all variables, observed and missing.  Precision and recall are specific to the missing actor links. 
    }
    \label{fig:densities}
\end{figure}
 

\begin{figure}[H]
    \centering    \includegraphics[width=11cm]{figs/quali_init_spca.png}
    \caption{Comparison of initial and final FPR and FNR, for cliques of neighbors of one missing actor obtained with the sparse PCA method. Position of dots are defined according to initial values, their color according to the final FPR and FNR. Sizes are proportional to the density of dots on a given position.}
    \label{fig:perfinit}
\end{figure}


 \begin{table}
\centering
\begin{tabular}{lrrrrrr}
  \hline
  & N & AUC & Precision & Recall & Cor. &t(s) \\ 
  \hline
  Major & 100 & 1 (0.00) & 1 (0.00) & 1 (0.01) & 0.86 (0.02)  & 1.28 (0.21) \\ 
  Medium & 132 & 1 (0.02) & 1 (0.00) & 0.99 (0.04) & 0.83 (0.02)  & 1.38 (0.46)  \\ 
  Minor &  68 & 0.98 (0.04) & 0.99 (0.03) & 0.96 (0.12) & 0.8 (0.04)& 1.56 (0.69) \\ 
   \hline
\end{tabular}
 \caption{\label{tab:oracle} Oracle procedure using true clique as starting point. The influence of the missing actor is measured with its degree, distinguishing three influence classes: \textit{Minor} (degree $\leq 5$), \textit{Medium} ($5<$ degree $\leq 7$) and \textit{Major} (degree $\geq 8$).  For each class of influence, the following quantities are reported:  number of simulated graphs (N), means and standard deviations of AUC, Precision, Recall, Correlation between missing actor inferred vector of means and original latent vector, and running times in seconds. AUC measures the retrieval of the dependence structure between all variables (observed and missing), whereas precision and recall are specific to the missing actor links.}
\end{table}


\section{Applications}  \label{sec:Appli}


%%%%%%%%%%%%%%%%%%%%%%%%%%%%%%%%%%%%%%%%%%%%%%%%%%%%%%%%%%%%%%
\subsection{Cross validation criterion for  model selection}
%\SR{
%As no model selection criteria has been designed for this problem yet, we  run a 10-fold cross-validation procedure, with pairwise composite likelihood (PCL) as adjustment criteria (see \citet{lindsay}). This choice of pairwise components is motivated from the fact that they keep track of the covariance structure, and the estimation of bivariate Poisson log-normal densities is made possible thanks to the \texttt{bipoilog} function of the \texttt{poilog} R package. \\
%
%The cross-validation procedure to estimate the pairwise composite likelihood is available in appendix \ref{CV}. The general idea is to compute the bivariate Poisson log-normal densities between all pair of species, conditional on a tree sampled as explained in appendix \ref{sampTrees} which defines the distribution parameters. Finally the procedure computes the average criteria  $$\displaystyle PCL(\Ybf)=\frac1V\sum_{v=1}^{V}\frac1B \sum_{b=1}^Bf_{PLN}(\Ybf; b,v),$$ where $f_{PLN}(\Ybf; b,v)=\sum_{\substack{i \in v\\ j < k}} \log p_{PLN}[(Y_{ij}^v, Y_{ik}^v) | T^b; \widehat{\theta},\widehat{\Sigma}_{T^b jk}]$, with $V=10$ and $B=10^2$. \\
%This is a computational greedy procedure that is not suited for a simulation study. It was applied to two empirical datasets in order to decide the number of missing actors in the model. The results, gathered in Figure \ref{fig:selec}, yield $r=1$ for the Barents Sea data set, and $r=2$ for the Fatala River one.}{}

The proposed model obviously raises the problem of choosing the number of missing actors $r$ (which may be zero). Variational-based inference often relies on approximate versions of the BIC or ICL criteria for model selection. Few theoretical guaranties exist about these approximate criteria and, in the present case, we observed that BIC and ICL penalizations did not yield consistent results. Therefore, we resort to $V$-fold cross validation to determine the number of missing actors. 

More specifically, we split the original dataset $\Ybf$ ($\Xbf$ is dropped here for the sake of clarity) into $V$ subsets with almost equal sizes $m_1, \dots m_V$ ($\sum_{v=1}^V m_v = n$), which we denote $\{\Ybf^v\}_{v = 1, \dots V}$. For each subset $v$, we define its complement $\Ybf^{-v}$ on which we fit a model with $r$ missing actors and get a parameter estimate $\Gammabf_r^{-v} = (\thetabf_r^{-v}, \sigmabf_r^{-v}, \betabf^{-v}_r, \Omegabf_r^{-v})$ and measure the fit of $\Gammabf_r^{-v}$ to the test dataset $\Ybf^v$. 

To avoid the integration over the $(p+r)$-dimensional Gaussian latent layer, we measure the fit with the pairwise composite likelihood \citep{lindsay}.
For any given tree $T$ and parameter $\Gammabf$, the bivariate Poisson log-normal pdf $p_{PLN}\left((Y_{ij}, Y_{ik}); \Gammabf, T \right)$ can be easily computed for any sample $i$ and pair of species $(j, k)$ with available tools such as the \texttt{poilog} R package \citep{ViS08} available on CRAN. The cross-validation criterion is defined as
$$
PCL_r(\Ybf) = \frac1V \sum_v \frac1B \sum_{b=1}^B \frac1{m_v} \sum_{i = 1}^{m_v} \sum_{j < k} \log p_{PLN}\left((Y^v_{ij}, Y^v_{ik}); \Gammabf_r^{-v}, T_{r, b}^{-v} \right)
$$
where the tree samples $\{T_{r, b}^{-v}\}_{b=1 \dots B}$ are iid according to $p_{\betabf_r^{-v}}(T)$. 

The sampling procedure for spanning trees is given in Appendix \ref{eq:sampTree}; the complete procedure for the calculation of $PCL_r(\Ybf)$ is described by Algorithm \ref{algo:model-selection}, given in Appendix \ref{sec:modSel}. Note that this criterion measures the fit of the model in terms of abundance prediction, whereas our interest is mostly focused on the inference of the dependency structure. In other words, our goal is identification, that is selecting the smallest model  and not the best model in terms of prediction \citep{arlot2010survey}.


We did not include this computationally greedy procedure in the simulation study but applied it to the two ecological datasets that will be described in the next two sections. The results, gathered in Figure \ref{fig:selec}, yield $r=1$ missing actor for the Barents Sea data set, and $r=2$ missing actors for the Fatala River one.

\begin{figure}[H]
    \centering
    \includegraphics[width=10cm]{figs/selec_model_applis.png}
    \caption{Pairwise composite likelihoods estimates of Barents and Fatala datasets for models including 0 to 3 missing actors.}
    \label{fig:selec}
\end{figure}

% \SR{
% A wider exploration of likely cliques is conducted thanks to a bootstrap approach with $200$ sub-samples. Each of the latter consists of 80\% of the data; sPCA is run on each of them and only the identified clique of the $r$ first principal components are stored, when the studied model involves $r$ missing actors. When $r>1$, the restriction on the clique sizes is lifted. The bootstrap thus yield 200 lists of $r$ initial cliques, from which only unique ones are kept. This approach is more time-consuming, and therefore only used on empirical datasets.
% }{
Regarding the initialization, we performed a wider exploration as compared to the simulation study. To enlarge the list of possible cliques, we applied a resampling version of the procedure described in Section \ref{sec:algoSpec}, and applied it to 200 sub-samples, each consisting in 80\% of the whole data set. This yielded 200 lists of $r$ initial cliques, from which duplicates were removed.
%}

%%%%%%%%%%%%%%%%%%%%%%%%%%%%%%%%%%%%%%%%%%%%%%%%%%%%%%%%%%%%%%%%%%%%%%%%%%%%%%%%
\subsection{Barents Sea}
%data availability ? précisions sur les données, elles viennent d'où

% \SR{30 species of fish were counted in 89 sites of the Barents Sea (shrimp survey in the period April-May 1997, available at  \url{https://www.fbbva.es/microsite/multivariate-statistics/data.html})}{
The dataset was first published by \cite{FNA06} and consists of the abundance of 30 fish species measured in 89 sites in the Barents See in April-May 1997. In addition to abundances, the water temperature was measured in each site. The complete dataset is available at \url{www.fbbva.es/microsite/multivariate-statistics/data.html}. 
Fishes distributions are known to be greatly linked with the temperature. 
Hence to illustrate our methodology, 
% \SR{models were fit\CA{}{ted} without any covariates and with one missing actor, which we denote $h$. $\rm{Cor}(k,temp)$ denotes the absolute Pearson's correlation between $M_k$ (estimated vector of means of node $k$ along all 89 sites) and the temperature.}{
we present the results of the model fitted without any covariate (that is not accounting for the temperature), but including one missing actor (as suggested by Figure \ref{fig:selec}). To assess the ability of the proposed methodology to retrieve the influence of temperature as a missing actor, we report the empirical correlation between the temperature and the conditional expectation of the missing actor $M_h$, which we denote $\corHTemp$.
 
% Runing time
The resampling initialization procedure yielded in 14 different cliques, for each of which a VEM algorithm was run: the mean running time was $6.63$mins with deviation $0.70$ mins. 

% Dependency structure
The edge probabilities involving node $h$ as an endpoint were either very close to 0 or very close to 1, yielding a total of 6 highly probable neighbors of $h$. Figure \ref{fig:barents_adj} shows that many direct interactions are inferred between the corresponding 6 species in absence of a missing actor, which vanish when it is introduced. It also shows that accounting for this actor has only a local effect and that the direct interactions among the other species are preserved, which is consistent with our notion of a missing actor.

\begin{figure}[H]
    \centering
    \includegraphics[width=10cm]{figs/Barents_mat_comp.png} \\\vspace{-2cm}
    \includegraphics[width=10cm]{figs/Barents_net_comp3.png}
    \caption{\textit{Top left:} adjacency matrix of the Barents Sea fishes interaction network for $r=0$  missing actor. The inferred neighbors are gathered in the last 6 columns, so that their interactions are observable in the upper-right corner. \textit{Top right:} adjacency matrix for $r=1$  missing actor. The last column gathers the interactions of the inferred missing actor. \textit{Bottom}: Inferred interaction network with $r=0$ (left) and $r=1$ (right). Colored nodes refer to the inferred neighbors (blue) of the missing actor (yellow). The edges width are proportional to their probability.}
    \label{fig:barents_adj}
\end{figure}

%\begin{figure}[H]
%    \centering
%    \includegraphics[width=10cm]{missing_article/Fig/Barents_net_comp3.png}
%    \caption{Barents Sea fishes interaction network with $r=0$ (left) and $r=1$ (right). Colored nodes refer to the inferred neighbors (blue) of the missing actor (yellow).}
%    \label{fig:barents_net}
%\end{figure}

% Interpretation
In terms of interpretation, Figure \ref{fig:barents_temp} shows that the missing actor is highly correlated with the temperature. It also appears that the abundances of the species neighbor to the missing actor are much more correlated with the temperature (mean correlation = 0.78, sd = .06) than the abundances of the non-neighbor species (mean correlation = 0.46, sd = .27). This example shows the ability of the method to recover an underlying effect that would not be recorded in the data.

\begin{figure}
    \centering
    \includegraphics[width=5cm]{figs/Barents_MH_temp_white.png}
    \caption{Missing actor estimated vector of means $M_h$ as a function of the temperature. $\corHTemp=0.85$.}
    \label{fig:barents_temp}
\end{figure}

% \SR{
% Table \ref{tab:barents} then gathers the mean of correlations to the temperature of neighbors and non-neighbors. It appears that neighbors to the missing actor are significantly more correlated to the temperature than other nodes are.
% \begin{table}[ht]
% \centering
% \begin{tabular}{rlrr}
%   \hline
%   $P_{hk}$ & $\corKTemp$  \\ 
%   \hline
%  $<0.5$  & 0.44 (0.25)\\ 
%   $\geq 0.5$ & 0.80 (0.06) \\ 
%   \hline
% \end{tabular}
% \caption{Mean and standard deviation of $\corKTemp$ in relation with node $k$ being inferred as a neighbor ($P_{hk}\geq 0.5$) of missing actor $h$ or not ($P_{hk}< 0.5$).}
% \label{tab:barents}
% \end{table}
% }{
%} 

%\begin{figure}
%    \centering
%    \includegraphics[width=9cm]{missing_article/Fig/Barents_mat_comp.png}
%    \caption{\textit{Left:} adjacency matrix of Barents Sea fishes interaction network for $r=0$. The inferred neighbors are gathered in the last 6 columns, so that their interactions are observable in the upper-right corner. \textit{Right:} adjacency matrix of Barents Sea fishes interaction network for $r=1$. The last column gathers the interactions of the inferred missing actor.}
%    \label{fig:my_label}
%\end{figure}

%\begin{figure}[H]
%    \centering
%    \includegraphics[width=10cm]{missing_article/Fig/Barents_net_comp3.png}
%    \caption{Barents Sea fishes interaction network with $r=0$ (left) and $r=1$ (right). Colored nodes refer to the inferred neighbors (blue) of the missing actor (yellow).}
%    \label{fig:my_label}
%\end{figure}

%%%%%%%%%%%%%%%%%%%%%%%%%%%%%%%%%%%%%%%%%%%%%%%%%%%%%%%%%%%%%%%%%%%%%%%%%%%%%%%%
\subsection{Fatala River}
% \SR{
% The R package \texttt{ade4}  provides with the baran95 dataset which gathers the abundances of 33 species of fish on 90 locations of the Fatala River in Guinea. Data sampled between June 1993 and February 1994, which we organize in dry and rainy seasons; dates and sites are available as dataset categorical covariates.
% %
% Again models are fit with no covariates, and here two missing actors are inferred, which we denote $h1$ and $h2$. As models include more than one missing actor, the best VEM is selected under the constraint $\log sd(M_h)\geq -20$ for $h\in \{h1, h2\}$. This constraint ensures that the corresponding marginal variance of each missing actor is not too high, which would mean that the algorithm did not learn a lot about them. In other words, this ensures the algorithm succeeds in finding the desired number of missing actors. \\
% %
% The bootstrap method gives 60 unique  lists of two possible initial cliques and the mean running time for a convergence with precision $1e-3$ is $11.33$ min with deviation $1.47$; 14 did not reach convergence and the algorithm stopped after 100 iterations. \\
% %
% The best fit gives vectors of means $Mh1$ and $Mh2$ corresponding two each missing actor. Figure \ref{fig:Fatala} shows the plot of $Mh1$ against $Mh2$ colored with either one of the available covariates. It is very interesting to see that $Mh1$ is linked to the Site covariate, as it clearly separates kilometer 3 from kilometer 46 which are the first and last locations. On the other hand $Mh2$ seems linked with the Season covariate, even if the separation is less obvious.
% }{
\cite{baran1995dynamique} collected the abundances of 33 fish species in 90 sites along the Fatala River in Guinea between June 1993 and February 1994. The data are available from the R package \texttt{ade4} on CRAN \citep{dray2007ade4}, along with the date and site of collection, from which we deduce the season (dry or rainy). Again the model was fitted without any covariates, but with two missing actors, as suggested by Figure \ref{fig:selec}. \\
%
The resampling initialization procedure yielded in 60 different cliques, for each of which a VEM algorithm was run: the mean running time was $11.33$ min (sd = $1.47$ mn). 14 VEM did not reach convergence (with tolerance $\varepsilon = 1e-3$) after 100 iterations. We filtered out the results obtained from the different initializations, when the algorithm obviously ended in a degenerate solution ($\Var(M_h) < \exp(-20)$). \\ 
%
% %\SR{
% Figure \ref{fig:Fatala} shows the scatterplot of estimated conditional mean of the two missing actors $(M_{h_1}, M_{h_2})$ in each site, colored with either one of the available covariates (site and season). $M_{h_1}$ is obviously linked to the site and separates most upstream locations (kilometer 3) from most downstream locations (kilometer 46). On the other hand $M_{h_2}$ seems linked with the Season covariate, even if the separation is less obvious. Similarly, the second missing actor shows a clear relation with the season. \\
% % 
% Again, the retrieved missing actor each correspond to an underlying effect that rules fish species abundances. The clear separation between the two effects is reinforced by the assumption the missing actors are independent from each other, which obviously holds for locations ans seasons. \\
% }{
Figure \ref{fig:Fatala} shows the scatterplot of the estimated conditional mean of the two missing actors $(M_{h_1}, M_{h_2})$ in each site, colored with either one of the available covariates (site and season). The missing actor $h_1$ is obviously linked to the site and separates most upstream locations (kilometer 3) from most downstream locations (kilometer 46). This actor has 11 highly probable neighbor species.
% \textcolor{red}{comparer les anova Poisson pour les voisins / pas voisins}
Again, this retrieved missing actor corresponds to an underlying effect (in this case: geography) that rules fish species abundances. \\
% 
The second missing actor seems to be linked with the season but with a less clear separation. Also the variability of $M_{h_2}$ is much smaller than this of $M_{h_1}$. This effect is therefore questionable, which brings us back to model selection. As mentioned above, we used a procedure based on cross-validation, which may  be prone to select too complex model \citep{shao1993linear,friedman2001elements,arlot2010survey}. The definition of a grounded model selection criterion for structure inference in presence of missing actors remains open.
%}

%\textcolor{red}{[C'est un peu dommage de finir comme ça mais bon...]}
 
\begin{figure}[h]
    \centering
    \includegraphics[width=12cm]{figs/Fatala_MH2.png}
    \caption{Estimated means $M_{h_1}$ and $M_{h_2}$ of the two inferred missing actors. Left column: scatterplots $M_{h_1}$ vs $M_{h_2}$ with site (top) and season (bottom) color code. Right: distribution of the estimated means across sites. Top right: distribution of $M_{h_1}$ in each location, bottom right: distribution of $M_{h_2}$ in each season.}
    \label{fig:Fatala}
\end{figure}



\begin{subappendices}
%\addtocontents{toc}{\setcounter{tocdepth{-1}}
\section{Published supplements}
  \subsection{Algebraic Tools} \label{app:tools}
 We here present some algebraic results about spanning tree structures which are used during the computations. Theorem \ref{thm:MTT}, Lemma \ref{lem:Meila} as well as Lemma \ref{lem:Kirshner} use the notion of Laplacian matrix  $\Qbf$ of a symmetric matrix $\Wbf=[w_{jk} ]_{1\leq j,k\leq p}$, which is defined as follows :
 
\[
 [\Qbf]_{jk}  =\begin{cases}
    -w_{jk}  & 1\leq j<k \leq p\\
    \sum_{u=1}^p w_{ju} & 1\leq j=k \leq p.
    \end{cases}
\]
 
We further denote $\Wbf^{uv}$ the matrix $\Wbf$ deprived from its $u$th row and $v$th column and we remind that the $(u, v)$-minor of $\Wbf$ is the determinant of this deprived matrix, that is $|\Wbf^{uv}|$.
The following Theorem \ref{thm:MTT} is the extension of Kirchhoff's Theorem to the case of weighted graphs \citep{matrixtree,MeilaJaak}.\\
\begin{theorem}[Matrix Tree Theorem] \label{thm:MTT}
    For any symmetric weight matrix W with all positive entries, the sum over all spanning trees of the product of the weights of their edges is equal to any minor of its Laplacian. That is, for any $1 \leq u, v \leq p$,
   \[
    W := \sum_{T\in\mathcal{T}} \prod_{(j, k)\in T} w_{jk} = |\Qbf^{uv}|.
    \]\\
\end{theorem}    

In the following, without loss of generality, we will choose $\Qbf^{11}$. As an extension of this result, \cite{MeilaJaak} provide a close form expression for the derivative of $W$ with respect to each entry of $\Wbf$. 

\begin{lemma} [\cite{MeilaJaak}] \label{lem:Meila}
    Define the entries of the symmetric matrix $\Mbf$ as
 \[    
 [\Mbf]_{jk} =\begin{cases}
    \left[(\Qbf^{11})^{-1}\right]_{jj} + \left[(\Qbf^{11})^{-1}\right]_{kk} -2\left[(\Qbf^{11})^{-1}\right]_{jk} & 1< j<k \leq p\\
    \left[(\Qbf^{11})^{-1}\right]_{jj} & k=1, 1< j \leq p  \\
    0 &  j=k .
    \end{cases}
\]
it then holds that $$\partial_{w_{jk}} W = [\Mbf]_{jk}  \times W.$$\\
\end{lemma}

\cite{kirshner} build on Lemma \ref{lem:Meila} to provide an efficient computation of all edges probabilities.
\begin{lemma} [\cite{kirshner}] \label{lem:Kirshner}
    Let $p_W$ be a distribution on the space of spanning trees, such that $p_W(T)=\prod_{kl\in T} w_{kl} / W$, where $W$ is defined as in Theorem \ref{thm:MTT}. Taking the symmetric matrix $\Mbf$ as defined in Lemma  \ref{lem:Meila}, the probability for an edge $kl$ to be in the tree $T^*$ writes:
 
$$\mathds{P}\{kl\in T^*\} = \sum_{T\in \mathcal{T}} p_W(T)= w_{kl}\: \Mbf_{kl}$$
\end{lemma}



\tocless\subsection{Computations} \label{app:comput}
\subsubsection[Update of tree parameter vector]{Update of $\betabf$.} \label{up:beta}
As in \citet{MRA20}, the update of $\betabf$ is such that:
$$\betabf^{t+1}  = \arg\max_\betabf \; \Esp_{g^t} \left[ \log p_\betabf(T) \right].
$$
By definition of $p_\betabf(T)$:
$$\Esp_{g^t} \left[ \log p_\betabf(T) \right] = \sum_{kl} P^t_{kl} \log \beta_{kl} - \log B\;,
\qquad
B=\sum_{T\in \mathcal{T}}\prod_{kl\in T} \beta_{kl}.$$
Computing the derivative with respect to the edge weight $\beta_{kl}$ gives:
\begin{align*}
\partial_{\beta_{kl}}\Esp_{g^t} \left[ \log p_\betabf(T) \right] &=\frac{P_{kl}^t}{\beta_{kl}} - \frac{\partial_{\beta_{kl}} B^t }{B^t} 
\end{align*}
According to Lemma \ref{lem:Meila}: $\partial_{\beta_{kl}} B^t  = [\boldsymbol{M}]_{kl} \times B$. Finally setting the derivative to 0 yields the update formula $
\beta^{t+1}_{kl} 
= \frac{P^t_{kl}}{ M(\betabf^t)_{kl}}$.

\subsubsection[Update of Gaussian tree precision matrix]{Update of $\Omega_T$} \label{up:omega}
The update of $\Omegabf_T$ respects
$$\Omegabf^{t+1}  = \arg\max_\Omegabf \; \Esp_{q^t} \left[ \log p_{\Omegabf}(\Ubf \mid T) \right].$$
This is a problem of parameter optimisation in the context of Gaussian Graphical Models (GGM).
In what follows, for any $q\times q$  matrix $A$, $A_{[kl]}$ will refer to the bloc $kl$ of $A$: $A_{[kl]}=(a_{ij})_{\{i,j\}\in\{k,l\}}$.   $[A_{[kl]}]^q$ will then denote the matrix obtained by filling up with zero entries to obtain full dimension $q\times q$, so that:
$$([A_{[kl]}]^q )_{ij}=\left\{ \begin{array}{rl}
a_{ij} & \text{if } \{i,j\}\in\{k,l\}\\
0 &  \text{if } \{i,j\}\in\{1,..., q\}_{\setminus kl}
\end{array}\right.$$
In its proposition 5.9, \citet{Lau96} states that in a  GGM with $p$ variables and associated with the decomposable graph $\mathcal{G}$, the maximum likelihood of the precision matrix exists if and only if $n > \max_{C\in \mathcal{C}} |C|$. It is then given as 
$$\widehat{\Omega}=n\left(\sum_{C\in \mathcal{C}} [SSD_{[C]}\,^{-1}]^p - \sum_{S\in \mathcal{S}} \nu(S)\,[SSD_{[S]}\,^{-1}]^p \right)$$
where $\mathcal{C}$ is the set of cliques and $\mathcal{S}$ the set of separators of $\mathcal{G}$, with associated multiplicities $\nu(S)$.\\


In our context, $\mathcal{G}$ is a spanning tree and so all cliques are edges and separators are nodes. The multiplicity of a given node $k$ as a separator in the graph is  $\nu(k) = d(k)-1$, where $d(k)$ is its degree. Therefore the estimator  $\widehat{\Omega}_T$  writes as the following 
\begin{align*}
\widehat{\Omega}_T &= n  \sum_{kl\in T}   [(SSD_{[kl]})^{-1}]^{p+r} - n\sum_k (d(k)-1)[(SSD_{kk})^{-1}]^{p+r}\\
&=n \sum_{kl\in T}  [(SSD_{[kl]})^{-1} - (SSD_{kk})^{-1} -  (SSD_{ll})^{-1} ]^{p+r} + n\sum_k[(SSD_{kk})^{-1}]^{p+r}
\end{align*}
As $SSD$ has diagonal $n$, the expression simplifies. Denoting $I_d$ the identity matrix of dimension $d$ we obtain:
$$\widehat{\Omega}_T =n\sum_{kl\in T} [(SSD_{[kl]})^{-1} -\frac{1}{n} I_2]^{p+r}+ I_{p+r}.$$

Detailing each bloc matrices as follows gives the update formulas in (\ref{omegaT}):
\[
n\times [(SSD_{[kl]})^{-1} - \frac{1}{n}I_2] = \frac{1}{1-(ssd_{kl}/n)^2}
\left(\begin{array}{cc}
		(ssd_{kl}/n)^2   & -ssd_{kl}/n\\
		-ssd_{kl}/n& (ssd_{kl}/n)^2 
		\end{array}\right)
\]


\subsubsection[for the toc]{Determinant of $\Omegabf_T$.}
The determinant of a precision matrix of a GGM with a decomposable graph is expressed as follows \citep{Lau96}:
$$ |\Omega| =\dfrac{\prod_{C\in \mathcal{C}} |\Sigma_C|^{-1}}{\prod_{S\in \mathcal{S}} |\Sigma_S|^{-\nu(S)}},$$
where $\Sigma = \Omega^{-1}$. As $\Omegabf_T$ is tree-structured, its determinant factorizes on the edges of $T$. It is expressed with the correlation matrix $\Rbf_T$ as follows, denoting $d(k)$ the degree of node $k$:
\begin{align*}
|{\Omegabf}_T| &=\frac{\prod_{kl \in T} |{\Rbf}_{Tkl}|^{-1}}{\prod_k |{\Rbf}_{Tkk}|^{1-d(k)}} 
 \end{align*}
Using that $\Rbf_T$ has diagonal 1, we obtain for step $t+1$ of the algorithm:
$$|\Omegabf^{t+1}_{T}| = \Big(\prod_{kl \in T} |\Rbf_{T[kl]}^{t+1}|\Big)^{-1}.$$


\subsubsection{Numerical issues.} \label{app:numIssues}

\paragraph{Exact computations} Our algorithm requires the computation of determinants (from the Matrix Tree Theorem) and inverses (in Kirshner's formula) of Laplacian of weight matrices. As we deal with highly variable weights, numerical issues arise: infinite determinants or matrix numerically non-invertible due to either the maximal machine precision (about $1.7\cdot 10^{308}$), or with machine zero (about $2.2 \cdot 10^{-16}$). To enhance the precision of such computations, we rely on multiple-precision arithmetic which allows the digit of precision of numbers to be  limited only by the available memory instead of 64 bits. We implemented matrix inversion and log-determinant computation using both, symbolic computation and multiple precision arithmetic, relying on the \texttt{gmp} R package available on CRAN, which uses \citep{lucas2020package}, the C library GMP (GNU Multiple Precision Arithmetic). 

\paragraph{Tempering parameter $\alpha$} \label{alpha}
\begin{description}
\item[definition]Weights $\widetilde{\beta}$ are mechanically linked to the quantity of data available $n$. To avoid reaching maximal precision when computing the determinant, a tempering parameter $\alpha$ is applied to every quantity proportional to $n$, so that the actual update performed is $$\log \widetilde{\beta}_{kl} = \log \beta_{kl} - \alpha(\frac{n}{2}\log|\widehat{\Rbf}_{Tkl}| + \widehat{\omega}_{Tkl} [M^\intercal M]_{kl}).$$
\item[Heuristic for an upper bound] The proposed algorithm requires the computation of the normalizing constant $\widetilde{B}$, which is the determinant of any minor of the Laplacian  of the $q\times q$ variational weights matrix $\betabft$. As these weights  mechanically increase with the quantity of available data $n$, this step is numerically very sensitive.  Hereafter we denote $|\Qbf^{uv}|$ this determinant and $\Delta$ the maximal machine precision. In order to ease the computations, we define the tempering parameter $\alpha$ as $$\log \widetilde{\beta}_{kl} = \log \beta_{kl} - \alpha(\frac{n}{2}\log|\widehat{\Rbf}_{Tkl}| + \widehat{\omega}_{Tkl} [M^\intercal M]_{kl})\;,\qquad \text{under constraint}\;\;\; |\Qbf^{uv}| \leq \Delta.$$

Let's first detail the expression for $\widetilde{\beta}_{kl}$. Following the definition of the $SSD$ matrix, and update formulas \eqref{omegaT} and \eqref{RT}, we obtain:
\begin{align*}
    \log \widetilde{\beta}_{kl} &=\log \beta_{kl} +\alpha \,n\left\{\frac{(ssd_{kl}/n)^2}{1-(ssd_{kl}/n)^2} -\frac12\log\big[1-(ssd_{kl}/n)^2\big]\right\}
\end{align*}
For large $n$, we thus have $$\widetilde{\beta}_{kl}\approx \exp \big[\alpha n \cdot C(ssd_{kl}/n)\big], \qquad \text{with }\; C(x)=x/(1-x) -\log(\sqrt{1-x}),\; x\in [0,1[.$$ 
We then define $C_{sup}$ such that $C_{sup} = C(ssd_{max})$, with $ ssd_{max}=\max\{ssd_{kl}, k\neq l\}$.
By definition, $\Qbf^{uv}$ is positive-definite, so its determinant is upper bounded by the product of its diagonal terms (Hadamard's inequality). Namely:
\begin{align*}
    |\Qbf^{uv}|&\leq \prod_{i=1}^{q-1} \Qbf^{uv}_{ii} \leq \prod_{i=1}^{q-1}\sum_{i=1}^{q-1} \exp (\alpha C_{sup} n)\\
    &\leq \left[(q-1)\exp(\alpha C_{sup} n)\right]^{q-1}
\end{align*}
Then applying the constraint yields:
\begin{align*}
    |\Qbf^{uv}| \leq \Delta \iff  \alpha \leq \frac{1}{C_{sup} n} \left[ \frac{1}{q-1}\log \Delta - \log(q-1)\right] 
\end{align*}

For $C_{sup}=0.8$, $n=200$ and $q=15$, we get $\alpha \leq 1.05\cdot 10^{-1}$.
\end{description}

%A heuristic for an upper bound of $\alpha$ is given in appendix \ref{alpha}.
%We provide a heuristic to set the parameter $\alpha$.


\tocless\subsection{Model selection and cross-validation} \label{sec:modSel}

%%%%%%%%%%%%%%%%%%%%%%%%%%%%%%%%%%%%%%%%%%%%%%%%%%%%%%%%%%%%%%%%%%%%%%%%%%%%%%%%%%%%%%%%
\subsubsection{Sampling spanning trees} \label{eq:sampTree}
%%%%%%%%%%%%%%%%%%%%%%%%%%%%%%%%%%%%%%%%%%%%%%%%%%%%%%%%%%%%%%%%%%%%%%%%%%%%%%%%%%%%%%%%%%%
Sampling non-uniform spanning trees (i.e. sampling $T$ from $p_\betabf$) is a research topic by itself, especially for large networks \citep[see][for a review]{DKP17}. For moderate size networks, a rejection algorithm \citep{Dev86} can be defined in the following way:
\begin{enumerate}
\item Sample $T$ from a distribution $q$, such that there exist a constant $M$, that ensures that, for all $T$, $M q(T) > p_\betabf(T)$;
\item Keep $T$ with probability $M^{-1} p_\betabf(T) / q(T)$ or try step 1 again.
\end{enumerate}
The efficiency of such an algorithm strongly relies on the choice of the proposal distribution. Here we adopt the following proposal:
\begin{enumerate}[label=\roman*]
\item Sample a connected graph $G$ with independent edges, each drawn with probability $Q_{jk} \propto P_{jk} = \Pr_\betabf\{ jk \in T\}$; 
\item Sample $T$ uniformly among the spanning trees of $G$.
\end{enumerate}
%
\paragraph{Evaluation of the proposal.}
To evaluate the proposal distribution for each sampled tree, we may observe that, the probability for a graph drawn from the proposal to contain a given tree $T$ is approximately
$$
{\Pr}_q\{G \ni T\} \approx \prod_{jk \in T} Q_{jk},
$$
the approximation being due to the connectivity constraint. This constraint can be almost surely satisfied by taking $Q_{jk}$'s large enough. So, denoting $|\Tcal(G)|$ the number of spanning trees in $G$, we have that
\begin{align*}
q(T) 
= \sum_{G \ni T} q(T \mid G) q(G)  = \sum_{G \ni T} \frac{q(G)}{|\Tcal(G)|} 
= {\Pr}_q\{G \ni T\} \; \Esp\left(|\Tcal(G)|^{-1} \mid G \ni T \right).
\end{align*}
The last expectation can be evaluated via Monte-Carlo, by sampling a series of graphs $G$ according to the proposal $q$ but forcing all edges from $T$ to be part of $G$. 
%
\paragraph{Upper bounding constant $M$.}
To evaluate the upper bounding constant $M$, we may observe that finding the tree $T^*$ such that
$$
m_\betabf 
:= \frac{{\Pr}_q\{G \ni T^*\}}{p_\betabf(T^*)}
= \min_{T \in \Tcal} \frac{{\Pr}_q\{G \ni T\}}{p_\betabf(T)} = \min_{T \in \Tcal} \prod_{jk \in T} \frac{Q_{jk}}{\beta_{jk}}
$$
is a minimum spanning tree problem. Then, obviously, for any tree $T$: ${\Pr}_q\{G \ni T\} \geq m_\betabf p_\betabf(T)$.
Now, because the maximum number of spanning trees within a graph is $p^{p-2}$, we have
$$
M q(T)
= M \sum_{G \ni T} \frac{q(G)}{|\Tcal(G)|} 
\geq \frac{M}{p^{p-2}} \sum_{G \ni T} q(G)
= \frac{M}{p^{p-2}} {\Pr}_q\{G \ni T\}
\geq M \frac{m_\betabf}{p^{p-2}}  p_\betabf(T)
$$
So we may set $M = p^{p-2} / m_\betabf$. Still, in practice, this bounds turns out to be far too large and needs to be tuned down to preserve computational efficiency.

%%%%%%%%%%%%%%%%%%%%%%%%%%%%%%%%%%%%%%%%%%%%%%%%%%%%%%%%%%%%%%%%%%%%%%%%%%%%%%%%%%%%%%%%%%%
 
\subsubsection{Cross-validation for model selection} \label{eq:cvAlgo}
\label{CV}

The cross-validation procedure to estimate the pairwise composite likelihood is given in Algorithm \ref{algo:model-selection}. In practice $V=10$ and $B = 100$.


\begin{algorithm}%[H]
\caption{Cross-validation for model selection with $r$ missing actors}
\label{algo:model-selection}
%  \dontprintsemicolon
  \CommentSty{// 0. INITIALIZATION}\; 
  Divide the dataset $\Ybf$ into $V$ subset $\Ybf^1, \dots \Ybf^V$;
  \BlankLine
  \For{$v \in \{1,\cdots, V\}$}{
    \BlankLine
    \CommentSty{// 1.   Apply the VEM algorithm to the train dataset $\Ybf^{-v}$}\; 
    $\Gammabf_r^{-v} \leftarrow (\thetabf_r^{-v}, \sigmabf_r^{-v}, \betabf^{-v}_r, \Omegabf_r^{-v})$
    \CommentSty{// 2. MONTE CARLO APPROXIMATION OF COMPLETE LOG-LIKELIHOOD EXPECTATION}\;
    \For{$b \in \{1,\cdots, B\}$}{
    \CommentSty{// 2.1 Draw tree (see Section \ref{eq:sampTree})}\; 
     $ T_{r, b}^{-v}  \sim p_{\betabf^{-v}_r}$ 
    \BlankLine
     \CommentSty{// 2.2. Build  the precision matrix having non-nul entries determined by $ T_{r, b}^{-v} $ and values stored in $\Omegabf_r^{-v}$, and its diagonal terms according to \eqref{omegaT}}\;
     $\Omegabf_{T^b}\leftarrow f( T_{r, b}^{-v} , \Omegabf_r^{-v} )$
     \BlankLine
      \CommentSty{// 2.3. Compute the marginal variance matrix }\;
      $\Sigmabf_{T^bO} \leftarrow  \Omegabf_{T^bOO} - \Omegabf_{T^bOH} \Omegabf_{T^bHH}^{-1} \Omegabf_{T^bHO}$;
          \BlankLine 
     \CommentSty{// 2.4. Compute the bivariate Poisson log-normal density in test sites}\;
     \For{site $i \in v$}{
        \For{pairs of species $(j,k)$}{
        $p_{PLN}\left((Y^v_{ij}, Y^v_{ik}); \Gammabf_r^{-v}, T_{r, b}^{-v} \right)$ with means $\xbf_i^\intercal \thetabf_{r, j}^{-v}$ and $\xbf_i^\intercal \thetabf_{r, k}^{-v}$ and variance matrix $[\Sigmabf_{T^bO}]_{[jk, jk]}$
%        $\log p_{PLN}[(Y_{ij}^v, Y_{ik}^v) | T^b; \widehat{\theta},\widehat{\Sigma}_{T^b jk}]$, 
         }}
       \CommentSty{// 2.5.  Compute the average}\;
        $$
        PCL_{rvb}(\Ybf^v, \Gammabf_r^{-v}, T^b) = \frac1{m_v} \sum_{i = 1}^{m_v} \sum_{j < k} \log p_{PLN}\left((Y^v_{ij}, Y^v_{ik}); \Gammabf_r^{-v}, T_{r, b}^{-v} \right)
        $$
%        $f_{PLN}(\Ybf; b,v)=\sum_{\substack{i \in v\\ j < k}} \log p_{PLN}[(Y_{ij}^v, Y_{ik}^v) | T^b; \widehat{\theta},\widehat{\Sigma}_{T^b jk}]$
    }
    \BlankLine        
  }  
   \CommentSty{// 3. AVERAGE OVER SUBSETS}\; 
$$
PCL_r(\Ybf) = \frac1V \sum_v PCL_{rv}(\Ybf^v, \Gammabf_r^{-v}) .
$$
  \BlankLine
\end{algorithm}

  %%%%%%%%%%%%%%%%%%%%%%%%%%%%%%%%%%%%%%%%%%%%%%%%%%%%%%%%%%%%%%%%%%%%%%%%%%%%%%%%%%%%%%%%%%%
%The cross-validation procedure to estimate the pairwise composite likelihood is as follows:
%\begin{enumerate}
%\item Divide the dataset $\Ybf$ into $V$ subset $\Ybf^1, \dots \Ybf^V$;
%\item For each subset $\Ybf^v$, do:
%\begin{enumerate}
%    \item Apply the VEM algorithm to the train dataset $\Ybf^{-v}$ to get the estimates $\Gammabf_r^{-v} = (\thetabf_r^{-v}, \sigmabf_r^{-v}, \betabf^{-v}_r, \Omegabf_r^{-v})$
%    \item Repeat $B$ times:
%        \begin{enumerate}
%        \item Draw tree $T^b \sim p_{\betabf^{-v}_r}$ using the procedure described in Section \ref{eq:sampTree};
%        \item Build $\Omegabf_{T^b}$ as the precision matrix having non-nul entries determined by $T^b$ and values stored in $\Omegabf_r^{-v}$, and its diagonal terms according to \eqref{omegaT};
%        \item Compute the marginal variance matrix $\Sigmabf_{T^bO} =  \Omegabf_{T^bOO} - \Omegabf_{T^bOH} \Omegabf_{T^bHH}^{-1} \Omegabf_{T^bHO}$;
%        \item For each site of test data $\Ybf^{v}$ and each pair of observed species, compute the bivariate Poisson log-normal density  
%        $p_{PLN}\left((Y^v_{ij}, Y^v_{ik}); \Gammabf_r^{-v}, T_{r, b}^{-v} \right)$ with means $\xbf_i^\intercal \thetabf_{r, j}^{-v}$ and $\xbf_i^\intercal \thetabf_{r, k}^{-v}$ and variance matrix $[\Sigmabf_{T^bO}]_{[jk, jk]}$
%        $\log p_{PLN}[(Y_{ij}^v, Y_{ik}^v) | T^b; \widehat{\theta},\widehat{\Sigma}_{T^b jk}]$, 
%        and compute the mean
%        $$
%        PCL_{rvb}(\Ybf^v, \Gammabf_r^{-v}, T^b) = \frac1{m_v} \sum_{i = 1}^{m_v} \sum_{j < k} \log p_{PLN}\left((Y^v_{ij}, Y^v_{ik}); \Gammabf_r^{-v}, T_{r, b}^{-v} \right)
%        $$
%%        $f_{PLN}(\Ybf; b,v)=\sum_{\substack{i \in v\\ j < k}} \log p_{PLN}[(Y_{ij}^v, Y_{ik}^v) | T^b; \widehat{\theta},\widehat{\Sigma}_{T^b jk}]$
%        \end{enumerate}
%    \item Average over the trees
%    $$
%    PCL_{rv}(\Ybf^v, \Gammabf_r^{-v}) = \frac1B \sum_{b=1}^B PCL_{rvb}(\Ybf^v, \Gammabf_r^{-v}, T^b) 
%    $$
%    \end{enumerate} 
%\item Average over the subsets
%$$
%PCL_r(\Ybf) = \frac1V \sum_v PCL_{rv}(\Ybf^v, \Gammabf_r^{-v}) .
%$$
%    \item Compute the average criteria defined as: $$\displaystyle PCL(\Ybf)=\frac1V\sum_{v=1}^{V}\frac1B \sum_{b=1}^Bf_{PLN}(\Ybf; b,v)$$
%\end{enumerate}


  

\section{Clique initialization}
\section{Comparison with PLN-network and EMtree}
This section compares the network inference methods from count data which build on the estimation of the PLN model as described in \citet{CMR18}: PLN-network \citep{CMR19}, EMtree and nestor. The R package PLN-network performs a sparse estimation of the precision matrix using the glasso. Details about EMtree and nestor are available respectively in Chapter 2 and Chapter 3 of this work.
\tocless\subsection{EMtree and nestor}
\subsubsection*{Different models}

The aim of EMtree is the network inference, and as such it focuses on the update of the tree distribution parameters (the edges weights $\betab$). It thus performs network inference without ever considering the precision matrix, which is the main difference with nestor.  Indeed, the inference of missing actors requires the estimation of their means $M_H$, which directly depends on terms of the proxy for the precision matrix ($\xbar{\Omegabf}$). $M_H$ is actually the parameter through which passes the information from the data to the network parameters $\betabft$, therefore the computation of the edges probabilities is also different. 

EMtree approximates the edges probabilities conditional on  $\Ybf$ by applying the Matrix Tree theorem on the matrix $\betabf \,\odot \,\boldsymbol{\psi}$, where $\boldsymbol{\psi}=\left((1-\rho_{kl}^2)^{-n/2}\right)_{kl}$ with $\rho_{kl}$ the correlation between variables $k$ and $l$. Transposed in the variational inference framework, edges probabilities conditional on $\Ybf$ are approximated by the edges probabilities computed from the distribution $g(T)$, which approximates $p(T\mid\Ybf)$. The distribution $g(T)$ is parametrized with the  variational edges weights $\betabft$, which update formula for the edges $kl$ at step $t+1$ is given in equation \eqref{update:beta}: $$ \betat^{t+1}_{kl} = \beta^{t+1}_{kl} \left|\Rbf_{[kl]}^{t+1}\right|^{-n/2} \exp\left(-\omega^{t+1}_{kl} \left[(\Mbf^t)^\intercal \Mbf^t\right]_{kl}\right).$$
Now noticing that $ \left|\Rbf_{[kl]}\right|^{-n/2} =\psi_{kl}$,  we can link this formula to that of the quantity used by EMtree to compute probabilities. We see that the difference between the two strategies is the term $\exp\left(-\omega^{t+1}_{kl} \left[(\Mbf^t)^\intercal \Mbf^t\right]_{kl}\right)$, which stems from the modeling difference of EMtree and nestor.

\subsubsection*{Different behaviors}
\begin{itemize}
\item nestor is more complete in its estimation, probabilities are discriminant. But numerically sensitive
\item EMtree is more robust
\end{itemize}

\tocless\subsection{Performance comparison}
\subsubsection*{Experiment}
We compared the three methods ability to infer networks based on AUC,  precision and recall criteria. We simulated 100 datasets with $n=200$ samples  and $p=15$ species for erdos and cluster structures, and tested two density levels ($3/p$ and $5/p$). With density $3/p$, nestor degenerated for 25 clusters and 5 erdos, and for only 2 clusters with density $5/p$.  Those datasets were filtered from the following results.

\subsubsection*{Results}
\begin{figure}
\centering
\includegraphics[width=12cm]{figs/AUC_PLN_EM_VEM.png}
\caption{Comparison of PLN-network, EMtree and nestor ability to infer networks with the AUC criteria. 100 datasets were simulated for each parameter settings.}
\label{compar:auc}
\end{figure}
\begin{figure}
\centering
\includegraphics[width=12cm]{figs/precrec_PLN_EM_VEM.png}
\caption{Mean Precision-Recall curves for PLN-network, EMtree and nestor }
\label{compar:precrec}
\end{figure}

with $min.ratio=1\cdot 10^{-3}$ for PLNnetwork
\section{Vignette for nestor}

nestor (Network inference from Species counTs with missingactORs) is an R package for the inference of species interaction networks from their observed abundances, while accounting for possible unobserved missing actors in the data.

\subsection{Simulation and preparation}\label{simulation-and-preparation}

\texttt{nestor} can simulate data with with missing actors with the
function \texttt{missing\_from\_scratch()}. It requires the desired type
of dependency structure (scale-free, erdos, tree or cluster) and the
number of missing actors \texttt{r}. Here is an example with
\texttt{r=1} for the scale-free structure:

\begin{Shaded}
\begin{Highlighting}[]
\KeywordTok{library}\NormalTok{(nestor)}
\NormalTok{p=}\DecValTok{10}
\NormalTok{r=}\DecValTok{1}
\NormalTok{n=}\DecValTok{100}
\NormalTok{data=}\KeywordTok{missing_from_scratch}\NormalTok{(n, p, r,}\DataTypeTok{type=}\StringTok{"scale-free"}\NormalTok{, }\DataTypeTok{plot=}\OtherTok{TRUE}\NormalTok{)}
\end{Highlighting}
\end{Shaded}

\begin{center}\includegraphics[width=0.3\linewidth]{nestorArticle/man/figures/README-unnamed-chunk-2-1} \end{center}

The original clique of the missing actor neighbors is available in the
value \texttt{TC}:

\begin{Shaded}
\begin{Highlighting}[]
\NormalTok{data}\OperatorTok{\$}\NormalTok{TC}
\CommentTok{#> [[1]]}
\CommentTok{#> [1]  1  4  5  6 10}
\end{Highlighting}
\end{Shaded}

The data is then prepared for analysis with the first step of the
procedure: fit the PLN model. The \texttt{norm\_PLN()} function is a
wraper to \texttt{PLNmodels::PLN()} which normalizes all the necessary
outputs, namely the mean, variance and correlation matrices of the model
latent Gaussian layer corresponding to observed species.

\begin{Shaded}
\begin{Highlighting}[]
\NormalTok{PLNfit<-}\KeywordTok{norm_PLN}\NormalTok{(data}\OperatorTok{\$}\NormalTok{Y)}
\NormalTok{MO<-PLNfit}\OperatorTok{\$}\NormalTok{MO}
\NormalTok{SO<-PLNfit}\OperatorTok{\$}\NormalTok{SO}
\NormalTok{sigma_obs=PLNfit}\OperatorTok{\$}\NormalTok{sigma_obs}
\end{Highlighting}
\end{Shaded}

\subsection{Inference}\label{inference}

\subsubsection{Single clique
initialization}\label{single-clique-initialization}

\texttt{nestor} then needs to be initialized. This requires to find an
initial clique of neighbors for the missing actor, for example using the
\texttt{FitSparsePCA()} function:

\begin{Shaded}
\begin{Highlighting}[]
\NormalTok{initClique =}\StringTok{ }\KeywordTok{FitSparsePCA}\NormalTok{(data}\OperatorTok{\$}\NormalTok{Y,}\DataTypeTok{r=}\DecValTok{1}\NormalTok{, }\DataTypeTok{min.size =} \DecValTok{3}\NormalTok{)}\OperatorTok{\$}\NormalTok{cliques}
\NormalTok{initClique}
\CommentTok{#> [[1]]}
\CommentTok{#> [1]  2  5  7  9 10}
\end{Highlighting}
\end{Shaded}

The \texttt{min.size} parameter defines the minimal size of the output
clique. The function \texttt{init\_mclust()} is also available for
finding a clique, it uses the package \texttt{mclust}.

Once an initial clique has been found, the algorithm can be initialized.
This is the aim of the function \texttt{initVEM()}, which initializes
all required parameters. This function builds one initialization from
one initial clique. We initialize with the clique previously identified:

\begin{Shaded}
\begin{Highlighting}[]
\NormalTok{initList =}\StringTok{ }\KeywordTok{initVEM}\NormalTok{(data}\OperatorTok{\$}\NormalTok{Y, }\DataTypeTok{cliqueList=}\NormalTok{initClique, sigma_obs, MO, }\DataTypeTok{r=}\DecValTok{1}\NormalTok{ )}
\end{Highlighting}
\end{Shaded}

Then to set the tempering parameter \texttt{alpha}, we can look at the
output of the \texttt{alphaMax()} function.

\begin{Shaded}
\begin{Highlighting}[]
\KeywordTok{alphaMax}\NormalTok{(p}\OperatorTok{+}\NormalTok{r, n)}
\CommentTok{#> [1] 0.3000768}
\end{Highlighting}
\end{Shaded}

The actual tempering parameter should be lower than the upper bound
given by \texttt{alphaMax()}. Here we set \texttt{alpha} to \(0.1\). The
core function \texttt{nestor()} can now be run as follows:

\begin{Shaded}
\begin{Highlighting}[]
\NormalTok{fit =}\StringTok{ }\KeywordTok{nestor}\NormalTok{(data}\OperatorTok{\$}\NormalTok{Y, MO,SO, }\DataTypeTok{initList=}\NormalTok{initList, }\DataTypeTok{alpha=}\FloatTok{0.1}\NormalTok{, }\DataTypeTok{eps=}\FloatTok{1e-3}\NormalTok{, }
           \DataTypeTok{maxIter=}\DecValTok{30}\NormalTok{)}
\CommentTok{#> }
\CommentTok{#> nestor ran in 1.078secs and 24 iterations.}
\end{Highlighting}
\end{Shaded}

The object \texttt{fit} contains inferred means and variances of the
complete data, as well as edges weight and probability matrices.

This package contains several visualization functions.
\texttt{plotPerf()} gives a quick overview of the fit performance
compared to initial graph:

\begin{Shaded}
\begin{Highlighting}[]
\KeywordTok{plotPerf}\NormalTok{(fit}\OperatorTok{\$}\NormalTok{Pg, data}\OperatorTok{\$}\NormalTok{G,}\DataTypeTok{r=}\DecValTok{1}\NormalTok{)}
\end{Highlighting}
\end{Shaded}

\begin{center}\includegraphics[width=0.8\linewidth]{nestorArticle/man/figures/README-unnamed-chunk-9-1} \end{center}

The convergence of \texttt{nestor()} can be checked with the plotting
function \texttt{plotConv()}:

\begin{Shaded}
\begin{Highlighting}[]
\KeywordTok{plotConv}\NormalTok{(}\DataTypeTok{nestorFit =}\NormalTok{ fit)}
\end{Highlighting}
\end{Shaded}

\begin{center}\includegraphics[width=0.8\linewidth]{nestorArticle/man/figures/README-unnamed-chunk-10-1} \end{center}

\subsubsection{Initialization with a list of
cliques}\label{initialization-with-a-list-of-cliques}

The fit of the \texttt{nestor()} function is very sensitive to the
initialization, and so it is recommanded to try several initial cliques.
Several functions are available for finding a list of possible starting
points:

\begin{itemize}
\tightlist
\item
  \texttt{init\_blockmodels()} uses package \texttt{blockmodels},
\item
  \texttt{boot\_FitSparsePCA()} is a bootstraped version using
  \texttt{sparsepca},
\item
  \texttt{complement\_spca()} looks in the complement of the
  \texttt{sparsepca} output.
\end{itemize}

Here we use the \texttt{complement\_spca()} function, which runs
\texttt{sparsepca} and returns the cliques corresponding to the
\texttt{k} first principal components as well as their complement.

\begin{Shaded}
\begin{Highlighting}[]
\NormalTok{six_cliques =}\StringTok{ }\KeywordTok{complement_spca}\NormalTok{(data}\OperatorTok{\$}\NormalTok{Y, }\DataTypeTok{k=}\DecValTok{3}\NormalTok{) }
\NormalTok{six_cliques}
\CommentTok{#> [[1]]}
\CommentTok{#> [[1]][[1]]}
\CommentTok{#> [1] 2 9}
\CommentTok{#> }
\CommentTok{#> [[2]]}
\CommentTok{#> [[2]][[1]]}
\CommentTok{#> [1]  7 10}
\CommentTok{#> }
\CommentTok{#> [[3]]}
\CommentTok{#> [[3]][[1]]}
\CommentTok{#> [1]  2  5 10}
\CommentTok{#> }
\CommentTok{#> [[4]]}
\CommentTok{#> [[4]][[1]]}
\CommentTok{#> [1]  1  3  4  5  6  7  8 10}
\CommentTok{#> }
\CommentTok{#> [[5]]}
\CommentTok{#> [[5]][[1]]}
\CommentTok{#> [1] 1 2 3 4 5 6 8 9}
\CommentTok{#> }
\CommentTok{#> [[6]]}
\CommentTok{#> [[6]][[1]]}
\CommentTok{#> [1] 1 3 4 6 7 8 9}
\end{Highlighting}
\end{Shaded}

This package provides with a parllel procedure for the computation of
several fits of \texttt{nestor()} corresponding to a list of possible
cliques, with the function \texttt{List.nestor()}. Below is an example
with the list of six cliques previously obtained with the
\texttt{complement\_spca()} function:

\begin{Shaded}
\begin{Highlighting}[]
\NormalTok{fitList=}\KeywordTok{List.nestor}\NormalTok{(six_cliques, data}\OperatorTok{\$}\NormalTok{Y, sigma_obs, MO,SO,}\DataTypeTok{r=}\DecValTok{1}\NormalTok{,}
\DataTypeTok{eps=}\FloatTok{1e-3}\NormalTok{,} \DataTypeTok{maxIter =} \DecValTok{50}\NormalTok{, }\DataTypeTok{alpha=}\FloatTok{0.1}\NormalTok{)}
\end{Highlighting}
\end{Shaded}

The object \texttt{fitList} is simply the list of all the
\texttt{nestor()} fits. This procedure aborts in case of degenerated
behaviour, which happens when the provided clique is too far from truth.
Wrong fits can be identified by their ouput size:

\begin{Shaded}
\begin{Highlighting}[]
\KeywordTok{do.call}\NormalTok{(rbind,}\KeywordTok{lapply}\NormalTok{(fitList, length))}
\CommentTok{#>      [,1]}
\CommentTok{#> [1,]    3}
\CommentTok{#> [2,]   12}
\CommentTok{#> [3,]    3}
\CommentTok{#> [4,]   12}
\CommentTok{#> [5,]   12}
\CommentTok{#> [6,]   12}
\end{Highlighting}
\end{Shaded}

Finally we can assess the performance of each converged fit with their
AUC, precision and recall regarding the hidden node \texttt{h}, and the
correlation between the inferred means and the original latent Gaussian
vector of \texttt{h}.

\begin{Shaded}
\begin{Highlighting}[]
\KeywordTok{do.call}\NormalTok{(rbind,}\KeywordTok{lapply}\NormalTok{(fitList, }\ControlFlowTok{function}\NormalTok{(vem)\{}
  \ControlFlowTok{if}\NormalTok{(}\KeywordTok{length}\NormalTok{(vem)}\OperatorTok{>}\DecValTok{4}\NormalTok{)\{}
\NormalTok{    perf=}\KeywordTok{ppvtpr}\NormalTok{(vem}\OperatorTok{\$}\NormalTok{Pg, data}\OperatorTok{\$}\NormalTok{G, }\DataTypeTok{r=}\NormalTok{r)}
    \KeywordTok{c}\NormalTok{(}\DataTypeTok{auc=}\KeywordTok{auc}\NormalTok{(vem}\OperatorTok{\$}\NormalTok{Pg, data}\OperatorTok{\$}\NormalTok{G),}\DataTypeTok{precH=}\NormalTok{perf}\OperatorTok{\$}\NormalTok{PPVH, }\DataTypeTok{recH=}\NormalTok{perf}\OperatorTok{\$}\NormalTok{TPRH, }
    \DataTypeTok{corMH=}\KeywordTok{cor}\NormalTok{(vem}\OperatorTok{\$}\NormalTok{M[,p}\OperatorTok{+}\NormalTok{r], data}\OperatorTok{\$}\NormalTok{UH))}
\NormalTok{  \}}
\NormalTok{\})) }\OperatorTok\StringTok{ }\KeywordTok{as_tibble}\NormalTok{()  }
\CommentTok{#> # A tibble: 4 x 4}
\CommentTok{#>     auc precH  recH  corMH}
\CommentTok{#>   <dbl> <dbl> <dbl>  <dbl>}
\CommentTok{#> 1  0.81  0.6   0.5  -0.785}
\CommentTok{#> 2  0.93  0.75  1    -0.815}
\CommentTok{#> 3  0.76  0.5   0.67 -0.715}
\CommentTok{#> 4  0.69  0.43  0.5  -0.585}
\end{Highlighting}
\end{Shaded}


 
\end{subappendices}