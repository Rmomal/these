

\documentclass[12pt]{phd}
 %%%%%%%%% PAckage %%%%%%%%%
%% intro 

\usepackage{enumerate}
\usepackage{smartdiagram}
\usepackage{graphicx,amsmath,amssymb,stmaryrd, amsthm}
\usepackage{ulem}
\usepackage{pifont}
\usepackage{amsfonts}
\usepackage{fancyhdr} 
\usepackage{color} % où xcolor selon l'installation
\usepackage{mdframed}
\usepackage{multirow} %% Pour mettre un texte sur plusieurs rangées
\usepackage{multicol} %% Pour mettre un texte sur plusieurs colonnes
\usepackage{tikz}
\usepackage[absolute]{textpos} 
\usepackage{colortbl}
\usepackage{array}
\usepackage{lineno}
%% JMVA 
\usepackage{enumitem,hyperref,ifthen,dsfont}

%%SAGMB
\usepackage{microtype}
\usepackage{epstopdf}

%%block
\usepackage{multirow}
\setlength\parindent{0pt}

% Functions
\def\independenT#1#2{\mathrel{\rlap{$#1 #2$}\mkern2mu{#1 #2}}}
\DeclareMathOperator*{\argmax}{arg\,max}
\DeclareMathOperator*{\argmin}{arg\,min}
%\DeclareMathOperator*{\Esp}{\mathbb{E}}
%\DeclareMathOperator*{\Var}{\mathbb{V}}
\DeclareMathOperator*{\Cov}{\mathbb{C}\text{ov}}
\DeclareMathOperator*{\prob}{\mathds{P}}
\DeclareMathOperator*{\probt}{\widetilde{\mathds{P}}}
\DeclareMathOperator{\sign}{sign}
\def\lambdaMax{\lambda_{\rm{max}}}
\def\lambdaMin{\lambda_{\rm{min}}}

% math
\newtheorem{theorem}{Theorem}[chapter]
\newtheorem{prop}[theorem]{Proposition} 
\newtheorem{lemma}[theorem]{Lemma}
\newtheorem{definition}{Definition}[chapter]
\newtheorem{remark}{Remark}[chapter]
\newtheorem{example}{Example}[chapter]
\newcommand{\Esp}{{\mathds{E}}}
\newcommand{\Var}{{\mathds{V}}}
\newcommand{\tr}[1]{\ensuremath{\rm tr}\left( #1 \right)}
\newcommand{\ve}{\ensuremath{\rm vec}}
\newcommand{\entr}{\mathcal{H}}
\newcommand{\corHTemp}{{\rho(H, \text{temp})}}
\newcommand{\corKTemp}{{\rho(k, \text{temp})}}
\newcommand{\STAB}[1]{\begin{tabular}{@{}c@{}}#1\end{tabular}}
\newcommand*\xbar[1]{%
   \hbox{%
     \vbox{%
       \hrule height 0.5pt % The actual bar
       \kern0.5ex%         % Distance between bar and symbol
       \hbox{%
         \kern-0.1em%      % Shortening on the left side
         \ensuremath{#1}%
         \kern-0.1em%      % Shortening on the right side
       }%
     }%
   }%
} 


% Notations pairs
\newcommand\jk{{jk}}

% Notations cal
\newcommand\Ccal{\mathcal{C}}
\newcommand\Ncal{\mathcal{N}}
\newcommand\Pcal{\mathcal{P}}
\newcommand\Tcal{\mathcal{T}}
\newcommand\G{\mathcal{G}}
%\newcommand\Ccal{{\mathcal{C}}}
\newcommand\Hcal{{\mathcal{H}}}
\newcommand\Jcal{{\mathcal{J}}}
%\newcommand\Ncal{{\mathcal{N}}}
%\newcommand\Pcal{{\mathcal{P}}}
%\newcommand\Tcal{{\mathcal{T}}}
\newcommand{\bound}{{\mathcal{J}}} 

% Notations bf
\newcommand\gammab{{\boldsymbol{\gamma}}}
\newcommand\betab{{\boldsymbol{\beta}}}
\newcommand\thetab{{\boldsymbol{\theta}}}
\newcommand\lambdab{{\boldsymbol{\lambda}}}
\newcommand\Lambdab{{\boldsymbol{\Lambda}}}
\newcommand\Sigmab{{\boldsymbol{\Sigma}}}
\newcommand\Omegab{{\boldsymbol{\Omega}}}
\newcommand\cst{\text{cst}}
\newcommand\Ob{{\bf O}}
\newcommand\Hb{{\bf H}}
\newcommand\Mbf{{\bf M}}
\newcommand\Qbf{{\bf Q}}
\newcommand\Abf{{\bf A}}
\newcommand\Wbf{{\bf W}}
\newcommand\Mb{{\bf M}}
\newcommand\Qb{{\bf Q}}
\newcommand\Wb{{\bf W}}
\newcommand\Xb{{\bf X}}
\newcommand\xb{{\boldsymbol{x}}}
\newcommand\Yb{{\bf Y}}
\newcommand\Zb{{\bf Z}}
\newcommand\zb{{\boldsymbol{z}}}
\newcommand\yb{{\boldsymbol{y}}}
\newcommand\Ibb{\mathbb{I}}
\newcommand{\betabf}{{\boldsymbol{\beta}}}
\newcommand{\thetabf}{{\boldsymbol{\theta}}}
\newcommand{\sigmabf}{{\boldsymbol{\sigma}}}
\newcommand{\Omegabf}{{\boldsymbol{\Omega}}}
\newcommand{\Sigmabf}{{\boldsymbol{\Sigma}}}
\newcommand{\Gammabf}{{\boldsymbol{\Gamma}}}
\newcommand{\zerobf}{{\boldsymbol{0}}}
\newcommand{\Xbf}{{\boldsymbol{X}}}
\newcommand{\xbf}{{\boldsymbol{x}}}
\newcommand{\Ybf}{{\boldsymbol{Y}}}
\newcommand{\Zbf}{{\boldsymbol{Z}}}
\newcommand{\hbf}{{\boldsymbol{h}}}
\newcommand{\Hbf}{{\boldsymbol{H}}}
\newcommand{\Ubf}{{\boldsymbol{U}}}
%\newcommand{\Mbf}{{\boldsymbol{M}}}
%\newcommand{\Qbf}{{\boldsymbol{Q}}}
\newcommand{\Rbf}{{\boldsymbol{R}}}
\newcommand{\Sbf}{{\boldsymbol{S}}}
\newcommand{\mbf}{{\boldsymbol{m}}}

% Notations tilde
\newcommand\Pt{\widetilde{P}}
\newcommand\pt{\widetilde{p}}
\newcommand\et{\widetilde{\mathds{E}}}
\newcommand\e{{\mathds{E}}}
\newcommand{\betabft}{{\widetilde{\betabf}}}
\newcommand{\Mbft}{{\widetilde{\Mbf}}}
\newcommand{\Sbft}{{\widetilde{\Sbf}}}
\newcommand\mt{\widetilde{m}}
\newcommand\St{\widetilde{S}}
\newcommand\mbt{\widetilde{\bf m}}
\newcommand\Sbt{\widetilde{\bf S}}
\newcommand{\betat}{{\widetilde{\beta}}}
\newcommand{\Bt}{{\widetilde{B}}}
%\newcommand{\Pt}{{\widetilde{P}}}

% TikZ
\newcommand{\nodesize}{0.4em}
\newcommand{\edgeunit}{6*\nodesize}
\tikzstyle{covariate}=[draw, rectangle, minimum width=\nodesize, minimum height=\nodesize, inner sep=0, color=black]
\tikzstyle{covmiss}=[draw, minimum width=\nodesize, minimum height=\nodesize, inner sep=0, color=gray, text=gray]
\tikzstyle{observed}=[draw, circle, minimum width=\nodesize, inner sep=0, color=black]
\tikzstyle{edge}=[-, line width=1pt, color=black]
\tikzstyle{edgemiss}=[-, line width=1pt, dashed, color=gray]
\tikzstyle{basic}=[draw, circle, minimum width=\nodesize, inner sep=0, color=black, fill=black]
\newcommand{\length}{1}
 \tikzstyle{clique}=[draw, rectangle, minimum width=2*\nodesize, minimum height=2*\nodesize, inner sep=0, color=black]
 \tikzstyle{variable}=[scale=0.9,rectangle,draw=white,transform shape,fill=white,font=\Large]

%coloring
\newcommand{\review}[1]{\textcolor{black}{#1}}
\newcommand{\validSR}[1]{\textcolor{black}{#1}}
\newcommand{\questSR}[1]{\textcolor{black}{[#1]}}
\newcommand{\newmodif}[1]{\textcolor{black}{#1}}
\newcommand{\modif}[1]{\textcolor{black}{#1}}
\newcommand{\remove}[1]{\textcolor{gray}{#1}}
\newcommand\independent{\protect\mathpalette{\protect\independenT}{\perp}}

 
\numberwithin{figure}{chapter}
\backgroundsetup{contents={}}
\newcommand{\nocontentsline}[3]{}
\newcommand{\tocless}[2]{\bgroup\let\addcontentsline=\nocontentsline#1{#2}\egroup}
\setcounter{chapter}{-1}

\begin{document}
\frontmatter

\begin{titlepage}


%\thispagestyle{empty}

\newgeometry{left=7.5cm,bottom=2cm, top=1cm, right=1cm}

\tikz[remember picture,overlay] \node[opacity=1,inner sep=0pt] at (-28mm,-135mm){\includegraphics{Bandeau_UPaS.png}};

% fonte sans empattement pour la page de titre
\fontencoding{T1}

\fontfamily{fvs}\fontseries{m}\selectfont


%*****************************************************
%******** NUMÉRO D'ORDRE DE LA THÈSE À COMPLÉTER *****
%******** POUR LE SECOND DÉPOT                   *****
%*****************************************************

\color{white}

\begin{picture}(0,0)

\put(-150,-735){\rotatebox{90}{NNT: 2020UPASM017}}
\end{picture}
 
%*****************************************************
%**  LOGO  ÉTABLISSEMENT PARTENAIRE SI COTUTELLE
%**  CHANGER L'IMAGE PAR DÉFAUT **
%*****************************************************
%\vspace{-10mm} % à ajuster en fonction de la hauteur du logo
%\flushright \includegraphics[scale=1]{logo2.png}




%*****************************************************
%******************** TITRE **************************
%*****************************************************
\flushright
\vspace{10mm} % à régler éventuellement
\color{Prune}
\fontfamily{fvs}\fontseries{m}\fontsize{22}{26}\selectfont
Network inference from incomplete abundance data
%*****************************************************

%\fontfamily{fvs}\fontseries{m}\fontsize{8}{12}\selectfont
\normalsize
\vspace{1.5cm}

\color{black}
\textbf{Thèse de doctorat de l'Université Paris-Saclay}

\vspace{15mm}

École doctorale n$^{\circ}$ 574, mathématiques Hadamard (EDMH)\\
\small Spécialité de doctorat : Mathématiques appliquées\\
\footnotesize Unité de recherche : Université Paris-Saclay, AgroParisTech, INRAE, UMR MIA-Paris, 75005, Paris, France.\\
\footnotesize Référent :  Faculté des sciences d’Orsay
\vspace{15mm}

\textbf{Thèse présentée et soutenue en visioconférence totale, le 12/11/2020, par}\\
\bigskip
\Large {\color{Prune} \textbf{Raphaëlle MOMAL}}


%************************************
\vspace{\fill} % ALIGNER LE TABLEAU EN BAS DE PAGE
%************************************

\flushleft \small \textbf{Composition du jury:}
\bigskip



\scriptsize
\begin{tabular}{|p{8cm}l}
\arrayrulecolor{Prune}
\textbf{Camille Coron} &  Examinatrice \\ 
Maître de Conférences, Université Paris-Saclay   &   \\ 
\textbf{Stéphane Dray} &  Examinateur \\ 
Directeur de Recherches, Université Claude Bernard Lyon 1 &   \\ 
\textbf{Florence Forbes} &  Rapportrice \\ 
Directrice de Recherches, Inria Grenoble Rhône-Alpes  &   \\ 
\textbf{Otso Ovaskainen} &  Rapporteur \\ 
Professor, University of Helsinky  &   \\ 
\textbf{Viet-Chi Tran} &   Président du jury\\ 
Professeur des Universités, Université Gustave-Eiffel & \\
\end{tabular} 

\medskip
\begin{tabular}{|p{8cm}l}\arrayrulecolor{white}
\textbf{Stéphane Robin} &   Directeur\\ 
Directeur de recherches, Université Paris-Saclay & \\
\textbf{Christophe Ambroise} &   Codirecteur\\ 
Professeur des Universités, Université Paris-Saclay  &   \\ 


\end{tabular} 


\end{titlepage}

 \clearemptydoublepage
\setcounter{tocdepth}{1}
\selectlanguage{english}
\dominitoc
\nomtcrule
 \clearemptydoublepage
 \newpage
 \chapter*{Remerciements}
 
  Merci Stéphane et Christophe pour ces trois dernières années (et un peu plus), durant lesquelles vous avez su accueillir mes doutes et mes erreurs avec patience et grand renforts de sourires. Ça a été un grand honneur de travailler avec vous deux, j'ai énormément appris et j'espère qu'on aura l'occasion de faire durer le plaisir encore un peu.\\


 Je remercie Florence Forbes et Otso Ovaskainen d'avoir rapporté cette thèse. Je remercie également Camille Coron, Chi Tran et Stéphane Dray d'avoir accepté de faire partie du jury.  Merci  aussi à Matthieu Authier et Mahendra Mariadassou pour des discussions intéressantes et enrichissantes.\\
 
 Je n'aurais jamais cru faire une thèse en mathématiques un jour. Enfin si, quand François Coquet m'a amené doucement à l'idée. Merci François de m'avoir poussée à ne pas me contenter de ce que j'avais sous les yeux. J'ai vraiment pris ma décision lors de mon premier contact avec la recherche. Merci à Andrea Rau d'avoir fait de cette expérience une révélation.\\

 
 On m'avait prévenu que j'arrivais dans un labo un peu spécial, un labo pas comme les autres.  L'Agro c'est comme une grande bourrasque d'été. Ça réchauffe, ça fait rire, ça décoiffe. Un lieu chaleureux et dynamique, peuplé de gens bourrés de talents et d'humour. Je remercie du fond du coeur toute cette grande famille, en commençant par les parents : Liliane, Éric, Céline, Julien, Marie-Laure, Sophie, Isabelle, Laure, Julie, Sarah, Pierre B,  Tristan, Maud, Jade, Joon, Erica, Christelle, Christophe D, Marie-Pierre. Un énorme merci aux  générations précédentes \sout{d'enfants} de doctorants que j'ai eu la chance de croiser : Anna, Marie C, Paul, Yann, Mathieu, Thimothée et Rana. \\
 
Annarosa, Martina, merci les filles !! Merci d'avoir rendu ce dernier bureau drôle et studieux juste ce qu'il faut, je n'aurais voulu partager les (quelques) craquages de dernière année avec personne d'autre. Ne changez rien vous êtes parfaites. Merci Claire pour tous tes passages (trop courts), pour les orangettes, et ces magnifiques (c'est le mot !) duos entre flûtistes. Je te souhaite le meilleur pour tes prochaines aventures. Antoine, merci de diffuser la bonne parole de R tidy, grâce à toi nous sommes de plus en plus nombreux dans la paroisse. Merci Félix pour ces moments de calme autour d'un bon thé, et ces week-ends en Bourgogne dont on se souviendra longtemps (qui a dit qu'une chaudière était nécessaire ? Une cheminée suffit !). Merci Saint-Clair de nous rappeler qu'on peut avoir vécu mille vies avant de se lancer dans une thèse ; amuse-toi bien dans la suite de cette aventure. Je remercie aussi les nouveaux que je n'ai pas eu le temps de mieux connaître : Wencan, Gaspard, Tâm et Marina. À vous la suite ! \\
Gabriel (j'ai failli te mettre dans les doctorants, je sais que ça te fait plaisir), merci pour toutes ces histoires, pour tes imitations d'hamsters et de poires astringentes, pour ces moments d'éclats sur isc, et pour ton expertise dans tous les domaines de la vie qu'en tant que petit doctorant on ne comprends pas toujours très bien. Pierre G merci pour tes éclats de voix (notamment de Julien Clerc), pour ta \sout{patience} tolérance à puyo puyo et pour nous rappeler qu'il est urgent de rire.  


Marie est une personne extraordinaire qui a le pouvoir d'augmenter le niveau de bonheur autour d'elle. Elle a l'effet secondaire d'induire une accoutumance assez sévère. Ayant la chance d'avoir partagé son bureau pendant 2 ans, puis d'avoir récidivé aux cours de théâtre, aux soirées et aux week-ends en Normandie, je suis complètement addict, je l'avoue sans rougir. Mais ce traitement de fond fait de moi un être humain plus heureux.  Alors merci Marie, je serais toujours là pour toi. 


Un grand merci à Charles (et les chichis), Luc, Nico et Joachim pour vos folies respectives. Marine, après toutes ces années tu continues à me suivre. Merci pour toutes ces aventures, hâte de la prochaine ! Merci aussi aux copains sans lesquelles je ne serais pas là aujourd'hui :  Jihane, Syrielle, Florian, Rocket, Clément et Laure.


Je remercie enfin ma famille sur laquelle je sais pouvoir compter. Merci beaucoup à mes frères et soeurs pour vos encouragements, Alain, Hélène, Ludo, les belles-soeurs de choc au top Sylvie et Amélie. Je remercie également toute ma belle-famille pour son grand coeur.  Je remercie infiniment mes parents d'être des esprits libres, indépendants et originaux, constantes sources d'inspiration pour moi. Merci maman pour ta spontanétité, ta joie et ton air marin. Merci papa pour ta sérénité, ta poésie et tes couleurs.


Mes derniers mots vont à Sylvain. Merci d'avoir eu la brillante idée de participer à un certain séminaire, là-haut sur une certaine montagne. Merci de m'honorer chaque jour de ta présence, de ton soutien infaillible et de ton petit air moqueur sans lequel le monde n'aurait plus aucun sens.
% \nomtcrule
%\clearemptydoublepage
\tableofcontents 
 \mainmatter



\chapter{Introduction}

\section*{Context}
Networks are objects representing relationships between entities. They are useful to comprehend systems joint organization and behaviors, leading to discoveries that would not have been possible by analyzing the entities separately. Applications are numerous in life sciences, among which genes regulatory networks (genomics), community assembly mechanisms (community ecology), or pathobiome organization (microbiology). Networks can be built from observed interactions, as is the case of host-parasite, plant-pollinator or trophic networks in ecology, or protein-protein interaction networks in genomics. However this strategy, which we call network reconstruction, limits the identified interactions to observable ones only, where others would be interesting too (e.g. competition for food or space, shelter sharing, etc.).\\

 
\begin{figure}[H]
\centering
\includegraphics[width=0.7\linewidth]{figs/plancton.png}
\captionsetup{labelformat=empty}
\caption{Integrated plankton community network related to carbon export at 150m \citep{GCB16}.}
\label{PPI}
\end{figure}



This work is interested in species network inference, which is the art of identifying interactions from observed measures on a set of species. Network inference necessarily relies on a mathematical definition of species interactions, allowing to detect a broad range of interactions. Their biological meaning is unknown and would have to be identified by experts later on. 


A first idea of a mathematical species interaction is the correlation between the species abundances. However using correlation results in dense networks proving hard to analyze, as spurious edges appear between two variables correlated to the same third one \footnote{e.g. the number of covid 19 cases detected correlates with both the real number of cases and the number of tests done on the population, which induces a spurious correlation between the two latter where obviously there is no direct effect of one on the other.}.  Instead, using conditional dependence relationships between species provides with  a clear separation between direct and indirect effects, and therefore yields sparse and easy to interpret networks.


Graphical models are the dedicated mathematical framework for the modeling and inference of such networks, for they graphically represent a multivariate random variable conditional dependencies. Gaussian Graphical Models (GGM) in particular present with specific theoretical and algebraic properties which facilitate the inference. GGM  have been widely used on continuous data. \\

However measures on species are often counts, as is usually the case in ecology and in experiments using high-throughput sequencing technologies in genomics and microbiology. A way to go for network inference from count data with a distinction between direct and indirect effects is then to adopt a modeling for the count allowing the use of the GGM framework. To obtain interpretable results, the model should also account for measured experimental offsets and covariates. Furthermore, the observed measures on species are also very likely to be incomplete as it is difficult to know in advance all the factors governing a phenomenon. This causes the species interaction network to present spurious edges between the species which should be linked to the unobserved actor (species or covariate). A partial observation of the data thus provides with a marginal network instead of the complete one, leading to biased further interpretations and analyzes.\\


\section*{Objectives and outline} 
The aim of the present work is to develop a methodology for the network inference from incomplete abundance data. This task was divided into two sub-objectives. 
First, develop a method for network inference from abundance data. To this aim we model counts  using the Poisson log-normal distribution, taking advantage of the estimation procedure developed by \citet{CMR18}. This specific distribution includes a latent layer of Gaussian parameters, within which the inference of the species interaction network is performed. Following \citet{MeilaJaak} and \citet{SR17}, the inference is carried out using tree averaging, allowing for a complete and efficient exploration of the space of spanning tree graphs, and yielding edges probabilities.

Then, this work includes missing actors in the model to account for incomplete data. There we model the missing actors as additional latent variables of the model Gaussian latent layer. The Gaussian graphical model parameters maximum likelihood estimators detailed in \citet{Lau96} are adapted to the specific case of spanning tree structures, and applied within a variational Expectation Maximization algorithm.

  \subsection*{Chapter 1}
The first chapter covers in details the mathematical and technical background used in Chapters 2 and 3. It defines the general framework of graphical models and its link with conditional independence relationships. The particular properties of the Gaussian graphical models are then presented, along with its maximum likelihood estimators. Two network inference methods are detailed: the penalized regularization which estimates the precision matrix in a sparse manner, and tree averaging which efficiently explores the super-exponential space of spanning-tree structures. This chapter then draws a state of the art of the strategies for the modelization of multivariate counts.

   \subsection*{Chapter 2}
   Chapter 2 details the proposed methodology for the inference of species interaction networks from measures of joint abundances. Counts are modeled in a hierarchical manner: a spanning-tree graph $T$ is first drawn, then parameters $\Zbf$ are modeled  conditionally on $T$ as a multivariate Gaussian faithful to $T$, and finally counts $\Ybf$ follow a Poisson log-normal distribution with parameters $\Zbf$. This model thus involves two latent layers of parameters: $\Zbf$ and $T$, and can be described by the following graph:
 
 \begin{center}
	\begin{tikzpicture}	
      \tikzstyle{every edge}=[-,>=stealth',auto,thin,draw]
		\node (A1) at (0*\length, 0*\length) {$T$};
		\node (A2) at (1*\length, 0*\length) {$\Zbf$};
		\node (A3) at (2*\length, 0*\length) {$\Ybf$};
		\draw (A1) edge [->](A2);
        \draw (A2) edge [->] (A3);
	\end{tikzpicture} 
   \end{center}
 
The model inference estimates the tree distribution using an EM algorithm, which had not been done before. The proposed methodology is implemented in the R package EMtree (\url{github.com/Rmomal/EMtree}). It is compared to state-of-the-art approaches and applied to two empirical datasets from ecology and microbiology. This chapter has been published in the journal \textit{Methods in Ecology and Evolution} \citep{MRA20}. The presented appendices are extended with a vignette showing usage examples of the EMtree package.

    \subsection*{Chapter 3}
This chapter presents an extended version of the previous model developed in Chapter 2 to include possible missing actors. The Gaussian latent layer is assumed to involve additional unobserved variables. The Gaussian layer is considered in its normalized form and denoted $\Ubf$.  This model thus involves three latent layers: $T$, $\Ubf_O$ where "O" stands for "observed", and $\Ubf_H$ where "H" stands for "hidden", and can be described by the following graph:
 
 \begin{center}
	\begin{tikzpicture}	
      \tikzstyle{every edge}=[-,>=stealth',auto,thin,draw]
		\node (A1) at (0.625*\length, 2*\length) {$T$};
		\node (A2) at (0*\length, 1*\length) {$\Ubf_O$};
		\node (A3) at (1.25*\length, 1*\length) {$\Ubf_H   $};
		\node (A4) at (0*\length, 0*\length) {$\Ybf$};
		\draw (A1) edge [->](A2);
        \draw (A1) edge [->] (A3);
        \draw (A2) edge  (A3);
        \draw (A2) edge [->](A4);
	\end{tikzpicture} 
	  \end{center}
 
 The model is estimated with a variational EM algorithm, which takes advantage of the average on trees to use the adaptation to the context of spanning trees of the maximum likelihood estimators of GGM parameters detailed in \citet{Lau96}. The developed procedure is implemented in the R package nestor (\url{github.com/Rmomal/nestor}) and illustrated on two  empirical datasets from ecology. \\

This chapter has been submitted for publication in a statistical journal. The submitted supplementary material is enriched with a vignette showing usage examples of nestor, a section presenting different initialization methods and finally a comparison of nestor, EMtree and  PLN-network which is another method building on the PLN distribution but uses a penalized approach for the network inference.

  \subsection*{Chapter 4}
This final chapter introduces some perspectives of this work. After summarizing  the specifics of the developed methodology and discussing unresolved issues, natural extensions of the model are presented. They first include extensions about measures on the inferred network, with a method for estimating the latent layer precision matrix,  and a strategy  to use tree distributions for comparing networks. Then perspectives about the data at hand are discussed, namely how to handle other data types or datasets presenting with spatial dependencies. Finally a model for the network inference not resorting to a latent layer is presented. %using only discrete distributions is presented.
\newpage
 \section*{Notations}
 
 \begin{description}
 \item[Operations:]  \begin{itemize}
     \item[]
 \item[] $|\cdot|$ : matrix determinant
 \item[] $\odot$ : Hadamard product
 \end{itemize}
 \item[Matrices:] \begin{itemize}
     \item[]
 \item[] $\Ybf$ : observed counts
 \item[] $\Zbf$ :  latent Gaussian parameters 
 \item[] $\Ubf$ :  latent normalized Gaussian parameters 
 \item[] $\Xbf$ : covariates 
 \item[] $O$ :  measured offsets
 \end{itemize}
 \item[Dimensions:]\begin{itemize}
     \item[]
 \item[] $n$ : samples
 \item[] $p$ : observed species
 \item[] $r$ : unobserved actors
 \item[] $d$ : covariates
 \end{itemize}
 \end{description} 
\clearemptydoublepage
\ActivateBG 
\chapter{Mathematical Framework}
 %network inference in ecology
\section{Networks in ecology}
\section{Mathematical problem}
\section{Technical elements}
 
\chapter{Network Inference from Abundance Data}

\vspace{1cm}
This Chapter is based on the article \textit{Tree-based Inference of Species Interaction Networks from Abundance data} published in Methods in Ecology and Evolution \citep{MRA20}. It details an original network inference method from species observed abundance data using the Poisson log-normal model and tree averaging. The developed methodology is compared to existing alternative methods for network inference, from ecology and genomics. After an enriching discussion with Nicolas Clark (MRFcov R package), covariates adjustment has been corrected, as well as simulation parameters of the Scale-free structures. Consequently Fig.~\ref{TPFN} has been updated and Fig.~\ref{SF50}  added. The appendix has been updated accordingly. This chapter completes the published material, and adds a vignette giving usage examples of the R package implementing the developed method.


%nouveaux suppléments vignette
\section{Introduction}
There is a growing awareness of biotic interactions being crucial components of biodiversity and relevant descriptors of ecosystems \citep{valiente,jordanoS}. 
Such interactions can be conveniently represented by networks, which have been increasingly studied and used in recent years for describing and understanding living systems in ecology \citep{poisot}, microbiology \citep{faust} or genomics \citep{evans}. 
Observing species interactions is a laborious task which restricts them to certain categories (e.g.  pollination\validSR{, predation, parasitism}), while many other \validSR{types of} interactions may be hard to observe and key in the system organization (e.g. communication, shelter sharing). Many efforts have been devoted in the last decade to get a more complete picture of the biotic interactions existing between species living in the same niche: \validSR{all these interactions can be gathered in a so-called \textit{species interaction network}}.

\paragraph{Network reconstruction.} 
A first attempt consists in using observed interactions  to predict other possible links based on species traits matching \citep[see e.g.][]{OlF15,BGT16,WeG17,GrW18}. The  interaction strength can also be predicted \citep{WeO13}. This  can be viewed as a prediction task, and modern approaches arising from signal processing and machine learning have been also proposed \citep{DPL17,SPW17,DPD17}. We name these approaches {\sl network reconstruction} to distinguish them from {\sl network inference}, which is the problem we consider in this article.

\paragraph{Network inference.} 
Network inference approaches also aim at retrieving the interactions among species, but do not rely on observed interactions and therefore, remain agnostic as for their type. Such approaches have been developed in many domains ranging from cell biology \citep[][to infer gene regulatory networks]{Fri04} to neurosciences \citep[][to deciphere brain connectivity structures]{ZhC18}. In ecology, it will typically aim at inferring the set of biotic interactions linking species from the same guild.  As  summarized in Fig.~\ref{fig:networkinference}, network inference takes as input measures on species \validSR{(here abundances)} at similar sites, and returns a network of species direct interactions. The importance of distinguishing between direct interaction and indirect association between species is explained in \citet{PWT19}. 

\validSR{Species not engaged in biotic interactions can appear  linked if they respond similarly or oppositely to an abiotic effect (spurious interaction). Therefore network inference must account for environmental covariates.} Fig.~\ref{fig:graphmodel} illustrates this phenomenon: \validSR{ in ($c$) } species (1 and 4) are not in direct interaction, but are affected by the variations of the same environmental covariate $x$. ($d$) displays the network when $x$ is not accounted for: a spurious edge appears between species.


\begin{figure}[H]
    \centering
    \begin{tabular}{ccccc}
        {\tt \begin{tabular}{rr}
date & site \\
%\hline
apr93 & km03 \\
apr93 & km03 \\
apr93 & km03 \\
apr93 & km03 \\
apr93 & km17 \\
apr93 & km17 \\
\vdots & \vdots
\end{tabular} } & \qquad &
        {\tt \begin{tabular}{rrrrr}
EFI & ELA & GDE & GME  \\
%\hline
 71 &   1 &   5 &   6    \\
118 &   2 &   3 &   0     \\
 69 &   0 &   6 &   2    \\
 56 &   0 &   0 &   0   \\
  0 &   1 &   1 &   0  \\
  0 &   0 &   2 &   0 \\
\vdots & \vdots & \vdots & \vdots  
\end{tabular}
%
%\begin{tabular}{rrrrrrr}
%\dots & EFI & ELA & GDE & GME & HFA & \dots \\
%\hline
%&  71 &   1 &   5 &   6 &   0 & \\
%& 118 &   2 &   3 &   0 &   0 & \\
%&  69 &   0 &   6 &   2 &   0 & \\
%&  56 &   0 &   0 &   0 &   0 & \\
%&   0 &   1 &   1 &   0 &   0 & \\
%&   0 &   0 &   2 &   0 &   0 & \\
%& \vdots & \vdots & \vdots & \vdots & \vdots
%\end{tabular} } & \qquad &
        \begin{tabular}{c}
        \includegraphics[width=.3\linewidth]{figs/barans2plot.png}
        \end{tabular} \\
        ($a$) covariates & & ($b$) species abundances & & ($c$) inferred network \\
    \end{tabular}
    \caption{Aim of species interaction network inference from abundance data. Data sample from the Fatala river dataset (see Section \ref{sec:datasets}). }
    \label{fig:networkinference}
\end{figure}

\paragraph{Joint species distribution models.} \validSR{ The rationale behind network inference} is that interactions between species must affect their joint distribution in a series of similar sites. 
Such approaches necessarily rely on a {\sl joint} species distribution model (JSDM), as opposed to species distribution models \citep{SDM} where species are traditionally considered as disconnected entities. 
A JSDM is a probabilistic model describing the species simultaneous presence/absence \citep{Har15,OTN17} or joint abundances \citep{PHW18,PWT19}. An important feature of JSDMs is to include environmental covariates to account for abiotic interactions. \\
Recently, latent variable models have received attention in community ecology as they provide a convenient way to model the dependence structure between species  \citep{WBO15}. The JSDM proposed by \cite{PHW18,PWT19} involves a latent layer. So does the Poisson log-Normal model \citep[PLN,][]{AiH89}, which combines generalized linear models to account for covariates and offsets, and a Gaussian latent structure to describe the species interactions. It can be seen as a multivariate mixed model, in which correlated random effects encode the dependency between the species abundances. 

\begin{figure}[H]
    \centering
    \begin{tabular}{ccccccc}
        \input{figs/FigGraphModel-Connected} & \qquad &
          \begin{tikzpicture}
  \node[observed] (1) at (-0.5*\edgeunit,  .5*\edgeunit) {$1$};
  \node[observed] (2) at (-0.5*\edgeunit, -.5*\edgeunit) {$2$};
  \node[observed] (3) at ( 0.5*\edgeunit, -.5*\edgeunit) {$3$};
  \node[observed] (4) at ( 0.5*\edgeunit,  .5*\edgeunit) {$4$};
  \draw[edge] (1) to (2); \draw[edge] (2) to (3);  \draw[edge] (1) to (3); 
  \end{tikzpicture}
 & \qquad &
          \begin{tikzpicture}
  \node[observed] (1) at (-0.5*\edgeunit,  .5*\edgeunit) {$1$};
  \node[observed] (2) at (-0.5*\edgeunit, -.5*\edgeunit) {$2$};
  \node[observed] (3) at ( 0.5*\edgeunit, -.5*\edgeunit) {$3$};
  \node[observed] (4) at ( 0.5*\edgeunit,  .5*\edgeunit) {$4$};
  \node[covariate] (x) at (0.0*\edgeunit,  .0*\edgeunit) {$x$};
  \draw[edge] (2) to (3); \draw[edge] (3) to (4); 
  \draw[edge] (x) to (1); \draw[edge] (x) to (4);
  \end{tikzpicture}
 & \qquad &
          \begin{tikzpicture}
  \node[observed] (1) at (-0.5*\edgeunit,  .5*\edgeunit) {$1$};
  \node[observed] (2) at (-0.5*\edgeunit, -.5*\edgeunit) {$2$};
  \node[observed] (3) at ( 0.5*\edgeunit, -.5*\edgeunit) {$3$};
  \node[observed] (4) at ( 0.5*\edgeunit,  .5*\edgeunit) {$4$};
  \node[covmiss] (x) at (0.0*\edgeunit,  .0*\edgeunit) {$x$};
  \draw[edge] (2) to (3); \draw[edge] (3) to (4); 
  \draw[edgemiss] (x) to (1); \draw[edgemiss] (x) to (4);
  \draw[edge] (1) to (4); 
  \end{tikzpicture}
 \\
        ($a$) connected & & ($b$) disconnected & & ($c$) with covariate & & ($d$) missing covariate 
    \end{tabular}
    \caption{Examples of graphical models. \validSR{($a$) All species are dependent, ($b$) 4 is independent from all others, ($c$)  1 and 4 are independent conditional on  $x$, ($d$) not accounting for  $x$ induces a spurious dependence between  1 and 4.}}
    \label{fig:graphmodel}
\end{figure}

\paragraph{Graphical models: a generic framework for network inference.}
Although they describe the dependence structure between the distributions of all the species from a same niche, JSDM are not sufficient to perform network inference as they do not distinguish  indirect associations from direct interactions \citep{DBD18}. Graphical models \citep{Lau96} provide a probabilistic framework to do so and, in the same time, a formal definition of the network to be inferred. This formalism is therefore especially appealing for the inference of species interaction networks \citep{PWT19}.
In an undirected graphical model \citep[which is the same as a Markov random field:][]{CWL18}, two species are connected if they are {\sl dependent} conditional on all other species, that is if the variations of their respective abundances would still be  correlated if ever both the environmental conditions and the abundances of all other species were kept fixed. Two species are unconnected if they are {\sl independent} conditional on all other species: the observed correlation between them only results from a series of links with other species \citep{morueta2016network} \validSR{or environmental effects} . 
Fig.~\ref{fig:graphmodel} illustrates the concept of conditional dependence/independence with toy graphical models. In ($a$), the network is connected so all species are interdependent: an association exists between any two of them. However, 1 is only directly interacting with 2 which mediates its association with 3 and 4: 1 is independent from them conditional on 2.

In ($b$), the network is disconnected: species 4 is independent from all others. This illustrates that graphical models enjoy all the desirable properties to represent interactions between species in an interpretable manner, so that they can be used as the mathematical counterpart of species interaction networks.

\modif{\paragraph{Network inference: the general problem.}
Network inference methods attempts to retrieve the graphical model underlying the distribution of  \review{abundance} data. In every domains, network inference is impeded by the huge number of possible graphs for a given set of nodes, which increases super-exponentially with the latter (more than $10^{13}$ undirected graphs can be drawn between 10 nodes, and more than $10^{57}$ between 20). The exploration of the graph space is therefore intractable from a combinatorial point of view. To reduce the search space, a common and reasonable assumption is that a relatively small fraction of species pairs are in direct interaction: the network is sparse. In the case of continuous observations, one of the most popular  approaches is the graphical lasso \citep[glasso:][]{GLasso} which takes advantage of the properties of Gaussian graphical models (GGM) to efficiently infer a sparse network.   
Alternatively, tree-based approaches have been proposed: \cite{ChowLiu} first made the too stringent assumption that the network is made of a single spanning tree (that is connecting all nodes without any loop, as in Fig.~\ref{fig:treeaveraging}).
More recent approaches introduced by \cite{,MeilaJaak} and \cite{kirshner} rely on efficient algebraic tools to average over all possible tree-structured graphical models. The inferred network resulting from such an averaging procedure is not restricted to be a tree: species or groups of species can be isolated (e.g. Fig.~\ref{fig:networkinference}), and loops can appear (e.g. Fig.~\ref{fig:treeaveraging}). }

\paragraph{Network inference from species abundance data.}
This work focuses on network inference based on abundance data, and not only their presence/absence \citep[as considered in][]{OHS10,CWL18}.
Network inference from species abundance measures is a notoriously difficult problem \citep{ulrich2010null}, not only because network inference is complex, but also because it has to account for the data specificities. Abundance data may spread over a wide range of values and often result from sampling efforts (sample and/or species-specific), making them difficult to compare. 
Obviously, count data do not directly fit the Gaussian framework but many network inference methods dedicated to abundance data actually rely on a latent Gaussian structure (see Section \ref{altmethods}).
 

\paragraph{Contribution.}
In the present work, we adopt a model-based approach to perform network inference from abundance data. To accommodate the data specificities we use a PLN model, which includes the over-dispersion of the observed counts as well as the sampling effort. Importantly, the PLN model \validSR{allows us} to account for abiotic effects and avoid the detection of spurious interactions between species.  \\
As for the network inference, we adopt a tree-based approach \citep[as opposed to][which also use a PLN model but resort to glasso]{MInt}, which provides a probability for each edge to be actually part of the underlying graphical model.

\paragraph{Outline.}
We introduce the method EMtree, which combines two (variational) \validSR{Expectation-Maximization (EM)} algorithms to estimate the model parameters. Importantly, our approach provides the probability for each possible edge to be part of the interaction network. We evaluate our approach on both synthetic and ecological datasets. An R package implementing EMtree is available on GitHub \url{https://github.com/Rmomal/EMtree}.  

%\input{framework}

\section{Material and methods}
\subsection{Model} \label{sec:model} %%%%%%%%%%%%%%%%%%%%%%%%%%%%%%%%%%%%%%%%%%%%%%%%%%%%%%%%%%%%%%%%%%%%%%%%%%%%%%%%
\section{Model} \label{sec:Model}
%%%%%%%%%%%%%%%%%%%%%%%%%%%%%%%%%%%%%%%%%%%%%%%%%%%%%%%%%%%%%%%%%%%%%%%%%%%%%%%%
%%%%%%%%%%%%%%%%%%%%%%%%%%%%%%%%%%%%%%%%%%%%%%%%%%%%%%%%%%%%%%%%%%%%%%%%%%%%%%%%

%%%%%%%%%%%%%%%%%%%%%%%%%%%%%%%%%%%%%%%%%%%%%%%%%%%%%%%%%%%%%%%%%%%%%%%%%%%%%%%%
\subsection{Poisson log-normal and tree-shaped graphical models}
%%%%%%%%%%%%%%%%%%%%%%%%%%%%%%%%%%%%%%%%%%%%%%%%%%%%%%%%%%%%%%%%%%%%%%%%%%%%%%%%

%%%%%%%%%%%%%%%%%%%%%%%%%%%%%%%%%%%%%%%%%%%%%%%%%%%%%%%%%%%%%%%%%%%%%%%%%%%%%%%%
\subsubsection*{Poisson log-normal model.} 
We start with a reminder on the multivariate Poisson log-normal model, with the example of abundance data. The abundances of $p$ species observed on $n$ sites are gathered in the $n \times p$ matrix $\Ybf$ where $Y_ {ij}$ is the count of species $j$ in site $i$, and the row $i$ of $\Ybf$, denoted $\Ybf_i$, is the abundance vector collected on site $i$. A covariate vector $\xbf_i $ with dimension $d$ is also measured on each site $i$ and all covariates are gathered in the $n \times d$ matrix  $\boldsymbol X$. The PLN model states that a (latent) Gaussian vector $\Ubf_i$ of size $p$ with variance matrix $\Rbf = (\rho_{kl})_{kl}$ is associated with each site:
\begin{equation} \label{eq:PLN-Z}
\{\Ubf_i\}_{1 \leq i \leq n} \text{ iid}, \qquad 
\Ubf_1 \sim \Ncal_p(\zerobf, \Rbf),
\end{equation}
the sites being assumed to be independent. To ensure identifiability, we let the diagonal of $\Rbf$ be made of 1's, so $\Rbf$ is actually a correlation matrix.
All latent vectors $\Ubf_i$ are gathered in the $n \times p$ matrix $\Ubf$. The PLN model further assumes that species abundances in all sites are conditionally independent, and that their respective distribution only depends on the environment and the associated latent variable:
\begin{equation} \label{eq:PLN-Y.Z}
\{Y_{ij}\}_{1 \leq i \leq n, 1 \leq j \leq p} \mid \Ubf \text{ independent}, \quad 
Y_{ij} \mid U_{ij} \sim \Pcal\left(\exp(o_{ij} + \xbf_i^\intercal \thetabf_j + \sigma_j U_{ij})\right),
\end{equation}
where $o_{ij}$ is a known offset term which typically accounts for the sampling effort, and $\sigma_j$ is the latent standard deviation associated with species $j$. The vector $d \times 1$ of regression coefficients $\thetabf_j$ describes the environmental effects on species $j$. An important feature of the PLN model is that the sign of the correlation between the observed counts is the same as this of correlation between the latent variables \citep{AiH89}: $\text{sign}(\text{Cor}(Y_{ij}, Y_{ik})) = \text{sign}(\text{Cor}(U_{ij}, U_{ik}))$. 
% The dependence between the species abundances is entirely controlled by the latent dependency structure encoded in the precision matrix $\Omegabf:=\Rbf^{-1}$.

%%%%%%%%%%%%%%%%%%%%%%%%%%%%%%%%%%%%%%%%%%%%%%%%%%%%%%%%%%%%%%%%%%%%%%%%%%%%%%%%
\subsubsection*{Tree-shaped graphical models.} 
Network inference relies on the assumption that few species are directly dependent on one another, meaning that the underlying graphical model is sparse. In the framework of the PLN model, the graphical model of interest rules the distribution of the latent vectors $\Ubf_i$ and is  encoded in the precision matrix $\Omegabf:=\Rbf^{-1}$. A way to foster sparsity is to impose $\Omegabf$ to be faithful to a spanning tree $T$, that is: $\Ubf_1 \sim \Ncal_p(\zerobf, \Omegabf_T^{-1})$ where the non-zero terms of $\Omegabf_T$ correspond to the edges of the tree $T$ . However this hypothesis is very restrictive  as it allows only $p-1$ links among $p$ species \citep{ChowLiu}. A more flexible approach consists in assuming that the latent vectors are drawn from a mixture of Gaussian distributions, each faithful to a tree $T$ \citep{MixtTrees,MeilaJaak,kirshner,SRS19}:
\begin{equation} \label{eq:mixt-Z}
\Ubf_1 \sim \sum_{T \in \Tcal_p} p(T) \Ncal_p(\zerobf, \Omegabf_T^{-1}),
\end{equation}
where $\Tcal_p$ is the set of spanning trees with $p$ nodes.
We further assume that the tree distribution $\{p(T)\}_{T \in \Tcal_p}$ can be written as a product over the edges:
\begin{equation} \label{eq:prob-T}
p(T) = B^{-1} \prod_{jk \in T} \beta_{jk}, \qquad
\text{with} \quad B = \sum_{T \in \Tcal_p} \prod_{jk \in T} \beta_{jk}.
\end{equation}
The weights $\beta_{jk}$ are gathered in the $p \times p$ symmetric matrix $\betabf$ with diagonal zero. Observe that these weights are defined up to a multiplicative constant, so that only $p(p-1)/2 - 1$ of them may vary independently. This PLN model with latent tree-shaped dependency structure is similar to that considered by \cite{MRA20}.

%%%%%%%%%%%%%%%%%%%%%%%%%%%%%%%%%%%%%%%%%%%%%%%%%%%%%%%%%%%%%%%%%%%%%%%%%%%%%%%%
\subsection{Introducing the missing actor} \label{sec:missActor}
%%%%%%%%%%%%%%%%%%%%%%%%%%%%%%%%%%%%%%%%%%%%%%%%%%%%%%%%%%%%%%%%%%%%%%%%%%%%%%%%

%%%%%%%%%%%%%%%%%%%%%%%%%%%%%%%%%%%%%%%%%%%%%%%%%%%%%%%%%%%%%%%%%%%%%%%%%%%%%%%%
\subsubsection*{PLN model with missing actors.} 
We now introduce the concept of missing actors, which corresponds to variables that are involved in the graphical model but are not associated with observed variables. To involve such actors in the model, we assume that a complete latent vector $\Ubf_i$ with dimension $p+r$ is associated with site $i$, where $r$ is the number of missing actors. This complete vector can be decomposed as $\Ubf_i^\intercal = [\Ubf_{Oi}^\intercal \; \Ubf_{Hi}^\intercal]$ where $\Ubf_{Oi}$ (with dimension $p$) corresponds to observed species and $\Ubf_{Hi}$ (with dimension $r$) corresponds to the missing actors.
The complete $n \times (p+r)$ latent matrix $\Ubf$ can be decomposed in the same way as $\Ubf = [\Ubf_O \; \Ubf_H]$, $\Ubf_O$ and $\Ubf_H$ having dimension $n \times p$ and $n \times r$, respectively. \\ 
The model we consider states that
\begin{enumerate}[label=\roman*]
\item the complete latent vectors $\Ubf_i$ are all iid and distributed according to a mixture similar to \eqref{eq:mixt-Z} and \eqref{eq:prob-T} but with Gaussian distributions (and matrices $\Omegabf_T$ and $\betabf$) of dimension $(p+r)$, and trees drawn from $\Tcal_{p+r}$;
\item  the abundances $Y_{ij}$ of the  $p$ observed species are distributed according to \eqref{eq:PLN-Y.Z}, replacing $\Ubf$ with $\Ubf_O$,
\end{enumerate}

\begin{figure}[H]
 \begin{center}
	\begin{tikzpicture}	
      \tikzstyle{every edge}=[-,>=stealth',auto,thin,draw]
		\node (A1) at (0.625*\length, 2*\length) {$T$};
		\node (A2) at (0*\length, 1*\length) {$\Ubf_O$};
		\node (A3) at (1.25*\length, 1*\length) {$\Ubf_H   $};
		\node (A4) at (0*\length, 0*\length) {$\Ybf$};
		\draw (A1) edge [->] (A2);
        \draw (A1) edge [->] (A3);
        \draw (A2) edge  (A3);
        \draw (A2) edge [->] (A4);
	\end{tikzpicture} 
 \caption{Graphical model linking the count data $\Ybf$, the latent layer of Gaussian parameters $\Ubf=(\Ubf_O,\Ubf_H)$, and the latent tree $T$.}
  \label{fig:MGmodel}
    \end{center}
\end{figure}

In the sequel, we shall refer to the elements of $\Ubf_O$ and $\Ubf_H$ respectively as 'observed' and 'hidden' (or 'missing') latent variables, whereas obviously none of them are actually observed. Figure \ref{fig:MGmodel} displays the graphical model of the quadruplet $(T, \Ubf_O, \Ubf_H, \Ybf)$. The observed data $\Ybf$ still arise from an PLN model, but the graphical model of the observed latent $\Ubf_O$ may not be sparse due to the marginalization over the hidden latent $\Ubf_H$. Our main goal is to infer the dependency structure of the complete latent vectors, that is to estimate the elements of the matrices $\Omegabf_T$ and the edges weights $\betabf$. The latent dependency structure is similar to this considered by \cite{RAR19}, but the inference strategy much differs, because of the additional hidden layer.

%%%%%%%%%%%%%%%%%%%%%%%%%%%%%%%%%%%%%%%%%%%%%%%%%%%%%%%%%%%%%%%%%%%%%%%%%%%%%%%%
\subsubsection*{Identifiability restriction.} 
The proposed model only makes sense because the graphical model of the complete latent vectors $\Ubf_i^\intercal = [\Ubf_{Oi}^\intercal \; \Ubf_{Hi}^\intercal]$ is supposed to be sparse. Missing actors could obviously not be identified from a regular PLN model, without restriction on the precision matrix $\Omegabf$, as only the marginal precision matrix of the $\Ubf_{Oi}$ could be recovered. Still, to ensure identifiability we impose the same restriction as \cite{RAR19} that  missing latent variables are not connected with each other (the block corresponding to $\Ubf_H \times \Ubf_H$ is diagonal in each $\Omegabf_{T}$).
%\CA{However \SR{here}{} we do not need the additional assumption that all precision matrices $\Omegabf_T$ borrow their non-null elements from a same matrix.}{} \SR{}{[{\sl Faut-il mettre cette dernière phrase alors que rien ne le suggère ? Ou alors préciser, 'as opposed to \cite{RAR19}'}]}
%\begin{description}
%\item[(A)] All the precision matrices $\Omegabf_T$ (for $T \in \Tcal_{p+r}$) borrow their elements from a same matrix $\Omegabf$. Namely:
%$$\forall T \in \Tcal_{p+r}, \qquad [\Omegabf_T]_{j,k} = 
%\left\{ \begin{array}{ll}
%    [\Omegabf]_{j, k} & \text{if } (j, k) \in T \\
%    0 & \text{otherwise}.
%\end{array}\right.$$
%and their conditional variance is set to one. Namely, 
%$$\forall p+1 \leq h, \ell \leq p+r, \qquad [\Omegabf]_{h, \ell} = 
%\left\{ \begin{array}{ll}
%    1 & \text{if } h = \ell \\
%    0 & \text{otherwise}.
%\end{array}\right.$$
 
%\end{description}
%Assumption ({\bf A}) obviously avoids to infer $|\Tcal_{p+r}| = (p+r)^{p+r-2}$ independent (sparse) precision matrices. Assumption ({\bf B}) ensures identifiability, especially regarding the scaling of the missing latent variables.

%To summarize, the model defined in Section \ref{sec:missActor} involves $p d$ regressions coefficients (gathered in $\thetabf$), $
%(p+r)(p+r+1)/2 - r(r+1)/2 = 
%p(p+1)/2 + r p$ conditional covariances (gathered in $\Omegabf$), and $(p+r)(p+r-1)/2 - 1$ independent edge weights (gathered in $\betabf$).

%Finally, the original PLN model only concerns covariates $\Ybf$ and $\Ubf_O$. The use of a dependency structure with mixture of trees allows a sparse and efficient inference, and missing actors are accounted for in covariate $\Ubf_H$.
\subsection{Inference with EMtree} \label{sec:inference} We now describe how to infer the model parameters. We gather the edges weights into the $p \times p$ matrix $\betab$ and the vectors of regression coefficients into a $d \times p$ matrix $\thetab$. The $p \times p$ matrix $\Sigmab$ contains the variances and covariances between the coordinates of each latent vector $\Zb_i$. Hence, the set of parameters to be inferred is $(\betab, \Sigmab, \thetab)$.

\paragraph{Likelihood.} 
The model described above is an incomplete data model, as it involves two hidden layers: the random tree $T$ and the latent Gaussian vectors $\Zb_i$. The most classical approach to achieve maximum likelihood inference in this context is to use the Expectation-Maximization algorithm \citep[EM:][]{DLR77}. Rather than the likelihood of the observed data $p(\Yb)$, the EM algorithm deals with the often more tractable likelihood $p(T, \Zb, \Yb)$ of the complete data (which consists of both the observed and the latent variables). It can be decomposed as 
 
\begin{equation} \label{eq:PTZY}
    p_{\betab, \Sigmab, \thetab}(T, \Zb, \Yb) = p_{\betab}(T) \times p_{\Sigmab}(\Zb \; | \; T) \times p_{\thetab}(\Yb \; | \; \Zb),
\end{equation}
 
where the subscripts indicate on which parameter each distribution depends. \\
Observe that the dependency structure between the species is only involved in the first two terms, whereas the third term only depends on the regression coefficients $\thetab$. 
We take advantage of this decomposition to propose a two-stage estimation algorithm. The first stage deals with the observed layer $p_{\thetab}(\Yb \; | \; \Zb)$, the second with the two hidden layers $p_{\betab}(T)$ and  $p_{\Sigmab}(\Zb \; | \; T)$. The network inference itself takes place in the second step.

\paragraph{Inference in the observed layer.} 
The variational EM (VEM) algorithm that provides an estimate of the regression coefficients matrix $\thetab$ is described in Appendix \ref{app:VEM} (along with a reminder on EM and VEM). It also provides the (approximate) conditional means $\Esp(Z_{ij} | \Yb_i)$, variances $\Var(Z_{ij} | \Yb_i)$ and covariances $\Cov(Z_{ij}, Z_{ik} | \Yb_i)$ required for the inference in the hidden layer. As a consequence, this first step provides the estimates $\widehat{\thetab}$ and $\widehat{\Sigmab}$.

\paragraph{Inference in the hidden layer.} The second step is dedicated to the estimation of $\betab$. The EM algorithm actually deals with the conditional expectation of the complete log-likelihood, namely $\Esp\left(\log p_{\betab, \Sigmab, \thetab}(T, \Zb, \Yb) \; | \; \Yb\right)$. 
As shown in Appendix \ref{app:EM}, this  reduces to
 
\begin{equation} \label{expectation}
    \Esp\left(\log p_{\betab, \Sigmab, \thetab}(T, \Zb, \Yb) \; | \; \Yb\right)
    \simeq
    \sum_{1 \leq j < k \leq p} P_\jk \log \left(\beta_\jk \widehat{\psi}_\jk\right) - \log B + \cst
\end{equation}
 
where $\widehat{\psi}_\jk$ is the estimate of $\psi_\jk$ defined in Eq.~\eqref{eq:pZfact}, and the '$\cst$' term depends on $\thetab$ and $\Sigmab$ but not on $\betab$. 
$P_\jk$ is the approximate conditional probability (given the data) for the edge $(j, k)$ to be part of the network:
$P_\jk \simeq \prob\{jk \in T \; | \; Y\}$.
It is also shown in Appendix~\ref{app:EM} that $\widehat{\psi}_\jk = (1-\widehat{\rho}_\jk^2)^{-n/2}$, where the estimated correlation $\widehat{\rho}_\jk$ depends on the conditional mean, variance and covariances of the $Z_{ij}$'s provided by the first step.
 Eq.~\eqref{expectation} is maximized via an EM algorithm iterating the calculation of the $P_\jk$ and the maximization with respect to the $\beta_\jk$:
\begin{description}

\item[Expectation step: Computing the $P_\jk$ with tree averaging.] The conditional probability of an edge is simply the sum of the conditional probabilities of the trees that contain this edge. Hence, computing $P_\jk$ amounts to averaging over all spanning trees.
Fig.~\ref{fig:treeaveraging} illustrates the principle of tree averaging for a toy network with $p=4$ nodes. Here, five arbitrary spanning trees $t_1$ to $t_5$ (among the $p^{p-2} = 16$ spanning trees) are displayed, with their respective conditional probability $p(T \mid Y)$. 
The edge $(1, 3)$ has a high conditional probability $P_{13}$ because it is part of likely trees such as $t_3$ and $t_4$, whereas $P_{23}$ is small because the edge $(2, 3)$ is only part of unlikely trees (e.g. $t_1$, $t_2$). \\
Averaging over all spanning trees at the cost of a determinant calculus (i.e. with complexity $O(p^3)$) is possible using the Matrix Tree theorem \citep[][recalled as Theorem~\ref{thm:MTT2} in Appendix~\ref{app:MTT}]{matrixtree}. 
\citet{kirshner} further shows that all the $P_\jk$'s can be computed at once with the same complexity $O(p^3)$, although the calculation may lead to numerical instabilities for large $n$ and $p$.



\begin{figure}%[H]
   \begin{center}
    \begin{tabular}{cccccc}
        \input{figs/FigTreeAveraging-p4-tree1-seed2} &
        \input{figs/FigTreeAveraging-p4-tree2-seed2} &
        \input{figs/FigTreeAveraging-p4-tree3-seed2} &
        \input{figs/FigTreeAveraging-p4-tree4-seed2} &
        \input{figs/FigTreeAveraging-p4-tree5-seed2} \\
        $t_1: 2.1\%$ & 
        $t_2: 3.5\%$ & 
        $t_3: 34.1\%$ & 
        $t_4: 15.6\%$ & 
        $t_5:  <.1\%$ \\ \\
        & 
        \input{figs/FigTreeAveraging-p4-avgtree-seed2} &
        \qquad \qquad &
        \input{figs/FigTreeAveraging-p4-graph-seed2} \\
        \multicolumn{3}{c}{Edge conditional probabilities} & Estimated graph \\
    \end{tabular}
    \caption{Tree averaging principle. 
    \textit{Top:} 5 spanning trees with 4 nodes  $(t_1, \dots t_5)$, with their respective conditional probability given the data $P(T = t \mid Y)$.
    \textit{Bottom left:} Weighted graph resulting from tree averaging. Each edge  has width proportional to its conditional probability. \textit{ Bottom right:} Estimated graph (obtained by thresholding edge probabilities) is not a tree.}
    \label{fig:treeaveraging}
   \end{center}
\end{figure}

\item[Maximization step: Estimating the $\beta_\jk$.] 
This step is not straightforward, as the normalizing constant $B = \sum_T \prod_{jk \in T} \beta_\jk$ involves all $\beta_\jk$'s. We propose an exact maximization built upon the Matrix Tree theorem (see Appendix~\ref{app:EM}). 
\end{description}


\paragraph{Algorithm output: edge scoring and network inference} 
%As a side product, 
EMtree provides the (approximate) conditional probability $P_\jk$ for each edge $(j, k)$ to be part of the network. 
Whenever an actual inferred network $\widehat{G}$ is needed (e.g. for a graphical purpose), it can be obtained by thresholding the $P_\jk$ (see Fig.~\ref{fig:treeaveraging}, bottom right). Because we are dealing with trees, a natural threshold is the density of a spanning tree, which is  $2/p$.
More robust results can be obtained using a resampling procedure similar to the stability selection proposed by \citet{LRW10}. It simply consists in sampling a series of subsamples $s = 1 \dots S$, to get an estimate $\widehat{G}^s$ from each of them and to collect the selection frequency for each edge. Again, these edge selection frequencies can be thresholded if needed.
\subsection{Simulation and illustrations} Because network inference is an unsupervised problem (as opposed to network reconstruction), we compare the accuracy of the methods described above on synthetic abundance datasets, for which the true underlying network is known.

\subsubsection{Alternative inference methods} \label{altmethods}
 
We consider network inference methods dedicated to both 
metagenomics (SPIEC-EASI, gCoda and MInt) and ecology (MRFcov, ecoCopula). All methods can handle count data and rely on some (implicit) Gaussian setting. SPIEC-EASI \citep{kurtz}, gCoda \citep{gcoda} and MRFcov \citep{CWL18} resort to data transformation to fit a Gaussian framework. MInt \citep{MInt} considers a Poisson mixed model similar to the one  we consider and ecoCopula \citep{PWT19} defines a multivariate count distribution, the dependency structure of which is encoded in a Gaussian copula. These methods all rely on a Gaussian graphical model (GGM) or a Gaussian copula, so that the network inference problem amounts to estimating a sparse version of the inverse covariance matrix (also named {\sl precision} matrix). 
 

\paragraph{Edge scoring.}
 These methods build upon glasso penalization \citep{FHT08}. For each edge, there exists a minimal penalty value above which it is eliminated from the network. The higher this minimal penalty, the more reliable the edge in the network, so it can be used as a score reflecting the importance of an edge. Only SpiecEasi and gCoda provide unthresholded quantities (namely the glasso regularization path) that can be used for edge scoring; the other methods only return their optimal graph.
 \modif{\paragraph{Covariates.} Only MInt, MRFcov and ecoCopula may include covariates. In order to draw a fair comparison, we give SPIEC-EASI and gCoda access to the covariate information by feeding them with residuals of the linear regression of the transformed data onto the covariates.}

\subsubsection{Comparison criteria}
 
\paragraph{False Discovery Rate (FDR) and density ratio criteria.}
Inferred networks are mostly useful to detect potential interactions between species, which then need to be studied by experts to determine their exact nature. Falsely including an edge lead to meaningless interpretation or useless validation experiments. 

A network with a few reliable edges will be preferred to one having more edges with a larger risk of possible false discoveries. Therefore we choose the FDR as an evaluation criterion, which should be close to 0. Comparing FDR's only makes sense for networks with similar densities. We then compute the ratio between the densities of the inferred and the true network ({\sl density ratio}).

\paragraph{Area Under the Curve (AUC) criterion.}
The AUC criterion allows to evaluate the inferences quality without resorting to any threshold. It evaluates the probability for a method to score the presence of a present edge higher than that of an absent one; it should be close to 1. Note that this criterion cannot be computed for MRFcov, ecoCopula and MInt as they provide a unique 
\modif{network}. 




%%%%%%%%%%%%%%%%%%%%%%%%%%%%%%%%%%%%%%%%%%%%%%%%%%%%%%%%%%
%%%%%%%%%%%%%%%%%%%%%%%%%%%%%%%%%%%%%%%%%%%%%%%%%%%%%%%%%%

\subsubsection{Simulation design}
 
\paragraph{Simulated graphs.}
We consider three typical graph structures: Scale-free, Erdös (short for Erdös-Reyni) and Cluster.
Scale-free structure bears the closest similarity to the tree one, with almost the same density and no loops; it is popular in social networks and in genomics as it corresponds to a preferential-attachment behavior. 
It is simulated following the Barb\'{a}si-Albert model as implemented in the \textit{huge} R package \citep{huge}. The degree distribution of Scale-free structure follows a power law, which constrains the edges probabilities such that the network density cannot be controlled.
Erdös structure is the most even as the edges all have the same existence probability. It is a step away from the tree as it may contain loops and its density can be increased arbitrarily.
Cluster structure spreads edges into highly connected clusters, with few connections between the clusters; the \textit{ratio} parameter controls the intra/inter connection probability ratio.




\paragraph{Simulated counts.}
The datasets are simulated under the Poisson mixed model described in Eq.~\eqref{eq:pY.Z}. We first build the covariance matrix $\Sigma_G$ associated with a graph $G$ following \citet{huge} and randomly choosing the sign of the link, so that in our simulations we consider both positive and negative interactions. For each site $i$, we simulate $\Zb_i \sim \Ncal(0,\Sigma_G)$, then use these parameters together with a set of covariates to generate count data $\Yb$. We use three covariates (one continuous, one ordinal and one categorical), with their regression coefficients $\theta$ drawn from a standard uniform distribution to create heterogeneity in environmental response across species.

\paragraph{Experiments.}
For each set of parameters and type of structure we generate 100 graphs, simulate a dataset under a heterogeneous environment and infer the dependency structure using EMtree, gCoda, SpiecEasi MInt, ecoCopula and MRFcov (the three latter only for Exp. 1). 
The settings of all methods are set to default, except for ecoCopula for which we use the "AIC" selection criterion ("BIC" gives too many empty results).
All computation times are obtained with a 2.5 GH Intel Core 17 processor and 8G of RAM.

\begin{description}
\item[Exp. 1: effect of the data dimensions on the inferred network.] We compare performances in terms of FDR and density ratios on two scenarios: \textit{easy} ($n=100$, $p=20$), and \textit{hard} ($n=50$, $p=30$). The network density for Erdös and Cluster structures is set to $\log(p)/p$.
\item[Exp. 2: effect of the network structure on edge rankings.] AUC measures are collected for alternate variations of $n$ and $p$ to get a general idea of each performance. For comparison's sake, the same density is fixed for all structures in this case,  so that only $n$ and $p$ vary in turn; the scale-free structure imposes a common density of $2/p$. The default values are $n=100$, $p=20$. 
\item[Exp. 3:  effect of the graph density on edge rankings.] AUC measures are collected for variations of $n$ and $p$ with a density of $5/p$ (5 neighbors per node on average), and for variations of density parameters. The default values are $n=100$, $p=20$.
\end{description}


 \subsubsection{Illustrations} \label{sec:datasets}
 
The first application deals with fish population measurements in the estuary of the Fatala River, Guinea, \citep[][available in the R package \textit{ade4}]{baran1995dynamique}. The data consists of 95 count samples of 33 fish species, and two covariates {\it date} and {\it site}. 
We infer the network using four models including no covariates, either one  or both covariates (i.e. respectively the \textit{null}, \textit{site}, \textit{date} and \textit{site+date} models)

The second example is a metabarcoding experiment designed to study oak powdery mildew \citep{jakuch}, caused by the fungal pathogen \textit{Erysiphe alphitoides} (Ea). To study the pathobiome of oak leaves, measurements were done on three trees with different infection status. The resulting dataset is composed of 116 count samples of 114 fungal and bacterial  operational taxonomic units (OTUs) of oak leaves, including the Ea agent.  The original raw data are available at \url{https://www.ebi.ac.uk/ena/data/view/PRJEB7319}. Several covariates are available, among which the tree status, the orientation of the branch, and three covariates measuring the distances of oak leaves to the ground (D1), to the base of the branch (D2), and to the tree trunk (D3). The experiment used different depths of coverage for bacteria and fungi, which we account for via the offset term. We fitted three Poisson mixed models including either none, the tree status or all of the covariates (i.e. respectively \textit{null}, \textit{tree}, and \textit{tree+D1+D2+D3} models).

To further analyze the inferred networks, we use the betweenness  centrality \citep{centrality}, a centrality measure popular in social network analysis. It measures a node's ability to act as a bridge in the network. High betweenness scores  identify sensitive nodes that can efficiently describe a network structure. We compute these using the R package \textit{igraph}.





 
\section{Results}
\subsection{On simulated data} \subsubsection{Effect of dataset dimensions}
\label{adverse}

Behaviors are compared on an easy setting ($n=100$, $p=20$) and a hard setting ($n=50$, $p=30$). Fig.~\ref{TPFN} displays FDR and density ratio measures for all methods on the different cases.  Detailed values of medians and standard-deviations are given in Tables \ref{medFDR} and \ref{meddens} in appendix. The behaviour of methods remains virtually the same across Erdös and Cluster structures. Scale-free structure appears to entail a greater difficulty for all methods with median FDR above $75\%$, except for EMtree which stays at $30\%$ in the hard setting. These poor performance are due to the inferred networks being too dense compared to the original Scale-free graphs. Another experiment with the Scale-free structure is detailed in Fig.~\ref{SF50}, where the number of species is fixed to $50$, and the number of samples is $100$ or $50$.  Under such setting the behavior of ecoCopula and MRFcov greatly improves with FDR at about $30\%$, where EMtree shows about $40\%$ in median (detailed values available in Table~\ref{perfSF50} in appendix).\\
The greater difficulty affects all methods. Density ratios either increase (MInt, SpiecEasi) or decrease (gCoda, ecoCopula, MRFcov). In the first case, FDRs tend to increase as well (e.g. $40\%$ increase for MInt in Erdös and Cluster structures), where a decrease in density ratio yields more empty results (e.g. in Table \ref{empty} $25\%$ of empty graphs for ecoCopula with  Erdös and Cluster structures, $15\%$ with Scale-free structures for $p$ fixed to $50$). EMtree seems to remain stable as for the density ratio, however it shows an increase in FDR measures of about $20\%$ for all structures.


Considering FDRs and density ratios combined, EMtree appears to be the method with the lower FDR which maintains a density ratio reasonably close to 1. As a consequence, the proposed methodology compares well to existing tools on problems with varying difficulties. EMtree is also comparable on running times. Table~\ref{timesTPFN} shows that for Erdös and Cluster it is the third quicker method in easy cases and the second in hard ones. Table \ref{timeSF} (in appendix) shows that on hard scale-free problems ($p=30$ and $p=50$ with $n=50$) EMtree is the quicker method, and third otherwise.

Interestingly, in easy cases when the network density is well estimated, methods yield FDR of $10\%-30\%$ in median. This reminds that network inference from abundance data is a difficult task, and that perfect inference of the network remains an out-of-reach goal. 
 
%\begin{figure}
%    \centering
%    \includegraphics[width=\linewidth]{figs/panel_TPFN_signed.png}
%    \caption{FDR and density ratio measures for all methods at two different difficulty levels and 100 networks of each type. White squares and black plain lines represent medians and quartiles respectively. \small{\textit{ecoCopula selection method: AIC. Number of subsamples for SpiecEasi and EMtree: $S=20$. SpiecEasi and gCoda: $lambda.min.ratio=0.001$,  $nlambda=100$.}}}
%    \label{TPFN}
%\end{figure}

\begin{figure}
    \centering
    \includegraphics[width=\linewidth]{figs/correct_TPFN.png}
    \caption{FDR and density ratio measures for all methods at two different difficulty levels and 100 networks of each type. White squares and black plain lines represent medians and quartiles respectively. \small{\textit{ecoCopula selection method: AIC. Number of subsamples for SpiecEasi and EMtree: $S=20$. SpiecEasi and gCoda: $lambda.min.ratio=0.001$,  $nlambda=100$.}}}
    \label{TPFN}
\end{figure}
\begin{figure}
    \centering
    \includegraphics[width=\linewidth]{figs/BadSF_TPFN_title.png}
    \caption{FDR and density ratio measures for all methods on scale-free graphs with $p=50$. White squares and black plain lines represent medians and quartiles respectively. \small{\textit{ecoCopula selection method: AIC. Number of subsamples for SpiecEasi and EMtree: $S=20$. SpiecEasi and gCoda: $lambda.min.ratio=0.001$,  $nlambda=100$.}}}
    \label{SF50}
\end{figure}

\begin{table}
\centering
 
\begin{tabular}{l|rrrrrr}
 & \multicolumn{1}{c}{SpiecEasi} & \multicolumn{1}{c}{gCoda} & \multicolumn{1}{c}{ecoCopula} & \multicolumn{1}{c}{MRFcov} & \multicolumn{1}{c}{MInt} & \multicolumn{1}{c}{EMtree} \\ 
  \hline
Easy &  19.99(4.19) & 0.1(0.05) & 4.2(0.24) & 5.76(0.35) & 54(26.9) & 4.44(0.64) \\ 
  Hard &  24.29(5.07) & 0.5(0.24) & 8.19(0.16) & 5.52(2.98) & 33.87(18.37) & 3.29(0.32)  \\ 
   \hline
\end{tabular}
\caption{Median and standard-deviation running-time values (in seconds) for Cluster and Erdös structures, including resampling with $S=20$ for SpiecEasi and EMtree.}
\label{timesTPFN}
\end{table}




%%%%%%%%%%%%%%%%%%%%%%%%%%%%%%%%%%%%
%%%%%%%%%%%%%%%%%%%%%%%%%%%%%%%%%%%%

\subsubsection{Effect of network structure}

As expected for a fixed $p$, the higher the number of observations $n$, the better the performance for all methods and structures. Interestingly, the same happens when $p$ increases for a fixed $n=100$ (except for SpiecEasi).
EMtree performs well on Scale-free structures (Fig.~\ref{SFAUC}) which was also expected; the other methods performance worsen compared to other structures. When lowering $n$ to 30, EMtree performance deteriorates along with $p$, yet remaining above $70\%$ in median in the extreme case where $p=n$ (Fig.~\ref{SFAUC}, right). The structure being Erdös or Cluster, each method is affected in the same way by an increase of $n$ or $p$ (Fig.~\ref{panelErdClust}). Besides, increasing the difference between the two structures by tuning up the \textit{ratio} parameter has no effect. Overall EMtree performs better than gCoda and SpiecEasi on all the studied configurations. Running times are summarized in Table~\ref{timeNP}. EMtree is about 10 times slower than gCoda (4 for small $n$), and 4 times faster than SpiecEasi. The high standard deviation for small $n$ seems to be due to gCoda struggling with Scale-free structures.
  
\begin{figure}[H]
    \centering
    \includegraphics[width=0.7\linewidth]{figs/panel_SF.png}
    \caption{Effect of Scale-free structure on AUC medians and inter-quartile intervals for parameters $n$ and $p$.}
      \label{SFAUC}
\end{figure}


\begin{table}[H]
\centering
\begin{tabular}{l|rr|rr}
 & \multicolumn{1}{c}{$n < 50$} & \multicolumn{1}{c}{$n\geq 50$}  & \multicolumn{1}{c}{$p < 20$} & \multicolumn{1}{c}{$p\geq 20$} \\  
 \hline
  EMtree    &   0.44 (0.14)	 &   0.60 (0.17) &   0.41 (0.13) &   0.76 (0.21)   \\ 
  gCoda     &   0.11 (26.8)	 &   0.05 (0.05) &   0.05 (0.04) &   0.09 (0.54)   \\ 
  SpiecEasi &   2.09 (0.26)	 &   2.37 (0.28) &   2.42 (0.27) &   2.42 (0.26)   \\ 
   \hline
\end{tabular}
\caption{Median and standard-deviation of running times for each method in seconds, for $n$ and $p$ parameters.}
\label{timeNP}
\end{table}

%%%%%%%%%%%%%%%%%%%%%%%%%%%%%%%%%%%%
%%%%%%%%%%%%%%%%%%%%%%%%%%%%%%%%%%%%

\subsubsection{Effect of network density}
The comparison of top and bottom panels of Fig.~\ref{panelErdClust} shows that network inference gets harder as the network gets denser, whatever the method and the structure of the true graph. Running times are not affected (Table \ref{timeDenser}).
Fig.~\ref{varyDens} shows that EMtree performance does not deteriorate faster than that of other methods, demonstrating that the tree hypothesis is not a limitation.


 \begin{figure}[H]
  \centering
   \includegraphics[width=\linewidth]{figs/panel_npFav.png}
  \includegraphics[width=\linewidth]{figs/panel_dense.png}
  \caption{Effect of Erdös and Cluster structures on AUC medians and inter-quartile intervals for parameters $n$, $p$ and $ratio$. \textit{Top}: densities set to $2/p$, \textit{bottom}: densities set to $5/p$.}
  \label{panelErdClust}
\end{figure}

\begin{figure}[H]
 \centering
  \includegraphics[width=0.6\linewidth]{figs/panel_dens_seuils.png}
  \caption{AUC median and inter-quartile intervals for parameters controlling the number of edges in Erdös (\textit{edge probability}) and in Cluster (\textit{density}) structures, $p=20$, $n=100$. The two vertical dotted lines are the $3/p$ and $5/p$ values.}
  \label{varyDens}
\end{figure}
 
 \label{sec:simul}
\subsection{Illustrations} In this section we emphasize the importance of covariates for network inference. Accounting for environmental effects changes the structure of all inferred networks we present; nodes with the highest betweenness scores are highlighted to spot these changes. Most frequently, it results in reducing the number of edges (i.e. making the network sparser). However new edges can appear as well, as adjusting for a covariate also reduces the variability, which improves the detection power. In all examples, we used the resampling method described in Section \ref{sec:inference}, which provides edge selection frequencies. Eventually, we have to threshold these frequencies to draw actual networks; the value of the threshold obviously affects the density of the plotted networks (see Fig.~\ref{QETOak}). 


\subsubsection{Fish populations in the Fatala River estuary}
\label{barans}

Networks on Fig.~\ref{baransNets} suggest a predominant role of the \textit{site} covariate compared to the \textit{date}. Indeed, adjusting for the \textit{site} results in much sparser networks (Fig.~\ref{QETOak} in appendix). It deeply modifies the network structure: the \textit{site} network has 12 new edges and only 6 in common with the \textit{null} network. Besides, the  highlighted nodes only change when introducing the \textit{site} covariate.  This suggests that the environmental heterogeneity between the sites has a major effect on the \modif{variations of species abundances}, while the effect of the date of sampling is moderate.



  \begin{figure}[H]
      \includegraphics[width=\linewidth]{figs/BaransNets.png}
      \caption{Interaction networks of Fatala River fishes inferred when adjusting for none, both or either one of the covariates among \textit{site} and \textit{date}. Highlighted nodes spot the highest betweenness centrality scores. Widths are proportional to selection frequencies. $S=100$, $f'=90\%$. }
    \label{baransNets}
  \end{figure}
  

%%%%%%%%%%%%%%%%%%%%%%%%%%%%%%%%%
\subsubsection{Oak powdery mildew}  
\label{oak}

When providing the inference with more information (tree status, distances), the structure of the resulting network is significantly modified. Nodes with high betweenness scores differ from one model to another. There is an important gap in density between the \textit{null} model and the others, starting from a $25\%$ selection threshold (Fig.~\ref{QETOak} in appendix). From a more biological point of view, the features of the pathogen node are greatly modified too: its betweenness score is among the smallest in the \textit{null} network (quantile $16\%$), and among the highest in the two other networks (quantiles $93\%$ and $96\%$). Its connections to the other nodes vary as well. 
Accounting for covariates results in less interactions with the pathogen but a greater role of the latter in the pathobiome organization.


 \begin{figure}[H]
    \centering
    \includegraphics[width=\linewidth]{figs/OakProbNets.png}
    \caption{Pathogen interaction networks on oak leaves inferred with EMtree when adjusting for none, the \textit{tree} covariate or \textit{tree} and distances. Bigger nodes represent OTUs with highest betweenness values, colors differentiate fungal and bacterial OTUs. Widths are proportional to selection frequencies. $S=100$, $f'=90\%$ .  }
    \label{oakNets}
\end{figure}

Using the dataset restricted to infected samples (39 observations for 114 OTUs) and correcting for the leaves position in the tree (proxy for their abiotic environment), \citet{jakuch} identifies a list of 26 OTUs likely to be directly interacting with the pathogen. Running EMtree on the same restricted dataset with the same correction yields a good concordance with  edge selection frequencies, as shown in Fig.~\ref{otujak}.


\begin{figure}[H]
    \centering
    \includegraphics[width=9.5cm]{figs/EAneighbors.png}
    \caption{EMtree selection frequencies of pathogen neighbors compared to \citet{jakuch} results, computed on infected samples and adjusting for the leaf position (100 subs-samples). }
    \label{otujak}
\end{figure}

 
 \label{sec:illustration}

\section{Discussion} The inference of species interaction network is a challenging task, for which a series of methods have been proposed in the past years. Abundance data seems to be a promising source of information for this purpose. Here we adopt the formalism of graphical models to define a probabilistic model-based framework for the inference of such networks from abundance data.
Using a model-based approach offers several important advantages. First, it enables easy and explicit integration of environmental and experimental effects.  These could be modeled in a more flexible way using generalized additive models, which include non-linear effects \citep{hastie2017generalized}. 
Then, as it also relies on a formal statistical definition of a \textsl{species interaction network} in the context of graphical models, accounting for abiotic effects and modeling species interactions are two clearly defined and distinguished goals. Finally, all the underlying assumptions are explicitly stated in the model definition itself, and can therefore be discussed and criticized. \\



We developed an efficient method to infer sparse networks, which combines a multivariate Poisson mixed model for the joint distribution of abundances, with an averaging over all spanning trees to efficiently infer direct species interactions. As we do consider a mixture over all spanning trees, our approach remains flexible and can infer most types of statistical dependencies. An EM algorithm (EMtree) maximizes the likelihood of the result and returns each edge probability to be part of the network. An optional resampling step increases network robustness.

\modif{A simulation study in a heterogeneous environment  demonstrates  that EMtree  compares very well to alternative approaches. The proposed model can take all kind of covariates into account, which when ignored  can have  dramatic effects  on the inferred network structure, as showed here on empirical datasets.  Experiments on simulated data and illustrations also demonstrate that EMtree  computational cost remains very reasonable.}

\modif{Alternative methods used in this work all rely on an optimized threshold to tell an edge presence. This particular threshold is obtained after testing a grid of possible values which all yield a different network, and altogether build a path. Making this path available to the user is useful, as the final threshold might need modification and it gives the possibility to build edges scores  and get more than a binary result. We found few recent approaches doing this, which prevented us to study their performance in a way that did not impose a threshold.}\\

The proposed methodology could be extended in several ways.
\modif{Species abundances and interactions indeed vary across space, and depend on local conditions \citep{PCM12,PSG15}. This can either be considered as nuisance parameter or as feature of interest. In the first case, the method could be extended to account for the spatial autocorrelation of sampling sites, to obtain a "regional" interaction network corrected for this effect, i.e. assuming the network is the same in all sites. If of interest, variation across space and local conditions could be studied by comparing networks inferred from the different sampling locations. Networks comparison is a wide and interesting question and tools lack to check which edges are shared by a set of networks. The approach introduced by \citet{SR17} could be adapted to  EMtree framework.} Lastly,  It is also very likely that not all covariates nor even all species have been measured or observed. Another extension may therefore be to detect ignored covariates or missing species. To this purpose EMtree could probably be combined with the approach developed by \citet{RAR19} to identify missing actors. 
\begin{subappendices}
\section{Completed supplements}
\subsection{Variational EM in the observed layer} \label{app:VEM} \modif{\paragraph{A reminder on EM and VEM.} Expectation-Maximisation \citep[EM:][]{DLR77} has become the standard algorithm for the maximum likelihood inference of latent variable models. Denoting $\gammab$ the unknown parameter, $\Yb$ the observed variables and $\Hb$ the latent variables, the aim of EM is to maximise the {\sl observed} (log-)likelihood $\log p_\gammab(\Yb)$. 
In the model defined in Section \ref{sec:model}, the set of parameter to estimate is $\gammab = (\betab, \Sigmab, \thetab)$ and the latent variables are $\Hb = (\Zb, T)$.
Because the {\sl complete} (log-)likelihood $\log p_\gammab(\Yb, \Hb)$ is often much easier to handle, EM alternatively evaluates the conditional distribution of the latent variables $p_\gammab(\Hb \mid \Yb)$ (E step) and updates the parameter estimates by maximizing the conditional expectation of the complete log-likelihood (M step).}

\modif{Unfortunately, for many models, the conditional distribution $p_\gammab(\Hb |\Yb)$ is intractable. The variational EM (VEM) algorithm has been designed to deal with such cases. Briefly speaking, the E step (during which the intractable conditional distribution should be evaluated) is replaced with a VE step, during which an approximate distribution $\pt(\Hb) \simeq p_\gammab(\Hb \mid \Yb)$ is determined. Actually, the VEM algorithm maximizes a lower bound of the genuine log-likelihood, similar to this given in Eq.~\eqref{eq:LowerBound} \citep[see][for an introduction]{OrW10,BKM17}.}

\paragraph{Application to the Poisson log-normal model.} 
To estimate the fixed regression parameters gathered in $\thetab$, we resort to a surrogate model where the entries of the abundance matrix $\Yb$ still have the conditional distribution given in Eq.~\eqref{eq:pY.Z}, but where the distribution of the $\Zb_i$ is not constrained to be faithful to a specific graphical model. Namely, the latent vectors $\Zb_i$ are only supposed to be independent and identically distributed (iid) Gaussian with distribution $\Ncal(0, \Sigmab)$, without any restriction on $\Sigmab$. 

This surrogate model is actually a Poisson log-normal model as introduced by \cite{AiH89}, the parameters of which can be estimated using a variational approximation similar to this introduced in \cite{CMR18}. 
%Namely, instead of maximizing the log-likelihood $\log p(\Yb)$ with respect to the parameters $\thetab$ and $\Sigmab$ using a regular EM algorithm, we maximize the lower bound of it
\modif{More specifically, we maximize with respect to the parameters $\thetab$ and $\Sigmab$ the following lower bound of the log-likelihood $\log p(\Yb)$:}
\begin{linenomath} 
\begin{equation}\label{eq:LowerBound}
\mathcal{J}(\Yb; \thetab, \Sigmab, \pt) := \log p_{\thetab, \Sigmab}(\Yb) - KL\left(\pt(\Zb) || p_{\thetab, \Sigmab}(\Zb \mid \Yb)\right),
\end{equation}
\end{linenomath}
where $KL(q||p)$ stands for Küllback-Leibler divergence between distributions $q$ and $p$ and where the approximate distribution $\pt(\Zb)$ 
%(which approximates $\pt_{\thetab, \Sigmab}(\Zb \mid \Yb)$) 
is chosen to be Gaussian. This means that each conditional distribution $p(\Zb_i \mid \Yb_i)$ is approximated with a normal distribution $\Ncal(\mbt_i, \Sbt_i)$. As shown in \cite{CMR18}, $\mathcal{J}(\Yb, \thetab, \Sigmab, \pt)$ is bi-concave in $(\thetab, \Sigmab)$ and $\{(\mbt_i, \Sbt_i)_i\}$, so that gradient ascent can be used. The {\tt PLNmodels} R-package --available on CRAN-- provides an efficient implementation of it.

The entries of the $\mbt_i$ and $\Sbt_i$ provide us with approximations of the conditional expectation,  variance and covariance of the $Z_{ij}$ conditionally on the $\Yb$, which we use to get the estimates $\widehat{\sigma}_j^2$ and $\widehat{\rho}_{jk}$ given in Eq.~\eqref{eq:sigma.rho}. More specifically, we use $\Esp(Z_{ij} \mid \Yb_i) \simeq \mt_{ij}$, $\Esp(Z^2_{ij} \mid \Yb_i) \simeq \mt_{ij}^2 + \St_{i,jj}$ and $\Esp(Z_{ij} Z_{ik} \mid \Yb_i) \simeq \mt_{ij} \mt_{ik} + \St_{i, jk}$.
\subsection{EM in the latent layer} \label{app:EM} %%%%%%%%%%%%%%%%%%%%%%%%%%%%%%%%%%%%%%%%%%%%%%%%%%%%%%%%%%%%%%%%%%%%%%%%%%%%%%%%%%%%%%%%%%%%
\subsubsection*{Complete log-likelihood conditional expectation}
%%%%%%%%%%%%%%%%%%%%%%%%%%%%%%%%%%%%%%%%%%%%%%%%%%%%%%%%%%%%%%%%%%%%%%%%%%%%%%%%%%%%%%%%%%%%
Because of the specific form given in Eq.~\eqref{eq:pZfact}, and because the $\Zb_i\mid T$  are Gaussian, we have that
 
\begin{align}
    \label{eq:logpZ.T}
    \log{p_\Sigmab}(\Zb \mid T)
    & = \sum_{j=1}^p \sum_{i=1}^n \log P(Z_{ij} \mid T) 
    + \sum_{(j, k) \in T} \sum_{i=1}^n \log \left(\frac{P(Z_{ij}, Z_{ik})}{P(Z_{ij})P(Z_{ik})}\right) \nonumber \\
    & = - \frac{n}2 \log \sigma_j^2 -\frac12 \sum_{j=1}^p \sum_{i=1}^n \frac{Z_{ij}^2}{\sigma_j^2}
    - \frac{n}2 \sum_{(j, k) \in T} \log (1- \rho_{jk}^2) \\
    & \quad - \frac12 \sum_{(j, k) \in T} \frac1{1- \rho_{jk}^2} \sum_{i=1}^n \left(
    \rho^2_{jk} \frac{Z_{ij}^2}{\sigma_j^2} + \rho^2_{jk} \frac{Z_{ik}^2}{\sigma_k^2} - 2\rho_{jk} \frac{Z_{ij} Z_{ik}}{\sigma_j \sigma_k}
    \right)
    + \text{cst} \nonumber 
\end{align}
 

where the constant term does not depend on any unknown parameter. 
In the EM algorithm, we have to maximize the conditional expectation of Eq.~\eqref{eq:logpZ.T} with respect to the variances $\sigma_j^2$ and the correlation coefficients $\rho_{jk}$. The resulting estimates take the usual forms, but with the conditional moments of the $Z_{ij}$, that is
 
\begin{equation} \label{eq:sigma.rho}
\widehat{\sigma}_j^2  = \frac1n \sum_i \Esp(Z^2_{ij}  \mid  \Yb),
\qquad
\widehat{\rho}_{jk}   = \frac1n \sum_i \Esp(Z_{ij} Z_{ik}  \mid  \Yb) \left/ (\widehat{\sigma}_j \widehat{\sigma}_k) \right. .
\end{equation}
 
which do not depend on T. 
The maximized conditional expectation of Eq.~\eqref{eq:logpZ.T} becomes
 
\begin{equation} \label{eq:ElogpZ.T.Y}
    \Esp\left(\log p_{\widehat{\Sigmab}}(\Zb \mid T) \mid \Yb\right)
    = - \frac{n}2 \log \widehat{\sigma}_j^2 
    - \frac{n}2 \sum_{(j, k) \in T} \log (1- \widehat{\rho}_{jk}^2)
    + \text{cst}.
\end{equation}
 
We are left with the writing of the conditional expectation of the first two terms of the logarithm of Eq.~\eqref{eq:PTZY}, once optimized in $\Sigmab$. Combining Eq.~\eqref{eq:pT} and Eq.~\eqref{eq:ElogpZ.T.Y}, and noticing that the probability for an edge to be part of the graph is the sum of the probability of all the trees than contain this edge, we get (denoting $\log \widehat{\psi}_{jk} = (1- \widehat{\rho}_{jk}^2)^{-n/2}$)

\begin{align*}
    \Esp\left(\log p_\betab (T) + \log p_{\widehat{\Sigmab}}(\Zb \mid T) \mid \Yb\right)
    & = \sum_{T \in \Tcal} p(T \mid \Yb) \left(\log p_\betab(T) + \log p_{\widehat{\Sigmab}}(\Zb \mid T) \right) \\
    & = -\log B + \sum_{T \in \Tcal} p(T \mid \Yb) \sum_{(j, k) \in T} \left(\log \beta_{jk} +\log \widehat{\psi}_{jk} \right) + \text{cst} \\
    & = -\log B + \sum_{(j, k)} \prob\{(j, k) \in T \mid \Yb\} \left(\log \beta_{jk} + \log \widehat{\psi}_{jk} \right) + \text{cst},
\end{align*}
 
which gives Eq.~\eqref{expectation}.\\
 
As explained in the section above, we approximate expectations and probabilities conditional on $\Yb$ by their variational approximation.
%\textcolor{red}{
%Because $T$ is independent of $Y$ given $Z$, we have that
%$$
%\log p(T \mid Y) = \log  \e_{Z \mid Y} \left(p(T, Z)\right), 
%$$
%the variational approximation of which is 
%$
%\log \et \left(p(T, Z)\right)
%\simeq \et \left(\log p(T, Z)\right)
%$,
%so we define
%$$
%\log \pt(T \mid Y)
%:= \et \left(\log p(T, Z)\right)
%= \sum_{j, k \in T} \log (\beta_{jk} \psi_{jk}) + \text{cst}
%$$
%(where the constant term does not depend on $T$), that is
%$
%\pt(T \mid Y) 
%= \left. \prod_{j, k \in T} \beta_{jk} \psi_{jk} \right/ C,
%$
%where $C$ is the normalizing constant:
%$C = \sum_T \prod_{j, k \in T} \beta_{jk} \psi_{jk}$.
%}
This provides us with the approximate conditional distribution of the tree $T$ given the data $\Yb$:
 
$$
\pt(T  \mid  \Yb) = \left. \prod_{jk \in T} \beta_{jk} \widehat{\psi}_{jk}  \right/ C,
$$
 
where $C$ is the normalizing constant: $C = \sum_T \prod_{j, k \in T} \beta_{jk} \widehat{\psi}_{jk}$. The intuition behind this approximation is the following: according to Eq. \eqref{eq:pT}, the marginal probability a tree $T$ is proportional to the product of the weights $\beta_{jk}$ of its edges. The conditional distribution probability of tree is proportional to the same product, the weights $\beta_{jk}$ being updated as $\beta_{jk} \widehat{\psi}_{jk}$, where $\widehat{\psi}_{jk}$ summarizes the information brought by the data about the edge ($j, k$).


%%%%%%%%%%%%%%%%%%%%%%%%%%%%%%%%%%%%%%%%%%%%%%%%%%%%%%%%%%%%%%%%%%%%%%%%%%%%%%%%%%%%%%%%%%%%
\subsubsection*{Steps E and M}
%%%%%%%%%%%%%%%%%%%%%%%%%%%%%%%%%%%%%%%%%%%%%%%%%%%%%%%%%%%%%%%%%%%%%%%%%%%%%%%%%%%%%%%%%%%%
\begin{description}
 \item[E step:]
From the above computation we get the following approximation:
 
 
$$\mathds{P}(\{j,k\}\in T  \mid \Yb) \simeq 1 - \sum_{T:jk\notin T} \pt(T \mid \Yb),$$
 
and so we define $p_{jk}$ as follows:  
$$P_{jk}= 1 - \frac{\sum_{T:jk\notin T} \prod_{j, k \in T} \beta_{jk} \psi_{jk}}{\sum_T \prod_{j, k \in T} \beta_{jk} \psi_{jk}}.$$
 
 $P_{jk}$ can be computed with Theorem \ref{thm:MTT}, letting $[\Wb^h]_\jk = \beta^h_\jk \widehat{\psi}_\jk$ and $\Wb^h_{\setminus \jk} = \Wb^h$ except for the entries $(j, k)$ and $(k, j)$ which are set to 0. The modification of $\Wb^h_{\setminus \jk}$ with respect to $\Wb^h$ amounts to set to zero the weight product, and so the probability, for any tree $T$ containing the edge $(j, k)$. As a consequence, we get
 
$$
 P^{h+1}_\jk = 1 - \left|Q_{uv}^*(\Wb^h_{\setminus \jk})\right| \left/ \left|Q_{uv}^*(\Wb^h)\right| \right..
 $$
 
 \item[M step:] Applying Lemma \ref{lem:Meila} to the weight matrix $\betab$, the derivative of $B$ with respect to $\beta_\jk$ is 
 
$$
 \partial_{\beta_\jk} B = [\Mb(\betab)]_\jk \times B
 $$
 
 then the derivative of \eqref{expectation} with respect to $\beta_\jk$ is null for
 $\beta^{h+1}_\jk = P_\jk^{h+1} \left/ [\Mb(\betab^h)]_\jk \right.$.
 \end{description}

\subsection{Matrix tree theorem}\label{app:MTT} For any matrix $\Wb$, we denote its entry in row $u$ and column $v$ by $[\Wb]_{uv}$. We define the Laplacian matrix $\Qb$ of a symmetric matrix $\Wb=[w_\jk ]_{1\leq j,k\leq p}$ as follows :
\begin{linenomath*}
\[
 [\Qb]_\jk =\begin{cases}
    -w_\jk  & 1\leq j<k \leq p\\
    \sum_{u=1}^p w_{ju} & 1\leq j=k \leq p.
    \end{cases}
\]
\end{linenomath*}
We further denote $\Wb^{uv}$ the matrix $\Wb$ deprived from its $u$th row and $v$th column and we remind that the $(u, v)$-minor of $\Wb$ is the determinant of this deprived matrix, that is $|\Wb^{uv}|$.

\begin{theorem}[Matrix Tree Theorem  \cite{matrixtree,MeilaJaak}] \label{thm:MTT2}
    For any symmetric weight matrix W, the sum over all spanning trees of the product of the weights of their edges is equal to any minor of its Laplacian. That is, for any $1 \leq u, v \leq p$,
   \begin{linenomath*}
   \[
    W := \sum_{T\in\mathcal{T}} \prod_{(j, k)\in T} w_\jk  = |\Qb^{uv}|.
    \]
    \end{linenomath*}
\end{theorem}    

In the following, without loss of generality, we will choose $\Qb^{pp}$. As an extension of this result, \cite{MeilaJaak} provide a close form expression for the derivative of $W$ with respect to each entry of $\Wb$. 

\begin{lemma} [\cite{MeilaJaak}] \label{lem:Meila2}
    Define the entries of the symmetric matrix $\Mb$ as
\begin{linenomath*}
\[    
 [\Mb]_\jk =\begin{cases}
    \left[(\Qb^{pp})^{-1}\right]_{jj} + \left[(\Qb^{pp})^{-1}\right]_{kk} -2\left[(\Qb^{pp})^{-1}\right]_\jk & 1\leq j<k < p\\
    \left[(\Qb^{pp})^{-1}\right]_{jj} & k=p, 1\leq j \leq p  \\
    0 & 1\leq j=k \leq p.
    \end{cases}
\]
\end{linenomath*}
it holds that
\begin{linenomath*}
$$
\partial_{w_\jk} W = [\Mb]_\jk  \times W.
$$
\end{linenomath*}
\end{lemma}

\subsection{Results}\label{app:results} \subsubsection{Simulations: dataset dimensions}
\vspace{1.5cm}
\begin{table}[H]
\centering
\begin{tabular}{l|l|rrrrrr}
\multicolumn{2}{l|}{} & \multicolumn{1}{c}{SpiecEasi} & \multicolumn{1}{c}{gCoda} & \multicolumn{1}{c}{ecoCopula} & \multicolumn{1}{c}{MRFcov} & \multicolumn{1}{c}{MInt} & \multicolumn{1}{c}{EMtree} \\ 
\hline
\multirow{3}{*}{{\rotatebox[origin=c]{90}{Easy}}} 
    & Cluster & 0.86  (0.20) & 0  (0.08) & 0.33  (0.14) & 0.27 (0.13) & 0.38  (0.17) & 0.12  (0.09) \\ 
    & Erdös   & 0.86  (0.21) & 0  (0.15) & 0.29  (0.15) &0.25 (0.14)& 0.38  (0.15) & 0.12  (0.08) \\ 
    & Scale-free & 0.82 (0.15) & 0.84 (0.07) & 0.85 (0.02) & 0.8 (0.03) & 0.85 (0.04) & 0.19 (0.11)  \\  \hline
\multirow{3}{*}{{\rotatebox[origin=c]{90}{Hard}}} & Cluster  & 0.88 (0.12) & 0 (0.2) & 0.15 (0.18) & 0.27 (0.19) & 0.77 (0.09) & 0.39 (0.09) \\ 
 & Erdös.     & 0.88 (0.11) & 0 (0.24) & 0 (0.15) & 0.32 (0.2) & 0.77 (0.1) & 0.39 (0.09) \\ 
 & Scale-free &0.83 (0.1) & 0.88 (0.05) & 0.86 (0.02) & 0.8 (0.03) & 0.85 (0.04) & 0.33 (0.09)  \\ \hline
\end{tabular}
\caption{Medians and standard-deviation of FDR computed on 100 graphs of each type (\textit{easy}: $n=100, p=20$, \textit{hard}: $n=50, p=30$)}
\label{medFDR}
\end{table}

\begin{table}[ht]
\centering
\begin{tabular}{l|l|rrrrrr}
\multicolumn{2}{l|}{} & \multicolumn{1}{c}{SpiecEasi} & \multicolumn{1}{c}{gCoda} & \multicolumn{1}{c}{ecoCopula} & \multicolumn{1}{c}{MRFcov} & \multicolumn{1}{c}{MInt} & \multicolumn{1}{c}{EMtree} \\ \hline
\multirow{3}{*}{{\rotatebox[origin=c]{90}{Easy}}} &Cluster &0.16 (0.11) & 0.05 (0.07) & 1.04 (0.48) & 0.62 (0.25) & 0.3 (0.13) & 0.81 (0.17) \\ 
& Erdös &0.15 (0.09) & 0.06 (0.08) & 0.95 (0.5) & 0.57 (0.26) & 0.3 (0.14) & 0.65 (0.12) \\ 
 & Scale-free & 0.42 (0.17) & 1.97 (0.82) & 6.05 (0.4) & 4.11 (0.34) & 2(0.54) & 1.05 (0.1) \\ \hline
\multirow{3}{*}{{\rotatebox[origin=c]{90}{Hard}}}  & Cluster &  0.21 (0.08) & 0.02 (0.03) & 0.02 (0.17) & 0.15 (0.09) & 0.68 (0.3) & 0.49 (0.11)  \\ 
 & Erdös & 0.21 (0.08) & 0.02 (0.02) & 0 (0.18) & 0.15 (0.08) & 0.66 (0.25) & 0.52 (0.1)  \\ 
 & Scale-free &  0.55 (0.12) & 2.16 (0.88) & 6.14 (0.47) & 3.38 (0.29) & 1.86 (0.32) & 1.03 (0.09)  \\ 
   \hline
\end{tabular}

\caption{Medians and standard-deviation of density ratio computed on 100 graphs of each type (\textit{easy}: $n=100, p=20$, \textit{hard}: $n=50, p=30$)}
\label{meddens}
\end{table}

\begin{table}[ht]
\centering
\begin{tabular}{l|l|rrrrrr}
\multicolumn{2}{l|}{} & \multicolumn{1}{c}{SpiecEasi} & \multicolumn{1}{c}{gCoda} & \multicolumn{1}{c}{ecoCopula} & \multicolumn{1}{c}{MRFcov} & \multicolumn{1}{c}{MInt} & \multicolumn{1}{c}{EMtree} \\ \hline
\multirow{3}{*}{{\rotatebox[origin=c]{90}{Easy}}} & Cluster & 1.77 & 13.89 & 1.74 & 0 & 0 & 0 \\
 & Erdös &0.68 & 11.95 & 0.99 & 0 & 0.83 & 0  \\
 & Scale-free & 0 & 0 & 0 & 0 & 0 & 0\\ \hline
\multirow{3}{*}{{\rotatebox[origin=c]{90}{Hard}}} & Cluster &  0 & 14.05 & 23.40 & 0 & 0 & 0\\
 & Erdös &0 & 20.85 & 27.28 & 0.60 & 0 & 0  \\
 & Scale-free &  0 & 0.43 & 0 & 0 & 0 & 0  \\ \hline
\end{tabular}
\caption{Percentage of empty networks computed on 100 graphs of each type (\textit{easy}: $n=100, p=20$, \textit{hard}: $n=50, p=30$)}
\label{empty}
\end{table}

%\newpage
\begin{table}[ht]
\centering
\begin{tabular}{l|l|rrrrrr}
   n & Criteria (\%) & SpiecEasi & gCoda & ecoCopula & MRFcov & MInt & EMtree \\ 
  \hline
\multirow{3}{*}{{$100$}} & FDR& 0.93(0.04) & 0(0.04) & 0.33(0.11) & 0.29(0.08) & 0.85(0.04) & 0.41(0.08) \\ 
  & density ratio & 0.62(0.13) & 0.08(0.07) & 0.92(0.3) & 0.56(0.14) & 1.97(0.54) & 1.08(0.08) \\ 
  & empty graphs & 0 & 1.88 & 0 & 0 & 0 & 0 \\  \hline
 \multirow{3}{*}{{$50$}} & FDR & 0.94(0.05) & 0(0.13) & 0(0.16) & 0.33(0.16) & 0.85(0.04) & 0.62(0.06) \\ 
 & density ratio & 0.61(0.13) & 0.04(0.03) & 0.08(0.24) & 0.22(0.1) & 1.86(0.32) & 1.14(0.11) \\  
  & empty graphs & 0& 5.97 & 15.46 & 0 & 0 & 0 \\  \hline
\end{tabular}
\caption{Medians and standard-deviation of FDR and density ratio critera, as well as percentage of empty networks computed on 100 scale-free graphs with $p=50$ nodes and $n=100$ or $n=50$ samples.}
\label{perfSF50}
\end{table}

\begin{table}[ht]
\centering
\begin{tabular}{ll|rrrrrr}
 
n & p & SpiecEasi & gCoda & ecoCopula & MRFcov & MInt & EMtree \\ 
  \hline
  100 & 20 & 15.92(0.05) & 0.34(0.79) & 4.84(0.24) & 5.62(0.79) & 66.83(36.03) & 5.06(0.74) \\    
  50 & 30   & 16.3(0.06) & 21.93(32.44) & 11.27(0.84) & 5.97(1.45) & 73.73(31.57) & 3.71(1.11) \\   \hline
  100 & 50 & 20.48(1.4) & 1.07(0.24) & 24.99(0.32) & 15.53(0.12) & 50.51(28.52) & 20.54(2.42) \\   
  50 & 50   & 20.01(2.76) & 44.21(14.78) & 32.75(1.39) & 27.32(1.8) & 55.96(23.26) & 11.54(0.66) \\ \hline
\end{tabular}
\caption{Median and standard-deviation of running times in seconds  of methods for the inference of scale-free structures with different values of the number of samples $n$ and of the number of species $p$. }
\label{timeSF}
\end{table}

\vspace{1.5cm}

\begin{figure}[H]
    \centering
    \includegraphics[width=10cm]{figs/S_effect.png}
    \caption{FDR and density ratio measures of EMtree with varying values of number of sub-samples $S$ (Erdös structure).}
    \label{Seffect}
\end{figure}


\begin{table}[ht]
\centering
\begin{tabular}{l|rrrrrr}
  $S$ & \multicolumn{1}{c}{1} & \multicolumn{1}{c}{2} & \multicolumn{1}{c}{10} & \multicolumn{1}{c}{20} & \multicolumn{1}{c}{50} & \multicolumn{1}{c}{150} \\ \hline
  Easy & 0.66  (0.15) & 1.86  (0.23) & 7.00  (0.81) & 12.29  (1.27) & 29.50  (3.39) & 87.30  (10.36) \\ 
  Hard & 0.45  (0.12) & 1.44  (0.14) & 5.06  (0.78) & 8.97  (0.87) & 23.35  (2.40) & 69.29  (10.83) \\ 
   \hline
\end{tabular}
\caption{Median and standard-deviation running-time values in seconds of EMtree with different values of the number of sub-samples $S$ for the inference of Erdös structures.}
\label{timesS}
\end{table}

\subsubsection{Simulations: network density}

\begin{table}[H]
\centering
\begin{tabular}{l|rr|rr}
    & \multicolumn{1}{c}{$n < 50$} & \multicolumn{1}{c}{$n\geq 50$} & \multicolumn{1}{c}{$p < 20$} & \multicolumn{1}{c}{$p\geq 20$} \\  \hline
  EMtree    &   0.41 (0.11)	&   0.60 (0.15) &   0.38 (0.12) &    0.71 (0.21)      \\ 
  gCoda     &   0.12 (0.47)	&   0.07 (0.03) &   0.05 (0.03) &    0.09 (0.06)     \\ 
  SpiecEasi &   2.41 (0.25)	&   2.41 (0.25) &   2.39 (0.25) &    2.42 (0.25)      \\ 
   \hline
\end{tabular}
\caption{Median and standard-deviation of running times for each method in seconds, for $n$ and $p$ parameters. corresponding to Erdös and cluster structures with $5/p$ densities.}
\label{timeDenser}
\end{table}
 
\subsubsection{Illustrations: edge frequency threshold}

\begin{figure}[H]
    \centering
    \includegraphics[width=\linewidth]{figs/QET_twoDataSets.png}
    \caption{Quantity of selected edges as a function of the selection threshold (\textit{left}: Fatala fishes, \textit{right}: oak mildew.)}
    \label{QETOak}
\end{figure}

The curves displayed on Fig. \ref{QETOak} are very smooth, which illustrates the difficulty of setting this threshold.


%\subsubsection{Illustrations: Fatala River fishes}
%\label{names_Baran}
%\paragraph{Species names with highest betweenness scores:}
%  13: Galeoides decadactylus; 19: Liza grandisquamis, 22:  Pseudotolithus brachygnatus, 25: Pellonula leonensis, 27:  Polydactylus quadrifilis, 30: Pseudotolithus typus, 32: Tylochromis intermedius.

\newpage
\section{Vignette for EMtree}EMtree implements an EM algorithm for the inference of interaction networks from abundance data. It uses averages over spanning trees within a Poisson log-normal model. In addition to functions performing the network inference, this package includes functions for count data simulation under the Poisson log-normal model which Gaussian layer of parameters is faithful to a desired graph structure. EMtree also provides with network visualization functions.


\subsection{\texorpdfstring{Data simulation with
EMtree}{Data simulation with EMtree}}\label{data-simulation-with-emtree}

The EMtree package provides with simulation functions for
graphs as well as for count data under the Poisson log-Normal model.
Four types of graphs are available in the \texttt{generator\_graph()}
function: erdos (Erdös-Reyni), cluster, scale-free (from the
\texttt{huge} package) and spanning tree (from the \texttt{vegan}
package).

\begin{Shaded}
\begin{Highlighting}[]
\NormalTok{p=}\DecValTok{20}
\NormalTok{cluster=}\KeywordTok{generator_graph}\NormalTok{(p, }\DataTypeTok{graph=}\StringTok{"cluster"}\NormalTok{,}\DataTypeTok{dens=}\FloatTok{0.4}\NormalTok{, }\DataTypeTok{r=}\DecValTok{10}\NormalTok{)}
\NormalTok{erdos=}\KeywordTok{generator_graph}\NormalTok{(p, }\DataTypeTok{graph=}\StringTok{"erdos"}\NormalTok{,}\DataTypeTok{dens=}\FloatTok{0.3}\NormalTok{)}
\NormalTok{scaleF=}\KeywordTok{generator_graph}\NormalTok{(p, }\DataTypeTok{graph=}\StringTok{"scale-free"}\NormalTok{)}
\NormalTok{tree=}\KeywordTok{generator_graph}\NormalTok{(p, }\DataTypeTok{graph=}\StringTok{"tree"}\NormalTok{)}
\end{Highlighting}
\end{Shaded}

The output of \texttt{generator\_graph()} is an adjacency matrix, which
can be given as an input to \texttt{generator\_param()} to define a
positive-definite precision matrix \texttt{omega} with signed entries,
and its corresponding variance-covariance matrix \texttt{sigma}.

\begin{Shaded}
\begin{Highlighting}[]
\NormalTok{cluster_param=}\KeywordTok{generator_param}\NormalTok{(cluster, }\DataTypeTok{signed =} \OtherTok{TRUE}\NormalTok{)}
\end{Highlighting}
\end{Shaded}

Then it is possible to simulate \texttt{n} samples of multivariate
counts under the PLN model with \texttt{generator\_PLN()}. Here we simulate $Y$ with the covariate matrix $X$ which includes an intercept and $x_1 \sim \Ncal(0,0.5)$ .

\begin{Shaded}
\begin{Highlighting}[]
\NormalTok{n =}\StringTok{ }\DecValTok{100}
\NormalTok{X =}\StringTok{ }\KeywordTok{data.frame}\NormalTok{(}\DataTypeTok{int=}\DecValTok{1}\NormalTok{,}\DataTypeTok{x1=}\KeywordTok{rnorm}\NormalTok{(n, }\DecValTok{0}\NormalTok{,}\FloatTok{0.5}\NormalTok{))}
\NormalTok{Y =}\StringTok{ }\KeywordTok{generator_PLN}\NormalTok{(cluster_param}\OperatorTok{\$}\NormalTok{sigma,}\DataTypeTok{covariates =}\StringTok{ }\NormalTok{X, }\DataTypeTok{n =} \NormalTok{n}\NormalTok{)}
\end{Highlighting}
\end{Shaded}

\begin{center}\includegraphics[width=0.5\linewidth]{EMtree/vignetteEMtree_files/figure-latex/unnamed-chunk-58-1} \end{center}

\subsection{Inference of the Fatala fishes
network}\label{inference-of-the-fatala-fishes-network}

This is a basic example detailing how to infer a network, using fishes
counts from the Fatala River available in the Barans95 dataset (\texttt{ade4}
package).

\subsubsection{Fatala fishes dataset}\label{fatala-fishes-dataset}

The data is composed of 33 species abundances measures in 95 samples.
The available covariates are the site and date of the samples.

\begin{Shaded}
\begin{Highlighting}[]
\KeywordTok{library}\NormalTok{(ade4)}
\KeywordTok{library}\NormalTok{(tibble)}
\KeywordTok{data}\NormalTok{(baran95)}
\NormalTok{Y =}\StringTok{ }\KeywordTok{as.matrix}\NormalTok{(baran95}\OperatorTok{\$}\NormalTok{fau)}
\NormalTok{X =}\StringTok{ }\KeywordTok{as_tibble}\NormalTok{(baran95}\OperatorTok{\$}\NormalTok{plan)}
\NormalTok{n =}\StringTok{ }\KeywordTok{nrow}\NormalTok{(Y)}
\NormalTok{p =}\StringTok{ }\KeywordTok{ncol}\NormalTok{(Y)}
\end{Highlighting}
\end{Shaded}

\subsubsection{Network inference}\label{network-inference}

EMtree infers a network from either a correlation matrix of a
multivariate Gaussian, or an object created by PLNmodels from count
data. Here we first create a \texttt{PLNmodels} object with
the \texttt{PLN()} function:

\begin{Shaded}
\begin{Highlighting}[]
\KeywordTok{library}\NormalTok{(PLNmodels)}
\NormalTok{PLNfit<-}\KeywordTok{PLN}\NormalTok{(Y }\OperatorTok{~}\StringTok{ }\NormalTok{X}\OperatorTok{\$}\NormalTok{site)}
\end{Highlighting}
\end{Shaded}

\begin{verbatim}
## 
##  Initialization...
##  Adjusting a PLN model with full covariance model
##  Post-treatments...
##  DONE!
\end{verbatim}

And then run \texttt{EMtree()}:

\begin{Shaded}
\begin{Highlighting}[]
\KeywordTok{library}\NormalTok{(EMtree)}
\NormalTok{EMtreeFit<-}\KeywordTok{EMtree}\NormalTok{(PLNfit,  }\DataTypeTok{maxIter =} \DecValTok{20}\NormalTok{, }\DataTypeTok{plot=}\OtherTok{TRUE}\NormalTok{, }\DataTypeTok{verbatim=}\OtherTok{FALSE}\NormalTok{)}
\end{Highlighting}
\end{Shaded}

\begin{center}\includegraphics[width=0.5\linewidth]{EMtree/vignetteEMtree_files/figure-latex/output-1} \end{center}

\begin{Shaded}
\begin{Highlighting}[]
\KeywordTok{str}\NormalTok{(EMtreeFit)}
\end{Highlighting}
\end{Shaded}

\begin{verbatim}
## List of 6
##  $ edges_prob  : num [1:33, 1:33] 0 0.00696 0.02016 0.14686 0.03192 ...
##  $ edges_weight: num [1:33, 1:33] 0 0.000946 0.000946 0.000948 0.000947 ...
##  $ logpY       : num [1:8] -45.2 -45.1 -45.1 -45.1 -45.1 ...
##  $ maxIter     : num 8
##  $ norm.cst    : num 2.07e-50
##  $ timeEM      : 'difftime' num 0.353778123855591
##   ..- attr(*, "units")= chr "secs"
\end{verbatim}

To get a network from a fit of \texttt{EMtree()}, the probabilities
stored in \texttt{edges\_prob} can be thresholded. We suggest the
$2/p$ threshold, which is the probability of an edge in a tree with uniform weights:

\begin{Shaded}
\begin{Highlighting}[]
\NormalTok{probs<-}\StringTok{ }\NormalTok{EMtreeFit}\OperatorTok{\$}\NormalTok{edges_prob}
\NormalTok{net<-}\DecValTok{1}\OperatorTok{*}\NormalTok{(probs}\OperatorTok{>}\DecValTok{2}\OperatorTok{/}\NormalTok{p)}
\end{Highlighting}
\end{Shaded}

To improve the robstness, the function \texttt{ResampleEMtree()}
implements a statibility selection of EMtree on S sub-samples. This
function uses parallel computations with \texttt{mclapply()}. The output
\texttt{Pmat} gathers all the infered edges probabilities for each
sub-sample.

\begin{Shaded}
\begin{Highlighting}[]
\NormalTok{ResampEmtreeFit<-}\KeywordTok{ResampleEMtree}\NormalTok{(}\DataTypeTok{counts=}\NormalTok{Y, }\DataTypeTok{covar_matrix =}\NormalTok{ X}\OperatorTok{\$}\NormalTok{site , }
\DataTypeTok{S=}\DecValTok{10}\NormalTok{, }\DataTypeTok{maxIter=}\DecValTok{20}\NormalTok{,}\DataTypeTok{cond.tol=}\FloatTok{1e-8}\NormalTok{, }\DataTypeTok{cores=}\DecValTok{1}\NormalTok{)}
\end{Highlighting}
\end{Shaded}

\begin{verbatim}
## Computing  10 probability matrices with 1 core(s)...  5.78 secs
\end{verbatim}

\begin{Shaded}
\begin{Highlighting}[]
\KeywordTok{str}\NormalTok{(ResampEmtreeFit}\OperatorTok{\$}\NormalTok{Pmat)}
\end{Highlighting}
\end{Shaded}

\begin{verbatim}
##  num [1:10, 1:528] 0.01361 0.01728 0.00562 0.00348 0.00364 ...
\end{verbatim}

Edges selection frequencies can be derived from the \texttt{Pmat} output
with the function \texttt{freq\_selec()}. A final network can then be
obtained by thresholding the frequencies, to keep for example edges that
are selected in more than $80\%$ of sub-samples:

\begin{Shaded}
\begin{Highlighting}[]
\NormalTok{freqs<-}\KeywordTok{freq_selec}\NormalTok{(ResampEmtreeFit}\OperatorTok{\$}\NormalTok{Pmat,}\DataTypeTok{Pt=}\DecValTok{2}\OperatorTok{/}\NormalTok{p)} \CommentTok{# thresh. probabilities}
\NormalTok{resampNet<-}\DecValTok{1}\OperatorTok{*}\NormalTok{(freqs}\OperatorTok{>}\FloatTok{0.8}\NormalTok{)}\CommentTok{# thresh. frequencies}
\end{Highlighting}
\end{Shaded}

\begin{center}\includegraphics[width=0.5\linewidth]{EMtree/vignetteEMtree_files/figure-latex/unnamed-chunk-9-1} \end{center}

\begin{Shaded}
\begin{Highlighting}[]
\KeywordTok{table}\NormalTok{(net, resampNet)}
\end{Highlighting}
\end{Shaded}

\begin{verbatim}
##    resampNet
## net   0   1
##   0 873   0
##   1 112 104
\end{verbatim}

\subsubsection{Infer networks under several
models:}\label{infer-networks-under-several-models}

The aim of function \texttt{ComparEMtree()} is to run network inference
with different covariates specifications. It uses
\texttt{ResampleEMtree()} and adjust the different models specified in
\texttt{model\_names} as follows:

\begin{Shaded}
\begin{Highlighting}[]
\NormalTok{tested_models=}\KeywordTok{list}\NormalTok{(}\DecValTok{1}\NormalTok{,}\DecValTok{2}\NormalTok{,}\KeywordTok{c}\NormalTok{(}\DecValTok{1}\NormalTok{,}\DecValTok{2}\NormalTok{))}
\NormalTok{models_names=}\KeywordTok{c}\NormalTok{(}\StringTok{"date"}\NormalTok{,}\StringTok{"site"}\NormalTok{,}\StringTok{"date + site"}\NormalTok{)}
\NormalTok{compare_models<-}\KeywordTok{ComparEMtree}\NormalTok{(Y, X, }\DataTypeTok{models=}\NormalTok{tested_models, }
\DataTypeTok{m_names=}\NormalTok{models_names, }\DataTypeTok{Pt=}\DecValTok{2}\OperatorTok{/}\NormalTok{p}\NormalTok{,  }\DataTypeTok{S=}\DecValTok{3}\NormalTok{, }\DataTypeTok{maxIter=}\DecValTok{5}\NormalTok{,}\DataTypeTok{cond.tol=}\FloatTok{1e-8}\NormalTok{,}\DataTypeTok{cores=}\DecValTok{1}\NormalTok{)}
\end{Highlighting}
\end{Shaded}

\begin{verbatim}
## model date : Computing  3 probability matrices with 1 core(s)... 
##   3.68 secs
## model site : Computing  3 probability matrices with 1 core(s)... 
##   1.5 secs
## model date + site : Computing  3 probability matrices with 1 core(s)... 
##   3.95 secs
\end{verbatim}

The output of \texttt{ComparEMtree()} is a tibble in long format, which
gathers information of all interactions in all tested models.

\begin{Shaded}
\begin{Highlighting}[]
\KeywordTok{head}\NormalTok{(compare_models,}\DecValTok{4}\NormalTok{)}
\end{Highlighting}
\end{Shaded}

\begin{verbatim}
## # A tibble: 4 x 4
##   node1 node2 model weight
##   <chr> <chr> <chr>  <dbl>
## 1 1     2     date       0
## 2 1     3     date       0
## 3 2     3     date       0
## 4 1     4     date       0
\end{verbatim}

\subsection{Visualizations}\label{visuals}

The EMtree package provides with easy plotting functions for network
visualizations. They build from the \texttt{ggraph} and
\texttt{tidygraph} packages.

\subsubsection{Simple networks}\label{simple-networks}

The function \texttt{draw\_network()} takes a weighted matrix as input,
and represents a network with edges widths proportional to the input
weights. Several layouts are available (see the \texttt{ggraph}
documentation). Nodes possessing among the highest betweenness
centrality measure can be highlighted with the parameter
\texttt{btw\_rank}.

Example from the adjacency matrices simulated earlier (cluster, spanning-tree, scale-free and erdos):

\begin{center}\includegraphics[width=0.8\linewidth]{EMtree/vignetteEMtree_files/figure-latex/unnamed-chunk-12-1} \end{center}

Weighted matrices can be edge probability matrices. For example, the Fatala fishes weighted network adjusted on the covariate
\textit{Site} is:

\begin{Shaded}
\begin{Highlighting}[]
\NormalTok{probs[probs}\OperatorTok{<}\DecValTok{2}\OperatorTok{/}\NormalTok{p]=}\DecValTok{0} \CommentTok{# threshold needed for representation clarity}
\KeywordTok{draw_network}\NormalTok{(probs,}\DataTypeTok{title=}\StringTok{"Site"}\NormalTok{, }\DataTypeTok{pal_edges=}\StringTok{"dodgerblue3"}\NormalTok{, }
\DataTypeTok{layout=}\StringTok{"nicely"}\NormalTok{,}\DataTypeTok{btw_rank=}\DecValTok{3}\NormalTok{)}\OperatorTok{\$}\NormalTok{G}
\end{Highlighting}
\end{Shaded}

\begin{center}\includegraphics[width=0.5\linewidth]{EMtree/vignetteEMtree_files/figure-latex/unnamed-chunk-13-1} \end{center}

\subsubsection{Several
networks}\label{facets-of-several-networks}

The function \texttt{compare\_graphs()} draws a facet plot of the output
networks from \texttt{ComparEMtree()}. Comparing network by eye is
difficult, in particular choosing the right layout to do so is often
troublesome. Here by default, the circle layout is used so that
differences in density and sensitive nodes are easily identified.

\begin{Shaded}
\begin{Highlighting}[]
\KeywordTok{compare_graphs}\NormalTok{(compare_models,}\DataTypeTok{shade=}\OtherTok{TRUE}\NormalTok{)}\OperatorTok{\$}\NormalTok{G}
\end{Highlighting}
\end{Shaded}

\begin{center}\includegraphics[width=0.8\linewidth]{EMtree/vignetteEMtree_files/figure-latex/unnamed-chunk-14-1} \end{center}

However, if another layout is preferred the nodes position is preserved
along the facet and defined by choosing the \texttt{base\_model}.

\begin{Shaded}
\begin{Highlighting}[]
\KeywordTok{compare_graphs}\NormalTok{(compare_models,}\DataTypeTok{shade=}\OtherTok{FALSE}\NormalTok{, }\DataTypeTok{layout=}\StringTok{"nicely"}\NormalTok{, }\DataTypeTok{curv=}\FloatTok{0.1}\NormalTok{, }
\DataTypeTok{base_model=}\StringTok{"site"}\NormalTok{)}\OperatorTok{\$}\NormalTok{G}
\end{Highlighting}
\end{Shaded}

\begin{center}\includegraphics[width=0.8\linewidth]{EMtree/vignetteEMtree_files/figure-latex/unnamed-chunk-15-1} \end{center}

 
\end{subappendices}


 
\chapter{Inference with Missing Actor}

%chap3

 % intro
%Les réseaux :
%- [ ] C’est quoi
%- [ ] À quoi ça sert (les enjeux, appliqué)
%- [ ] Quelles sont les questions qui se posent (que dit la recherche)
%
%- [ ] Lauritzen pour les nuls
%
%L’inférence de réseaux :
%- [ ] Quelle méthode pour quel réseau
%- [ ] Auxquelles on se compare, pourquoi elles sont comme ça comment elles marchent et ce qui manque

  \section*{Biological context}
 \subsection*{Networks}
 A network is an intuitive object which anyone can easily relate to. It is first of all a graphical tool representing the links between different entities. This helps understand how a system organizes and have a direct image of it. It is also an analysis tool which can unravel sensible information about the system, its structure and the different roles in its organization. Networks are versatile tools that are  used in many domains (e.g. sociology, linguistics, computer sciences, neurosciences, climatology, psychology, etc.) and can take various forms to adapt to each problem.  They can be directed or undirected. Some link entities with multiple kinds of edges (multidimensional), or have different layers (multiplex), others link groups of disconnected objects (multipartite). In biology, the most typical networks simply represent the species (nodes) and their relationships (edges).

Various types of species interactions are studied with networks. For example, their is a rich literature of networks for plant-pollinator and host-parasite relationships in ecology. These species interactions are clearly defined and directly observed in the field. Contacts of pollination or parasitism are counted and networks constructed from these interaction abundances.
\begin{figure}
\centering
\includegraphics[width=0.7\linewidth]{figs/pocock.png}
\caption{Species interaction networks at Norwood Farm, Somerset, UK \citep{PED12,BRM13}.}
\label{pocock}
\end{figure}
 However many mechanisms cannot be observed and may not be well defined. One way to discover them may then be to resort to a more mathematical definition of species interactions. Working with the latter allows the study of community assembly mechanisms with the inference of networks representing guilds of species in community ecology. This type of network is extensively used in genomics for protein-protein interaction network, or gene regulatory networks, or in microbiology to study the output of a metabarcoding experiment assessing the composition of a microbiome. 
 
 \begin{figure}
\centering
\includegraphics[width=0.7\linewidth]{figs/PPInetwork.png}
\caption{Protein-protein interactions between genes involved in schizophrenia \citep{GTH16}.}
\label{PPI}
\end{figure}

  \subsection*{Statistical interactions}
The correlations between the species measures first come to mind as a statistical characterization of an interaction. These are easily obtained, yet their corresponding networks are hard to interpret. Indeed, two covariates correlating with a same third one will appear correlated, even if they have no direct effect on each other\footnote{e.g. the number of covid 19 cases detected correlates with both the real number of cases and the number of tests done on the population, which induces a spurious correlation between the two latter where obviously there is no direct effect of one on the other.}. This phenomena of spurious correlations complicates both the analysis and the interpretation by inducing a very high number of edges  which cannot be categorized as direct or indirect associations between the species. 
 
 Conditional dependencies are then very useful measures of interaction. They describe dependencies between each pair of species conditional on all others. That is, all other species measures kept fixed, their measures should still be correlated. A link between two species can then be interpreted as a direct association. This yields a network of species conditional dependence (link) and independence (absence of link), which is interpretable and falls within the well-studied mathematical framework of graphical models.
 
 \subsection*{Measures on species}
 Networks of statistical interactions are obtained from datasets of repeated measures on a set of species, which can be of various types. Measures can be continuous, as for example the output of gene expression profiling experiments using DNA microarrays, which  are  fluorescences measures from targeted genes. Using a Gaussian approximation, these measures of genes expressions can be used to derive genes regulatory networks.  Measures can also be binary, as in co-occurrence data in ecology, which record the joint presences and absences of a set of species in several sites. \citet{CAM16}.
%développer les données de co-occurence

Abundance data  are joint counts of species in a series of sites (also known as assemblages data in ecology). Recent technologies made this type of data increasingly available. Assemblages data were rare in ecology as they implied intensive sampling efforts, which is now greatly facilitated by camera traps and sensors. In metagenomics, high-throughput sequencing technologies for metabarcoding experiments made it possible  to get joint counts of pseudo-species (operational taxonomic units (OTUs)) abundances.  Both domains work with the same type of output: a dataset of joint (pseudo-)species abundances from different sites or samples.\\
%développer les metabarcoding et OTU


Once the data has been collected, it is very likely that not all species or covariates were observed: there exists missing actors and data is incomplete. In the network, the existence of a missing actor translates into appearance of edges between all the species it should connect with, creating dense cliques of species which are not actually conditionally dependent on one another. A second objective of this work is to take missing actors into account during network inference in order to get more accurate interpretations.

\section*{Network inference}
% ce qui existe et motivation du sujet
\section*{Objectives}
 \subsection*{Graphical models and Trees}
The dedicated framework for the representation of conditional dependency structures are graphical models. Gaussian Graphical Models (GGM) in particular provide with appealing algebraic properties for network inference, which are detailed in \citet{Lau96}. Exploring the set of possible graphs is a non-ending task, and we chose to reduce the searching space to that of spanning trees. This is the sparsest connected structure, and enjoys specific algebraic properties allowing to sum on all possible spanning trees at the cost of a determinant calculus. Our network inference methodology relies on spanning trees and \citet{Lau96} maximum likelihood estimators for multivariate Gaussian graphical models.

%caser le missing actor qq part
 \subsection*{Modeling abundance data}
 As this ideal framework of GGM is not directly applicable to abundance data, there exist two possible ways to proceed: either apply a Gaussian transformation to the data, or rely on Gaussian latent vectors in the framework of Joint Species Distribution Models (JSDM). Our methodology resorts to the latter, and more specifically to the Poisson log-normal distribution to model the counts. This distribution allows easy handling of both covariates and offsets, as well as take overdispersion into account thanks to its Gaussian random parameters. 
 
 
 \subsection*{Estimation procedure} %or algorithm
The central model we adopt involves a Gaussian layer of parameters which is a mixture on all spanning trees. Each component of the mixture is a Gaussian distribution which dependency structure is a spanning tree. This represents two hidden layers of parameters, which we first estimates with an Estimation-Maximization (EM) algorithm. Then to infer missing actors of the network, we resort to a variational (VEM) approach.

\section*{Outline} 
  \subsection*{Chapter 1}
The first chapter details the mathematical tools and technical results for the statistical modeling and the network inference used in Chapters 2 and 3.

   \subsection*{Chapter 2}
   This chapter details a method to infer undirected networks representing conditional statistical dependencies between the species from their joint abundance measures.  The proposed methodology, implemented in the R package \texttt{EMtree},  is compared to state of the art approaches and applied to two empirical datasets from ecology and metagenomics. This chapter has been published as an article in the journal \textit{Methods in Ecology and Evolution} \citep{MRA20}.
   
    \subsection*{Chapter 3}
This chapter details a variational approach to the inference of a missing actor in the network. The reconstruction of missing actor(s) is implemented in the R package \texttt{nestor} and illustrated on two  empirical datasets from ecology. This chapter has been submitted for publication in the \textit{Journal of the Royal Statistical Society: series C (applied statistics)}.
 
  \subsection*{Chapter 4}
This final chapter introduces some natural perspectives of this work. After concluding on the specifics of the developed methodology, natural extension are presented, including network comparison. The inclusion of spatialized data is discussed, as well as the possibility of network inference directly from the observed counts.
 \section*{Notations}
 
 \begin{description}
 \item[Operations:]  \begin{itemize}
     \item[]
 \item[] $|\cdot|$ : matrix determinant
 \item[] $\odot$ : Hadamard product
 \end{itemize}
 \item[Variables:] \begin{itemize}
     \item[]
 \item[] $\Ybf$ :  matrix  of observed counts
 \item[] $\Zbf$ : matrix of latent Gaussian parameters 
 \item[] $\Ubf$ : matrix of latent normalized Gaussian parameters 
 \item[] $\Xbf$ : matrix of covariates 
 \item[] $O$ : matrix of measured offsets
 \end{itemize}
 \item[Dimensions:]\begin{itemize}
     \item[]
 \item[] $n$ : number of samples
 \item[] $p$: number of observed species
 \item[] $r$: number of unobserved species
 \item[] $d$: number of covariates
 \end{itemize}
 \end{description} 
 

%%%%%%%%%%%%%%%%%%%%%%%%%%%%%%%%%%%%%%%%%%%%%%%%%%%%%%%%%%%%%%%%%%%%%%%%%%%%%%%%
\section{Model} \label{sec:Model}
%%%%%%%%%%%%%%%%%%%%%%%%%%%%%%%%%%%%%%%%%%%%%%%%%%%%%%%%%%%%%%%%%%%%%%%%%%%%%%%%
%%%%%%%%%%%%%%%%%%%%%%%%%%%%%%%%%%%%%%%%%%%%%%%%%%%%%%%%%%%%%%%%%%%%%%%%%%%%%%%%

%%%%%%%%%%%%%%%%%%%%%%%%%%%%%%%%%%%%%%%%%%%%%%%%%%%%%%%%%%%%%%%%%%%%%%%%%%%%%%%%
\subsection{Poisson log-normal and tree-shaped graphical models}
%%%%%%%%%%%%%%%%%%%%%%%%%%%%%%%%%%%%%%%%%%%%%%%%%%%%%%%%%%%%%%%%%%%%%%%%%%%%%%%%

%%%%%%%%%%%%%%%%%%%%%%%%%%%%%%%%%%%%%%%%%%%%%%%%%%%%%%%%%%%%%%%%%%%%%%%%%%%%%%%%
\subsubsection*{Poisson log-normal model.} 
We start with a reminder on the multivariate Poisson log-normal model, with the example of abundance data. The abundances of $p$ species observed on $n$ sites are gathered in the $n \times p$ matrix $\Ybf$ where $Y_ {ij}$ is the count of species $j$ in site $i$, and the row $i$ of $\Ybf$, denoted $\Ybf_i$, is the abundance vector collected on site $i$. A covariate vector $\xbf_i $ with dimension $d$ is also measured on each site $i$ and all covariates are gathered in the $n \times d$ matrix  $\boldsymbol X$. The PLN model states that a (latent) Gaussian vector $\Ubf_i$ of size $p$ with variance matrix $\Rbf = (\rho_{kl})_{kl}$ is associated with each site:
\begin{equation} \label{eq:PLN-Z}
\{\Ubf_i\}_{1 \leq i \leq n} \text{ iid}, \qquad 
\Ubf_1 \sim \Ncal_p(\zerobf, \Rbf),
\end{equation}
the sites being assumed to be independent. To ensure identifiability, we let the diagonal of $\Rbf$ be made of 1's, so $\Rbf$ is actually a correlation matrix.
All latent vectors $\Ubf_i$ are gathered in the $n \times p$ matrix $\Ubf$. The PLN model further assumes that species abundances in all sites are conditionally independent, and that their respective distribution only depends on the environment and the associated latent variable:
\begin{equation} \label{eq:PLN-Y.Z}
\{Y_{ij}\}_{1 \leq i \leq n, 1 \leq j \leq p} \mid \Ubf \text{ independent}, \quad 
Y_{ij} \mid U_{ij} \sim \Pcal\left(\exp(o_{ij} + \xbf_i^\intercal \thetabf_j + \sigma_j U_{ij})\right),
\end{equation}
where $o_{ij}$ is a known offset term which typically accounts for the sampling effort, and $\sigma_j$ is the latent standard deviation associated with species $j$. The vector $d \times 1$ of regression coefficients $\thetabf_j$ describes the environmental effects on species $j$. An important feature of the PLN model is that the sign of the correlation between the observed counts is the same as this of correlation between the latent variables \citep{AiH89}: $\text{sign}(\text{Cor}(Y_{ij}, Y_{ik})) = \text{sign}(\text{Cor}(U_{ij}, U_{ik}))$. 
% The dependence between the species abundances is entirely controlled by the latent dependency structure encoded in the precision matrix $\Omegabf:=\Rbf^{-1}$.

%%%%%%%%%%%%%%%%%%%%%%%%%%%%%%%%%%%%%%%%%%%%%%%%%%%%%%%%%%%%%%%%%%%%%%%%%%%%%%%%
\subsubsection*{Tree-shaped graphical models.} 
Network inference relies on the assumption that few species are directly dependent on one another, meaning that the underlying graphical model is sparse. In the framework of the PLN model, the graphical model of interest rules the distribution of the latent vectors $\Ubf_i$ and is  encoded in the precision matrix $\Omegabf:=\Rbf^{-1}$. A way to foster sparsity is to impose $\Omegabf$ to be faithful to a spanning tree $T$, that is: $\Ubf_1 \sim \Ncal_p(\zerobf, \Omegabf_T^{-1})$ where the non-zero terms of $\Omegabf_T$ correspond to the edges of the tree $T$ . However this hypothesis is very restrictive  as it allows only $p-1$ links among $p$ species \citep{ChowLiu}. A more flexible approach consists in assuming that the latent vectors are drawn from a mixture of Gaussian distributions, each faithful to a tree $T$ \citep{MixtTrees,MeilaJaak,kirshner,SRS19}:
\begin{equation} \label{eq:mixt-Z}
\Ubf_1 \sim \sum_{T \in \Tcal_p} p(T) \Ncal_p(\zerobf, \Omegabf_T^{-1}),
\end{equation}
where $\Tcal_p$ is the set of spanning trees with $p$ nodes.
We further assume that the tree distribution $\{p(T)\}_{T \in \Tcal_p}$ can be written as a product over the edges:
\begin{equation} \label{eq:prob-T}
p(T) = B^{-1} \prod_{jk \in T} \beta_{jk}, \qquad
\text{with} \quad B = \sum_{T \in \Tcal_p} \prod_{jk \in T} \beta_{jk}.
\end{equation}
The weights $\beta_{jk}$ are gathered in the $p \times p$ symmetric matrix $\betabf$ with diagonal zero. Observe that these weights are defined up to a multiplicative constant, so that only $p(p-1)/2 - 1$ of them may vary independently. This PLN model with latent tree-shaped dependency structure is similar to that considered by \cite{MRA20}.

%%%%%%%%%%%%%%%%%%%%%%%%%%%%%%%%%%%%%%%%%%%%%%%%%%%%%%%%%%%%%%%%%%%%%%%%%%%%%%%%
\subsection{Introducing the missing actor} \label{sec:missActor}
%%%%%%%%%%%%%%%%%%%%%%%%%%%%%%%%%%%%%%%%%%%%%%%%%%%%%%%%%%%%%%%%%%%%%%%%%%%%%%%%

%%%%%%%%%%%%%%%%%%%%%%%%%%%%%%%%%%%%%%%%%%%%%%%%%%%%%%%%%%%%%%%%%%%%%%%%%%%%%%%%
\subsubsection*{PLN model with missing actors.} 
We now introduce the concept of missing actors, which corresponds to variables that are involved in the graphical model but are not associated with observed variables. To involve such actors in the model, we assume that a complete latent vector $\Ubf_i$ with dimension $p+r$ is associated with site $i$, where $r$ is the number of missing actors. This complete vector can be decomposed as $\Ubf_i^\intercal = [\Ubf_{Oi}^\intercal \; \Ubf_{Hi}^\intercal]$ where $\Ubf_{Oi}$ (with dimension $p$) corresponds to observed species and $\Ubf_{Hi}$ (with dimension $r$) corresponds to the missing actors.
The complete $n \times (p+r)$ latent matrix $\Ubf$ can be decomposed in the same way as $\Ubf = [\Ubf_O \; \Ubf_H]$, $\Ubf_O$ and $\Ubf_H$ having dimension $n \times p$ and $n \times r$, respectively. \\ 
The model we consider states that
\begin{enumerate}[label=\roman*]
\item the complete latent vectors $\Ubf_i$ are all iid and distributed according to a mixture similar to \eqref{eq:mixt-Z} and \eqref{eq:prob-T} but with Gaussian distributions (and matrices $\Omegabf_T$ and $\betabf$) of dimension $(p+r)$, and trees drawn from $\Tcal_{p+r}$;
\item  the abundances $Y_{ij}$ of the  $p$ observed species are distributed according to \eqref{eq:PLN-Y.Z}, replacing $\Ubf$ with $\Ubf_O$,
\end{enumerate}

\begin{figure}[H]
 \begin{center}
	\begin{tikzpicture}	
      \tikzstyle{every edge}=[-,>=stealth',auto,thin,draw]
		\node (A1) at (0.625*\length, 2*\length) {$T$};
		\node (A2) at (0*\length, 1*\length) {$\Ubf_O$};
		\node (A3) at (1.25*\length, 1*\length) {$\Ubf_H   $};
		\node (A4) at (0*\length, 0*\length) {$\Ybf$};
		\draw (A1) edge [->] (A2);
        \draw (A1) edge [->] (A3);
        \draw (A2) edge  (A3);
        \draw (A2) edge [->] (A4);
	\end{tikzpicture} 
 \caption{Graphical model linking the count data $\Ybf$, the latent layer of Gaussian parameters $\Ubf=(\Ubf_O,\Ubf_H)$, and the latent tree $T$.}
  \label{fig:MGmodel}
    \end{center}
\end{figure}

In the sequel, we shall refer to the elements of $\Ubf_O$ and $\Ubf_H$ respectively as 'observed' and 'hidden' (or 'missing') latent variables, whereas obviously none of them are actually observed. Figure \ref{fig:MGmodel} displays the graphical model of the quadruplet $(T, \Ubf_O, \Ubf_H, \Ybf)$. The observed data $\Ybf$ still arise from an PLN model, but the graphical model of the observed latent $\Ubf_O$ may not be sparse due to the marginalization over the hidden latent $\Ubf_H$. Our main goal is to infer the dependency structure of the complete latent vectors, that is to estimate the elements of the matrices $\Omegabf_T$ and the edges weights $\betabf$. The latent dependency structure is similar to this considered by \cite{RAR19}, but the inference strategy much differs, because of the additional hidden layer.

%%%%%%%%%%%%%%%%%%%%%%%%%%%%%%%%%%%%%%%%%%%%%%%%%%%%%%%%%%%%%%%%%%%%%%%%%%%%%%%%
\subsubsection*{Identifiability restriction.} 
The proposed model only makes sense because the graphical model of the complete latent vectors $\Ubf_i^\intercal = [\Ubf_{Oi}^\intercal \; \Ubf_{Hi}^\intercal]$ is supposed to be sparse. Missing actors could obviously not be identified from a regular PLN model, without restriction on the precision matrix $\Omegabf$, as only the marginal precision matrix of the $\Ubf_{Oi}$ could be recovered. Still, to ensure identifiability we impose the same restriction as \cite{RAR19} that  missing latent variables are not connected with each other (the block corresponding to $\Ubf_H \times \Ubf_H$ is diagonal in each $\Omegabf_{T}$).
%\CA{However \SR{here}{} we do not need the additional assumption that all precision matrices $\Omegabf_T$ borrow their non-null elements from a same matrix.}{} \SR{}{[{\sl Faut-il mettre cette dernière phrase alors que rien ne le suggère ? Ou alors préciser, 'as opposed to \cite{RAR19}'}]}
%\begin{description}
%\item[(A)] All the precision matrices $\Omegabf_T$ (for $T \in \Tcal_{p+r}$) borrow their elements from a same matrix $\Omegabf$. Namely:
%$$\forall T \in \Tcal_{p+r}, \qquad [\Omegabf_T]_{j,k} = 
%\left\{ \begin{array}{ll}
%    [\Omegabf]_{j, k} & \text{if } (j, k) \in T \\
%    0 & \text{otherwise}.
%\end{array}\right.$$
%and their conditional variance is set to one. Namely, 
%$$\forall p+1 \leq h, \ell \leq p+r, \qquad [\Omegabf]_{h, \ell} = 
%\left\{ \begin{array}{ll}
%    1 & \text{if } h = \ell \\
%    0 & \text{otherwise}.
%\end{array}\right.$$
 
%\end{description}
%Assumption ({\bf A}) obviously avoids to infer $|\Tcal_{p+r}| = (p+r)^{p+r-2}$ independent (sparse) precision matrices. Assumption ({\bf B}) ensures identifiability, especially regarding the scaling of the missing latent variables.

%To summarize, the model defined in Section \ref{sec:missActor} involves $p d$ regressions coefficients (gathered in $\thetabf$), $
%(p+r)(p+r+1)/2 - r(r+1)/2 = 
%p(p+1)/2 + r p$ conditional covariances (gathered in $\Omegabf$), and $(p+r)(p+r-1)/2 - 1$ independent edge weights (gathered in $\betabf$).

%Finally, the original PLN model only concerns covariates $\Ybf$ and $\Ubf_O$. The use of a dependency structure with mixture of trees allows a sparse and efficient inference, and missing actors are accounted for in covariate $\Ubf_H$.


We now describe how to infer the model parameters. We gather the edges weights into the $p \times p$ matrix $\betab$ and the vectors of regression coefficients into a $d \times p$ matrix $\thetab$. The $p \times p$ matrix $\Sigmab$ contains the variances and covariances between the coordinates of each latent vector $\Zb_i$. Hence, the set of parameters to be inferred is $(\betab, \Sigmab, \thetab)$.

\paragraph{Likelihood.} 
The model described above is an incomplete data model, as it involves two hidden layers: the random tree $T$ and the latent Gaussian vectors $\Zb_i$. The most classical approach to achieve maximum likelihood inference in this context is to use the Expectation-Maximization algorithm \citep[EM:][]{DLR77}. Rather than the likelihood of the observed data $p(\Yb)$, the EM algorithm deals with the often more tractable likelihood $p(T, \Zb, \Yb)$ of the complete data (which consists of both the observed and the latent variables). It can be decomposed as 
 
\begin{equation} \label{eq:PTZY}
    p_{\betab, \Sigmab, \thetab}(T, \Zb, \Yb) = p_{\betab}(T) \times p_{\Sigmab}(\Zb \; | \; T) \times p_{\thetab}(\Yb \; | \; \Zb),
\end{equation}
 
where the subscripts indicate on which parameter each distribution depends. \\
Observe that the dependency structure between the species is only involved in the first two terms, whereas the third term only depends on the regression coefficients $\thetab$. 
We take advantage of this decomposition to propose a two-stage estimation algorithm. The first stage deals with the observed layer $p_{\thetab}(\Yb \; | \; \Zb)$, the second with the two hidden layers $p_{\betab}(T)$ and  $p_{\Sigmab}(\Zb \; | \; T)$. The network inference itself takes place in the second step.

\paragraph{Inference in the observed layer.} 
The variational EM (VEM) algorithm that provides an estimate of the regression coefficients matrix $\thetab$ is described in Appendix \ref{app:VEM} (along with a reminder on EM and VEM). It also provides the (approximate) conditional means $\Esp(Z_{ij} | \Yb_i)$, variances $\Var(Z_{ij} | \Yb_i)$ and covariances $\Cov(Z_{ij}, Z_{ik} | \Yb_i)$ required for the inference in the hidden layer. As a consequence, this first step provides the estimates $\widehat{\thetab}$ and $\widehat{\Sigmab}$.

\paragraph{Inference in the hidden layer.} The second step is dedicated to the estimation of $\betab$. The EM algorithm actually deals with the conditional expectation of the complete log-likelihood, namely $\Esp\left(\log p_{\betab, \Sigmab, \thetab}(T, \Zb, \Yb) \; | \; \Yb\right)$. 
As shown in Appendix \ref{app:EM}, this  reduces to
 
\begin{equation} \label{expectation}
    \Esp\left(\log p_{\betab, \Sigmab, \thetab}(T, \Zb, \Yb) \; | \; \Yb\right)
    \simeq
    \sum_{1 \leq j < k \leq p} P_\jk \log \left(\beta_\jk \widehat{\psi}_\jk\right) - \log B + \cst
\end{equation}
 
where $\widehat{\psi}_\jk$ is the estimate of $\psi_\jk$ defined in Eq.~\eqref{eq:pZfact}, and the '$\cst$' term depends on $\thetab$ and $\Sigmab$ but not on $\betab$. 
$P_\jk$ is the approximate conditional probability (given the data) for the edge $(j, k)$ to be part of the network:
$P_\jk \simeq \prob\{jk \in T \; | \; Y\}$.
It is also shown in Appendix~\ref{app:EM} that $\widehat{\psi}_\jk = (1-\widehat{\rho}_\jk^2)^{-n/2}$, where the estimated correlation $\widehat{\rho}_\jk$ depends on the conditional mean, variance and covariances of the $Z_{ij}$'s provided by the first step.
 Eq.~\eqref{expectation} is maximized via an EM algorithm iterating the calculation of the $P_\jk$ and the maximization with respect to the $\beta_\jk$:
\begin{description}

\item[Expectation step: Computing the $P_\jk$ with tree averaging.] The conditional probability of an edge is simply the sum of the conditional probabilities of the trees that contain this edge. Hence, computing $P_\jk$ amounts to averaging over all spanning trees.
Fig.~\ref{fig:treeaveraging} illustrates the principle of tree averaging for a toy network with $p=4$ nodes. Here, five arbitrary spanning trees $t_1$ to $t_5$ (among the $p^{p-2} = 16$ spanning trees) are displayed, with their respective conditional probability $p(T \mid Y)$. 
The edge $(1, 3)$ has a high conditional probability $P_{13}$ because it is part of likely trees such as $t_3$ and $t_4$, whereas $P_{23}$ is small because the edge $(2, 3)$ is only part of unlikely trees (e.g. $t_1$, $t_2$). \\
Averaging over all spanning trees at the cost of a determinant calculus (i.e. with complexity $O(p^3)$) is possible using the Matrix Tree theorem \citep[][recalled as Theorem~\ref{thm:MTT2} in Appendix~\ref{app:MTT}]{matrixtree}. 
\citet{kirshner} further shows that all the $P_\jk$'s can be computed at once with the same complexity $O(p^3)$, although the calculation may lead to numerical instabilities for large $n$ and $p$.



\begin{figure}%[H]
   \begin{center}
    \begin{tabular}{cccccc}
        \input{figs/FigTreeAveraging-p4-tree1-seed2} &
        \input{figs/FigTreeAveraging-p4-tree2-seed2} &
        \input{figs/FigTreeAveraging-p4-tree3-seed2} &
        \input{figs/FigTreeAveraging-p4-tree4-seed2} &
        \input{figs/FigTreeAveraging-p4-tree5-seed2} \\
        $t_1: 2.1\%$ & 
        $t_2: 3.5\%$ & 
        $t_3: 34.1\%$ & 
        $t_4: 15.6\%$ & 
        $t_5:  <.1\%$ \\ \\
        & 
        \input{figs/FigTreeAveraging-p4-avgtree-seed2} &
        \qquad \qquad &
        \input{figs/FigTreeAveraging-p4-graph-seed2} \\
        \multicolumn{3}{c}{Edge conditional probabilities} & Estimated graph \\
    \end{tabular}
    \caption{Tree averaging principle. 
    \textit{Top:} 5 spanning trees with 4 nodes  $(t_1, \dots t_5)$, with their respective conditional probability given the data $P(T = t \mid Y)$.
    \textit{Bottom left:} Weighted graph resulting from tree averaging. Each edge  has width proportional to its conditional probability. \textit{ Bottom right:} Estimated graph (obtained by thresholding edge probabilities) is not a tree.}
    \label{fig:treeaveraging}
   \end{center}
\end{figure}

\item[Maximization step: Estimating the $\beta_\jk$.] 
This step is not straightforward, as the normalizing constant $B = \sum_T \prod_{jk \in T} \beta_\jk$ involves all $\beta_\jk$'s. We propose an exact maximization built upon the Matrix Tree theorem (see Appendix~\ref{app:EM}). 
\end{description}


\paragraph{Algorithm output: edge scoring and network inference} 
%As a side product, 
EMtree provides the (approximate) conditional probability $P_\jk$ for each edge $(j, k)$ to be part of the network. 
Whenever an actual inferred network $\widehat{G}$ is needed (e.g. for a graphical purpose), it can be obtained by thresholding the $P_\jk$ (see Fig.~\ref{fig:treeaveraging}, bottom right). Because we are dealing with trees, a natural threshold is the density of a spanning tree, which is  $2/p$.
More robust results can be obtained using a resampling procedure similar to the stability selection proposed by \citet{LRW10}. It simply consists in sampling a series of subsamples $s = 1 \dots S$, to get an estimate $\widehat{G}^s$ from each of them and to collect the selection frequency for each edge. Again, these edge selection frequencies can be thresholded if needed.


\section{Simulations} \label{sec:Simul}
%%%%%%%%%%%%%%%%%%%%%%%%%%%%%%
\subsection{Count datasets}
For the simulation study, 300 count datasets of $15$ species in total including one missing actor are generated, thus $p=14$ and $r=1$. 
Data is generated as follows.
We generate a scale-free structure $\mathcal{G}$ (which degree distribution is a power law) with $p+1$ nodes using the R package \texttt{huge} \citep{zhao2012huge} available on CRAN. 
The missing species $h$ is chosen as the one with highest degree. We measure the {\sl influence} of the missing actor with its degree, distinguishing three influence classes: \textit{Minor} (degree $\leq 5$), \textit{Medium} ($5<$ degree $\leq 7$) and \textit{Major} (degree $\geq 8$). For each replicate, the latent layer $\Ubf$ and the observed abundances $\Ybf$ are simulated according to the model defined in Section \ref{sec:Model}.


\subsection{Experiment \& Measures}
% \SR{
% For each simulated dataset, the set of 4 likely initial cliques is identified as detailed in Section 
%\ref{init}. 
%Simulated datasets involve a unique missing actor. A quick exploration of the space of likely cliques consists in keeping the two first principal components of sPCA, and their complements. This way, we obtain a set of 4 $C_h$ candidates, which size we impose to be between 2 and $(p-1)$ (the missing actor should have at least two neighbors, but not all the network).
%}{
For each simulated dataset, the VEM algorithm is initialized as described in Section \ref{init}. 
More specifically and because we only look for one missing actor, we consider the cliques corresponding to each of the first two principal components of sPCA, and their respective complements, which provides us with four cliques.
Then four VEM algorithms, as described in Section \ref{algo}, are run starting from each of the four candidate cliques, and the one yielding the highest lower bound $\Jcal$ is kept. 
For all simulations, we set the precision of the convergence criterion to $\varepsilon=10^{-3}$, the tempering parameter to $\alpha=0.1$ and the maximal number of iterations to $100$. 
The  inference quality is assessed regarding the global network inference, the missing actor's position in the network, and its values along the $n$ sites. We refer to this first procedure as the \textit{blind} procedure. Additionally, we define the \textit{oracle} procedure as running the VEM with the set of true neighbors of the missing actor as initial clique.\\

For each procedure, a general measure of the whole network inference quality is first given by comparing the inferred edge probabilities to the original dependency structure. This is done using the Area Under the ROC Curve (AUC) criteria.  Then, to be more specific and target the neighbors of node $h$ specifically, the probabilities of edges involving $h$ are transformed into binary values using the 0.5 threshold. The values are then compared to the original links of $h$ and yield quantities of true/false ($T$/$P$) positives/negatives ($P$/$N$), from which are built the \textit{precision} (also known as the positive predictive value, ${TP}/({TP+FP)}$) and the \textit{recall} (also known as the true positive rate, ${TP}/({TP+FN})$) criteria. Finally, we assess the ability to reconstruct the missing actor across the sites  by computing the absolute correlation between its inferred vector of means ($M_h$) and its original latent Gaussian vector $\Ubf_h$.


%%%%%%%%%%%%%%%%%%%%%%%%%%%%%%
\subsection{Results}
Simulations performance measures are gathered in Table \ref{tab:perf} and Table \ref{tab:oracle} for blind and oracle procedures respectively. The distributions of the quality measures are displayed in Figure \ref{fig:densities}. \\

Table \ref{tab:perf} shows the network is well inferred, as all AUC means are above 0.85, with almost perfect inference when the influence of the missing actor is major. Its neighbors and values per site are very well retrieved in these cases with mean recall values above 0.9 and mean correlation above 0.8, with a great confidence in the algorithm outputs as mean precision is above 0.95. However, there exists a clear deterioration of all performance as the influence decreases with lower means are greater deviations, down to about 0.6 mean values for all measures when the influence is minor. Moreover, the algorithm takes more and more time to converge as the influence decreases, although it stays at about $3s$ for minor cases which is reasonable. Figure \ref{fig:densities} shows that as the influence decreases, the densities present with several modes and dilute towards 0, illustrating that even if some networks are still well-inferred, there also are more and more cases where the algorithm fails. In particular, the performance decrease of medium cases seems to be only due to a greater number of failed inferences.\\

All these elements point to minor cases being harder problems to solve, unsurprisingly. Yet as oracle results show in Table \ref{tab:oracle}, it is possible to carry out almost-perfect inference in all cases, if the algorithm is initialized with the true clique; the deterioration is still present in all measures, but stays marginal. Thus the harsh decrease in the blind procedures seems to be mainly due to the proposed initialization method failing at correctly finding some of the small cliques of neighbors.\\

%\textcolor{red}{Faut-il autant s'etendre sur ce point ? Pour l'instant, nous semblons dire que l'initialisation fait tout et que la solution que nous proposons ne fonctionne que dans les cas facile...}
%\textcolor{red}{Si on le garde : ajouter un titre 'About initialization'?}

% point sur les FN dans les initializations
\paragraph{About intialization.}
Figure \ref{fig:perfinit} compares the initialization quality and the corresponding final inferred neighbors, in terms of initial (-i) and final (-f) false negative (FNR, also 1-TPR) and positive rates (FPR). It clearly appears that final measures  mostly increase with false negatives of the initial clique. This means that not including a neighbor in the initialization is much worse for the inference than falsely including a node. The increase of FNR-f is bigger than that of FPR-f, meaning that a wrong initialization leads to a set of inferred neighbors which most part can be trusted, but which will be largely incomplete. This advocates for bigger initialization cliques when no prior information is available.


\begin{table}
\centering
\begin{tabular}{lrrrrrr}
  \hline
  & N & AUC & Precision & Recall & Correlation & Time (s) \\ 
  \hline
 Major & 100 & 0.98 (0.06) & 0.96 (0.14) & 0.94 (0.17) & 0.83 (0.10)& 2.36 (0.91)  \\ 
 Medium & 132 & 0.93 (0.12) & 0.83 (0.26) & 0.81 (0.30) & 0.73 (0.17)& 2.69 (1.15)  \\ 
 Minor &  68 & 0.89 (0.10) & 0.61 (0.34) & 0.66 (0.36) & 0.59 (0.21) & 3.08 (1.14) \\ 
   \hline
\end{tabular}
\caption{\label{tab:perf}Blind procedure using cliques from initialization. The influence of the missing actor is measured with its degree, distinguishing three influence classes: \textit{Minor} (degree $\leq 5$), \textit{Medium} ($5<$ degree $\leq 7$) and \textit{Major} (degree $\geq 8$).  For each class of influence, the following quantities are reported:  number of simulated graphs (N), means and standard deviations of AUC, Precision, Recall, Correlation between missing actor inferred vector of means and original latent vector, and running times in seconds. AUC measures the retrieval of the dependence structure between all variables (observed and missing), whereas precision and recall are specific to the missing actor links.} 

\end{table}
 
 

\begin{figure}[H]
    \centering
    \includegraphics[width=11cm]{figs/simu_densities.png}
    \caption{
    %\CA{}
    The influence of the missing actor is measured with its degree, distinguishing three influence classes: \textit{Minor} (degree $\leq 5$), \textit{Medium} ($5<$ degree $\leq 7$) and \textit{Major} (degree $\geq 8$). 
    %\CA{Distributions of performance measures}
    The distributions of performance measures are displayed for each class of influence: AUC measures the retrieval of the dependence structure between all variables, observed and missing.  Precision and recall are specific to the missing actor links. 
    }
    \label{fig:densities}
\end{figure}
 

\begin{figure}[H]
    \centering    \includegraphics[width=11cm]{figs/quali_init_spca.png}
    \caption{Comparison of initial and final FPR and FNR, for cliques of neighbors of one missing actor obtained with the sparse PCA method. Position of dots are defined according to initial values, their color according to the final FPR and FNR. Sizes are proportional to the density of dots on a given position.}
    \label{fig:perfinit}
\end{figure}


 \begin{table}
\centering
\begin{tabular}{lrrrrrr}
  \hline
  & N & AUC & Precision & Recall & Cor. &t(s) \\ 
  \hline
  Major & 100 & 1 (0.00) & 1 (0.00) & 1 (0.01) & 0.86 (0.02)  & 1.28 (0.21) \\ 
  Medium & 132 & 1 (0.02) & 1 (0.00) & 0.99 (0.04) & 0.83 (0.02)  & 1.38 (0.46)  \\ 
  Minor &  68 & 0.98 (0.04) & 0.99 (0.03) & 0.96 (0.12) & 0.8 (0.04)& 1.56 (0.69) \\ 
   \hline
\end{tabular}
 \caption{\label{tab:oracle} Oracle procedure using true clique as starting point. The influence of the missing actor is measured with its degree, distinguishing three influence classes: \textit{Minor} (degree $\leq 5$), \textit{Medium} ($5<$ degree $\leq 7$) and \textit{Major} (degree $\geq 8$).  For each class of influence, the following quantities are reported:  number of simulated graphs (N), means and standard deviations of AUC, Precision, Recall, Correlation between missing actor inferred vector of means and original latent vector, and running times in seconds. AUC measures the retrieval of the dependence structure between all variables (observed and missing), whereas precision and recall are specific to the missing actor links.}
\end{table}


\section{Applications}  \label{sec:Appli}


%%%%%%%%%%%%%%%%%%%%%%%%%%%%%%%%%%%%%%%%%%%%%%%%%%%%%%%%%%%%%%
\subsection{Cross validation criterion for  model selection}
%\SR{
%As no model selection criteria has been designed for this problem yet, we  run a 10-fold cross-validation procedure, with pairwise composite likelihood (PCL) as adjustment criteria (see \citet{lindsay}). This choice of pairwise components is motivated from the fact that they keep track of the covariance structure, and the estimation of bivariate Poisson log-normal densities is made possible thanks to the \texttt{bipoilog} function of the \texttt{poilog} R package. \\
%
%The cross-validation procedure to estimate the pairwise composite likelihood is available in appendix \ref{CV}. The general idea is to compute the bivariate Poisson log-normal densities between all pair of species, conditional on a tree sampled as explained in appendix \ref{sampTrees} which defines the distribution parameters. Finally the procedure computes the average criteria  $$\displaystyle PCL(\Ybf)=\frac1V\sum_{v=1}^{V}\frac1B \sum_{b=1}^Bf_{PLN}(\Ybf; b,v),$$ where $f_{PLN}(\Ybf; b,v)=\sum_{\substack{i \in v\\ j < k}} \log p_{PLN}[(Y_{ij}^v, Y_{ik}^v) | T^b; \widehat{\theta},\widehat{\Sigma}_{T^b jk}]$, with $V=10$ and $B=10^2$. \\
%This is a computational greedy procedure that is not suited for a simulation study. It was applied to two empirical datasets in order to decide the number of missing actors in the model. The results, gathered in Figure \ref{fig:selec}, yield $r=1$ for the Barents Sea data set, and $r=2$ for the Fatala River one.}{}

The proposed model obviously raises the problem of choosing the number of missing actors $r$ (which may be zero). Variational-based inference often relies on approximate versions of the BIC or ICL criteria for model selection. Few theoretical guaranties exist about these approximate criteria and, in the present case, we observed that BIC and ICL penalizations did not yield consistent results. Therefore, we resort to $V$-fold cross validation to determine the number of missing actors. 

More specifically, we split the original dataset $\Ybf$ ($\Xbf$ is dropped here for the sake of clarity) into $V$ subsets with almost equal sizes $m_1, \dots m_V$ ($\sum_{v=1}^V m_v = n$), which we denote $\{\Ybf^v\}_{v = 1, \dots V}$. For each subset $v$, we define its complement $\Ybf^{-v}$ on which we fit a model with $r$ missing actors and get a parameter estimate $\Gammabf_r^{-v} = (\thetabf_r^{-v}, \sigmabf_r^{-v}, \betabf^{-v}_r, \Omegabf_r^{-v})$ and measure the fit of $\Gammabf_r^{-v}$ to the test dataset $\Ybf^v$. 

To avoid the integration over the $(p+r)$-dimensional Gaussian latent layer, we measure the fit with the pairwise composite likelihood \citep{lindsay}.
For any given tree $T$ and parameter $\Gammabf$, the bivariate Poisson log-normal pdf $p_{PLN}\left((Y_{ij}, Y_{ik}); \Gammabf, T \right)$ can be easily computed for any sample $i$ and pair of species $(j, k)$ with available tools such as the \texttt{poilog} R package \citep{ViS08} available on CRAN. The cross-validation criterion is defined as
$$
PCL_r(\Ybf) = \frac1V \sum_v \frac1B \sum_{b=1}^B \frac1{m_v} \sum_{i = 1}^{m_v} \sum_{j < k} \log p_{PLN}\left((Y^v_{ij}, Y^v_{ik}); \Gammabf_r^{-v}, T_{r, b}^{-v} \right)
$$
where the tree samples $\{T_{r, b}^{-v}\}_{b=1 \dots B}$ are iid according to $p_{\betabf_r^{-v}}(T)$. 

The sampling procedure for spanning trees is given in Appendix \ref{eq:sampTree}; the complete procedure for the calculation of $PCL_r(\Ybf)$ is described by Algorithm \ref{algo:model-selection}, given in Appendix \ref{sec:modSel}. Note that this criterion measures the fit of the model in terms of abundance prediction, whereas our interest is mostly focused on the inference of the dependency structure. In other words, our goal is identification, that is selecting the smallest model  and not the best model in terms of prediction \citep{arlot2010survey}.


We did not include this computationally greedy procedure in the simulation study but applied it to the two ecological datasets that will be described in the next two sections. The results, gathered in Figure \ref{fig:selec}, yield $r=1$ missing actor for the Barents Sea data set, and $r=2$ missing actors for the Fatala River one.

\begin{figure}[H]
    \centering
    \includegraphics[width=10cm]{figs/selec_model_applis.png}
    \caption{Pairwise composite likelihoods estimates of Barents and Fatala datasets for models including 0 to 3 missing actors.}
    \label{fig:selec}
\end{figure}

% \SR{
% A wider exploration of likely cliques is conducted thanks to a bootstrap approach with $200$ sub-samples. Each of the latter consists of 80\% of the data; sPCA is run on each of them and only the identified clique of the $r$ first principal components are stored, when the studied model involves $r$ missing actors. When $r>1$, the restriction on the clique sizes is lifted. The bootstrap thus yield 200 lists of $r$ initial cliques, from which only unique ones are kept. This approach is more time-consuming, and therefore only used on empirical datasets.
% }{
Regarding the initialization, we performed a wider exploration as compared to the simulation study. To enlarge the list of possible cliques, we applied a resampling version of the procedure described in Section \ref{sec:algoSpec}, and applied it to 200 sub-samples, each consisting in 80\% of the whole data set. This yielded 200 lists of $r$ initial cliques, from which duplicates were removed.
%}

%%%%%%%%%%%%%%%%%%%%%%%%%%%%%%%%%%%%%%%%%%%%%%%%%%%%%%%%%%%%%%%%%%%%%%%%%%%%%%%%
\subsection{Barents Sea}
%data availability ? précisions sur les données, elles viennent d'où

% \SR{30 species of fish were counted in 89 sites of the Barents Sea (shrimp survey in the period April-May 1997, available at  \url{https://www.fbbva.es/microsite/multivariate-statistics/data.html})}{
The dataset was first published by \cite{FNA06} and consists of the abundance of 30 fish species measured in 89 sites in the Barents See in April-May 1997. In addition to abundances, the water temperature was measured in each site. The complete dataset is available at \url{www.fbbva.es/microsite/multivariate-statistics/data.html}. 
Fishes distributions are known to be greatly linked with the temperature. 
Hence to illustrate our methodology, 
% \SR{models were fit\CA{}{ted} without any covariates and with one missing actor, which we denote $h$. $\rm{Cor}(k,temp)$ denotes the absolute Pearson's correlation between $M_k$ (estimated vector of means of node $k$ along all 89 sites) and the temperature.}{
we present the results of the model fitted without any covariate (that is not accounting for the temperature), but including one missing actor (as suggested by Figure \ref{fig:selec}). To assess the ability of the proposed methodology to retrieve the influence of temperature as a missing actor, we report the empirical correlation between the temperature and the conditional expectation of the missing actor $M_h$, which we denote $\corHTemp$.
 
% Runing time
The resampling initialization procedure yielded in 14 different cliques, for each of which a VEM algorithm was run: the mean running time was $6.63$mins with deviation $0.70$ mins. 

% Dependency structure
The edge probabilities involving node $h$ as an endpoint were either very close to 0 or very close to 1, yielding a total of 6 highly probable neighbors of $h$. Figure \ref{fig:barents_adj} shows that many direct interactions are inferred between the corresponding 6 species in absence of a missing actor, which vanish when it is introduced. It also shows that accounting for this actor has only a local effect and that the direct interactions among the other species are preserved, which is consistent with our notion of a missing actor.

\begin{figure}[H]
    \centering
    \includegraphics[width=10cm]{figs/Barents_mat_comp.png} \\\vspace{-2cm}
    \includegraphics[width=10cm]{figs/Barents_net_comp3.png}
    \caption{\textit{Top left:} adjacency matrix of the Barents Sea fishes interaction network for $r=0$  missing actor. The inferred neighbors are gathered in the last 6 columns, so that their interactions are observable in the upper-right corner. \textit{Top right:} adjacency matrix for $r=1$  missing actor. The last column gathers the interactions of the inferred missing actor. \textit{Bottom}: Inferred interaction network with $r=0$ (left) and $r=1$ (right). Colored nodes refer to the inferred neighbors (blue) of the missing actor (yellow). The edges width are proportional to their probability.}
    \label{fig:barents_adj}
\end{figure}

%\begin{figure}[H]
%    \centering
%    \includegraphics[width=10cm]{missing_article/Fig/Barents_net_comp3.png}
%    \caption{Barents Sea fishes interaction network with $r=0$ (left) and $r=1$ (right). Colored nodes refer to the inferred neighbors (blue) of the missing actor (yellow).}
%    \label{fig:barents_net}
%\end{figure}

% Interpretation
In terms of interpretation, Figure \ref{fig:barents_temp} shows that the missing actor is highly correlated with the temperature. It also appears that the abundances of the species neighbor to the missing actor are much more correlated with the temperature (mean correlation = 0.78, sd = .06) than the abundances of the non-neighbor species (mean correlation = 0.46, sd = .27). This example shows the ability of the method to recover an underlying effect that would not be recorded in the data.

\begin{figure}
    \centering
    \includegraphics[width=5cm]{figs/Barents_MH_temp_white.png}
    \caption{Missing actor estimated vector of means $M_h$ as a function of the temperature. $\corHTemp=0.85$.}
    \label{fig:barents_temp}
\end{figure}

% \SR{
% Table \ref{tab:barents} then gathers the mean of correlations to the temperature of neighbors and non-neighbors. It appears that neighbors to the missing actor are significantly more correlated to the temperature than other nodes are.
% \begin{table}[ht]
% \centering
% \begin{tabular}{rlrr}
%   \hline
%   $P_{hk}$ & $\corKTemp$  \\ 
%   \hline
%  $<0.5$  & 0.44 (0.25)\\ 
%   $\geq 0.5$ & 0.80 (0.06) \\ 
%   \hline
% \end{tabular}
% \caption{Mean and standard deviation of $\corKTemp$ in relation with node $k$ being inferred as a neighbor ($P_{hk}\geq 0.5$) of missing actor $h$ or not ($P_{hk}< 0.5$).}
% \label{tab:barents}
% \end{table}
% }{
%} 

%\begin{figure}
%    \centering
%    \includegraphics[width=9cm]{missing_article/Fig/Barents_mat_comp.png}
%    \caption{\textit{Left:} adjacency matrix of Barents Sea fishes interaction network for $r=0$. The inferred neighbors are gathered in the last 6 columns, so that their interactions are observable in the upper-right corner. \textit{Right:} adjacency matrix of Barents Sea fishes interaction network for $r=1$. The last column gathers the interactions of the inferred missing actor.}
%    \label{fig:my_label}
%\end{figure}

%\begin{figure}[H]
%    \centering
%    \includegraphics[width=10cm]{missing_article/Fig/Barents_net_comp3.png}
%    \caption{Barents Sea fishes interaction network with $r=0$ (left) and $r=1$ (right). Colored nodes refer to the inferred neighbors (blue) of the missing actor (yellow).}
%    \label{fig:my_label}
%\end{figure}

%%%%%%%%%%%%%%%%%%%%%%%%%%%%%%%%%%%%%%%%%%%%%%%%%%%%%%%%%%%%%%%%%%%%%%%%%%%%%%%%
\subsection{Fatala River}
% \SR{
% The R package \texttt{ade4}  provides with the baran95 dataset which gathers the abundances of 33 species of fish on 90 locations of the Fatala River in Guinea. Data sampled between June 1993 and February 1994, which we organize in dry and rainy seasons; dates and sites are available as dataset categorical covariates.
% %
% Again models are fit with no covariates, and here two missing actors are inferred, which we denote $h1$ and $h2$. As models include more than one missing actor, the best VEM is selected under the constraint $\log sd(M_h)\geq -20$ for $h\in \{h1, h2\}$. This constraint ensures that the corresponding marginal variance of each missing actor is not too high, which would mean that the algorithm did not learn a lot about them. In other words, this ensures the algorithm succeeds in finding the desired number of missing actors. \\
% %
% The bootstrap method gives 60 unique  lists of two possible initial cliques and the mean running time for a convergence with precision $1e-3$ is $11.33$ min with deviation $1.47$; 14 did not reach convergence and the algorithm stopped after 100 iterations. \\
% %
% The best fit gives vectors of means $Mh1$ and $Mh2$ corresponding two each missing actor. Figure \ref{fig:Fatala} shows the plot of $Mh1$ against $Mh2$ colored with either one of the available covariates. It is very interesting to see that $Mh1$ is linked to the Site covariate, as it clearly separates kilometer 3 from kilometer 46 which are the first and last locations. On the other hand $Mh2$ seems linked with the Season covariate, even if the separation is less obvious.
% }{
\cite{baran1995dynamique} collected the abundances of 33 fish species in 90 sites along the Fatala River in Guinea between June 1993 and February 1994. The data are available from the R package \texttt{ade4} on CRAN \citep{dray2007ade4}, along with the date and site of collection, from which we deduce the season (dry or rainy). Again the model was fitted without any covariates, but with two missing actors, as suggested by Figure \ref{fig:selec}. \\
%
The resampling initialization procedure yielded in 60 different cliques, for each of which a VEM algorithm was run: the mean running time was $11.33$ min (sd = $1.47$ mn). 14 VEM did not reach convergence (with tolerance $\varepsilon = 1e-3$) after 100 iterations. We filtered out the results obtained from the different initializations, when the algorithm obviously ended in a degenerate solution ($\Var(M_h) < \exp(-20)$). \\ 
%
% %\SR{
% Figure \ref{fig:Fatala} shows the scatterplot of estimated conditional mean of the two missing actors $(M_{h_1}, M_{h_2})$ in each site, colored with either one of the available covariates (site and season). $M_{h_1}$ is obviously linked to the site and separates most upstream locations (kilometer 3) from most downstream locations (kilometer 46). On the other hand $M_{h_2}$ seems linked with the Season covariate, even if the separation is less obvious. Similarly, the second missing actor shows a clear relation with the season. \\
% % 
% Again, the retrieved missing actor each correspond to an underlying effect that rules fish species abundances. The clear separation between the two effects is reinforced by the assumption the missing actors are independent from each other, which obviously holds for locations ans seasons. \\
% }{
Figure \ref{fig:Fatala} shows the scatterplot of the estimated conditional mean of the two missing actors $(M_{h_1}, M_{h_2})$ in each site, colored with either one of the available covariates (site and season). The missing actor $h_1$ is obviously linked to the site and separates most upstream locations (kilometer 3) from most downstream locations (kilometer 46). This actor has 11 highly probable neighbor species.
% \textcolor{red}{comparer les anova Poisson pour les voisins / pas voisins}
Again, this retrieved missing actor corresponds to an underlying effect (in this case: geography) that rules fish species abundances. \\
% 
The second missing actor seems to be linked with the season but with a less clear separation. Also the variability of $M_{h_2}$ is much smaller than this of $M_{h_1}$. This effect is therefore questionable, which brings us back to model selection. As mentioned above, we used a procedure based on cross-validation, which may  be prone to select too complex model \citep{shao1993linear,friedman2001elements,arlot2010survey}. The definition of a grounded model selection criterion for structure inference in presence of missing actors remains open.
%}

%\textcolor{red}{[C'est un peu dommage de finir comme ça mais bon...]}
 
\begin{figure}[h]
    \centering
    \includegraphics[width=12cm]{figs/Fatala_MH2.png}
    \caption{Estimated means $M_{h_1}$ and $M_{h_2}$ of the two inferred missing actors. Left column: scatterplots $M_{h_1}$ vs $M_{h_2}$ with site (top) and season (bottom) color code. Right: distribution of the estimated means across sites. Top right: distribution of $M_{h_1}$ in each location, bottom right: distribution of $M_{h_2}$ in each season.}
    \label{fig:Fatala}
\end{figure}



\begin{subappendices}
\section{Published supplements}
 \subsection{Algebraic Tools} \label{app:tools}
 We here present some algebraic results about spanning tree structures which are used during the computations. Theorem \ref{thm:MTT}, Lemma \ref{lem:Meila} as well as Lemma \ref{lem:Kirshner} use the notion of Laplacian matrix  $\Qbf$ of a symmetric matrix $\Wbf=[w_{jk} ]_{1\leq j,k\leq p}$, which is defined as follows :
 
\[
 [\Qbf]_{jk}  =\begin{cases}
    -w_{jk}  & 1\leq j<k \leq p\\
    \sum_{u=1}^p w_{ju} & 1\leq j=k \leq p.
    \end{cases}
\]
 
We further denote $\Wbf^{uv}$ the matrix $\Wbf$ deprived from its $u$th row and $v$th column and we remind that the $(u, v)$-minor of $\Wbf$ is the determinant of this deprived matrix, that is $|\Wbf^{uv}|$.
The following Theorem \ref{thm:MTT} is the extension of Kirchhoff's Theorem to the case of weighted graphs \citep{matrixtree,MeilaJaak}.\\
\begin{theorem}[Matrix Tree Theorem] \label{thm:MTT}
    For any symmetric weight matrix W with all positive entries, the sum over all spanning trees of the product of the weights of their edges is equal to any minor of its Laplacian. That is, for any $1 \leq u, v \leq p$,
   \[
    W := \sum_{T\in\mathcal{T}} \prod_{(j, k)\in T} w_{jk} = |\Qbf^{uv}|.
    \]\\
\end{theorem}    

In the following, without loss of generality, we will choose $\Qbf^{11}$. As an extension of this result, \cite{MeilaJaak} provide a close form expression for the derivative of $W$ with respect to each entry of $\Wbf$. 

\begin{lemma} [\cite{MeilaJaak}] \label{lem:Meila}
    Define the entries of the symmetric matrix $\Mbf$ as
 \[    
 [\Mbf]_{jk} =\begin{cases}
    \left[(\Qbf^{11})^{-1}\right]_{jj} + \left[(\Qbf^{11})^{-1}\right]_{kk} -2\left[(\Qbf^{11})^{-1}\right]_{jk} & 1< j<k \leq p\\
    \left[(\Qbf^{11})^{-1}\right]_{jj} & k=1, 1< j \leq p  \\
    0 &  j=k .
    \end{cases}
\]
it then holds that $$\partial_{w_{jk}} W = [\Mbf]_{jk}  \times W.$$\\
\end{lemma}

\cite{kirshner} build on Lemma \ref{lem:Meila} to provide an efficient computation of all edges probabilities.
\begin{lemma} [\cite{kirshner}] \label{lem:Kirshner}
    Let $p_W$ be a distribution on the space of spanning trees, such that $p_W(T)=\prod_{kl\in T} w_{kl} / W$, where $W$ is defined as in Theorem \ref{thm:MTT}. Taking the symmetric matrix $\Mbf$ as defined in Lemma  \ref{lem:Meila}, the probability for an edge $kl$ to be in the tree $T^*$ writes:
 
$$\mathds{P}\{kl\in T^*\} = \sum_{T\in \mathcal{T}} p_W(T)= w_{kl}\: \Mbf_{kl}$$
\end{lemma}



\tocless\subsection{Computations} \label{app:comput}
\subsubsection[Update of tree parameter vector]{Update of $\betabf$.} \label{up:beta}
As in \citet{MRA20}, the update of $\betabf$ is such that:
$$\betabf^{t+1}  = \arg\max_\betabf \; \Esp_{g^t} \left[ \log p_\betabf(T) \right].
$$
By definition of $p_\betabf(T)$:
$$\Esp_{g^t} \left[ \log p_\betabf(T) \right] = \sum_{kl} P^t_{kl} \log \beta_{kl} - \log B\;,
\qquad
B=\sum_{T\in \mathcal{T}}\prod_{kl\in T} \beta_{kl}.$$
Computing the derivative with respect to the edge weight $\beta_{kl}$ gives:
\begin{align*}
\partial_{\beta_{kl}}\Esp_{g^t} \left[ \log p_\betabf(T) \right] &=\frac{P_{kl}^t}{\beta_{kl}} - \frac{\partial_{\beta_{kl}} B^t }{B^t} 
\end{align*}
According to Lemma \ref{lem:Meila}: $\partial_{\beta_{kl}} B^t  = [\boldsymbol{M}]_{kl} \times B$. Finally setting the derivative to 0 yields the update formula $
\beta^{t+1}_{kl} 
= \frac{P^t_{kl}}{ M(\betabf^t)_{kl}}$.

\subsubsection[Update of Gaussian tree precision matrix]{Update of $\Omega_T$} \label{up:omega}
The update of $\Omegabf_T$ respects
$$\Omegabf^{t+1}  = \arg\max_\Omegabf \; \Esp_{q^t} \left[ \log p_{\Omegabf}(\Ubf \mid T) \right].$$
This is a problem of parameter optimisation in the context of Gaussian Graphical Models (GGM).
In what follows, for any $q\times q$  matrix $A$, $A_{[kl]}$ will refer to the bloc $kl$ of $A$: $A_{[kl]}=(a_{ij})_{\{i,j\}\in\{k,l\}}$.   $[A_{[kl]}]^q$ will then denote the matrix obtained by filling up with zero entries to obtain full dimension $q\times q$, so that:
$$([A_{[kl]}]^q )_{ij}=\left\{ \begin{array}{rl}
a_{ij} & \text{if } \{i,j\}\in\{k,l\}\\
0 &  \text{if } \{i,j\}\in\{1,..., q\}_{\setminus kl}
\end{array}\right.$$
In its proposition 5.9, \citet{Lau96} states that in a  GGM with $p$ variables and associated with the decomposable graph $\mathcal{G}$, the maximum likelihood of the precision matrix exists if and only if $n > \max_{C\in \mathcal{C}} |C|$. It is then given as 
$$\widehat{\Omega}=n\left(\sum_{C\in \mathcal{C}} [SSD_{[C]}\,^{-1}]^p - \sum_{S\in \mathcal{S}} \nu(S)\,[SSD_{[S]}\,^{-1}]^p \right)$$
where $\mathcal{C}$ is the set of cliques and $\mathcal{S}$ the set of separators of $\mathcal{G}$, with associated multiplicities $\nu(S)$.\\


In our context, $\mathcal{G}$ is a spanning tree and so all cliques are edges and separators are nodes. The multiplicity of a given node $k$ as a separator in the graph is  $\nu(k) = d(k)-1$, where $d(k)$ is its degree. Therefore the estimator  $\widehat{\Omega}_T$  writes as the following 
\begin{align*}
\widehat{\Omega}_T &= n  \sum_{kl\in T}   [(SSD_{[kl]})^{-1}]^{p+r} - n\sum_k (d(k)-1)[(SSD_{kk})^{-1}]^{p+r}\\
&=n \sum_{kl\in T}  [(SSD_{[kl]})^{-1} - (SSD_{kk})^{-1} -  (SSD_{ll})^{-1} ]^{p+r} + n\sum_k[(SSD_{kk})^{-1}]^{p+r}
\end{align*}
As $SSD$ has diagonal $n$, the expression simplifies. Denoting $I_d$ the identity matrix of dimension $d$ we obtain:
$$\widehat{\Omega}_T =n\sum_{kl\in T} [(SSD_{[kl]})^{-1} -\frac{1}{n} I_2]^{p+r}+ I_{p+r}.$$

Detailing each bloc matrices as follows gives the update formulas in (\ref{omegaT}):
\[
n\times [(SSD_{[kl]})^{-1} - \frac{1}{n}I_2] = \frac{1}{1-(ssd_{kl}/n)^2}
\left(\begin{array}{cc}
		(ssd_{kl}/n)^2   & -ssd_{kl}/n\\
		-ssd_{kl}/n& (ssd_{kl}/n)^2 
		\end{array}\right)
\]


\subsubsection[for the toc]{Determinant of $\Omegabf_T$.}
The determinant of a precision matrix of a GGM with a decomposable graph is expressed as follows \citep{Lau96}:
$$ |\Omega| =\dfrac{\prod_{C\in \mathcal{C}} |\Sigma_C|^{-1}}{\prod_{S\in \mathcal{S}} |\Sigma_S|^{-\nu(S)}},$$
where $\Sigma = \Omega^{-1}$. As $\Omegabf_T$ is tree-structured, its determinant factorizes on the edges of $T$. It is expressed with the correlation matrix $\Rbf_T$ as follows, denoting $d(k)$ the degree of node $k$:
\begin{align*}
|{\Omegabf}_T| &=\frac{\prod_{kl \in T} |{\Rbf}_{Tkl}|^{-1}}{\prod_k |{\Rbf}_{Tkk}|^{1-d(k)}} 
 \end{align*}
Using that $\Rbf_T$ has diagonal 1, we obtain for step $t+1$ of the algorithm:
$$|\Omegabf^{t+1}_{T}| = \Big(\prod_{kl \in T} |\Rbf_{T[kl]}^{t+1}|\Big)^{-1}.$$


\subsubsection{Numerical issues.} \label{app:numIssues}

\paragraph{Exact computations} Our algorithm requires the computation of determinants (from the Matrix Tree Theorem) and inverses (in Kirshner's formula) of Laplacian of weight matrices. As we deal with highly variable weights, numerical issues arise: infinite determinants or matrix numerically non-invertible due to either the maximal machine precision (about $1.7\cdot 10^{308}$), or with machine zero (about $2.2 \cdot 10^{-16}$). To enhance the precision of such computations, we rely on multiple-precision arithmetic which allows the digit of precision of numbers to be  limited only by the available memory instead of 64 bits. We implemented matrix inversion and log-determinant computation using both, symbolic computation and multiple precision arithmetic, relying on the \texttt{gmp} R package available on CRAN, which uses \citep{lucas2020package}, the C library GMP (GNU Multiple Precision Arithmetic). 

\paragraph{Tempering parameter $\alpha$} \label{alpha}
\begin{description}
\item[definition]Weights $\widetilde{\beta}$ are mechanically linked to the quantity of data available $n$. To avoid reaching maximal precision when computing the determinant, a tempering parameter $\alpha$ is applied to every quantity proportional to $n$, so that the actual update performed is $$\log \widetilde{\beta}_{kl} = \log \beta_{kl} - \alpha(\frac{n}{2}\log|\widehat{\Rbf}_{Tkl}| + \widehat{\omega}_{Tkl} [M^\intercal M]_{kl}).$$
\item[Heuristic for an upper bound] The proposed algorithm requires the computation of the normalizing constant $\widetilde{B}$, which is the determinant of any minor of the Laplacian  of the $q\times q$ variational weights matrix $\betabft$. As these weights  mechanically increase with the quantity of available data $n$, this step is numerically very sensitive.  Hereafter we denote $|\Qbf^{uv}|$ this determinant and $\Delta$ the maximal machine precision. In order to ease the computations, we define the tempering parameter $\alpha$ as $$\log \widetilde{\beta}_{kl} = \log \beta_{kl} - \alpha(\frac{n}{2}\log|\widehat{\Rbf}_{Tkl}| + \widehat{\omega}_{Tkl} [M^\intercal M]_{kl})\;,\qquad \text{under constraint}\;\;\; |\Qbf^{uv}| \leq \Delta.$$

Let's first detail the expression for $\widetilde{\beta}_{kl}$. Following the definition of the $SSD$ matrix, and update formulas \eqref{omegaT} and \eqref{RT}, we obtain:
\begin{align*}
    \log \widetilde{\beta}_{kl} &=\log \beta_{kl} +\alpha \,n\left\{\frac{(ssd_{kl}/n)^2}{1-(ssd_{kl}/n)^2} -\frac12\log\big[1-(ssd_{kl}/n)^2\big]\right\}
\end{align*}
For large $n$, we thus have $$\widetilde{\beta}_{kl}\approx \exp \big[\alpha n \cdot C(ssd_{kl}/n)\big], \qquad \text{with }\; C(x)=x/(1-x) -\log(\sqrt{1-x}),\; x\in [0,1[.$$ 
We then define $C_{sup}$ such that $C_{sup} = C(ssd_{max})$, with $ ssd_{max}=\max\{ssd_{kl}, k\neq l\}$.
By definition, $\Qbf^{uv}$ is positive-definite, so its determinant is upper bounded by the product of its diagonal terms (Hadamard's inequality). Namely:
\begin{align*}
    |\Qbf^{uv}|&\leq \prod_{i=1}^{q-1} \Qbf^{uv}_{ii} \leq \prod_{i=1}^{q-1}\sum_{i=1}^{q-1} \exp (\alpha C_{sup} n)\\
    &\leq \left[(q-1)\exp(\alpha C_{sup} n)\right]^{q-1}
\end{align*}
Then applying the constraint yields:
\begin{align*}
    |\Qbf^{uv}| \leq \Delta \iff  \alpha \leq \frac{1}{C_{sup} n} \left[ \frac{1}{q-1}\log \Delta - \log(q-1)\right] 
\end{align*}

For $C_{sup}=0.8$, $n=200$ and $q=15$, we get $\alpha \leq 1.05\cdot 10^{-1}$.
\end{description}

%A heuristic for an upper bound of $\alpha$ is given in appendix \ref{alpha}.
%We provide a heuristic to set the parameter $\alpha$.


\tocless\subsection{Model selection and cross-validation} \label{sec:modSel}

%%%%%%%%%%%%%%%%%%%%%%%%%%%%%%%%%%%%%%%%%%%%%%%%%%%%%%%%%%%%%%%%%%%%%%%%%%%%%%%%%%%%%%%%
\subsubsection{Sampling spanning trees} \label{eq:sampTree}
%%%%%%%%%%%%%%%%%%%%%%%%%%%%%%%%%%%%%%%%%%%%%%%%%%%%%%%%%%%%%%%%%%%%%%%%%%%%%%%%%%%%%%%%%%%
Sampling non-uniform spanning trees (i.e. sampling $T$ from $p_\betabf$) is a research topic by itself, especially for large networks \citep[see][for a review]{DKP17}. For moderate size networks, a rejection algorithm \citep{Dev86} can be defined in the following way:
\begin{enumerate}
\item Sample $T$ from a distribution $q$, such that there exist a constant $M$, that ensures that, for all $T$, $M q(T) > p_\betabf(T)$;
\item Keep $T$ with probability $M^{-1} p_\betabf(T) / q(T)$ or try step 1 again.
\end{enumerate}
The efficiency of such an algorithm strongly relies on the choice of the proposal distribution. Here we adopt the following proposal:
\begin{enumerate}[label=\roman*]
\item Sample a connected graph $G$ with independent edges, each drawn with probability $Q_{jk} \propto P_{jk} = \Pr_\betabf\{ jk \in T\}$; 
\item Sample $T$ uniformly among the spanning trees of $G$.
\end{enumerate}
%
\paragraph{Evaluation of the proposal.}
To evaluate the proposal distribution for each sampled tree, we may observe that, the probability for a graph drawn from the proposal to contain a given tree $T$ is approximately
$$
{\Pr}_q\{G \ni T\} \approx \prod_{jk \in T} Q_{jk},
$$
the approximation being due to the connectivity constraint. This constraint can be almost surely satisfied by taking $Q_{jk}$'s large enough. So, denoting $|\Tcal(G)|$ the number of spanning trees in $G$, we have that
\begin{align*}
q(T) 
= \sum_{G \ni T} q(T \mid G) q(G)  = \sum_{G \ni T} \frac{q(G)}{|\Tcal(G)|} 
= {\Pr}_q\{G \ni T\} \; \Esp\left(|\Tcal(G)|^{-1} \mid G \ni T \right).
\end{align*}
The last expectation can be evaluated via Monte-Carlo, by sampling a series of graphs $G$ according to the proposal $q$ but forcing all edges from $T$ to be part of $G$. 
%
\paragraph{Upper bounding constant $M$.}
To evaluate the upper bounding constant $M$, we may observe that finding the tree $T^*$ such that
$$
m_\betabf 
:= \frac{{\Pr}_q\{G \ni T^*\}}{p_\betabf(T^*)}
= \min_{T \in \Tcal} \frac{{\Pr}_q\{G \ni T\}}{p_\betabf(T)} = \min_{T \in \Tcal} \prod_{jk \in T} \frac{Q_{jk}}{\beta_{jk}}
$$
is a minimum spanning tree problem. Then, obviously, for any tree $T$: ${\Pr}_q\{G \ni T\} \geq m_\betabf p_\betabf(T)$.
Now, because the maximum number of spanning trees within a graph is $p^{p-2}$, we have
$$
M q(T)
= M \sum_{G \ni T} \frac{q(G)}{|\Tcal(G)|} 
\geq \frac{M}{p^{p-2}} \sum_{G \ni T} q(G)
= \frac{M}{p^{p-2}} {\Pr}_q\{G \ni T\}
\geq M \frac{m_\betabf}{p^{p-2}}  p_\betabf(T)
$$
So we may set $M = p^{p-2} / m_\betabf$. Still, in practice, this bounds turns out to be far too large and needs to be tuned down to preserve computational efficiency.

%%%%%%%%%%%%%%%%%%%%%%%%%%%%%%%%%%%%%%%%%%%%%%%%%%%%%%%%%%%%%%%%%%%%%%%%%%%%%%%%%%%%%%%%%%%
 
\subsubsection{Cross-validation for model selection} \label{eq:cvAlgo}
\label{CV}

The cross-validation procedure to estimate the pairwise composite likelihood is given in Algorithm \ref{algo:model-selection}. In practice $V=10$ and $B = 100$.


\begin{algorithm}%[H]
\caption{Cross-validation for model selection with $r$ missing actors}
\label{algo:model-selection}
%  \dontprintsemicolon
  \CommentSty{// 0. INITIALIZATION}\; 
  Divide the dataset $\Ybf$ into $V$ subset $\Ybf^1, \dots \Ybf^V$;
  \BlankLine
  \For{$v \in \{1,\cdots, V\}$}{
    \BlankLine
    \CommentSty{// 1.   Apply the VEM algorithm to the train dataset $\Ybf^{-v}$}\; 
    $\Gammabf_r^{-v} \leftarrow (\thetabf_r^{-v}, \sigmabf_r^{-v}, \betabf^{-v}_r, \Omegabf_r^{-v})$
    \CommentSty{// 2. MONTE CARLO APPROXIMATION OF COMPLETE LOG-LIKELIHOOD EXPECTATION}\;
    \For{$b \in \{1,\cdots, B\}$}{
    \CommentSty{// 2.1 Draw tree (see Section \ref{eq:sampTree})}\; 
     $ T_{r, b}^{-v}  \sim p_{\betabf^{-v}_r}$ 
    \BlankLine
     \CommentSty{// 2.2. Build  the precision matrix having non-nul entries determined by $ T_{r, b}^{-v} $ and values stored in $\Omegabf_r^{-v}$, and its diagonal terms according to \eqref{omegaT}}\;
     $\Omegabf_{T^b}\leftarrow f( T_{r, b}^{-v} , \Omegabf_r^{-v} )$
     \BlankLine
      \CommentSty{// 2.3. Compute the marginal variance matrix }\;
      $\Sigmabf_{T^bO} \leftarrow  \Omegabf_{T^bOO} - \Omegabf_{T^bOH} \Omegabf_{T^bHH}^{-1} \Omegabf_{T^bHO}$;
          \BlankLine 
     \CommentSty{// 2.4. Compute the bivariate Poisson log-normal density in test sites}\;
     \For{site $i \in v$}{
        \For{pairs of species $(j,k)$}{
        $p_{PLN}\left((Y^v_{ij}, Y^v_{ik}); \Gammabf_r^{-v}, T_{r, b}^{-v} \right)$ with means $\xbf_i^\intercal \thetabf_{r, j}^{-v}$ and $\xbf_i^\intercal \thetabf_{r, k}^{-v}$ and variance matrix $[\Sigmabf_{T^bO}]_{[jk, jk]}$
%        $\log p_{PLN}[(Y_{ij}^v, Y_{ik}^v) | T^b; \widehat{\theta},\widehat{\Sigma}_{T^b jk}]$, 
         }}
       \CommentSty{// 2.5.  Compute the average}\;
        $$
        PCL_{rvb}(\Ybf^v, \Gammabf_r^{-v}, T^b) = \frac1{m_v} \sum_{i = 1}^{m_v} \sum_{j < k} \log p_{PLN}\left((Y^v_{ij}, Y^v_{ik}); \Gammabf_r^{-v}, T_{r, b}^{-v} \right)
        $$
%        $f_{PLN}(\Ybf; b,v)=\sum_{\substack{i \in v\\ j < k}} \log p_{PLN}[(Y_{ij}^v, Y_{ik}^v) | T^b; \widehat{\theta},\widehat{\Sigma}_{T^b jk}]$
    }
    \BlankLine        
  }  
   \CommentSty{// 3. AVERAGE OVER SUBSETS}\; 
$$
PCL_r(\Ybf) = \frac1V \sum_v PCL_{rv}(\Ybf^v, \Gammabf_r^{-v}) .
$$
  \BlankLine
\end{algorithm}

  %%%%%%%%%%%%%%%%%%%%%%%%%%%%%%%%%%%%%%%%%%%%%%%%%%%%%%%%%%%%%%%%%%%%%%%%%%%%%%%%%%%%%%%%%%%
%The cross-validation procedure to estimate the pairwise composite likelihood is as follows:
%\begin{enumerate}
%\item Divide the dataset $\Ybf$ into $V$ subset $\Ybf^1, \dots \Ybf^V$;
%\item For each subset $\Ybf^v$, do:
%\begin{enumerate}
%    \item Apply the VEM algorithm to the train dataset $\Ybf^{-v}$ to get the estimates $\Gammabf_r^{-v} = (\thetabf_r^{-v}, \sigmabf_r^{-v}, \betabf^{-v}_r, \Omegabf_r^{-v})$
%    \item Repeat $B$ times:
%        \begin{enumerate}
%        \item Draw tree $T^b \sim p_{\betabf^{-v}_r}$ using the procedure described in Section \ref{eq:sampTree};
%        \item Build $\Omegabf_{T^b}$ as the precision matrix having non-nul entries determined by $T^b$ and values stored in $\Omegabf_r^{-v}$, and its diagonal terms according to \eqref{omegaT};
%        \item Compute the marginal variance matrix $\Sigmabf_{T^bO} =  \Omegabf_{T^bOO} - \Omegabf_{T^bOH} \Omegabf_{T^bHH}^{-1} \Omegabf_{T^bHO}$;
%        \item For each site of test data $\Ybf^{v}$ and each pair of observed species, compute the bivariate Poisson log-normal density  
%        $p_{PLN}\left((Y^v_{ij}, Y^v_{ik}); \Gammabf_r^{-v}, T_{r, b}^{-v} \right)$ with means $\xbf_i^\intercal \thetabf_{r, j}^{-v}$ and $\xbf_i^\intercal \thetabf_{r, k}^{-v}$ and variance matrix $[\Sigmabf_{T^bO}]_{[jk, jk]}$
%        $\log p_{PLN}[(Y_{ij}^v, Y_{ik}^v) | T^b; \widehat{\theta},\widehat{\Sigma}_{T^b jk}]$, 
%        and compute the mean
%        $$
%        PCL_{rvb}(\Ybf^v, \Gammabf_r^{-v}, T^b) = \frac1{m_v} \sum_{i = 1}^{m_v} \sum_{j < k} \log p_{PLN}\left((Y^v_{ij}, Y^v_{ik}); \Gammabf_r^{-v}, T_{r, b}^{-v} \right)
%        $$
%%        $f_{PLN}(\Ybf; b,v)=\sum_{\substack{i \in v\\ j < k}} \log p_{PLN}[(Y_{ij}^v, Y_{ik}^v) | T^b; \widehat{\theta},\widehat{\Sigma}_{T^b jk}]$
%        \end{enumerate}
%    \item Average over the trees
%    $$
%    PCL_{rv}(\Ybf^v, \Gammabf_r^{-v}) = \frac1B \sum_{b=1}^B PCL_{rvb}(\Ybf^v, \Gammabf_r^{-v}, T^b) 
%    $$
%    \end{enumerate} 
%\item Average over the subsets
%$$
%PCL_r(\Ybf) = \frac1V \sum_v PCL_{rv}(\Ybf^v, \Gammabf_r^{-v}) .
%$$
%    \item Compute the average criteria defined as: $$\displaystyle PCL(\Ybf)=\frac1V\sum_{v=1}^{V}\frac1B \sum_{b=1}^Bf_{PLN}(\Ybf; b,v)$$
%\end{enumerate}


  

\section{Clique initialization}
\section{Network inference from the PLN model}

\begin{figure}
\centering
\includegraphics[width=12cm]{figs/AUC_PLN_EM_VEM.png}
\end{figure}
\begin{figure}
\centering
\includegraphics[width=12cm]{figs/precrec_PLN_EM_VEM.png}
\end{figure}

with $min.ratio=1\cdot 10^{-3}$ for PLNnetwork
\section{Vignette for nestor}
\end{subappendices}

%\clearemptydoublepage
%\addtocontents{toc}{\protect\newpage}
\chapter{Perspectives}
\paragraph{ideas}

 
\section{Unresolved details of the presented approach}
\begin{itemize}
\item model selection: not working and why (pb théorique stat selection dans modele inference variationnel)
\item offsets: very important and not so obvious in some cases (pb modélisation, pas forcément facile à avoir)
\item un mot sur la clique, pb algo
\end{itemize} 

\section{natural extensions}
state the model clearly, as it holds for the whole section
\subsection{Other free results}
estimation de la vraie matrice omega à partir de sigma et de la structure inférée, transformée en graph chordal si besoin (sinon pas décomposable et pas lauritzen)
contribution d'une arête
\subsection{Network comparison}
comparing networks = comparing laws on trees, avec divergence KL symmétrisée par ex
(si temps pour exemple, MDS 2 premiers axes)

\subsection{Emission law}
 loi d'émission différente : données mixtes (popovic), données multidimensionnelles sur les noeuds (au lieu d'univarié) article martina, extension pour les comptages multiatribute gaussian gm
 
la mécanique d'inférence de réseau reste la même, parce qu'on conserve la couche Gaussienne. En revanche boulot pour estimation des paramètres de cette loi d'émission.

\subsection{Spatial dependence}
how to not model the spatial dependency:
\begin{itemize}
\item covariates. Because close sites are similar (environmental similarity)
\item missing actor
\end{itemize}
if it's an absolute necessity, modèle kronecker en regardant les sites 2 à 2 pour chaque espèces et les espèces 2 à 2 pour chaque site. Vraisemblance composite.
ovaskainen hmsc fait le job, à vérifier

\section{Network inference in the observed layer}
Sortie en sentiers non-gaussiens
\begin{itemize}
\item comment simuler des comptages avec structure de dépendance
\item par paires avec arbre maximal chow \& liu
\item factorization par paire avec n'importe quelle loi de comptage bivariée. 
\item Moyenne d'arbre possible et vraisemblance composite pour évaluer les modèles
\end{itemize}



%\selectlanguage{french}
\clearemptydoublepage
\DeactivateBG 
\chapter{Résumé}
\textit{This chapter is a summary of the present work, written in French.}\\

Les réseaux sont des objets qui permettent de graphiquement représenter des liens entre des entités.  Ces outils sont utilisés dans des domaines très variés, allant de l'informatique aux neurosciences, aux sciences sociales et à la biologie. Ce travail s'intéresse aux réseaux d'espèces en écologie et microbiologie (ou méta-génomique), et plus particulièrement à l'inférence des réseaux de dépendances conditionnelles entre espèces d'une même communauté partageant le même environnement. Les réseaux représentent alors les espèces par des noeuds, et leurs liens de dépendances par des arêtes.

 L'inférence de réseau considérée ici a pour point de départ des mesures répétées de comptages des espèces d'intérêt, soit un tableau de données de $n$ mesures discrètes de $p$ espèces. Il est important de distinguer ce cas de figure où le réseau n'est pas connu ni observé, de celui où il est possible d'observer les liens du réseau directement, et donc de reconstruire ce dernier via des comptages d'interactions observées, comme c'est classiquement le cas en écologie par exemple. Considérer les relations de dépendances conditionnelles permet à la fois d'obtenir des réseaux parcimonieux et interprétables en ne représentant que des liens directs entre espèces, et d'inférer des liens qui ne sont pas directement observables comme par exemple les dépendances entre des (pseudo-)espèces microscopiques (bactéries, champignons, protéines, virus, gènes, etc.) ou des liens dont la nature les rend difficilement identifiables (par exemple les relations de coopération ou de communication). \\

Les modèles graphiques sont le cadre mathématique des réseaux de dépendances conditionnelles.  Notamment, les modèles graphique gaussiens possèdent des propriétés particulières facilitant leur inférence. Ce cas particulier est un cadre très utilisé pour l'inférence de réseaux en biologie, cependant il n'est pas directement applicable à des données discrètes.  Le premier objectif de ce travail est ainsi de développer une méthodologie pour l'inférence de réseaux de dépendances conditionnelles à partir de données de comptages. Par ailleurs, pour assurer la validité des résultats il est nécessaire que les covariables expérimentales ainsi que les offsets mesurés (durées d'observation, profondeur de séquençage) soient inclus dans la modélisation des comptages. L'inférence de réseau en elle-même tire parti des propriétés algébriques des arbres couvrants pour réaliser une exploration efficace et exhaustive de l'espace des graphes réduit à celui de ces structures particulières.


Il est en outre possible que toutes les espèces ou covariables n'aient pas été mesurées lors de l'expérience. Un réseau inféré à partir de données incomplètes est alors un réseau marginal, qui présente mécaniquement des formations denses entre les espèces liées à un acteur non observé menant à des interprétations biaisées et partielles. Le second objectif de ce travail est  d'inclure de possibles acteurs manquants dans l'inférence, afin d'obtenir à la fois le réseau complet et des informations permettant de caractériser et de mieux comprendre les acteurs manquants. 


\section*{Chapitre 1}
Le premier chapitre  expose en détails les éléments de théorie invoqués pour la modélisation et l'inférence développées dans les chapitres suivants. Le cadre des modèles graphiques est tout d'abord abordé de manière générale et inspirée de \citet{Lau96}, avant d'introduire les arbres couvrants et leurs propriétés algébriques. Dans un second temps, les particularités des modèles graphiques gaussiens sont présentées, anisi que deux méthodes pour leur inférence. Après un bref exposé de l'inférence dans le cadre de données incomplètes, la dernière partie traite de l'inférence de réseaux à partir de données de comptage et présente plusieurs stratégies issues de la littérature en écologie des communautés et microbiologie.

\subsection*{Modèles graphiques}
\subsubsection*{Définitions générales.}
Un graphe $\G$ est constitué d'un ensemble de noeuds $V$ et d'un ensemble d'arêtes $E$. Pour un triplet $(A, B, S)$ de sous-ensembles disjoints de $V$, l'ensemble $S$ sépare $A$ et $B$ dans $\G$ si tout chemin allant de $A$ à $B$ intersecte $S$. La notion de séparation est essentielle  pour établir le lien entre graphe et relations d'indépendances conditionnelles au sein d'une variable aléatoire multi-variée. 

Soit $\G$ un graphe non-dirigé et $X=(X_v)_{v\in V}$ un vecteur aléatoire à valeurs dans un espace produit $\mathcal{X}=\otimes_{v\in V} \mathcal{X}_v$. Pour tout sous-ensemble $A$ de $V$, $X_A$ dénote $(X_v)_{v\in A}$.


\textbf{Propriété} (Markov globale). \textit{Une mesure de probabilité sur $\mathcal{X}$ satisfait la propriété de Markov globale  par rapport à $\G$ si pour tout triplet $(A, B, S)$ de sous-ensembles disjoints de $V$, on a
$$\text{S sépare A et B } \Rightarrow X_A\independent X_B \mid X_S.$$ }
Un modèle graphique pour $X$ est alors tout graphe tel que la distribution de $X$ soit globale Markov par rapport à ce graphe. La propriété de Markov globale est seulement une implication, ce qui signifie qu'il est autorisé qu'un modèle graphique n'inclue pas toutes les relations d'indépendances conditionnelles. Lorsque que la relation est une équivalence, la distribution de $X$ est dite fidèle Markov, et le graphe associé représente exactement toutes les relations d'indépendances conditionnelles dans $X$.


Si $S$ sépare $A$ et $B$ et si de plus tous les noeuds dans $S$ sont liés enter eux ($S$ est complet), le triplet $(A,B,S)$ est alors une décomposition propre de $\G$. Cette notion permet de définir un graphe décomposable.


\textbf{Définition} (Graphe décomposable). \textit{Un graphe $\G$ est dit décomposable s'il existe une décomposition propre $(A, B, C)$ sur $\G$, et si les sous-graphes définis par $A\cup B$ et $B\cup C$ sont eux-mêmes décomposables.}


Les modèles graphiques à graphes décomposables permettent de structurer l'écriture des paramètres de variance sur des ensembles de variables complets, ce qui facilite leur estimation. 

\subsubsection*{Arbres couvrants.}
Parmi l'ensemble des graphes, les arbres couvrants sont les structures les plus parcimonieuses, et les structures sans cycles les plus denses. Naturellement décomposables, ils sont pratiques à manipuler grâce à leurs propriétés algébriques. L'espace des graphes non-dirigés est de taille super-exponentielle ($2^{p(p-1)/2}$ graphes possibles pour $p$ noeuds). L'espace des arbres couvrants, bien que plus petit, reste combinatoirement grand ($p^{p-2}$ arbres pour $p$ noeuds). Il est cependant possible de sommer sur cet espace en $\mathcal{O}(p^3)$ opérations. Ce théorème est l'extension aux réels du théorème de Kirchhoff  \citep{matrixtree, MeilaJaak}, connu sous le nom de théorème arbre-matrice ou des mineurs égaux. \\


\textbf{Théorème} (Arbre-matrice). \textit{Soit une matrice de poids symétrique $\Wbf=(w_{jk})_{jk}$ dont les entrées sont dans $\mathds{R}^+$. Alors la somme sur l'ensemble des arbres couvrants du produit des poids de leurs arêtes est égal à n'importe quel mineur du Laplacien $\Qbf$ de $\Wbf$. Formellement, pour tout $1 \leq u, v \leq p$ :
   \[  W := \sum_{T\in\mathcal{T}} \prod_{jk\in T} w_{jk} = |\Qbf^{uv}|.  \]
   }
   
Ce théorème fait apparaître une forme somme-produit sur l'espace des arbres couvrants. Le lemme technique qui suit, établi par \citet{MeilaJaak}, permet de dériver une telle somme. On rappelle que le Laplacien d'une matrice est la matrice dont les termes diagonaux (resp. extra-diagonaux) sont les sommes par lignes (resp. l'opposé des termes extra-diagonaux) de la matrice de départ.\\

\textbf{Lemme} (Dérivation d'une somme-produit sur les arbres). \textit{Soit une matrice de poids symétrique $\Wbf=(w_{jk})_{jk}$ dont les entrées sont dans $\mathds{R}^+$. $\Qbf^{11}$ dénote le mineur [1,1] de son Laplacien. Soit alors la matrice $\Mbf$, définie termes à termes comme suit :
\[    
 [\Mbf(\Wbf)]_{jk} =\begin{cases}
    \left[(\Qbf^{11})^{-1}\right]_{jj} + \left[(\Qbf^{11})^{-1}\right]_{kk} -2\left[(\Qbf^{11})^{-1}\right]_{jk} & 1< j<k \leq p\\
    \left[(\Qbf^{11})^{-1}\right]_{jj} & k=1, 1< j \leq p  \\
    0 &  j=k .
    \end{cases}
\] 
Alors la dérivée partielle de $W = \sum_{T\in\mathcal{T}} \prod_{jk\in T} w_{jk}$ vaut :
$$\partial_{w_{jk}} W = [\Mbf]_{jk}  \times W.$$ }
   
Ces deux résultats rendent possible l'utilisation des arbres couvrants au sein de procédures d'inférence.
 %pas parlé de factorization ni de hammersley
\subsection*{Cas gaussien}
\subsubsection*{Un cadre populaire.}
Les modèles graphiques sont particulièrement utilisés dans le cas gaussien, notamment pour leur propriété de fidélité.\\

\textbf{Propriété} (Fidélité). \textit{Soit $X\sim \Ncal(\mu, \Sigmab)$ une variable aléatoire gaussienne multi-variée. Alors sa distribution est fidèle Markov au graphe $\G$ dont les arêtes sont les termes non nuls de sa matrice de précision $\Omegab = \Sigmab^{-1}$.\\}
  
  Ainsi dans le cas gaussien le graphe représentant exactement les relations d'indépendances conditionnelles est facilement défini. Dans le cadre des modèles graphiques gaussiens, \citet{Lau96} donne l'estimateurs du maximum de vraisemblance des termes de $\Sigmab$ correspondant aux arêtes du graphe. Si le graphe est de plus décomposable il existe un estimateur du maximum de vraisemblance pour $\Omegab$, et une estimation simplifiée de son déterminant en découle. Ces estimateurs ont une forme complexe, mais sont manipulables si la structure de graphe considérée est simple.\\
  
  \subsubsection*{Inférence pénalisée.}
  L'inférence d'un modèle graphique gaussien revient à identifier les éléments non-nuls de sa matrice de précision $\Omegab$. Classiquement, une estimation par régularisation pénalisée de $\Omegab$ est réalisée, connue sous le nom de \textit{graphical lasso} \citep{FHT08}. Cette approche maximise la log-vraisemblance pénalisée par la norme $\ell_1$ de $\Omegab$ :
 $$\argmax_{\Omegab\geq 0} \big\{\log |\Omegab| +\tr{\Yb^\intercal \Yb \Omegab} - \lambda ||\Omegab||_1\big\}, \qquad ||\Omegab||_1 = \sum_{j\neq k} |\omega_{jk}|.$$
  Cette stratégie vise une estimation parcimonieuse en introduisant des termes nuls dans $\Omegab$. Très utilisée, elle nécessite cependant une attention particulière pour la sélection du niveau de parcimonie, c'est à dire du paramètre de régularisation $\lambda$.\\
  
    \subsubsection*{Inférence par arbres.}
  Ce travail utilise une stratégie à base d'arbres couvrants. Une exploration de l'espace des arbres couvrants suppose que le graphe sous-jacent est un arbre aléatoire $T$. Ainsi contrairement à l'inférence pénalisée, le graphe est ici une variable latente du modèle. En définissant une distribution de probabilité sur cet espace, il est possible de définir un mélange d'arbre \citep{MixtTrees} pour des variables gaussiennes.
   
\textbf{Définition} (Mélange d'arbres gaussien). \textit{La distribution d'une variable aléatoire $\Ybf$ est un mélange d'arbres gaussien centré sur l'espace des arbres couvrants si elle s'écrit :
$$p(\Ybf) = \sum_{T\in \mathcal{T}} p(T) p(\Ybf \mid T), \qquad \Ybf\mid T \sim \Ncal(0, \Sigma_T).$$}

Dans l'expression d'un mélange d'arbres gaussien, la distribution gaussienne $p(\Ybf\mid T)$ s'écrit naturellement comme un produit sur les paires de noeuds. Donc, en choisissant une distribution $p(T)$ qui se factorise de la même manière une forme somme-produit apparaît, qui est calculable grâce au théorème arbre-matrice.

 Une telle distribution sur l'espace des arbres est par exemple la distribution décomposable, qui attribue un poids à chaque arête et pour laquelle la probabilité d'un arbre est proportionnelle au produit de ces poids. Cette distribution permet notamment de calculer facilement des probabilités d'arêtes. En utilisant la définition d'une probabilité d'arête comme la somme des probabilités des arbres contenant cette arête, il devient clair qu'adopter une distribution décomposable mène à une nouvelle forme somme-produit sur l'espace des arbres. Pour une matrice de poids $\Wbf=(w_{uv})_{uv}$, la probabilité que les noeuds $j$ et $k$ soient reliés dans T est :
$$\mathds{P}\{jk\in T\} = \sum_{\substack{T\in \mathcal{T} \\ T\ni jk}} p(T) \propto \sum_{\substack{T\in \mathcal{T} \\ T\ni jk}} \prod_{uv \in T} w_{uv}.$$
Chaque probabilité peut ensuite être calculée par le théorème arbre-matrice, ou en utilisant un résultat de \citet{kirshner} permettant de les calculer toutes en une seule opération.

 
\subsection*{Inférence de variables latentes}
L'algorithme Espérance-Maximisation (EM, \citet{DLR77}) permet de mener une inférence sur des données $\Ybf$ en présence de variables latentes $\Zbf$. C'est un algorithme itératif qui alterne deux étapes : une étape d'estimation (étape E) de l'espérance conditionnelle de la log-vraisemblance jointe des variables observées et latentes (complète), et une étape de maximisation (étape M) de cette quantité en les paramètres du modèle.

Lorsque la distribution des variables latentes conditionnellement aux observées $p(\Zbf\mid\Ybf)$ n'est pas disponible ou calculable, l'étape E a besoin d'être modifiée. L'approche variationnelle permet alors d'approximer $p(\Zbf\mid\Ybf)$ en définissant un ensemble de distributions autorisées $Q$, et une mesure de distance de la distribution approchée à la vraie distribution conditionnelle. L'algorithme est alors Variationnel EM (VEM), dont l'étape VE est un problème d'optimisation visant à trouver la distribution $q\in Q$ la plus proche de $p(\Zbf\mid\Ybf)$. L'algorithme VEM peut aussi être vu comme l'optimisation d'une borne inférieure de la vraisemblance, pénalisée par une mesure de la distance entre la distribution conditionnelle $p(\Zbf\mid\Ybf)$ et son approximation variationnelle $q(\Zb)$ .\\


\textbf{Propriété} (Approximation en champs moyen). \textit{Soit un algorithme VEM avec $Q$  l'ensemble des distributions factorisables et la divergence de Küllback-Leibler, appliqué aux données observées $\Ybf$ et à  $K$ variables latentes $\Zbf=\{z_1,...,z_K\}$. Alors la solution du problème d'optimisation de l'étape VE à l'itération $t+1$ pour la distribution marginale $q_k$ de $Z_k$ est telle que:
$$ q_k^{(t+1)}(z_k)  \propto \exp \left\{ \Esp_{q_{\setminus k}^t} \left[ \log p_{\thetabf^{t+1}}(\Ybf, \Zbf) \right] \right\}.$$}
Ce résultat, dû à \citet{beal}, est connu sous le nom d'approximation en champ moyen et permet une écriture claire des algorithmes VEM sous les conditions précisées.

\subsection*{Inférence à partir de données de comptages}
L'inférence de réseaux à partir de données de comptages nécessite la définition d'une distribution jointe permettant de modéliser les dépendances entre variables discrètes. Il existe peu d'options dans le domaine, et une manière de faire est d'utiliser les modèles linéaires mixtes  généralisés. Pour chaque échantillon $i$ et espèce $j$, un effet aléatoire $Z_{ij}$ gaussien est associé au comptage moyen $m_{ij}$ par une fonction de lien $g$. Une écriture générale de ce modèle prenant en compte l'offset $o_{ij}$ et le vecteur de covariables $\xb_i$ de coefficient $\thetab_j$ est la suivante :
 \begin{equation*}
 \left\{ \begin{array}{ll}
 &g(m_{ij}) = o_{ij}+\xb_i^\intercal  \thetab_j + Z_{ij},\\
&\Zb_i \sim \Ncal(0, \Sigmab),\\
 &\Ybf_{ij}\mid\Zbf_i \sim F(m_{ij}).
 \end{array} \right. 
 \end{equation*}
 
$\Ybf_{ij}$ est distribué selon $F$, de moyenne $m_{ij}$. Par la suite, c'est la matrice de variance-covariance  $\Sigmab$ des effets aléatoires gaussiens $\Zbf$ qui est étudiée, plutôt que les dépendances de $\Ybf$ directement. Il existe différentes stratégies pour se ramener au cadre gaussien en utilisant les modèles linéaires mixtes généralisés, et dans les chapitres suivants nous utilisons la distribution Poisson log-normale (PLN). Dans ce cas précis, $F$ est une distribution de Poisson et $g$ la fonction $\log$.



\section*{Chapitre 2}
Ce deuxième chapitre présente la méthode développée pour l'inférence de réseaux d'interactions d'espèces à partir de données de comptages. Les comptages $\Ybf$ sont modélisés avec la distribution Poisson log-normale. L'inférence du réseau en elle-même a lieu dans la couche des paramètres gaussiens $\Zbf$, et utilise un mélange d'arbres. Cela signifie que le graphe des dépendances conditionnelles sous-jacent à $\Zbf$ est supposé être un arbre latent $T$. Ce modèle hiérarchique à deux couches latentes peut se résumer ainsi : un arbre est d'abord tiré aléatoirement, puis les paramètres gaussiens sont tirés conditionnellement à $T$, et enfin les comptages $Y$ sont tirés conditionnellement aux $\Zbf$ selon une loi de Poisson.

 \begin{center}
	\begin{tikzpicture}	
      \tikzstyle{every edge}=[-,>=stealth',auto,thin,draw]
		\node (A1) at (0*\length, 0*\length) {$T$};
		\node (A2) at (1*\length, 0*\length) {$\Zbf$};
		\node (A3) at (2*\length, 0*\length) {$\Ybf$};
		\draw (A1) edge [->] (A2);
        \draw (A2) edge  [->] (A3);
	\end{tikzpicture} 
   \end{center}
   
   
   La stratégie d'inférence de réseaux mise en place vise la mise à jour de la distribution de l'arbre latent $T$ au travers des poids $\betabf$, et le calcul de probabilités d'arêtes.

\subsection*{Modèle}
Les comptages $\Ybf$ sont modélisés avec la distribution PLN. En notant $\Zbf$ les paramètres latents, $\xb_i$ le vecteur de covariables correspondant à l'échantillon $i$, $\thetab_j$ son coefficient pour l'espèce $j$ et $o_{ij}$ l'offset associé, le modèle s'écrit comme suit :
 $$ Y_{ij}\mid \Zbf_i\sim \Pcal (\exp (o_{ij}+\xb_i^\intercal \thetab_j + Z_{ij})), \;\;\; (Y_{ij} \independent) \mid \Zbf_i .$$
 
Conditionnellement à l'arbre $T$, les paramètres latents $\Zbf$ sont distribués selon la loi normale. La loi de $\Zbf\mid T$ est donc fidèle Markov à $T$ et s'écrit :
 $$\Zbf_i\mid T  \sim \Ncal (0, \Omegab_T), \;\;\;  \{\Zbf_i\}_i \text{ iid},$$
 
 où les termes non-nuls de $\Omegab_T$ correspondent aux arêtes de $T$.  Enfin, l'arbre $T$ est supposé suivre une loi décomposable sur ses arêtes, avec $\betabf$ la matrice des paramètres de poids et B la constante de normalisation :
 $$ T\sim \prod_{kl\in T} \beta_{kl} / B, \;\;\;  B= \sum_{T\in\mathcal{T}} \prod_{kl \in T} \beta_{kl}.$$
 
L'inférence du graphe des dépendances de la couche des $\Zbf$ repose alors sur un mélange d'arbres gaussien. C'est à dire que les paramètres latents suivent une distribution de mélange de gaussiennes indépendantes dont chaque composante est fidèle à un arbre de l'espace des arbres couvrants $\mathcal{T}$ :

$$\Zb_i \sim \sum_{T\in \mathcal{T}} p(T) \Ncal (\Zbf_i\mid T; 0, \Omegab_T).$$


\subsection*{Inference}
%Ce chapitre est une première approche du modèle décrit ci-dessus qui se concentre sur les paramètres de la loi de $T$ pour l'inférence du réseau. Ainsi plutôt que la distribution de mélange pour $\Zbf$, l'inférence utilise que $\Zbf_i$ est encore une gaussienne centrée, et considère sa matrice de covariance $\Sigma$. Les paramètres du modèle sont donc simplement $(\thetabf, \Sigma, \betabf)$.
%La vraisemblance du modèle s'écrit de la manière suivante :
%$$p_{\thetabf, \Sigma, \betabf}(T, \Zbf, \Ybf) = p_\betabf(T)\times p_\Sigma(\Zbf\mid T) \times p_\thetabf(\Ybf\mid\Zbf).$$

Les paramètres concernant directement le réseau sont les poids rassemblés dans $\betabf$. L'inférence du modèle comprend deux étapes qui séparent  l'estimation de $\betabf$ du reste des paramètres. La première étape utilise l'algorithme variationnel développé par \citet{CMR18} pour avoir accès aux estimateurs variationnels de  $\thetabf$ et des statistiques exhaustives relatives à $\Zbf$. Ces paramètres sont fixés pour la suite de l'inférence. La seconde étape est un algorithme EM qui a pour but l'estimation de $\betabf$, et le calcul des probabilités d'arêtes.

L'algorithm EM requiert le calcul de l'espérance conditionnelle de la log-vraisemblance complète. En notant $\widehat{\rho}_{jk}$ la corrélation estimée entre les variables $j$ et $k$, et $P_{jk}\simeq\mathds{P}\{jk\in T\mid \Ybf\}$ la probabilité d'arête conditionnelle estimée entre $j$ and $k$, cette espérance est approximée par la quantité ci-dessous, où le terme cst ne dépend pas de $\betabf$ :
$$\sum_{1\leq j < k \leq p} P_{jk} \log \left(\beta_{jk} (1-\widehat{\rho}_{jk}^2)^{-n/2} \right) -\log B + cst.$$


\begin{description}
\item[Étape E :] Les estimateurs des statistiques exhaustives de $\Zbf$ obtenus en première étape de l'inférence donnent accès aux $\widehat{\rho}_{jk}$. L'étape E consiste donc en le calcul des probabilités approchées $P_{jk}$. On considère pour cela la probabilité conditionnelle à $\Ybf$ de l'ensemble des arbres contenant l'arête $jk$, ce qui revient à appliquer le théorème arbre-matrice à une nouvelle matrice de poids, de terme général $\beta_{jk}(1-\widehat{\rho}_{jk}^2)^{-n/2}$.

\item[Étape M :] Cette étape  maximise la quantité précédente en $\betabf$. La forme close de la formule de mise à jour  pour $\betabf$ est obtenue  en utilisant un lemme de \citet{MeilaJaak}, qui définit une matrice $\Mbf(\betabf)$, fonction de $\betabf$. À l'itération $t+1$ de l'algorithme EM, la mise à jour est telle que:
$$\beta_{jk}^{t+1} = P_{jk}^{t+1} \left/ \left[\Mbf(\betabf^t)\right]_{jk} \right..$$
\end{description}

\subsection*{Simulations et applications}
La procédure d'inférence est implémentée dans le package R  EMtree, disponible sur GitHub (\url{https://github.com/Rmomal/EMtree}). Les performances de cette méthode ont été comparées à celles d'approches venant de la micro-biologie (SpiecEasi \citep{kurtz}, gCoda \citep{gcoda}, MInt \citep{MInt}) et de l'écologie des communautés (MRFcov \citep{CWL18}, ecoCopula \citep{PWT19}). Ces méthodes diffèrent sur la manière qu'elles ont de se ramener au cadre gaussien. Toutes ces méthodes infèrent ensuite le réseau en ayant recourt à une estimation pénalisée de la matrice de précision de la loi normale. %Elles diffèrent sur l'obtention de ces nouvelles données : SpiecEasi et gCoda opèrent une transformation continue, MInt considère la couche latente gaussienne du modèle PLN, enfin ecoCopula et MRFcov utilisent des copules gaussiennes.

%on peut en dire plus sur les graphes simulés
Lors de l'étude de simulations sur des graphes de densité et structure variables (cluster, Erdös-Reyni, scale-free), EMtree s'illustre comme la méthode  donnant le moins de faux positifs (fausses arêtes) et conservant une densité de réseau comparable à l'originale, tout en figurant parmi les algorithmes les plus rapides. %À noter toutefois une diminution des performances pour des réseaux scale-free de plus grande dimension, où ecoCopula et en particulier MRFcov semblent meilleurs.

Deux exemples d'application d'EMtree, un en écologie et un en méta-génomique, montrent l'importance de la prise en compte des covariables expérimentales dans le modèle. Concernant la deuxième application, EMtree retrouve des résultats obtenus précédemment dans \citet{jakuch}.


\section*{Chapitre 3}
Le chapitre 3 considère le modèle exposé précédemment dans le chapitre 2, dans le cas où la couche latente des paramètres gaussiens $\Zbf$ comprend des dimensions supplémentaires qui ne correspondent à aucune donnée observée. Ces dimensions supplémentaires représentent des acteurs, espèces ou covariables, qui ont une influence sur les données observées mais n'ont pas été mesurés. Ce sont des acteurs manquants, qui s'ils ne sont pas pris en compte génèrent des liens de dépendances conditionnelles entre les espèces dépendantes de chacun des acteurs. Le modèle précédent est modifié pour prendre en compte les dimensions supplémentaires indexées par $H$ de la couche latente, et considère la version normalisée de cette dernière, notée $\Ubf$. Le graphe suivant synthétise les dépendances entre variables, où $\Ubf_O$ sont les variables latentes des données observées, et $\Ubf_H$ celles des acteurs manquants :

\begin{center}
	\begin{tikzpicture}	
      \tikzstyle{every edge}=[-,>=stealth',auto,thin,draw]
		\node (A1) at (0.625*\length, 2*\length) {$T$};
		\node (A2) at (0*\length, 1*\length) {$\Ubf_O$};
		\node (A3) at (1.25*\length, 1*\length) {$\Ubf_H   $};
		\node (A4) at (0*\length, 0*\length) {$\Ybf$};
		\draw (A1) edge [->] (A2);
        \draw (A1) edge [->]  (A3);
        \draw (A2) edge  (A3);
        \draw (A2) edge [->]  (A4);
	\end{tikzpicture} 
\end{center}

C'est un modèle plus complexe que précédemment, comprenant trois couches latentes. Un algorithme VEM est développé pour l'inférence, qui permet d'obtenir des informations concernant les acteurs manquants, en plus d'inférer le réseau complet des dépendances conditionnelles.

\subsection*{Modèle}
Les comptages sont modélisés avec la distribution PLN comme détaillé au chapitre 2, mais en considérant la version normalisée de la couche latente gaussienne :
$$Y_{ij} \mid U_{ij} \sim \Pcal\left(\exp(o_{ij} + \xbf_i^\intercal \thetabf_j + \sigma_j U_{ij})\right), \qquad (Y_{ij}\independent )\mid \Ubf_i.$$

La couche latente gaussienne $\Ubf$ est composée de $\Ubf_O$, de dimension $p$ qui correspond aux $\Ybf$ observés, et $\Ubf_H$ qui rassemble les $r$ variables supplémentaires non-observées. 

Le reste du modèle est le même que précédemment, si ce n'est pour les dimensions supplémentaires : l'arbre $T$  est paramétré par une matrice de poids $\betabf$ de dimension $(p+r)\times (p+r)$, la distribution marginale de $\Ubf$ est un mélange gaussien sur l'espace des arbres couvrants de dimension $p+r$, dont chaque composante est fidèle à un arbre. On note $\Omegab$ l'ensemble des matrices de précision des gaussiennes fidèles à un arbre : $\Omegab = \{\Omegab_T, T\in \mathcal{T}\}$. 
 
\subsection*{Inférence}
La complexité supplémentaire de la structure latente de ce modèle demande une inférence variationnelle. L'inférence présentée ici maximise la borne inférieure suivante où $KL$ désigne la divergence de Küllback-Leibler:
$$\Jcal(\thetabf, \betabf, \Omegab; q)= \log p_{\thetabf, \betabf, \Omega}(\Ybf) - KL\left(q(\Ubf, T) \| p_{\thetabf, \betabf, \Omega}(\Ubf, T \mid \Ybf) \right).$$
Ci-dessus, $q(\Ubf, T)$ est la distribution approchée des variables latentes conditionnellement aux données observées : $q(\Ubf, T)\approx p(\Ubf, T \mid \Ybf)$. On suppose que $q$ est une distribution produit, ce qui permet de séparer les distributions marginales variationnelles de l'arbre et des variables gaussiennes comme suit:
$$q(\Ubf, T) = h(\Ubf) \,g(T).$$
La distribution $h$ est de plus elle-même supposée être un produit de $n$ gaussiennes à matrices de variance-covariances diagonales, à l'instar de \citet{CMR18}. Les paramètres de $h$ sont les matrices $M=[M_O, M_H]$ et $S=[S_O, S_H]$ de dimensions $n\times (p+r)$, rassemblant respectivement les moyennes et les variances de chaque composante du produit.
La distribution $g$ de l'arbre $T$ se factorise sur les arêtes de $T$, et ses paramètres de poids sont rassemblés dans la matrice $\widetilde{\betabf}$.

La procédure d'inférence tire à nouveau parti de l'estimation variationnelle du modèle PLN développée dans \citet{CMR18}. Cette estimation donne une approximation des paramètres $\thetab$, $M_O$ et $S_O$ qui sont considérés fixes pour la suite de l'inférence.

L'algorithme VEM développé a pour but l'estimation des poids $\betabf$ et itère les étapes suivantes:
\begin{description}
\item[Étape VE :] Cette étape maximise $\bound$ par rapport à $q$, ce qui revient à maximiser $\bound$ en les paramètres variationnels de $g$ et ceux restants de $h$, soit en $\widetilde{\betabf}$, $M_H$ et $S_H$. L'écriture de $q$ sous forme factorisée  permet d'aboutir à une approximation en champs moyen pour les estimations de $h$ et $g$. 
\item[Étape M :] Cette étape maximise $\bound$ en les paramètres restants de la log-vraisemblance complète, soit en $\betabf$ et $\Omegabf$. La mise à jour de $\beta$ est la même que dans le chapitre 2 et nécessite le calcul des probabilités d'arêtes. Dans le cas présent la distribution conditionnelle des arbres aux données est approchée par $g$, ce qui justifie de calculer les probabilités d'arêtes en appliquant la formule de \citet{kirshner} aux poids variationnels $\widetilde{\betabf}$. 

L'estimation des matrices $\Omegabf_T$ est plus complexe et applique au contexte des arbres couvrants  les formules du maximum de vraisemblance de \citet{Lau96}, établies dans le cadre des modèles graphiques gaussiens décomposables. Ces formules sont définies à partir de l'espérance de statistiques exhaustives, calculée à partir de $\Mbf$ et $\Sbf$.
\end{description}
Cette procédure est implémentée dans le package R nestor (\textit{Network inference from Species counTs with missing actORs}, \url{https://github.com/Rmomal/nestor}). Une étude de simulations révèle son efficacité et l'importance fondamentale de son initialisation.

\subsection*{Applications}
La procédure d'inférence donne une estimation du réseau complet et des paramètres du modèle. Cela rend disponible deux informations clés pour caractériser les acteurs manquants supposés du réseau, à savoir leur position dans le réseau, et l'estimation de leur moyenne variationnelle $M_H$ sur les différents sites d'échantillonnage. 

Un premier exemple d'application utilise des données de recensement de poissons dans la mer de Barents. Là, l'inférence de réseau avec $r=1$ acteur manquant donne un noeud dont le vecteur de moyennes $M_H$ est fortement corrélé à la covariable de température, tout comme ceux de ses voisins directs dans le réseau. Dans un second exemple sur des poissons du fleuve Fatala en Guinée, l'inférence est réalisée avec deux acteurs manquants. L'étude de leur moyenne variationnelle  montre que le premier est lié à la nature spatiale de ces donnée, et le second semble lié à la dimension temporelle de l'échantillonnage.

Ces applications  illustrent à la fois la validité de la méthode en comparant les acteurs inférés à des covariables d'importance, et une première approche pour la caractérisation des acteurs manquants.

 
\section*{Chapitre 4}
Le dernier chapitre présente des perspectives de ce travail. Après avoir résumé les spécificités et discuté des points sensibles de la méthodologie développée, des extensions naturelles du modèle sont présentées.  Les premières  portent sur le réseau inféré. Plus précisément, une méthode d'estimation de la matrice de précision $\Omegab$ de la couche latente est proposée, qui utilise les estimateurs de Lauritzen. L'estimation précise de cette matrice permet des interprétations intéressantes dans le domaine d'application en question, à propos de la force et du sens des interactions détectées. Le sujet de la comparaison de réseaux est abordé dans un second temps. La manière de faire générale est de résumer les réseaux à des vecteurs de mesures caractéristiques, puis de comparer ces vecteurs. Ici nous proposons de comparer les distributions d'arbres inférées sur les réseaux. 

Des perspectives sur les spécificités des données  disponibles sont ensuite discutées. La procédure d'inférence de réseaux développée a lieu au sein d'une couche gaussienne latente. Tant qu'elle est présente dans le modèle l'inférence de réseau reste la même, aussi la loi d'émission des données à partir de ces paramètres latents peut être différente. Ceci permet d'inférer des réseaux à partir de données de natures variées, pourvu que l'estimation des paramètres de cette loi soit disponible. Par ailleurs, les données peuvent présenter des dépendances spatiales, comme c'est souvent le cas en écologie. Il est possible de prendre en compte ces effets en moyenne, en ajustant par exemple des coordonnées spatiales au modèle. Cette première correction peut ne pas être suffisante. Nous présentons une manière d'introduire des paramètres de variance des effets spatiaux dans le modèle d'inférence de réseau directement, afin d'ajuster les effets spatiaux en variance.


Enfin, une perspective plus générale sur l'inférence de réseau à partir de données de comptages sans avoir recours à une couche latente gaussienne est présentée. Comme la vraisemblance d'une variable aléatoire conditionnellement à une structure d'arbre se factorise sur les arêtes de cet arbre, l'idée est d'inférer le réseau par un mélange d'arbres en utilisant une loi bivariée sur les comptages. La difficulté de cette approche réside dans l'estimation des paramètres.

\chapter*{Liste des publications}
\begin{itemize}
\item Momal, Raphaëlle, Stéphane Robin, and Christophe Ambroise. "Tree‐based inference of species interaction networks from abundance data." Methods in Ecology and Evolution 11.5 (2020): 621-632.\\
\item Momal, Raphaëlle, Stéphane Robin, and Christophe Ambroise. "Accounting for missing actors in interaction network inference from abundance data." arXiv preprint arXiv:2007.14299 (2020).
\end{itemize}
\bibliographystyle{plainnat}  
\bibliography{biblio.bib}

%%%%%%%%%%%%%%%%%%%%%%%%%%%%%%%%%%%%%%%%%%%%%%%%%%%%%%%%%%%%%%%
% 4eme de couverture
\clearemptydoublepage
\ifthispageodd{\newpage\thispagestyle{empty}\null\newpage}{}
\thispagestyle{empty}
\newgeometry{top=1.5cm, bottom=1.25cm, left=2cm, right=2cm}
\fontfamily{rm}\selectfont

\lhead{}
\rhead{}
\rfoot{}
\cfoot{}
\lfoot{}

\noindent 
%*****************************************************
%***** LOGO DE L'ED À CHANGER ÉVENTUELLEMENT *********
%*****************************************************
\includegraphics[height=2.45cm]{e.png}
\vspace{1.5cm}
%*****************************************************

\begin{mdframed}[linecolor=Prune,linewidth=1]
\vspace{-.25cm}
\paragraph*{Titre :} Inférence de réseaux à partir de données d'abondances incomplètes.
\begin{small}
\vspace{-.25cm}
\paragraph*{Mots clés :} Réseaux, Modèles Graphiques, Données d'abondances, Acteurs manquants, Algorithme VEM
\vspace{-.5cm}
\begin{multicols}{2}
\paragraph*{Résumé :}  

Les réseaux sont utilisés comme outils en microbiologie et en écologie pour représenter des relations entre espèces. Les modèles graphiques gaussiens sont le cadre mathématique dédié à l'inférence des réseaux de dépendances conditionnelles, qui permettent une séparation claires des effets directs et indirects. Cependant, les données observées sont souvent des comptages discrets qui ne permettent pas l'utilisation de ce modèle. Cette thèse développe une méthodologie pour l'inférence de réseaux à partir de données d'abondance d'espèces. La méthode repose sur une exploration efficace et exhaustive de l'espace des arbres couvrants dans un espace latent des comptages observés, rendue possible par les propriétés algébriques de ces structures.\\
Par ailleurs,  il est probable que les comptages observés dépendent d'acteurs non mesurés (espèces ou covariable).  Ce phénomène produit des arêtes supplémentaires dans le réseau marginal entre les espèces liées à l'acteur manquant dans le réseau complet, ce qui fausse la suite des analyses. Le second objectif de ce travail est de prendre en compte les acteurs manquants lors de l'inférence de réseau. Les paramètres du modèle proposé sont estimés par une approche variationnelle, qui fournit des éléments d'information pertinents à propos des données non observées.

\end{multicols}
\end{small}
\end{mdframed}

\begin{mdframed}[linecolor=Prune,linewidth=1]
\vspace{-.25cm}
\paragraph*{Title:} Network inference from incomplete abundance data.
\begin{small}
\vspace{-.25cm}
\paragraph*{Keywords:} Networks, Graphical Models, Abundance data, Missing Actors, VEM algorithm

\vspace{-.5cm}
\begin{multicols}{2}
\paragraph*{Abstract:} 

Networks are tools used to represent species relationships in microbiology and ecology. Gaussian Graphical Models provide with a mathematical framework for the inference of conditional dependency networks, which allow for a clear separation of direct and indirect effects. However observed data are often discrete counts and the inference cannot be directly performed with this model. This work develops a methodology for network inference from species observed abundances. The method relies on specific algebraic properties of spanning tree structures to perform an efficient and complete exploration of the space of spanning trees. The inference takes place in a latent space of the observed counts.\\
Then, observed abundances are likely to depend on unmeasured actors (e.g. species or covariate). This results in spurious edges in the marginal network between the species linked to the latter in the complete network, causing inaccurate further analysis. The second objective of this work is to account for  missing actors during network inference. To do so we adopt a variational approach yielding valuable insights about the missing actors.

\end{multicols}
\end{small}
\end{mdframed}

%************************************
\vspace{3cm} % ALIGNER EN BAS DE PAGE
%************************************
\fontfamily{fvs}\fontseries{m}\selectfont
\begin{tabular}{p{14cm}r}
\multirow{3}{16cm}[+0mm]{{\color{Prune} Université Paris-Saclay\\
Espace Technologique / Immeuble Discovery\\
Route de l’Orme aux Merisiers RD 128 / 91190 Saint-Aubin, France}} & %\multirow{3}{2.19cm}[+9mm]{\includegraphics[height=2.19cm]{e.pdf}}\\
\end{tabular}

\end{document}