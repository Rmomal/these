\subsubsection{Effect of dataset dimensions}
\label{adverse}

Behaviors are compared on an easy setting ($n=100$, $p=20$) and a hard setting ($n=50$, $p=30$). Fig.~\ref{TPFN} displays FDR and density ratio measures for all methods on the different cases.  Detailed values of medians and standard-deviations are given in Tables \ref{medFDR} and\ref{meddens}. The behaviour of methods remains virtually the same across Erdös and Cluster structures. Scale-free structure appears to entail a greater difficulty for all methods except ecoCopula: the FDR increases in easy cases of about $15\%$ for SpeacEasi, MRFcov and EMtree, and about $35\%$ for MInt.\\
The greater difficulty affects all methods. gCoda standard-deviation increases by $10\%$. MRFcov, EMtree and MInt show an increase in FDR of about $5\%$, $20\%$ and $30\%$ respectively. Density ratios overall decrease, especially for ecoCopula which ratio is close to 0 and yields a proportion of empty networks of $15$-$25\%$ (Table \ref{empty}).


Considering FDRs and density ratios combined, EMtree appears to be the method with the lower FDR which maintains a density ratio reasonably close to 1. As a consequence, the proposed methodology compares well to existing tools on problems with varying difficulties. EMtree is also comparable on running times. Table~\ref{timesTPFN} shows that for Erdös and Cluster it is the third quicker method in easy cases and the second in hard ones. Table \ref{timeSF} (in appendix) shows that on scale-free problems, EMtree is the second quicker method in hard cases, and is curiously slow on easy ones.

Interestingly, in easy cases when the network density is well estimated, methods yield FDR of $10\%-30\%$ in median. This reminds that network inference from abundance data is a difficult task, and that perfect inference of the network remains an out-of-reach goal. 
 
\begin{figure}
    \centering
    \includegraphics[width=\linewidth]{figs/panel_TPFN_signed.png}
    \caption{FDR and density ratio measures for all methods at two different difficulty levels and 100 networks of each type. White squares and black plain lines represent medians and quartiles respectively. \small{\textit{ecoCopula selection method: AIC. Number of subsamples for SpiecEasi and EMtree: $S=20$. SpiecEasi and gCoda: $lambda.min.ratio=0.001$,  $nlambda=100$.}}}
    \label{TPFN}
\end{figure}

\begin{table}[ht]
\centering
 
\begin{tabular}{l|rrrrrr}
 & \multicolumn{1}{c}{SpiecEasi} & \multicolumn{1}{c}{gCoda} & \multicolumn{1}{c}{ecoCopula} & \multicolumn{1}{c}{MRFcov} & \multicolumn{1}{c}{MInt} & \multicolumn{1}{c}{EMtree} \\ 
  \hline
Easy & 25.45  (1.87) & 0.11  (0.06) & 5.55  (0.64) & 34.51  (3.68) & 43.04  (19.76) & 11.72  (1.89) \\ 
  Hard & 28.43  (1.30) & 0.53  (0.25) & 9.6  (0.65) & 8.29  (0.36) & 33.77  (18.20) & 8.17  (0.50) \\ 
   \hline
\end{tabular}
\caption{Median and standard-deviation running-time values (in seconds) for Cluster and Erdös structures, including resampling with $S=20$ for SpiecEasi and EMtree.}
\label{timesTPFN}
\end{table}




%%%%%%%%%%%%%%%%%%%%%%%%%%%%%%%%%%%%
%%%%%%%%%%%%%%%%%%%%%%%%%%%%%%%%%%%%

\subsubsection{Effect of network structure}

As expected for a fixed $p$, the higher the number of observations $n$, the better the performance for all methods and structures. Interestingly, the same happens when $p$ increases for a fixed $n=100$ (except for SpiecEasi).
EMtree performs well on Scale-free structures (Fig.~\ref{SFAUC}) which was also expected; the other methods performance worsen compared to other structures. When lowering $n$ to 30, EMtree performance deteriorates along with $p$, yet remaining above $70\%$ in median in the extreme case where $p=n$ (Fig.~\ref{SFAUC}, right). The structure being Erdös or Cluster, each method is affected in the same way by an increase of $n$ or $p$ (Fig.~\ref{panelErdClust}). Besides, increasing the difference between the two structures by tuning up the \textit{ratio} parameter has no effect. Overall EMtree performs better than gCoda and SpiecEasi on all the studied configurations. Running times are summarized in Table~\ref{timeNP}. EMtree is about 10 times slower than gCoda (4 for small $n$), and 4 times faster than SpiecEasi. The high standard deviation for small $n$ seems to be due to gCoda struggling with Scale-free structures.
  
\begin{figure}[H]
    \centering
    \includegraphics[width=0.7\linewidth]{figs/panel_SF.png}
    \caption{Effect of Scale-free structure on AUC medians and inter-quartile intervals for parameters $n$ and $p$.}
      \label{SFAUC}
\end{figure}


\begin{table}[H]
\centering
\begin{tabular}{l|rr|rr}
 & \multicolumn{1}{c}{$n < 50$} & \multicolumn{1}{c}{$n\geq 50$}  & \multicolumn{1}{c}{$p < 20$} & \multicolumn{1}{c}{$p\geq 20$} \\  
 \hline
  EMtree    &   0.44 (0.14)	 &   0.60 (0.17) &   0.41 (0.13) &   0.76 (0.21)   \\ 
  gCoda     &   0.11 (26.8)	 &   0.05 (0.05) &   0.05 (0.04) &   0.09 (0.54)   \\ 
  SpiecEasi &   2.09 (0.26)	 &   2.37 (0.28) &   2.42 (0.27) &   2.42 (0.26)   \\ 
   \hline
\end{tabular}
\caption{Median and standard-deviation of running times for each method in seconds, for $n$ and $p$ parameters.}
\label{timeNP}
\end{table}

%%%%%%%%%%%%%%%%%%%%%%%%%%%%%%%%%%%%
%%%%%%%%%%%%%%%%%%%%%%%%%%%%%%%%%%%%

\subsubsection{Effect of network density}
The comparison of top and bottom panels of Fig.~\ref{panelErdClust} shows that network inference gets harder as the network gets denser, whatever the method and the structure of the true graph. Running times are not affected (Table \ref{timeDenser}).
Fig.~\ref{varyDens} shows that EMtree performance does not deteriorate faster than that of other methods, demonstrating that the tree hypothesis is not a limitation.


 \begin{figure}[H]
  \centering
   \includegraphics[width=\linewidth]{figs/panel_npFav.png}
  \includegraphics[width=\linewidth]{figs/panel_dense.png}
  \caption{Effect of Erdös and Cluster structures on AUC medians and inter-quartile intervals for parameters $n$, $p$ and $ratio$. \textit{Top}: densities set to $2/p$, \textit{bottom}: densities set to $5/p$.}
  \label{panelErdClust}
\end{figure}

\begin{figure}[H]
 \centering
  \includegraphics[width=0.6\linewidth]{figs/panel_dens_seuils.png}
  \caption{AUC median and inter-quartile intervals for parameters controlling the number of edges in Erdös (\textit{edge probability}) and in Cluster (\textit{density}) structures, $p=20$, $n=100$.}
  \label{varyDens}
\end{figure}
 
