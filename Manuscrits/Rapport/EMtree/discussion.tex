The inference of species interaction network is a challenging task, for which a series of methods have been proposed in the past years. Abundance data seems to be a promising source of information for this purpose. Here we adopt the formalism of graphical models to define a probabilistic model-based framework for the inference of such networks from abundance data.
Using a model-based approach offers several important advantages. First, it enables easy and explicit integration of environmental and experimental effects.  These could be modeled in a more flexible way using generalized additive models, which include non-linear effects \citep{hastie2017generalized}. 
Then, as it also relies on a formal statistical definition of a \textsl{species interaction network} in the context of graphical models, accounting for abiotic effects and modeling species interactions are two clearly defined and distinguished goals. Finally, all the underlying assumptions are explicitly stated in the model definition itself, and can therefore be discussed and criticized. \\



We developed an efficient method to infer sparse networks, which combines a multivariate Poisson mixed model for the joint distribution of abundances, with an averaging over all spanning trees to efficiently infer direct species interactions. As we do consider a mixture over all spanning trees, our approach remains flexible and can infer most types of statistical dependencies. An EM algorithm (EMtree) maximizes the likelihood of the result and returns each edge probability to be part of the network. An optional resampling step increases network robustness.

\modif{A simulation study in a heterogeneous environment  demonstrates  that EMtree  compares very well to alternative approaches. The proposed model can take all kind of covariates into account, which when ignored  can have  dramatic effects  on the inferred network structure, as showed here on empirical datasets.  Experiments on simulated data and illustrations also demonstrate that EMtree  computational cost remains very reasonable.}

\modif{Alternative methods used in this work all rely on an optimized threshold to tell an edge presence. This particular threshold is obtained after testing a grid of possible values which all yield a different network, and altogether build a path. Making this path available to the user is useful, as the final threshold might need modification and it gives the possibility to build edges scores  and get more than a binary result. We found few recent approaches doing this, which prevented us to study their performance in a way that did not impose a threshold.}\\

The proposed methodology could be extended in several ways.
\modif{Species abundances and interactions indeed vary across space, and depend on local conditions \citep{PCM12,PSG15}. This can either be considered as nuisance parameter or as feature of interest. In the first case, the method could be extended to account for the spatial autocorrelation of sampling sites, to obtain a "regional" interaction network corrected for this effect, i.e. assuming the network is the same in all sites. If of interest, variation across space and local conditions could be studied by comparing networks inferred from the different sampling locations. Networks comparison is a wide and interesting question and tools lack to check which edges are shared by a set of networks. The approach introduced by \citet{SR17} could be adapted to  EMtree framework.} Lastly,  It is also very likely that not all covariates nor even all species have been measured or observed. Another extension may therefore be to detect ignored covariates or missing species. To this purpose EMtree could probably be combined with the approach developed by \citet{RAR19} to identify missing actors. 