We now describe how to infer the model parameters. We gather the edges weights into the $p \times p$ matrix $\betab$ and the vectors of regression coefficients into a $d \times p$ matrix $\thetab$. The $p \times p$ matrix $\Sigmab$ contains the variances and covariances between the coordinates of each latent vector $\Zb_i$. Hence, the set of parameters to be inferred is $(\betab, \Sigmab, \thetab)$.

\paragraph{Likelihood.} 
The model described above is an incomplete data model, as it involves two hidden layers: the random tree $T$ and the latent Gaussian vectors $\Zb_i$. The most classical approach to achieve maximum likelihood inference in this context is to use the Expectation-Maximization algorithm \citep[EM:][]{DLR77}. Rather than the likelihood of the observed data $p(\Yb)$, the EM algorithm deals with the often more tractable likelihood $p(T, \Zb, \Yb)$ of the complete data (which consists of both the observed and the latent variables). It can be decomposed as 
 
\begin{equation} \label{eq:PTZY}
    p_{\betab, \Sigmab, \thetab}(T, \Zb, \Yb) = p_{\betab}(T) \times p_{\Sigmab}(\Zb \; | \; T) \times p_{\thetab}(\Yb \; | \; \Zb),
\end{equation}
 
where the subscripts indicate on which parameter each distribution depends. \\
Observe that the dependency structure between the species is only involved in the first two terms, whereas the third term only depends on the regression coefficients $\thetab$. 
We take advantage of this decomposition to propose a two-stage estimation algorithm. The first stage deals with the observed layer $p_{\thetab}(\Yb \; | \; \Zb)$, the second with the two hidden layers $p_{\betab}(T)$ and  $p_{\Sigmab}(\Zb \; | \; T)$. The network inference itself takes place in the second step.

\paragraph{Inference in the observed layer.} 
The variational EM (VEM) algorithm that provides an estimate of the regression coefficients matrix $\thetab$ is described in Appendix \ref{app:VEM} (along with a reminder on EM and VEM). It also provides the (approximate) conditional means $\Esp(Z_{ij} | \Yb_i)$, variances $\Var(Z_{ij} | \Yb_i)$ and covariances $\Cov(Z_{ij}, Z_{ik} | \Yb_i)$ required for the inference in the hidden layer. As a consequence, this first step provides the estimates $\widehat{\thetab}$ and $\widehat{\Sigmab}$.

\paragraph{Inference in the hidden layer.} The second step is dedicated to the estimation of $\betab$. The EM algorithm actually deals with the conditional expectation of the complete log-likelihood, namely $\Esp\left(\log p_{\betab, \Sigmab, \thetab}(T, \Zb, \Yb) \; | \; \Yb\right)$. 
As shown in Appendix \ref{app:EM}, this  reduces to
 
\begin{equation} \label{expectation}
    \Esp\left(\log p_{\betab, \Sigmab, \thetab}(T, \Zb, \Yb) \; | \; \Yb\right)
    \simeq
    \sum_{1 \leq j < k \leq p} P_\jk \log \left(\beta_\jk \widehat{\psi}_\jk\right) - \log B + \cst
\end{equation}
 
where $\widehat{\psi}_\jk$ is the estimate of $\psi_\jk$ defined in Eq.~\eqref{eq:pZfact}, and the '$\cst$' term depends on $\thetab$ and $\Sigmab$ but not on $\betab$. 
$P_\jk$ is the approximate conditional probability (given the data) for the edge $(j, k)$ to be part of the network:
$P_\jk \simeq \prob\{jk \in T \; | \; Y\}$.
It is also shown in Appendix~\ref{app:EM} that $\widehat{\psi}_\jk = (1-\widehat{\rho}_\jk^2)^{-n/2}$, where the estimated correlation $\widehat{\rho}_\jk$ depends on the conditional mean, variance and covariances of the $Z_{ij}$'s provided by the first step.
 Eq.~\eqref{expectation} is maximized via an EM algorithm iterating the calculation of the $P_\jk$ and the maximization with respect to the $\beta_\jk$:
\begin{description}

\item[Expectation step: Computing the $P_\jk$ with tree averaging.] The conditional probability of an edge is simply the sum of the conditional probabilities of the trees that contain this edge. Hence, computing $P_\jk$ amounts to averaging over all spanning trees.
Fig.~\ref{fig:treeaveraging} illustrates the principle of tree averaging for a toy network with $p=4$ nodes. Here, five arbitrary spanning trees $t_1$ to $t_5$ (among the $p^{p-2} = 16$ spanning trees) are displayed, with their respective conditional probability $p(T \mid Y)$. 
The edge $(1, 3)$ has a high conditional probability $P_{13}$ because it is part of likely trees such as $t_3$ and $t_4$, whereas $P_{23}$ is small because the edge $(2, 3)$ is only part of unlikely trees (e.g. $t_1$, $t_2$). \\
Averaging over all spanning trees at the cost of a determinant calculus (i.e. with complexity $O(p^3)$) is possible using the Matrix Tree theorem \citep[][recalled as Theorem~\ref{thm:MTT2} in Appendix~\ref{app:MTT}]{matrixtree}. 
\citet{kirshner} further shows that all the $P_\jk$'s can be computed at once with the same complexity $O(p^3)$, although the calculation may lead to numerical instabilities for large $n$ and $p$.



\begin{figure}%[H]
   \begin{center}
    \begin{tabular}{cccccc}
          \begin{tikzpicture}
  \node[observed] (1) at (0*\edgeunit, 0*\edgeunit) {$1$};
  \node[observed] (2) at (0*\edgeunit, 1*\edgeunit) {$2$};
  \node[observed] (3) at (1*\edgeunit, 1*\edgeunit) {$3$};
  \node[observed] (4) at (1*\edgeunit, 0*\edgeunit) {$4$};

  \draw[edge] (1) to (4); \draw[edge] (2) to (3); \draw[edge] (2) to (4);
  \end{tikzpicture}
 &
          \begin{tikzpicture}
  \node[observed] (1) at (0*\edgeunit, 0*\edgeunit) {$1$};
  \node[observed] (2) at (0*\edgeunit, 1*\edgeunit) {$2$};
  \node[observed] (3) at (1*\edgeunit, 1*\edgeunit) {$3$};
  \node[observed] (4) at (1*\edgeunit, 0*\edgeunit) {$4$};
  \draw[edge] (1) to (3); \draw[edge] (2) to (3); \draw[edge] (2) to (4);
  \end{tikzpicture}
 &
          \begin{tikzpicture}
  \node[observed] (1) at (0*\edgeunit, 0*\edgeunit) {$1$};
  \node[observed] (2) at (0*\edgeunit, 1*\edgeunit) {$2$};
  \node[observed] (3) at (1*\edgeunit, 1*\edgeunit) {$3$};
  \node[observed] (4) at (1*\edgeunit, 0*\edgeunit) {$4$};
  \draw[edge] (1) to (2); \draw[edge] (1) to (3); \draw[edge] (2) to (4);
  \end{tikzpicture}
 &
          \begin{tikzpicture}
  \node[observed] (1) at (0*\edgeunit, 0*\edgeunit) {$1$};
  \node[observed] (2) at (0*\edgeunit, 1*\edgeunit) {$2$};
  \node[observed] (3) at (1*\edgeunit, 1*\edgeunit) {$3$};
  \node[observed] (4) at (1*\edgeunit, 0*\edgeunit) {$4$};
  \draw[edge] (1) to (2); \draw[edge] (1) to (3); \draw[edge] (1) to (4);
  \end{tikzpicture}
 &
          \begin{tikzpicture}
  \node[observed] (1) at (0*\edgeunit, 0*\edgeunit) {$1$};
  \node[observed] (2) at (0*\edgeunit, 1*\edgeunit) {$2$};
  \node[observed] (3) at (1*\edgeunit, 1*\edgeunit) {$3$};
  \node[observed] (4) at (1*\edgeunit, 0*\edgeunit) {$4$};
  \draw[edge] (1) to (3); \draw[edge] (2) to (4); \draw[edge] (3) to (4);
  \end{tikzpicture}
 \\
        $t_1: 2.1\%$ & 
        $t_2: 3.5\%$ & 
        $t_3: 34.1\%$ & 
        $t_4: 15.6\%$ & 
        $t_5:  <.1\%$ \\ \\
        & 
          \begin{tikzpicture}
  \node[observed] (1) at (0*\edgeunit, 0*\edgeunit) {$1$};
  \node[observed] (2) at (0*\edgeunit, 1*\edgeunit) {$2$};
  \node[observed] (3) at (1*\edgeunit, 1*\edgeunit) {$3$};
  \node[observed] (4) at (1*\edgeunit, 0*\edgeunit) {$4$};

  \draw[-, line width=2pt, color=black] (1) to (2); 
  \draw[-, line width=4pt, color=black] (1) to (3); 
  \draw[-, line width=2pt, color=black] (1) to (4);
  \draw[-, line width=.5pt, color=black] (2) to (3); 
  \draw[-, line width=4pt, color=black] (2) to (4); 
  \draw[-, line width=.1pt, color=black] (3) to (4);
  \end{tikzpicture}

% > round(50*Pedge)/10
%      [,1] [,2] [,3] [,4]
% [1,]  0.0  2.6  4.8  3.0
% [2,]  2.6  0.0  0.5  4.1
% [3,]  4.8  0.5  0.0  0.0
% [4,]  3.0  4.1  0.0  0.0
% > Pedge
%           [,1]       [,2]         [,3]         [,4]
% [1,] 0.0000000 0.52246119 9.540435e-01 6.057683e-01
% [2,] 0.5224612 0.00000000 9.751246e-02 8.201690e-01
% [3,] 0.9540435 0.09751246 0.000000e+00 4.565667e-05
% [4,] 0.6057683 0.82016896 4.565667e-05 0.000000e+00
 &
        \qquad \qquad &
          \begin{tikzpicture}
  \node[observed] (1) at (0*\edgeunit, 0*\edgeunit) {$1$};
  \node[observed] (2) at (0*\edgeunit, 1*\edgeunit) {$2$};
  \node[observed] (3) at (1*\edgeunit, 1*\edgeunit) {$3$};
  \node[observed] (4) at (1*\edgeunit, 0*\edgeunit) {$4$};

  \draw[edge] (1) to (2); \draw[edge] (1) to (3); \draw[edge] (1) to (4); \draw[edge] (2) to (4);
  \end{tikzpicture}
 \\
        \multicolumn{3}{c}{Edge conditional probabilities} & Estimated graph \\
    \end{tabular}
    \caption{Tree averaging principle. 
    \textit{Top:} 5 spanning trees with 4 nodes  $(t_1, \dots t_5)$, with their respective conditional probability given the data $P(T = t \mid Y)$.
    \textit{Bottom left:} Weighted graph resulting from tree averaging. Each edge  has width proportional to its conditional probability. \textit{ Bottom right:} Estimated graph (obtained by thresholding edge probabilities) is not a tree.}
    \label{fig:treeaveraging}
   \end{center}
\end{figure}

\item[Maximization step: Estimating the $\beta_\jk$.] 
This step is not straightforward, as the normalizing constant $B = \sum_T \prod_{jk \in T} \beta_\jk$ involves all $\beta_\jk$'s. We propose an exact maximization built upon the Matrix Tree theorem (see Appendix~\ref{app:EM}). 
\end{description}


\paragraph{Algorithm output: edge scoring and network inference} 
%As a side product, 
EMtree provides the (approximate) conditional probability $P_\jk$ for each edge $(j, k)$ to be part of the network. 
Whenever an actual inferred network $\widehat{G}$ is needed (e.g. for a graphical purpose), it can be obtained by thresholding the $P_\jk$ (see Fig.~\ref{fig:treeaveraging}, bottom right). Because we are dealing with trees, a natural threshold is the density of a spanning tree, which is  $2/p$.
More robust results can be obtained using a resampling procedure similar to the stability selection proposed by \citet{LRW10}. It simply consists in sampling a series of subsamples $s = 1 \dots S$, to get an estimate $\widehat{G}^s$ from each of them and to collect the selection frequency for each edge. Again, these edge selection frequencies can be thresholded if needed.