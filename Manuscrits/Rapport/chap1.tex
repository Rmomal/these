%network inference in ecology
 \section{Graphical Models}
 
 \subsection{Definitions}
 A graph is defined as a pair $\G=(V,E)$ such that $V$ is a finite set of vertices, and the set of edges $E$ is a subset of $V\times V$ such that vertices are not linked to themselves and there are no multiple edges between two vertices. In the litterature, $V$ can be composed of both quantitative and qualitative variables. However here we  only use quantitative variables and so $V$ is called pure. The following definitions are adapted from \citet{Lau96}.
 \begin{definition}[Subgraphs]
 Let $A$ be a subset of the vertex set $V$. Then the subgraph $\G_A$ is obtained from $\G$ by keeping edges with both endpoints in $A$.
 \end{definition}
 
 \begin{definition}[Cliques and complete subsets]
 A subset of a graph is complete if all the vertices it contains are linked. A complete subset of maximal size is called a clique.
 \end{definition}
 
 \begin{definition}[proper decomposition]
 A triple $(A, B, C)$ of disjoint subsets of the pure vertex set $V$ of an undirected graph $\G$ form a decomposition of $\G$ if $V=A\cup B \cup C$, and if $C$ verifies:
 \begin{enumerate}[label=(\roman*)]
 \item $C$ is a complete subset of  $V$,
 \item $C$ separates $A$ from $B$.
 \end{enumerate}
 If $A$ and $B$ are both non-empty, the decomposition is proper.
 \end{definition}
 
 \begin{definition}[decomposable graph]
 An undirected graph is decomposable if it is complete, or if there exists a proper decomposition $(A, B, C)$ into decomposable subgraphs $\G_{A\cup B}$ and $\G_{B\cup C}$.
 \end{definition}
 This recursive definition thus states that a decomposable graph can be successively decomposed into its cliques.
 
 \begin{definition}[perfect sequence]
 Let $B_1,...,B_k$ be a sequence of subsets of the pure set of vertex $V$ of the undirected graph $\G$. Let $H_j=B_1\cup .. \cup B_j$, $S_j = H_{j-1} \cap B_J$. The sequence is perfect if it satisfies the following conditions:
 \begin{enumerate}[label=(\roman*)]
 \item for all $i>1$ there is a $j<i$ such that $S_i \subseteq B_j$ (running intersection property),
 \item each $S_i$ is a complete subset
 \end{enumerate}
 $H_j$ are called the histories and $S_j$ the separators.
 \end{definition}
% \begin{definition}[perfect numbering]
% A perfect numbering of the vertices $V$ of $\G$ is a numbering $\alpha_1,...,\alpha_k$ such that the sequence $B_1,...,B_k$ with 
% $$B_j = cl(\alpha_j) \cap \{\alpha_1,...,\alpha_j\}, j\geq 1 $$ is a perfect sequence. This implies that each $B_j$ is a complete subset.
% \end{definition}
 \begin{definition}[multiplicity of a separator]
 The multiplicity $\nu (S)$ of the separator $S$ is an index counting the number of times $S$ occurs in a perfect sequence.
 \end{definition}
 \citet{Lau96} further demonstrates that the set of cliques of an undirected decomposable graph can be numbered to form a perfect sequence. 
 
 \subsection{Characterization of conditional independance}
\begin{itemize}
\item conditional independance
\item markov properties
\item factorisation
\item hamersay clifford : conclusion si on trouve une factorisation, on a un graphe
\end{itemize}
 
 \subsection{Gaussian Graphical Models}
  \begin{itemize}
 \item magical factorization on non-nul terms of precision matrix
 \item utilisation des définitions pour les expressions de la densité et precision K et det K
 \item ML estimators
 \end{itemize}

  \subsection{Spanning trees}
  \begin{itemize}
  \item definition and properties (multiplicity)
  \item specific writing of ML estimators ? or in article
  \item algebra 
  \end{itemize}
  
 We define the Laplacian matrix $\Qbf$ of a symmetric matrix $\Wbf=[w_{jk} ]_{1\leq j,k\leq p}$ as follows :

\[
 [\Qbf]_{jk}  =\begin{cases}
    -w_{jk}  & 1\leq j<k \leq p\\
    \sum_{u=1}^p w_{ju} & 1\leq j=k \leq p.
    \end{cases}
\]
 
We further denote $\Wbf^{uv}$ the matrix $\Wbf$ deprived from its $u$th row and $v$th column and we remind that the $(u, v)$-minor of $\Wbf$ is the determinant of this deprived matrix, that is $|\Wbf^{uv}|$.

\begin{theorem}[Matrix Tree Theorem  \cite{matrixtree,MeilaJaak}] \label{thm:MTT}
    For any symmetric weight matrix W with all positive entries, the sum over all spanning trees of the product of the weights of their edges is equal to any minor of its Laplacian. That is, for any $1 \leq u, v \leq p$,
 
   \[
    W := \sum_{T\in\mathcal{T}} \prod_{(j, k)\in T} w_{jk} = |\Qbf^{uv}|.
    \]
   
\end{theorem}    

In the following, without loss of generality, we will choose $\Qbf^{11}$. As an extension of this result, \cite{MeilaJaak} provide a close form expression for the derivative of $W$ with respect to each entry of $\Wbf$. 

\begin{lemma} [\cite{MeilaJaak}] \label{lem:Meila}
    Define the entries of the symmetric matrix $\Mbf$ as
 
\[    
 [\Mbf]_{jk} =\begin{cases}
    \left[(\Qbf^{11})^{-1}\right]_{jj} + \left[(\Qbf^{11})^{-1}\right]_{kk} -2\left[(\Qbf^{11})^{-1}\right]_{jk} & 1< j<k \leq p\\
    \left[(\Qbf^{11})^{-1}\right]_{jj} & k=1, 1< j \leq p  \\
    0 &  j=k .
    \end{cases}
\]
 
it holds that
 
$$
\partial_{w_{jk}} W = [\Mbf]_{jk}  \times W.
$$
\end{lemma}
\begin{lemma} [\cite{kirshner}] \label{lem:Kirshner}
    Let $p_W$ be a distribution on the space of spanning trees, such that $p_W(T)=\prod_{kl\in T} w_{kl} / W$, where $W$ is defined as in Theorem \ref{thm:MTT}. Taking the symmetric matrix $\Mbf$ as defined in Lemma  \ref{lem:Meila}, the probability for an edge $kl$ to be in the tree $T$ writes:
 
$$\mathds{P}\{kl\in T\} = \sum_{T\in \mathcal{T}} p_W(T)= w_{kl}\: \Mbf_{kl}$$
\end{lemma}

  \section{Incomplete data inference}
  here different type of data that are incomplete: counts, gaussian latent variables, trees (discrete )
  partie sur modèles à variables latentes ?

 \subsection{Expectation-Maximization algorithm}
 
 \subsection{Variational version}
  
\section{Statistical modeling for network inference}
  inférence réseaux vs inférence des paramètres (on n'observe pas le réseau)
 \subsection{Modeling count data}
PLN
 JSDM ?
\subsection{State of the art}
 