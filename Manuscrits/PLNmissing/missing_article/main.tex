\documentclass[a4paper,11pt]{article}

\usepackage[utf8]{inputenc}
\usepackage{amsmath,amssymb,amsthm,mathrsfs,amsfonts,dsfont,bm} 
\usepackage[dvipsnames,table,xcdraw]{color,xcolor}
\usepackage{array,graphicx}
\usepackage{hyperref}
\usepackage{float, wrapfig}
\usepackage{empheq}
\usepackage[margin=2.5cm]{geometry}
\usepackage[toc]{appendix}
\usepackage{enumerate}
\usepackage{mathtools}  
\usepackage{multirow}
\usepackage{tikz}
\usepackage[round, sort]{natbib}
\usepackage{setspace}
\usepackage[affil-it]{authblk}
\usepackage[french,english]{babel}
\usetikzlibrary{shapes,backgrounds,arrows,automata,snakes,shadows,positioning, mindmap}

\newcommand{\edgeunit}{1.5}
\newcommand{\nodesize}{1em}
%\newcommand{\edgeunit}{4*\nodesize}
\newcommand{\length}{1}
\tikzstyle{covariate}=[draw, rectangle, minimum width=\nodesize, minimum height=\nodesize, inner sep=0, color=black]
\tikzstyle{covmiss}=[draw, minimum width=\nodesize, minimum height=\nodesize, inner sep=0, color=gray, text=gray]
\tikzstyle{observed}=[draw, circle, minimum width=\nodesize, inner sep=0, color=black]
\tikzstyle{edge}=[-, line width=1pt, color=black]
\tikzstyle{edgemiss}=[-, line width=1pt, dashed, color=gray]
\tikzstyle{variable}=[scale=0.9,rectangle,draw=white,transform shape,fill=white,font=\Large]
    

%%%%%%%%%%%%%%%%%%%%%%%%%%%%%%%%%%%%%%%%%%%%%
%% Environment
\renewcommand{\appendixname}{Annex}
\newtheorem{theorem}{Theorem}
\newtheorem{lemma}{Lemma}
\newtheorem{remark}{Remark}
\renewcommand{\abstractname}{Summary}
\newcommand{\review}[1]{\textcolor{black}{#1}}
\newcommand{\validSR}[1]{\textcolor{black}{#1}}
\newcommand{\questSR}[1]{\textcolor{black}{[#1]}}
\newcommand{\newmodif}[1]{\textcolor{black}{#1}}
\newcommand{\remove}[1]{\textcolor{gray}{#1}}
\newcommand\independent{\protect\mathpalette{\protect\independenT}{\perp}}
\newcommand{\STAB}[1]{\begin{tabular}{@{}c@{}}#1\end{tabular}}

% Functions
\def\independenT#1#2{\mathrel{\rlap{$#1#2$}\mkern2mu{#1#2}}}
\DeclareMathOperator*{\argmax}{arg\,max}
\DeclareMathOperator*{\Esp}{\mathbb{E}}
\DeclareMathOperator*{\Var}{\mathbb{V}}
\DeclareMathOperator*{\Cov}{\mathbb{C}\text{ov}}
\DeclareMathOperator*{\prob}{\mathds{P}}
\DeclareMathOperator*{\probt}{\widetilde{\mathds{P}}}

% Notations pairs
\newcommand\jk{{jk}}

% Notations cal
\newcommand\Ccal{\mathcal{C}}
\newcommand\Ncal{\mathcal{N}}
\newcommand\Pcal{\mathcal{P}}
\newcommand\Tcal{\mathcal{T}}
\newcommand\mt{\widetilde{m}}
\newcommand\St{\widetilde{S}}
\newcommand\mbt{\widetilde{\bf m}}
\newcommand\Sbt{\widetilde{\bf S}}

% Notations bf
\newcommand\gammab{{\boldsymbol{\gamma}}}
\newcommand\betab{{\boldsymbol{\beta}}}
\newcommand\thetab{{\boldsymbol{\theta}}}
\newcommand\Sigmab{{\boldsymbol{\Sigma}}}
\newcommand\cst{\text{cst}}
\newcommand\Ob{{\bf O}}
\newcommand\Hb{{\bf H}}
\newcommand\Mb{{\bf M}}
\newcommand\Qb{{\bf Q}}
\newcommand\Wb{{\bf W}}
\newcommand\Xb{{\bf X}}
\newcommand\xb{{\bf x}}
\newcommand\Yb{{\bf Y}}
\newcommand\Zb{{\bf Z}}
\newcommand\zb{{\bf z}}



% Notations tilde
\newcommand\Pt{\widetilde{P}}
\newcommand\pt{\widetilde{p}}
\newcommand\et{\widetilde{\mathds{E}}}
\newcommand\e{{\mathds{E}}}
% TikZ
\newcommand{\nodesize}{1em}
\newcommand{\edgeunit}{4*\nodesize}
\tikzstyle{covariate}=[draw, rectangle, minimum width=\nodesize, minimum height=\nodesize, inner sep=0, color=black]
\tikzstyle{covmiss}=[draw, minimum width=\nodesize, minimum height=\nodesize, inner sep=0, color=gray, text=gray]
\tikzstyle{observed}=[draw, circle, minimum width=\nodesize, inner sep=0, color=black]
\tikzstyle{edge}=[-, line width=1pt, color=black]
\tikzstyle{edgemiss}=[-, line width=1pt, dashed, color=gray]

\newcommand{\argmin}{\arg\!\min}
\newcommand{\argmax}{\arg\!\max}
\newcommand*\widefbox[1]{\fbox{\hspace{3em}#1\hspace{3em}}}
\newcommand*\lesswidefbox[1]{\fbox{\hspace{2em}#1\hspace{2em}}}
\newcommand{\entr}{\mathcal{H}}
\newcommand{\betabf}{\boldsymbol{\beta}}
\newcommand{\thetabf}{\boldsymbol{\theta}}
\newcommand{\mubf}{\boldsymbol{\mu}}
\newcommand{\Omegabf}{\boldsymbol{\Omega}}
\newcommand{\Sigmabf}{\boldsymbol{\Sigma}}
\newcommand{\zerobf}{\boldsymbol{0}}
\newcommand{\Xbf}{\boldsymbol{X}}
\newcommand{\xbf}{\boldsymbol{x}}
\newcommand{\Ybf}{\boldsymbol{Y}}
\newcommand{\Zbf}{\boldsymbol{Z}}
\newcommand{\Sbf}{\boldsymbol{S}}
\newcommand{\mbf}{\boldsymbol{m}}
\newcommand\Ncal{\mathcal{N}}
\newcommand\Pcal{\mathcal{P}}
\newcommand\Tcal{\mathcal{T}} 
\newcommand{\Esp}{\mathds{E}}
\newcommand{\bound}{\mathcal{J}}

\definecolor{coquelicot}{rgb}{1.0, 0.22, 0.0}
\newcommand{\modifRM}[1]{\textcolor{coquelicot}{#1}}
\newcommand{\modifSR}[2]{\textcolor{gray}{#1}\textcolor{blue}{#2}}
%%%%%%%%%%%%%%%%%%%%%%%%%%%%%%%%%%%%%%%%%%%%%

\title{\textbf{Prise en compte d'un acteur manquant dans l'inférence de réseaux d'interactions d'espèces par mélange d'arbres à partir de données de comptages}}

\author{Raphaëlle Momal$^1$%
  \thanks{Electronic address: \texttt{raphaelle.momal@agroparistech.fr}; Corresponding author}, \hspace{0.3cm} Stéphane Robin$^1$, \hspace{0.3cm} Christophe Ambroise$^2$}
\affil{1: Université Paris-Saclay, AgroParisTech, INRAE, UMR MIA-Paris, Paris, France.\\
2: Laboratoire de Mathématiques et Modélisation d'Évry, 23 bvd de France, Évry, France}

\date{\vspace{-5ex}}
%\date{\today}

\begin{document}
\setlength{\parindent}{0ex}
\maketitle
%%%%%%%%%%%%%%%%%%%%%%%%%%%%%%%%%%%%%%%%%%%%%
 

\selectlanguage{english}
\begin{abstract}
Network inference is used in many areas such as genomics or ecology to infer the structure of conditional independence between covariates, based on the measures of gene expression or species abundance for example. In many experiments, it is likely that not all covariates involved in the network were actually observed. Then observed samples are drawn from a distribution where some unobserved covariates were marginalized.

We introduce a generic staitstical model for network inference from abundance data with missing actor. The model includes fixed effects to take account of environmental covariates  and sampling efforts, as well as correlated random effects to encode species interactions. The correlation structure is that of a gaussian graphical model marginalized on one ore more covariates, corresponding to the missing actors. The inferred network is obtained by averaging on all spanning trees, in a computationnally efficient way.
\paragraph{Key-words: }  graphical models, network inference, missing actor, abundance data, Variational EM algorithm, matrix tree theorem, Poisson log-Normal model
\end{abstract}
 
% intro
%Les réseaux :
%- [ ] C’est quoi
%- [ ] À quoi ça sert (les enjeux, appliqué)
%- [ ] Quelles sont les questions qui se posent (que dit la recherche)
%
%- [ ] Lauritzen pour les nuls
%
%L’inférence de réseaux :
%- [ ] Quelle méthode pour quel réseau
%- [ ] Auxquelles on se compare, pourquoi elles sont comme ça comment elles marchent et ce qui manque

  \section*{Biological context}
 \subsection*{Networks}
 A network is an intuitive object which anyone can easily relate to. It is first of all a graphical tool representing the links between different entities. This helps understand how a system organizes and have a direct image of it. It is also an analysis tool which can unravel sensible information about the system, its structure and the different roles in its organization. Networks are versatile tools that are  used in many domains (e.g. sociology, linguistics, computer sciences, neurosciences, climatology, psychology, etc.) and can take various forms to adapt to each problem.  They can be directed or undirected. Some link entities with multiple kinds of edges (multidimensional), or have different layers (multiplex), others link groups of disconnected objects (multipartite). In biology, the most typical networks simply represent the species (nodes) and their relationships (edges).

Various types of species interactions are studied with networks. For example, their is a rich literature of networks for plant-pollinator and host-parasite relationships in ecology. These species interactions are clearly defined and directly observed in the field. Contacts of pollination or parasitism are counted and networks constructed from these interaction abundances.
\begin{figure}
\centering
\includegraphics[width=0.7\linewidth]{figs/pocock.png}
\caption{Species interaction networks at Norwood Farm, Somerset, UK \citep{PED12,BRM13}.}
\label{pocock}
\end{figure}
 However many mechanisms cannot be observed and may not be well defined. One way to discover them may then be to resort to a more mathematical definition of species interactions. Working with the latter allows the study of community assembly mechanisms with the inference of networks representing guilds of species in community ecology. This type of network is extensively used in genomics for protein-protein interaction network, or gene regulatory networks, or in microbiology to study the output of a metabarcoding experiment assessing the composition of a microbiome. 
 
 \begin{figure}
\centering
\includegraphics[width=0.7\linewidth]{figs/PPInetwork.png}
\caption{Protein-protein interactions between genes involved in schizophrenia \citep{GTH16}.}
\label{PPI}
\end{figure}

  \subsection*{Statistical interactions}
The correlations between the species measures first come to mind as a statistical characterization of an interaction. These are easily obtained, yet their corresponding networks are hard to interpret. Indeed, two covariates correlating with a same third one will appear correlated, even if they have no direct effect on each other\footnote{e.g. the number of covid 19 cases detected correlates with both the real number of cases and the number of tests done on the population, which induces a spurious correlation between the two latter where obviously there is no direct effect of one on the other.}. This phenomena of spurious correlations complicates both the analysis and the interpretation by inducing a very high number of edges  which cannot be categorized as direct or indirect associations between the species. 
 
 Conditional dependencies are then very useful measures of interaction. They describe dependencies between each pair of species conditional on all others. That is, all other species measures kept fixed, their measures should still be correlated. A link between two species can then be interpreted as a direct association. This yields a network of species conditional dependence (link) and independence (absence of link), which is interpretable and falls within the well-studied mathematical framework of graphical models.
 
 \subsection*{Measures on species}
 Networks of statistical interactions are obtained from datasets of repeated measures on a set of species, which can be of various types. Measures can be continuous, as for example the output of gene expression profiling experiments using DNA microarrays, which  are  fluorescences measures from targeted genes. Using a Gaussian approximation, these measures of genes expressions can be used to derive genes regulatory networks.  Measures can also be binary, as in co-occurrence data in ecology, which record the joint presences and absences of a set of species in several sites. \citet{CAM16}.
%développer les données de co-occurence

Abundance data  are joint counts of species in a series of sites (also known as assemblages data in ecology). Recent technologies made this type of data increasingly available. Assemblages data were rare in ecology as they implied intensive sampling efforts, which is now greatly facilitated by camera traps and sensors. In metagenomics, high-throughput sequencing technologies for metabarcoding experiments made it possible  to get joint counts of pseudo-species (operational taxonomic units (OTUs)) abundances.  Both domains work with the same type of output: a dataset of joint (pseudo-)species abundances from different sites or samples.\\
%développer les metabarcoding et OTU


Once the data has been collected, it is very likely that not all species or covariates were observed: there exists missing actors and data is incomplete. In the network, the existence of a missing actor translates into appearance of edges between all the species it should connect with, creating dense cliques of species which are not actually conditionally dependent on one another. A second objective of this work is to take missing actors into account during network inference in order to get more accurate interpretations.

\section*{Network inference}
% ce qui existe et motivation du sujet
\section*{Objectives}
 \subsection*{Graphical models and Trees}
The dedicated framework for the representation of conditional dependency structures are graphical models. Gaussian Graphical Models (GGM) in particular provide with appealing algebraic properties for network inference, which are detailed in \citet{Lau96}. Exploring the set of possible graphs is a non-ending task, and we chose to reduce the searching space to that of spanning trees. This is the sparsest connected structure, and enjoys specific algebraic properties allowing to sum on all possible spanning trees at the cost of a determinant calculus. Our network inference methodology relies on spanning trees and \citet{Lau96} maximum likelihood estimators for multivariate Gaussian graphical models.

%caser le missing actor qq part
 \subsection*{Modeling abundance data}
 As this ideal framework of GGM is not directly applicable to abundance data, there exist two possible ways to proceed: either apply a Gaussian transformation to the data, or rely on Gaussian latent vectors in the framework of Joint Species Distribution Models (JSDM). Our methodology resorts to the latter, and more specifically to the Poisson log-normal distribution to model the counts. This distribution allows easy handling of both covariates and offsets, as well as take overdispersion into account thanks to its Gaussian random parameters. 
 
 
 \subsection*{Estimation procedure} %or algorithm
The central model we adopt involves a Gaussian layer of parameters which is a mixture on all spanning trees. Each component of the mixture is a Gaussian distribution which dependency structure is a spanning tree. This represents two hidden layers of parameters, which we first estimates with an Estimation-Maximization (EM) algorithm. Then to infer missing actors of the network, we resort to a variational (VEM) approach.

\section*{Outline} 
  \subsection*{Chapter 1}
The first chapter details the mathematical tools and technical results for the statistical modeling and the network inference used in Chapters 2 and 3.

   \subsection*{Chapter 2}
   This chapter details a method to infer undirected networks representing conditional statistical dependencies between the species from their joint abundance measures.  The proposed methodology, implemented in the R package \texttt{EMtree},  is compared to state of the art approaches and applied to two empirical datasets from ecology and metagenomics. This chapter has been published as an article in the journal \textit{Methods in Ecology and Evolution} \citep{MRA20}.
   
    \subsection*{Chapter 3}
This chapter details a variational approach to the inference of a missing actor in the network. The reconstruction of missing actor(s) is implemented in the R package \texttt{nestor} and illustrated on two  empirical datasets from ecology. This chapter has been submitted for publication in the \textit{Journal of the Royal Statistical Society: series C (applied statistics)}.
 
  \subsection*{Chapter 4}
This final chapter introduces some natural perspectives of this work. After concluding on the specifics of the developed methodology, natural extension are presented, including network comparison. The inclusion of spatialized data is discussed, as well as the possibility of network inference directly from the observed counts.
 \section*{Notations}
 
 \begin{description}
 \item[Operations:]  \begin{itemize}
     \item[]
 \item[] $|\cdot|$ : matrix determinant
 \item[] $\odot$ : Hadamard product
 \end{itemize}
 \item[Variables:] \begin{itemize}
     \item[]
 \item[] $\Ybf$ :  matrix  of observed counts
 \item[] $\Zbf$ : matrix of latent Gaussian parameters 
 \item[] $\Ubf$ : matrix of latent normalized Gaussian parameters 
 \item[] $\Xbf$ : matrix of covariates 
 \item[] $O$ : matrix of measured offsets
 \end{itemize}
 \item[Dimensions:]\begin{itemize}
     \item[]
 \item[] $n$ : number of samples
 \item[] $p$: number of observed species
 \item[] $r$: number of unobserved species
 \item[] $d$: number of covariates
 \end{itemize}
 \end{description} 
%%%%%%%%%%%%%%%%%%%%%%%%%%%%%%%%%%%%%%%%%%%%%%%%%%%%%%%%%%%%%%%%%%%%%%%%%%%%%%%%
\section{Model} \label{sec:Model}
%%%%%%%%%%%%%%%%%%%%%%%%%%%%%%%%%%%%%%%%%%%%%%%%%%%%%%%%%%%%%%%%%%%%%%%%%%%%%%%%
%%%%%%%%%%%%%%%%%%%%%%%%%%%%%%%%%%%%%%%%%%%%%%%%%%%%%%%%%%%%%%%%%%%%%%%%%%%%%%%%

%%%%%%%%%%%%%%%%%%%%%%%%%%%%%%%%%%%%%%%%%%%%%%%%%%%%%%%%%%%%%%%%%%%%%%%%%%%%%%%%
\subsection{Poisson log-normal and tree-shaped graphical models}
%%%%%%%%%%%%%%%%%%%%%%%%%%%%%%%%%%%%%%%%%%%%%%%%%%%%%%%%%%%%%%%%%%%%%%%%%%%%%%%%

%%%%%%%%%%%%%%%%%%%%%%%%%%%%%%%%%%%%%%%%%%%%%%%%%%%%%%%%%%%%%%%%%%%%%%%%%%%%%%%%
\subsubsection*{Poisson log-normal model.} 
We start with a reminder on the multivariate Poisson log-normal model, with the example of abundance data. The abundances of $p$ species observed on $n$ sites are gathered in the $n \times p$ matrix $\Ybf$ where $Y_ {ij}$ is the count of species $j$ in site $i$, and the row $i$ of $\Ybf$, denoted $\Ybf_i$, is the abundance vector collected on site $i$. A covariate vector $\xbf_i $ with dimension $d$ is also measured on each site $i$ and all covariates are gathered in the $n \times d$ matrix  $\boldsymbol X$. The PLN model states that a (latent) Gaussian vector $\Ubf_i$ of size $p$ with variance matrix $\Rbf = (\rho_{kl})_{kl}$ is associated with each site:
\begin{equation} \label{eq:PLN-Z}
\{\Ubf_i\}_{1 \leq i \leq n} \text{ iid}, \qquad 
\Ubf_1 \sim \Ncal_p(\zerobf, \Rbf),
\end{equation}
the sites being assumed to be independent. To ensure identifiability, we let the diagonal of $\Rbf$ be made of 1's, so $\Rbf$ is actually a correlation matrix.
All latent vectors $\Ubf_i$ are gathered in the $n \times p$ matrix $\Ubf$. The PLN model further assumes that species abundances in all sites are conditionally independent, and that their respective distribution only depends on the environment and the associated latent variable:
\begin{equation} \label{eq:PLN-Y.Z}
\{Y_{ij}\}_{1 \leq i \leq n, 1 \leq j \leq p} \mid \Ubf \text{ independent}, \quad 
Y_{ij} \mid U_{ij} \sim \Pcal\left(\exp(o_{ij} + \xbf_i^\intercal \thetabf_j + \sigma_j U_{ij})\right),
\end{equation}
where $o_{ij}$ is a known offset term which typically accounts for the sampling effort, and $\sigma_j$ is the latent standard deviation associated with species $j$. The vector $d \times 1$ of regression coefficients $\thetabf_j$ describes the environmental effects on species $j$. An important feature of the PLN model is that the sign of the correlation between the observed counts is the same as this of correlation between the latent variables \citep{AiH89}: $\text{sign}(\text{Cor}(Y_{ij}, Y_{ik})) = \text{sign}(\text{Cor}(U_{ij}, U_{ik}))$. 
% The dependence between the species abundances is entirely controlled by the latent dependency structure encoded in the precision matrix $\Omegabf:=\Rbf^{-1}$.

%%%%%%%%%%%%%%%%%%%%%%%%%%%%%%%%%%%%%%%%%%%%%%%%%%%%%%%%%%%%%%%%%%%%%%%%%%%%%%%%
\subsubsection*{Tree-shaped graphical models.} 
Network inference relies on the assumption that few species are directly dependent on one another, meaning that the underlying graphical model is sparse. In the framework of the PLN model, the graphical model of interest rules the distribution of the latent vectors $\Ubf_i$ and is  encoded in the precision matrix $\Omegabf:=\Rbf^{-1}$. A way to foster sparsity is to impose $\Omegabf$ to be faithful to a spanning tree $T$, that is: $\Ubf_1 \sim \Ncal_p(\zerobf, \Omegabf_T^{-1})$ where the non-zero terms of $\Omegabf_T$ correspond to the edges of the tree $T$ . However this hypothesis is very restrictive  as it allows only $p-1$ links among $p$ species \citep{ChowLiu}. A more flexible approach consists in assuming that the latent vectors are drawn from a mixture of Gaussian distributions, each faithful to a tree $T$ \citep{MixtTrees,MeilaJaak,kirshner,SRS19}:
\begin{equation} \label{eq:mixt-Z}
\Ubf_1 \sim \sum_{T \in \Tcal_p} p(T) \Ncal_p(\zerobf, \Omegabf_T^{-1}),
\end{equation}
where $\Tcal_p$ is the set of spanning trees with $p$ nodes.
We further assume that the tree distribution $\{p(T)\}_{T \in \Tcal_p}$ can be written as a product over the edges:
\begin{equation} \label{eq:prob-T}
p(T) = B^{-1} \prod_{jk \in T} \beta_{jk}, \qquad
\text{with} \quad B = \sum_{T \in \Tcal_p} \prod_{jk \in T} \beta_{jk}.
\end{equation}
The weights $\beta_{jk}$ are gathered in the $p \times p$ symmetric matrix $\betabf$ with diagonal zero. Observe that these weights are defined up to a multiplicative constant, so that only $p(p-1)/2 - 1$ of them may vary independently. This PLN model with latent tree-shaped dependency structure is similar to that considered by \cite{MRA20}.

%%%%%%%%%%%%%%%%%%%%%%%%%%%%%%%%%%%%%%%%%%%%%%%%%%%%%%%%%%%%%%%%%%%%%%%%%%%%%%%%
\subsection{Introducing the missing actor} \label{sec:missActor}
%%%%%%%%%%%%%%%%%%%%%%%%%%%%%%%%%%%%%%%%%%%%%%%%%%%%%%%%%%%%%%%%%%%%%%%%%%%%%%%%

%%%%%%%%%%%%%%%%%%%%%%%%%%%%%%%%%%%%%%%%%%%%%%%%%%%%%%%%%%%%%%%%%%%%%%%%%%%%%%%%
\subsubsection*{PLN model with missing actors.} 
We now introduce the concept of missing actors, which corresponds to variables that are involved in the graphical model but are not associated with observed variables. To involve such actors in the model, we assume that a complete latent vector $\Ubf_i$ with dimension $p+r$ is associated with site $i$, where $r$ is the number of missing actors. This complete vector can be decomposed as $\Ubf_i^\intercal = [\Ubf_{Oi}^\intercal \; \Ubf_{Hi}^\intercal]$ where $\Ubf_{Oi}$ (with dimension $p$) corresponds to observed species and $\Ubf_{Hi}$ (with dimension $r$) corresponds to the missing actors.
The complete $n \times (p+r)$ latent matrix $\Ubf$ can be decomposed in the same way as $\Ubf = [\Ubf_O \; \Ubf_H]$, $\Ubf_O$ and $\Ubf_H$ having dimension $n \times p$ and $n \times r$, respectively. \\ 
The model we consider states that
\begin{enumerate}[label=\roman*]
\item the complete latent vectors $\Ubf_i$ are all iid and distributed according to a mixture similar to \eqref{eq:mixt-Z} and \eqref{eq:prob-T} but with Gaussian distributions (and matrices $\Omegabf_T$ and $\betabf$) of dimension $(p+r)$, and trees drawn from $\Tcal_{p+r}$;
\item  the abundances $Y_{ij}$ of the  $p$ observed species are distributed according to \eqref{eq:PLN-Y.Z}, replacing $\Ubf$ with $\Ubf_O$,
\end{enumerate}

\begin{figure}[H]
 \begin{center}
	\begin{tikzpicture}	
      \tikzstyle{every edge}=[-,>=stealth',auto,thin,draw]
		\node (A1) at (0.625*\length, 2*\length) {$T$};
		\node (A2) at (0*\length, 1*\length) {$\Ubf_O$};
		\node (A3) at (1.25*\length, 1*\length) {$\Ubf_H   $};
		\node (A4) at (0*\length, 0*\length) {$\Ybf$};
		\draw (A1) edge [->] (A2);
        \draw (A1) edge [->] (A3);
        \draw (A2) edge  (A3);
        \draw (A2) edge [->] (A4);
	\end{tikzpicture} 
 \caption{Graphical model linking the count data $\Ybf$, the latent layer of Gaussian parameters $\Ubf=(\Ubf_O,\Ubf_H)$, and the latent tree $T$.}
  \label{fig:MGmodel}
    \end{center}
\end{figure}

In the sequel, we shall refer to the elements of $\Ubf_O$ and $\Ubf_H$ respectively as 'observed' and 'hidden' (or 'missing') latent variables, whereas obviously none of them are actually observed. Figure \ref{fig:MGmodel} displays the graphical model of the quadruplet $(T, \Ubf_O, \Ubf_H, \Ybf)$. The observed data $\Ybf$ still arise from an PLN model, but the graphical model of the observed latent $\Ubf_O$ may not be sparse due to the marginalization over the hidden latent $\Ubf_H$. Our main goal is to infer the dependency structure of the complete latent vectors, that is to estimate the elements of the matrices $\Omegabf_T$ and the edges weights $\betabf$. The latent dependency structure is similar to this considered by \cite{RAR19}, but the inference strategy much differs, because of the additional hidden layer.

%%%%%%%%%%%%%%%%%%%%%%%%%%%%%%%%%%%%%%%%%%%%%%%%%%%%%%%%%%%%%%%%%%%%%%%%%%%%%%%%
\subsubsection*{Identifiability restriction.} 
The proposed model only makes sense because the graphical model of the complete latent vectors $\Ubf_i^\intercal = [\Ubf_{Oi}^\intercal \; \Ubf_{Hi}^\intercal]$ is supposed to be sparse. Missing actors could obviously not be identified from a regular PLN model, without restriction on the precision matrix $\Omegabf$, as only the marginal precision matrix of the $\Ubf_{Oi}$ could be recovered. Still, to ensure identifiability we impose the same restriction as \cite{RAR19} that  missing latent variables are not connected with each other (the block corresponding to $\Ubf_H \times \Ubf_H$ is diagonal in each $\Omegabf_{T}$).
%\CA{However \SR{here}{} we do not need the additional assumption that all precision matrices $\Omegabf_T$ borrow their non-null elements from a same matrix.}{} \SR{}{[{\sl Faut-il mettre cette dernière phrase alors que rien ne le suggère ? Ou alors préciser, 'as opposed to \cite{RAR19}'}]}
%\begin{description}
%\item[(A)] All the precision matrices $\Omegabf_T$ (for $T \in \Tcal_{p+r}$) borrow their elements from a same matrix $\Omegabf$. Namely:
%$$\forall T \in \Tcal_{p+r}, \qquad [\Omegabf_T]_{j,k} = 
%\left\{ \begin{array}{ll}
%    [\Omegabf]_{j, k} & \text{if } (j, k) \in T \\
%    0 & \text{otherwise}.
%\end{array}\right.$$
%and their conditional variance is set to one. Namely, 
%$$\forall p+1 \leq h, \ell \leq p+r, \qquad [\Omegabf]_{h, \ell} = 
%\left\{ \begin{array}{ll}
%    1 & \text{if } h = \ell \\
%    0 & \text{otherwise}.
%\end{array}\right.$$
 
%\end{description}
%Assumption ({\bf A}) obviously avoids to infer $|\Tcal_{p+r}| = (p+r)^{p+r-2}$ independent (sparse) precision matrices. Assumption ({\bf B}) ensures identifiability, especially regarding the scaling of the missing latent variables.

%To summarize, the model defined in Section \ref{sec:missActor} involves $p d$ regressions coefficients (gathered in $\thetabf$), $
%(p+r)(p+r+1)/2 - r(r+1)/2 = 
%p(p+1)/2 + r p$ conditional covariances (gathered in $\Omegabf$), and $(p+r)(p+r-1)/2 - 1$ independent edge weights (gathered in $\betabf$).

%Finally, the original PLN model only concerns covariates $\Ybf$ and $\Ubf_O$. The use of a dependency structure with mixture of trees allows a sparse and efficient inference, and missing actors are accounted for in covariate $\Ubf_H$.
We now describe how to infer the model parameters. We gather the edges weights into the $p \times p$ matrix $\betab$ and the vectors of regression coefficients into a $d \times p$ matrix $\thetab$. The $p \times p$ matrix $\Sigmab$ contains the variances and covariances between the coordinates of each latent vector $\Zb_i$. Hence, the set of parameters to be inferred is $(\betab, \Sigmab, \thetab)$.

\paragraph{Likelihood.} 
The model described above is an incomplete data model, as it involves two hidden layers: the random tree $T$ and the latent Gaussian vectors $\Zb_i$. The most classical approach to achieve maximum likelihood inference in this context is to use the Expectation-Maximization algorithm \citep[EM:][]{DLR77}. Rather than the likelihood of the observed data $p(\Yb)$, the EM algorithm deals with the often more tractable likelihood $p(T, \Zb, \Yb)$ of the complete data (which consists of both the observed and the latent variables). It can be decomposed as 
 
\begin{equation} \label{eq:PTZY}
    p_{\betab, \Sigmab, \thetab}(T, \Zb, \Yb) = p_{\betab}(T) \times p_{\Sigmab}(\Zb \; | \; T) \times p_{\thetab}(\Yb \; | \; \Zb),
\end{equation}
 
where the subscripts indicate on which parameter each distribution depends. \\
Observe that the dependency structure between the species is only involved in the first two terms, whereas the third term only depends on the regression coefficients $\thetab$. 
We take advantage of this decomposition to propose a two-stage estimation algorithm. The first stage deals with the observed layer $p_{\thetab}(\Yb \; | \; \Zb)$, the second with the two hidden layers $p_{\betab}(T)$ and  $p_{\Sigmab}(\Zb \; | \; T)$. The network inference itself takes place in the second step.

\paragraph{Inference in the observed layer.} 
The variational EM (VEM) algorithm that provides an estimate of the regression coefficients matrix $\thetab$ is described in Appendix \ref{app:VEM} (along with a reminder on EM and VEM). It also provides the (approximate) conditional means $\Esp(Z_{ij} | \Yb_i)$, variances $\Var(Z_{ij} | \Yb_i)$ and covariances $\Cov(Z_{ij}, Z_{ik} | \Yb_i)$ required for the inference in the hidden layer. As a consequence, this first step provides the estimates $\widehat{\thetab}$ and $\widehat{\Sigmab}$.

\paragraph{Inference in the hidden layer.} The second step is dedicated to the estimation of $\betab$. The EM algorithm actually deals with the conditional expectation of the complete log-likelihood, namely $\Esp\left(\log p_{\betab, \Sigmab, \thetab}(T, \Zb, \Yb) \; | \; \Yb\right)$. 
As shown in Appendix \ref{app:EM}, this  reduces to
 
\begin{equation} \label{expectation}
    \Esp\left(\log p_{\betab, \Sigmab, \thetab}(T, \Zb, \Yb) \; | \; \Yb\right)
    \simeq
    \sum_{1 \leq j < k \leq p} P_\jk \log \left(\beta_\jk \widehat{\psi}_\jk\right) - \log B + \cst
\end{equation}
 
where $\widehat{\psi}_\jk$ is the estimate of $\psi_\jk$ defined in Eq.~\eqref{eq:pZfact}, and the '$\cst$' term depends on $\thetab$ and $\Sigmab$ but not on $\betab$. 
$P_\jk$ is the approximate conditional probability (given the data) for the edge $(j, k)$ to be part of the network:
$P_\jk \simeq \prob\{jk \in T \; | \; Y\}$.
It is also shown in Appendix~\ref{app:EM} that $\widehat{\psi}_\jk = (1-\widehat{\rho}_\jk^2)^{-n/2}$, where the estimated correlation $\widehat{\rho}_\jk$ depends on the conditional mean, variance and covariances of the $Z_{ij}$'s provided by the first step.
 Eq.~\eqref{expectation} is maximized via an EM algorithm iterating the calculation of the $P_\jk$ and the maximization with respect to the $\beta_\jk$:
\begin{description}

\item[Expectation step: Computing the $P_\jk$ with tree averaging.] The conditional probability of an edge is simply the sum of the conditional probabilities of the trees that contain this edge. Hence, computing $P_\jk$ amounts to averaging over all spanning trees.
Fig.~\ref{fig:treeaveraging} illustrates the principle of tree averaging for a toy network with $p=4$ nodes. Here, five arbitrary spanning trees $t_1$ to $t_5$ (among the $p^{p-2} = 16$ spanning trees) are displayed, with their respective conditional probability $p(T \mid Y)$. 
The edge $(1, 3)$ has a high conditional probability $P_{13}$ because it is part of likely trees such as $t_3$ and $t_4$, whereas $P_{23}$ is small because the edge $(2, 3)$ is only part of unlikely trees (e.g. $t_1$, $t_2$). \\
Averaging over all spanning trees at the cost of a determinant calculus (i.e. with complexity $O(p^3)$) is possible using the Matrix Tree theorem \citep[][recalled as Theorem~\ref{thm:MTT2} in Appendix~\ref{app:MTT}]{matrixtree}. 
\citet{kirshner} further shows that all the $P_\jk$'s can be computed at once with the same complexity $O(p^3)$, although the calculation may lead to numerical instabilities for large $n$ and $p$.



\begin{figure}%[H]
   \begin{center}
    \begin{tabular}{cccccc}
        \input{figs/FigTreeAveraging-p4-tree1-seed2} &
        \input{figs/FigTreeAveraging-p4-tree2-seed2} &
        \input{figs/FigTreeAveraging-p4-tree3-seed2} &
        \input{figs/FigTreeAveraging-p4-tree4-seed2} &
        \input{figs/FigTreeAveraging-p4-tree5-seed2} \\
        $t_1: 2.1\%$ & 
        $t_2: 3.5\%$ & 
        $t_3: 34.1\%$ & 
        $t_4: 15.6\%$ & 
        $t_5:  <.1\%$ \\ \\
        & 
        \input{figs/FigTreeAveraging-p4-avgtree-seed2} &
        \qquad \qquad &
        \input{figs/FigTreeAveraging-p4-graph-seed2} \\
        \multicolumn{3}{c}{Edge conditional probabilities} & Estimated graph \\
    \end{tabular}
    \caption{Tree averaging principle. 
    \textit{Top:} 5 spanning trees with 4 nodes  $(t_1, \dots t_5)$, with their respective conditional probability given the data $P(T = t \mid Y)$.
    \textit{Bottom left:} Weighted graph resulting from tree averaging. Each edge  has width proportional to its conditional probability. \textit{ Bottom right:} Estimated graph (obtained by thresholding edge probabilities) is not a tree.}
    \label{fig:treeaveraging}
   \end{center}
\end{figure}

\item[Maximization step: Estimating the $\beta_\jk$.] 
This step is not straightforward, as the normalizing constant $B = \sum_T \prod_{jk \in T} \beta_\jk$ involves all $\beta_\jk$'s. We propose an exact maximization built upon the Matrix Tree theorem (see Appendix~\ref{app:EM}). 
\end{description}


\paragraph{Algorithm output: edge scoring and network inference} 
%As a side product, 
EMtree provides the (approximate) conditional probability $P_\jk$ for each edge $(j, k)$ to be part of the network. 
Whenever an actual inferred network $\widehat{G}$ is needed (e.g. for a graphical purpose), it can be obtained by thresholding the $P_\jk$ (see Fig.~\ref{fig:treeaveraging}, bottom right). Because we are dealing with trees, a natural threshold is the density of a spanning tree, which is  $2/p$.
More robust results can be obtained using a resampling procedure similar to the stability selection proposed by \citet{LRW10}. It simply consists in sampling a series of subsamples $s = 1 \dots S$, to get an estimate $\widehat{G}^s$ from each of them and to collect the selection frequency for each edge. Again, these edge selection frequencies can be thresholded if needed.
The inference of species interaction network is a challenging task, for which a series of methods have been proposed in the past years. Abundance data seems to be a promising source of information for this purpose. Here we adopt the formalism of graphical models to define a probabilistic model-based framework for the inference of such networks from abundance data.
Using a model-based approach offers several important advantages. First, it enables easy and explicit integration of environmental and experimental effects.  These could be modeled in a more flexible way using generalized additive models, which include non-linear effects \citep{hastie2017generalized}. 
Then, as it also relies on a formal statistical definition of a \textsl{species interaction network} in the context of graphical models, accounting for abiotic effects and modeling species interactions are two clearly defined and distinguished goals. Finally, all the underlying assumptions are explicitly stated in the model definition itself, and can therefore be discussed and criticized. \\



We developed an efficient method to infer sparse networks, which combines a multivariate Poisson mixed model for the joint distribution of abundances, with an averaging over all spanning trees to efficiently infer direct species interactions. As we do consider a mixture over all spanning trees, our approach remains flexible and can infer most types of statistical dependencies. An EM algorithm (EMtree) maximizes the likelihood of the result and returns each edge probability to be part of the network. An optional resampling step increases network robustness.

\modif{A simulation study in a heterogeneous environment  demonstrates  that EMtree  compares very well to alternative approaches. The proposed model can take all kind of covariates into account, which when ignored  can have  dramatic effects  on the inferred network structure, as showed here on empirical datasets.  Experiments on simulated data and illustrations also demonstrate that EMtree  computational cost remains very reasonable.}

\modif{Alternative methods used in this work all rely on an optimized threshold to tell an edge presence. This particular threshold is obtained after testing a grid of possible values which all yield a different network, and altogether build a path. Making this path available to the user is useful, as the final threshold might need modification and it gives the possibility to build edges scores  and get more than a binary result. We found few recent approaches doing this, which prevented us to study their performance in a way that did not impose a threshold.}\\

The proposed methodology could be extended in several ways.
\modif{Species abundances and interactions indeed vary across space, and depend on local conditions \citep{PCM12,PSG15}. This can either be considered as nuisance parameter or as feature of interest. In the first case, the method could be extended to account for the spatial autocorrelation of sampling sites, to obtain a "regional" interaction network corrected for this effect, i.e. assuming the network is the same in all sites. If of interest, variation across space and local conditions could be studied by comparing networks inferred from the different sampling locations. Networks comparison is a wide and interesting question and tools lack to check which edges are shared by a set of networks. The approach introduced by \citet{SR17} could be adapted to  EMtree framework.} Lastly,  It is also very likely that not all covariates nor even all species have been measured or observed. Another extension may therefore be to detect ignored covariates or missing species. To this purpose EMtree could probably be combined with the approach developed by \citet{RAR19} to identify missing actors. 

\begin{appendix}
\newpage
\bibliographystyle{apsrev} %tested plainnat
%\bibliographystyle{apsrev} % tested apsrev
\bibliography{biblio.bib}
\end{appendix}
\end{document}
