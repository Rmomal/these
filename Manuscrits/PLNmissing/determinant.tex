\documentclass[11pt,a4paper]{article}
\usepackage[utf8]{inputenc}
\usepackage{amsmath}
\usepackage{amsfonts}
\usepackage{amssymb}
\usepackage{graphicx}
\usepackage[left=2cm,right=2cm,top=2cm,bottom=2cm]{geometry}
 
\begin{document}

\section{Inégalité d'Hadamard pour le déterminant}
Le déterminant d'une matrice $W\in \mathcal{M}_{q,q}$ est borné par le produit des normes des colonnes :
$$\det(W) < \prod_i ||w_i|| $$

S'il existe une borne $b$ telle que $|w_{ij}|\leq b$ pour tout $i$ et $j$, alors
$$\det(W) \leq b^q  q^{q/2} $$

Le déterminant d'une matrice définie positive est majoré par le produit de ses termes diagonaux:
$$\det(W) \leq \prod_i w_{ii} $$

\section{Application au Matrix Tree}

On considère $W_g$ la matrice des poids $\tilde{\beta}$, défini par :
\begin{align*}
\log(\tilde{\beta}_{jk})& = \log\beta_{jk} + \frac{n\alpha}{2} (\log \varphi_{jk} - \omega_{jk} \tilde{\sigma}_{jk})\\
&= O(\alpha n)
\end{align*}

On souhaite borner le déterminant du Laplacien de $W_g$.

\begin{align*}
\det(\mathcal{L}_{uv}(W_g)) &\leq \prod_i  [\mathcal{L}_{uv}(W_g)]_{ii}\\
&\leq \prod_{i=1}^{q-1} \sum_{i=1}^q \exp(\alpha n)\\
& \leq (q \exp(\alpha n) )^{q-1}
\end{align*}

En notant $\Delta$ la précision machine, on souhaite donc 
$$(q \exp(\alpha n) )^{q-1} \leq \Delta $$
on aboutit à
$$ \alpha \leq \frac{1}{n} \left(\frac{1}{q-1} \log \Delta - \log q \right)$$

Pour q=15, $\alpha \approx 0.24$.
\end{document}