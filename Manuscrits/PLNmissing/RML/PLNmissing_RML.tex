\documentclass[11pt,a4paper]{article}
\usepackage[utf8]{inputenc}
\usepackage{amsmath, mathtools}
\usepackage{amsfonts, dsfont}
\usepackage{amssymb}
\usepackage{graphicx}
\usepackage{empheq}

\newcommand*\widefbox[1]{\fbox{\hspace{3em}#1\hspace{3em}}}

\newcommand*\lesswidefbox[1]{\fbox{\hspace{2em}#1\hspace{2em}}}

\providecommand\given{} % is redefined in \Set
\newcommand\SetSymbol[1][]{\nonscript\:#1\vert\nonscript\:}
%\usepackage[mathcal]{eucal}
\usepackage[left=2cm,right=2cm,top=2cm,bottom=2cm]{geometry}

\setlength\parindent{0pt}
\author{Raphaelle Momal}
\title{PLN with missing actor}


\newcommand{\Esp}{\mathds{E}}
\begin{document}
\maketitle


We had: 
$$ p(Y,Z,T) = p(T)p(Z|T)p(Y|Z)$$
$$ \Esp(\log p(Y,Z,T)|Y) \approx \sum_{1 \leq j < k \leq p} P_{jk} \log\left(\beta_{jk} \hat{\psi}_{jk}\right) - \log B + cst$$

We want to add a hidden layer of unobserved data, indexed by H.

$$ p(Y,Z_O,Z_H,T)$$
\section{PLNmissing peculiarities: Model and distributions}
$$\left\{\begin{array}{rl}
T & \sim\prod_{jk \in T} \beta_{jk}/B \\\\

Z|T& \sim\mathcal{N}(0,\Sigma_T)\\\\

Y|Z&\sim\mathcal{P}( \exp( Z+...) )
\end{array} \right.$$

\paragraph{Marginals}
\begin{align*}
(Z_O,Z_H) &\sim \sum_{T \in \mathcal{T}} p(T) \mathcal{N}(0,\Sigma_T) \\
\end{align*}

\paragraph{Conditional on T:}
\begin{align*}
(Z_O,Z_H)|T & \sim\mathcal{N}(0,\Sigma_T)\\
Z_O|T & \sim\mathcal{N}(0,\Omega_{Tm}^{-1})\\
Z_H|T & \sim\mathcal{N}(0,\Omega_{TH}^{-1})
\end{align*}
With $ \Omega_{Tm} =\Sigma_{TO}^{-1} =  \Omega_{TO} - \Omega_{TOH}\Omega_{TH}^{-1}\Omega_{THO}$


\paragraph{Conditional on observed data:\\}


$Z_O$ decouples $Z_H$ and $Y_O$, we will never need $Z_H|Y_O$.
$$ p(Z|Y_O) = p(Z_O,Z_H | Y_O) = p(Z_H|Z_O) \: p(Z_O|Y_O) $$


VEM on observed data: $$Z_O|Y_O \approx \widetilde{Z}_O \sim \mathcal{N}(m,S)$$

\underline{ Hypothesis:}
$$ Z_O|Y,T \sim \mathcal{N}(m_T,S_T)$$

Model for the hidden part: $$Z_H|Z_O,T \sim \mathcal{N}(\mu_{H|O,T}, \Omega_{TH|O}^{-1})$$ 

\begin{itemize}
\item $\Omega_{TH|O}^{-1} = \Sigma_{TH} -\Sigma_{THO}\Sigma_{TO}^{-1}\Sigma_{TOH}$, 

\item$ \displaystyle\begin{aligned}[t]
\mu_{H|O,T} &= \Sigma_{THO}\Sigma_{TO}^{-1}((Z_O|T)-\underbrace{\mu_{Z_O|T}}_{0}) \\
 &= -\Omega_{TH}^{-1}\Omega_{TOH}((Z_O|T))
\end{aligned}$\\
\end{itemize}


\paragraph{Joint:}
\begin{align*}
p(Y,Z,T)& = p(T) \: p(Z|T) \: p(Y|Z) \\
&= p(T)\: p(Z_O,Z_H|T) \: p(Y|Z_O,Z_H) \\
&= p(T) \: p(Z_O|T) \: p(Z_H | Z_O,T)  \: p(Y|Z_O)
\end{align*} 

\section{Inference strategy}
$$p(Y,Z,T)= p(T) \: p(Z_O|T) \: p(Z_H | Z_O,T)  \: \underbrace{p(Y|Z_O)}_{\text{do not depend on T}}$$
\subsection{Observed count data Y}
This part aims at maximizing the likelihood of count data under the PLN model. 
$$p(Y|Z_O,Z_H) = p(Y|Z_O)$$
\subsection{Latent Gaussian layer Z}
This part aims at maximizing the underlying random tree of the Gaussian latent layer.
$$p(Z_O,Z_H,T) = p(T) \: p(Z_O|T) \: p(Z_H | Z_O,T)$$
\section{EM algorithm: E step}

Remarks:
\begin{itemize}
\item When all is observed we do MLE with EMtree, and in formulas $\psi$ contained maximized expression for $p(Z \mid Y)$. Here with missing data MLE is not possible but we might end up with a $\Psi$ gathering information about $ p(Z_O,Z_H|T,Y_O)$
\item The PLN gives us sufficient statistic for the Gaussian latent layer.There should be no problem using Genevieve's formulas until the very end where they should be conditioned on $Y_O$.
\item There exist no $Y_H$ variable. This is because In the latent variable paradigm, observed vraiables are modelled thanks to latent unobserved ones. Hence observed count data $Y$ is only $Y_O$, modelled thanks to $Z$ which structure includes a  missing part.
\item how many times to count an edge/vertices in sums 
\item do not forget the sum on all samples
\end{itemize}



We do not separate the Z vector:
$$p(Z_O,Z_H,T) = p(T) \: p(Z_O,Z_H|T)$$

For $n$ samples:
\begin{align*}
Q= \sum_{i=1}^{n} \Esp[\log p(Z_O,Z_H,T) | Y ] &= \sum_i  \Esp[\log p(T)+\log p(Z_O,Z_H|T)|Y] \\
  &=\sum_i \Esp\left[\sum_{jk \in T} \log \beta_{jk} - \log B + \frac{1}{2} \log(|\Omega_T|) - \frac{1}{2} Z^T \Omega_T Z\,\middle\vert\,  Y\right]\\
\end{align*} 


\subsection{Determinent of $\Omega_T$:\\}
$\Omega_T$ is decomposable along the tree structure. Denoting $c\in\mathcal{C}$ the cliques, $s\in \mathcal{S}$ the separator with multiplicity $\nu(s)$ and $[.]^\Gamma$ the O-completed matrix of size $(p+r)\times(p+r)$:
\begin{align*}
\Omega_T &= \sum_{c\in \mathcal{C}} [\Omega_{Tc}]^\Gamma - \sum_{s \in\mathcal{S}} \nu(s)[\Omega_{Ts}]^\Gamma\\
&= \sum_{jk \in T} [\Omega_{Tjk}]^\Gamma - \sum_j (deg(j)-1)[\Omega_{Tjj}]^\Gamma
\end{align*}

Hence, following Lauritzen p145, for all $(j,k) \in \{1,...,p+r\}^2$:
\begin{align*}
|\Omega_T| &= \dfrac{\prod_{jk \in T} |\Omega_T[jk,jk]|}{\prod_j (\omega_{Tjj})^{deg(j)}} \times \prod_j \omega_{Tjj}\\
&=\prod_j \omega_{Tjj} \times \prod_{jk \in T} \dfrac{\omega_{Tjj}\omega_{Tkk}-(\omega_{Tjk})^2}{\omega_{Tjj}\omega_{Tkk}}
\end{align*}


\begin{empheq}[box=\widefbox]{align*}
\log|\Omega_T|  &= \sum_j \log(\omega_{Tjj}) + \sum_{jk \in T}\log\left( 1- \frac{\omega_{Tjk} ^2}{\omega_{Tjj}\omega_{Tkk}}\right)
\end{empheq}

 

\subsection{Squared form terms:\\}
\begin{align*}
\Esp\left[-\frac{1}{2}\sum_i Z^T \Omega_T Z\,\middle\vert\,  Y,T\right]& =  \sum_i \Esp\left[-\sum_{jk \in T} \omega_{Tjk} Z_{ij} Z_{ik} - \frac{1}{2} \sum_j \omega_{Tjj}Z^2_{ij} \,\middle\vert\,  Y,T\right]\\
&=-\sum_i \underbrace{ \left(\sum_{jk \in T}\omega_{Tjk} \Esp\left[ Z_{ij} Z_{ik} \,\middle\vert\,  Y,T \right] \right)}_{\text{Off-diagonal}}-  \frac{1}{2} \sum_i \underbrace{\left( \sum_j\omega_{Tjj}\Esp\left[Z^2_{ij} \,\middle\vert\,  Y,T\right]\right)}_{\text{Diagonal}}
\end{align*}
Let's not forget that
 $$ p(Z_O,Z_H \mid Y_O) = p(Z_H\mid Z_O) p(Z_O\mid Y_O)$$
 Which leads to 
 \begin{align*}
 \Esp(Z_jZ_h \mid Y)& = \int Z_jZ_hp(Z_H\mid Z_O)p(Z_O\mid Y) dZ_HdZ_O\\
 &=\int Z_j\left(\int Z_h p(Z_H\mid Z_O) dZ_H\right) p(Z_O\mid Y) dZ_O \\
 &=\Esp_{Z_O\mid Y}\left[Z_j \Esp_{Z_H\mid Z_O}[Z_h]\right]
 \end{align*}

\paragraph{Off-diagonal terms when $j\in O, h\in H$:}
\begin{align*}
\Esp\left[ Z_{ij} Z_{ih} \,\middle\vert\,  Y_i,T \right] &=\Esp_{Z_{ij}\mid Y_i}\left[Z_{ij} \Esp_{Z_{ih}\mid  Z_{ij},T}[Z_{ih}]\right]
\end{align*}
With $$\Esp_{Z_{ih}\mid Z_{ij},T}[Z_{ih}] = -\omega_{Th}^{-1}\omega_{Tjh} Z_{ij}$$
We get
\begin{align*}
\Esp\left[ Z_{ij} Z_{ih} \,\middle\vert\,  Y_i,T \right] &= -\omega_{Th}^{-1}\omega_{Tjh} \Esp_{Z_{ij}\mid Y_i, T}\left[Z_{ij}^2  \right]\\
&=- \omega_{Th}^{-1}\omega_{Tjh} (m_{Tij}^2 + S_{Tij})
\end{align*}

\paragraph{Off-diagonal terms  when $(j,k)\in O\times O$:}
\begin{align*}
\Esp\left[ Z_{ij} Z_{ik} \,\middle\vert\,  Y_i,T \right] &=  m_{Tij}m_{Tik}\\
& \approx\Esp\left[Z_{ij} Z_{ik}  \,\middle\vert\,  Y_i\right]\\
\end{align*}
\paragraph{Diagonal observed terms:}
\begin{align*}
\Esp\left[ Z_{ij}^2 \,\middle\vert\,  Y_i,T \right] &= S_{Tjj}
\end{align*}
\paragraph{Diagonal hidden terms:\\}
We condition on any (or all ?) observed Z:
 \begin{align*}
\Esp\left[ Z_{ih}^2 \,\middle\vert\,  Y_i,T \right] & = \Esp_{Z_{iO}\mid Y_i}\left[ \Esp_{Z_{ih}\mid  Z_{iO},T}[Z_{ih}^2]\right]\\
&= \Esp_{Z_{iO}\mid Y_i}\left[ \Esp_{Z_{ih}\mid  Z_{iO},T}[Z_{ih}]^2 + Var(Z_{ih} \mid Z_{iO}, T)\right]\\
&= \Esp_{Z_{iO}\mid Y_i}\left[ \omega_{Th}^{-2}(\Omega_{TOh}Z_{O})^2 + \omega_{Th}^{-1}\right]\\
&= \omega_{Th}^{-1}+\omega_{Th}^{-2}\Esp_{Z_{iO}\mid Y_i}\left[ \Omega_{TOh}^TZ_{O}Z_{O}^T\Omega_{TOh} \right]\\
&= \omega_{Th}^{-1}+\omega_{Th}^{-2} \times\Omega_{TOh}^T(m_{Ti}m_{Ti}^T+S_{Ti})\Omega_{TOh}
\end{align*}

Compute the bilinear form:\\
With $M_TM_T^T+ \sum_i(S_{Ti}) = n\Sigma_T$
\begin{align*}
\sum_i\Omega_{TOh}^T (m_{Ti}^Tm_{Ti}+S_{Ti})\Omega_{TOh} &= n\sum_{jk}\Sigma_{Tjk} \;\omega_{Tjh}\omega_{Tkh}\\
&=n \sum_j \Sigma_{Tjj} \; \omega_{Tjh}^2 +2n\sum_{j < k} \Sigma_{Tjk} \; \omega_{Tjh} \: \omega_{Tkh}
\end{align*}
 Finally 
 \begin{align*}
 \sum_i\Esp\left[ Z_{ih}^2 \,\middle\vert\,  Y_i,T \right] &= n\omega_{Th}^{-1}+n\omega_{Th}^{-2} \left( \sum_j \Sigma_{Tjj} \omega_{Tjh}^2 +2\sum_{j < k} \Sigma_{Tjk} \: \omega_{Tjh} \: \omega_{Tkh}\right)
 \end{align*}
 
 \paragraph{Final form of square form terms:}
\begin{align*} 
\Esp\left[-\frac{1}{2}\sum_i Z^T \Omega_T Z\,\middle\vert\,  Y,T\right]&=-\sum_{jk\in T}\omega_{Tjk}\left(\sum_i\Esp[Z_{ij}Z_{ik}\mid Y_i,T]\right) + \sum_{jh\in T} \omega_{Tjh}\left(\sum_i\Esp[Z_{ij}^2\mid Y_i,T]\right) -\frac{n}{2} \sum_h 1 \\
& \;\;\;\;  -\frac{1}{2} \sum_j \omega_{Tj}\left(\sum_i\Esp[Z_{ij}^2\mid Y_i,T]\right)-\frac{1}{2} \sum_h \omega_{Th}^{-1} \times\Omega_{TOh}^T\left(\sum_i\Esp[Z_OZ_O^T\mid Y_i,T]\right)\Omega_{TOh}\\
\end{align*}

\begin{empheq}[box=\lesswidefbox]{align*} 
\Esp\left[-\frac{1}{2}\sum_i Z^T \Omega_T Z\,\middle\vert\,  Y,T\right]&=-n\sum_{j < k}\omega_{Tjk}\Sigma_{Tjk}+n \sum_{j < h} \omega_{Tjh}\Sigma_{Tjj} -\frac{n}{2} \sum_h 1 \\
& \;\;\;\;  -\frac{n}{2} \sum_j \omega_{Tj}\Sigma_{Tjj}-\frac{n}{2} \sum_h \omega_{Th}^{-1} \left( \sum_j \Sigma_{Tjj} \; \omega_{Tjh}^2 +2\sum_{j < k} \Sigma_{Tjk} \; \omega_{Tjh} \: \omega_{Tkh}\right)\\
\end{empheq}
 
 

\section{EM algorithm: M step}
\subsection{Off-diagonal terms}
\begin{align*}
\frac{\partial \log(|\Omega_T|)}{\partial \omega_{jk}} &= \frac{-2 \: \omega_{jk}}{\omega_j \omega_k -\omega_{jk}^2};
\end{align*}

\subsubsection{OO block: $(j,k)\in O\times O$}

\begin{align*}
\frac{\partial Q}{\partial \omega_{jk}} &= \frac{n}{2}\times \frac{-2 \: \omega_{jk}}{\omega_j \omega_k -\omega_{jk}^2}-n\Sigma_{Tjk}
\end{align*}
Setting the derivate to 0 to find a maximum :
\begin{align*}
& \omega_{Tjk}^2 = \omega_{Tj}\omega_{Tk} +\frac{\omega_{Tjk}}{\Sigma_{Tjk}}\\
\iff&\Sigma_{Tjk} \omega_{Tjk}^2 -\omega_{Tjk}-\Sigma_{Tjk}\omega_{Tj}\omega_{Tk} =0\\
\iff &\hat{\omega}_{Tjk} =\left(1 \pm \sqrt{1+4\Sigma_{Tjk}^2\omega_{Tj}\omega_{Tk}}\right) / 2\Sigma_{Tjk}
\end{align*}

We have the second order derivative:
\begin{align*}
\partial^2_{\omega_{Tjk}^2}Q = -n\left(\frac{2\omega_{Tj}\omega_{Tk} + \omega_{Tjk}/\Sigma_{Tjk}}{\omega_{Tjk}^2/\Sigma_{Tjk}^2}\right) <0 &\iff \frac{\omega_{Tjk}}{\Sigma_{kj}}>0\\
& \iff 1 \pm \sqrt{1+4\Sigma_{Tjk}^2\omega_{Tj}\omega_{Tk}} >0\\
&\iff \hat{\omega}_{Tjk} = \left(1 + \sqrt{1+4\Sigma_{Tjk}^2\omega_{Tj}\omega_{Tk}}\right) / 2\Sigma_{Tjk}
\end{align*}


\subsubsection{OH block: $(j,h)\in O\times H$}

\begin{align*}
\frac{\partial Q}{\partial \omega_{jh}}  &= -n\left( \frac{\omega_{Tjh}}{\omega_{Tj}\omega_{Th}-\omega_{Tjh}^2}+(\omega_{Th}^{-1}\Sigma_{Tjj})\;\omega_{Tjh} + \omega_{Th}^{-1}\underbrace{\sum_{j<k}(\Sigma_{Tjk}\;\omega_{Tkh})}_{\text{\large{$\varphi$}}}-\Sigma_{Tjj}\right)
\end{align*}

The quantity $\varphi$ accounts for all other neighbors except node $j$ of a hidden node. 
Setting the derivative to 0 to find a maximum:
\begin{align*}
&\omega_{Tjh} + \omega_{Tjh}(\omega_{Tj}\omega_{Th} - \omega_{Tjh}^2)\omega_{Th}^{-1}\Sigma_{jj} = (\Sigma_{Tjj} - \omega_{Th}^{-1} \varphi)(\omega_{Tj}\omega_{Th} - \omega_{Tjh}^2)\\
\iff & -(\omega_{Th}^{-1} \Sigma_{jj})\omega_{Tjh}^3 - (\omega_{Th}^{-1}\varphi - \Sigma_{Tjj})\omega_{Tjh}^2 + (1+\omega_{Tj}\omega_{Th}\omega_{Th  }^{-1}\Sigma_{Tjj})\omega_{Tjh} + \omega_{Tj}\omega_{Th}(\omega_{Th}^{-1}\varphi - \Sigma_{Tjj}) = 0\\
\iff  & -(\omega_{Th}^{-1} \Sigma_{jj}) \;\omega_{Tjh}^3 + ( \Sigma_{Tjj} - \omega_{Th}^{-1}\varphi)\;\omega_{Tjh}^2 + (1+\omega_{Tj} \Sigma_{Tjj})\;\omega_{Tjh} + \omega_{Tj}(\varphi - \omega_{Th}\Sigma_{Tjj}) = 0
\end{align*}
\subsection{Diagonal terms}
\begin{align*}
\frac{\partial \log(|\Omega_T|)}{\partial \omega_{Tj}} &= \frac{1}{\omega_{Tj}} - \sum_{k} \frac{1}{\omega_{Tj} - \omega_{Tj}^2 \frac{\omega_{Tk}}{\omega_{Tjk}^2}}\\
&= \frac{1}{\omega_{Tj}}\left( 1 + \sum_{k} \frac{\omega_{Tjk} ^2}{\omega_{Tj}\omega_{Tk} - \omega_{Tjk} ^2}  \right);
\end{align*}

\subsubsection{Observed ones: $j\in O $}


\begin{align*}
\frac{\partial Q}{\partial \omega_{Tj}}  &= \frac{n}{2}\times \frac{1}{\omega_{Tj}}\left( 1 + \sum_{k>j} \frac{1}{\dfrac{\omega_{Tj}\omega_{Tk}}{\omega_{Tjk} ^2} - 1}  \right) -\frac{n}{2} \Sigma_{Tjj}
\end{align*}




\subsubsection{Hidden ones: $h\in H$}
\begin{align*}
\frac{\partial Q}{\partial \omega_{Th}}  &= \frac{n}{2}\times \frac{1}{\omega_{Th}}\left(1 + \sum_{h<k} \frac{1}{\frac{\omega_{Th}\omega_{Tk}}{\omega_{Thk}^2}-1} + \frac{1}{\omega_{Th}} \left(\sum_{j \in O} \Sigma_{Tjj}\omega_{Tjh}^2 + 2\sum_{j<k \in O^2} \Sigma_{Tjk} \omega_{Tjh}\omega_{Tkh}\right) \right) 
\end{align*}
%%%%%%%%%%%%%%%%%%%%%%%%%%%%%%%%%%%%%%%%%
%%%%%%%%%%%%%%%%%%%%%%%%%%%%%%%%%%%%%%%%%

\newpage
\section{Appendix}

\subsection{Generalities}


\subsubsection{Reminder}
Remind that:
\begin{itemize}
\item if $x\sim\mathcal{N}(m, \Sigma)$, then
$\Esp[x^TAx] = Tr(A\Sigma) + m^TAm$
\item if Y and X are Gaussian, $Y|X \sim \mathcal{N}(\mu_Y+\Sigma_{YX}\Sigma_{XX}^{-1}(x-\mu_X) , \Sigma_{YY} - \Sigma_{YX}\Sigma_{XX}^{-1}\Sigma_{XY})$\\
If Y and X are also centered then $\mu_{Y|X} = \Sigma_{YX}\Sigma_{XX}^{-1} X$
\end{itemize}


\subsubsection{Gaussian tree-structured variable}
When $Y$ is tree-structured:
$$ p(Y|T) = \prod_i  p(Y_i|T) \prod_{jk\in T} \underbrace{\psi_{jk}(Y_j,Y_k)}_{\text{emp. mutual  information}}$$

If $Y|T$ is also Gaussian:

$$\log(p(Y|T)) = -\frac{n}{2}\log(|\Sigma_T|) - \frac{1}{2} \sum_i\sum_{jk\in T} Y_{ij} \Omega_{jk} Y_{ik}$$

During the E step of the EM we need to compute $\Esp[Y^T\Omega Y|Y_O]$
\subsubsection{General considerations toward computation}
\begin{align*}
 \Esp[Y^T\Omega Y|Y_O] &=\Esp\left[\sum_{jk\in T } Y_j\Omega_{jk}Y_k|Y_O\right]&\\
& = \sum_{jk\in O^2} Y_i\Omega_{jk} Y_k & \text{[OO]}\\
&+\sum_{jk \in H^2} \Esp[Y_j\Omega_{jj}Y_j|Y_O] & \text{[HH]} \\
&+2 \sum_{j\in O, k \in H} \Esp(Y_j\Omega_{jk} Y_k |Y_O) & \text{[OH]}\\
\end{align*}

\paragraph{[OO]} We know everyone
\paragraph{[HH]} There are still the quadratic terms to compute. 
$$ \Omega_{kk} \Esp[Y_k^2|Y_O] = \Omega_{kk} \left( \Esp [Y_k|Y_O]^2 + \mathds{V}(Y_k|Y_O)\right)$$

\paragraph{[OH]}$$ Y_j\Omega_{jk}\Esp[Y_k|Y_O]$$

\subsubsection{Schur's complement}

\[
 \Sigma=
  \left( {\begin{array}{cc}
  K_O &  K_{OH}\\\\
  K_{HO} & K_H
  \end{array} } \right)^{-1} =
  \left( {\begin{array}{cc}
  \Sigma_O =( K_O - K_{HO}K_H^{-1}K_{OH})^{-1} &  - \Sigma_O K_{OH}K_H^{-1}\\\\
 -K_H^{-1}K_{HO}\Sigma_O & K_H^{-1}+K_H^{-1}K_{OH}\Sigma_OK_{HO}K_H^{-1}
  \end{array} } \right)
  \]
\subsection{Computations}

For any function $f$ we have using the VEM for PLN:
$$\Esp(f(Z)|Y_O)  \approx \int f(Z) p(Z_H|Z_O)\tilde{p}(Z_O) dZ_H dZ_O$$
$$\Esp(f(Z|T)|Y_O)  \approx \int f(Z|T) p(T|Z_O,Z_H) p(Z_H|Z_O)\tilde{p}(Z_O) dT dZ_H dZ_O$$
\subsubsection{ function of Z}
\begin{align*}
\Esp(f(Z)|Y_O) &=\int f(Z) p(Z|Y_O) dZ\\
&=\int f(Z) p(Z_H|Z_O) p(Z_O|Y_O) dZ_H dZ_O\\
&\approx \int f(Z) p(Z_H|Z_O)\tilde{p}(Z_O) dZ_H dZ_O
\end{align*}


 Here we consider $f(Z) = Z^T\Omega Z$.
\begin{align*}
\widehat{\Esp}[f(Z)|Z_O] &= \tilde{\Esp}(Z_O^T\Omega_{OO}Z_O)\\
& +\tilde{\Esp}\left[(\Sigma_{HO}\Sigma_{OO}^{-1} Z_O)^T \Omega_{HO}Z_O)\right] +\tilde{\Esp}\left[(Z_O)^T \Omega_{OH} (\Sigma_{HO}\Sigma_{OO}^{-1} Z_O)\right]\\
&+ \tilde{\Esp}\left[\Esp_{Z_H|Z_O} [Z_H\Omega_{HH}Z_H]\right]
\end{align*}


\subsubsection{A function of $Z|T$}
\begin{align*}
 p(Z|T,Y_O) &= \frac{p(Z,T|Y_O)}{p(T|Y_O)}\\
 &= \frac{p(T|Z,Y_O) p(Z|Y_O)}{p(T|Y_O)}\\
 & \propto p(T|Z_O,Z_H) p(Z_H|Z_O) p(Z_O|Y_O)\\
 &\approx  p(T|Z_O,Z_H) p(Z_H|Z_O)  \tilde{p}(Z_O) 
\end{align*}
\begin{align*}
\Esp(f(Z|T)|Y_O) &=\int f(Z) p(Z|T,Y_O) dZ\\
&\approx \int f(Z) p(T|Z_O,Z_H)p(Z_H|Z_O)\tilde{p}(Z_O) dT dZ_H dZ_O
\end{align*}


\end{document}